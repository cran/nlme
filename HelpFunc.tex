\documentclass[pdftex]{article} \usepackage{url,graphicx}
\usepackage{times} \usepackage{alltt} \usepackage{amsbsy}
\usepackage{harvard} \usepackage{fancyheadings}
%\usepackage{color}
%\definecolor{Blue}{rgb}{0,0,0.8}
%\definecolor{Red}{rgb}{0.7,0,0}
\usepackage{S_PDF}
%\usepackage[bookmarksopen]{hyperref}
\usepackage{hyperref}
\hypersetup{pdftitle=lme and nlme}
\hypersetup{pdfauthor=J. C. Pinheiro and D. M. Bates}
\graphicspath{{figs/}}
\DeclareGraphicsExtensions{.pdf}
\setkeys{Gin}{width=\textwidth}

\newcommand{\HRule}{\rule{\linewidth}{2mm}}
\newcommand{\Hrule}{\rule{\linewidth}{1mm}}
\newcommand{\bm}[1]{{\boldsymbol {#1}}}
\newcommand{\bS}{\bm\Sigma}
\newcommand{\bL}{{\bm L}}
\renewcommand{\Twiddle}{\mbox{\(\sim\)}}

\begin{document}
% These have to come after AtBeginDocument used by hyperref.
\pdfpagewidth=175mm
\pdfpageheight=245mm

\thispagestyle{empty}
\vspace*{\stretch{1}}
\begin{flushleft}
  {\Large {\bf Functions and Methods for Mixed-Effects Models}}
\end{flushleft}
\HRule
{\large
\begin{flushright}
   Help Files \\
   Version 3.0\\
   June 1999
\end{flushright}
}

\vspace*{\stretch{2}}

\begin{flushleft}
  {\large \bf by Jos\'e C. Pinheiro and Douglas M. Bates}\\
  \Hrule\\
  Bell Labs, Lucent Technologies and University of Wisconsin --- Madison
\end{flushleft}

\newpage

\begin{Helpfile}{[.pdMat}{Subscript a pdMat Object}
This method function extracts sub-matrices from the positive-definite
matrix represented by \Co{x}.
\begin{Example}
x[i, j, drop]
x[i, j] <- value
\end{Example}
\begin{Argument}{ARGUMENTS}
\item[\Co{x:}]
an object inheriting from class \Co{pdMat} representing a
positive-definite matrix.
\item[\Co{i, j:}]
optional subscripts applying respectively to the rows
and columns of the positive-definite matrix represented by
\Co{object}. When \Co{i} (\Co{j}) is omitted, all rows (columns)
are extracted.
\item[\Co{drop:}]
a logical value. If \Co{TRUE}, single rows or columns are
converted to vectors. If \Co{FALSE} the returned value retains its
matrix representation.
\item[\Co{value:}]
a vector, or matrix, with the replacement values for the
relevant piece of the matrix represented by \Co{x}.
\end{Argument}
\Paragraph{VALUE}
if \Co{i} and \Co{j} are identical, the returned value will be
\Co{pdMat} object with the same class as \Co{x}. Otherwise, the
returned value will be a matrix. In the case a single row (or column)
is selected, the returned value may be converted to a vector,
according to the rules above.
\Paragraph{SEE ALSO}
\Co{[}, \Co{pdMat}
\need 15pt
\Paragraph{EXAMPLE}
\vspace{-16pt} 
\begin{Example}
pd1 <- pdSymm(diag(3))
pd1[1, drop = F]
pd1[1:2, 1:2] <- 3 * diag(2)
\end{Example}
\end{Helpfile}
\begin{Helpfile}{ACF}{Autocorrelation Function}
This function is generic; method functions can be written to handle
specific classes of objects. Classes which already have methods for
this function include: \Co{gls} and  \Co{lme}.
\begin{Example}
ACF(object, maxLag, ...)
\end{Example}
\begin{Argument}{ARGUMENTS}
\item[\Co{object:}]
any object from which an autocorrelation function can be
obtained. Generally an object resulting from a model fit, from which
residuals can be extracted.
\item[\Co{maxLag:}]
maximum lag for which the autocorrelation should be
calculated.
\item[\Co{...:}]
some methods for this generic require additional
arguments.
\end{Argument}
\Paragraph{VALUE}
will depend on the method function used; see the appropriate documentation.
\Paragraph{REFERENCES}
Box, G.E.P., Jenkins, G.M., and Reinsel G.C. (1994) "Time Series
Analysis: Forecasting and Control", 3rd Edition, Holden-Day.
\Paragraph{SEE ALSO}
\Co{ACF.gls}, \Co{ACF.lme}
\need 15pt
\Paragraph{EXAMPLE}
\vspace{-16pt}
\begin{Example}
## see the method function documentation
\end{Example}
\end{Helpfile}
\begin{Helpfile}{ACF.gls}{Autocorrelation Function for gls Residuals}
This method function calculates the empirical autocorrelation function
for the residuals from an \Co{gls} fit. If a grouping variable is
specified in \Co{form}, the autocorrelation values
are calculated using pairs of residuals within the same group;
otherwise all possible residual pairs are used. The autocorrelation
function is useful for investigating serial correlation models for
equally spaced data.
\begin{Example}
ACF(object, maxLag, resType, form, na.action)
\end{Example}
\begin{Argument}{ARGUMENTS}
\item[\Co{object:}]
an object inheriting from class \Co{gls}, representing
a generalized least squares fitted  model.
\item[\Co{maxLag:}]
an optional integer giving the maximum lag for which the
autocorrelation should be calculated. Defaults to maximum lag in the
residuals.
\item[\Co{resType:}]
an optional character string specifying the type of
residuals to be used. If \Co{"response"}, the "raw" residuals
(observed - fitted) are used; else, if \Co{"pearson"}, the
standardized residuals (raw residuals divided by the corresponding
standard errors) are used; else, if \Co{"normalized"}, the
normalized residuals (standardized residuals pre-multiplied by the
inverse square-root factor of the estimated error correlation
matrix) are used. Partial matching of arguments is used, so only the
first character needs to be provided. Defaults to \Co{"pearson"}.
\item[\Co{form:}]
an optional one sided formula of the form \Co{{\Twiddle} t}, or
\Co{{\Twiddle} t | g}, specifying a time covariate \Co{t} and,  optionally, a
grouping factor \Co{g}. The time covariate must be integer
valued. When a grouping factor is present in 
\Co{form}, the autocorrelations are calculated using residual pairs
within the same group. Defaults to \Co{{\Twiddle} 1}, which corresponds to
using the order of the observations in the data as a covariate, and
no groups.
\item[\Co{na.action:}]
a function that indicates what should happen when the
data contain \Co{NA}s.  The default action (\Co{na.fail}) causes
\Co{ACF.gls} to print an error message and terminate if there are any
incomplete observations.
\end{Argument}
\Paragraph{VALUE}
a data frame with columns \Co{lag} and \Co{ACF} representing,
respectively, the lag between residuals within a pair and the corresponding
empirical autocorrelation. The returned value inherits from class
\Co{ACF}.
\Paragraph{REFERENCES}
Box, G.E.P., Jenkins, G.M., and Reinsel G.C. (1994) "Time Series
Analysis: Forecasting and Control", 3rd Edition, Holden-Day.
\Paragraph{SEE ALSO}
\Co{ACF.gls}, \Co{plot.ACF}
\need 15pt
\Paragraph{EXAMPLE}
\vspace{-16pt}
\begin{Example}
fm1 <- gls(follicles {\Twiddle} sin(2*pi*Time) + cos(2*pi*Time), Ovary)
ACF(fm1, form = {\Twiddle} 1 | Mare)
\end{Example}
\end{Helpfile}
\begin{Helpfile}{ACF.lme}{Autocorrelation Function for lme Residuals}
This method function calculates the empirical autocorrelation function
for the within-group residuals from an \Co{lme} fit. The
autocorrelation values are calculated using pairs of residuals within
the innermost group level. The autocorrelation function is  useful for
investigating serial correlation models for equally spaced data.
\begin{Example}
ACF(object, maxLag, resType)
\end{Example}
\begin{Argument}{ARGUMENTS}
\item[\Co{object:}]
an object inheriting from class \Co{lme}, representing
a fitted linear mixed-effects model.
\item[\Co{maxLag:}]
an optional integer giving the maximum lag for which the
auntocorrelation should be calculated. Defaults to maximum lag in the
withnin-group residuals.
\item[\Co{resType:}]
an optional character string specifying the type of
residuals to be used. If \Co{"response"}, the "raw" residuals
(observed - fitted) are used; else, if \Co{"pearson"}, the
standardized residuals (raw residuals divided by the corresponding
standard errors) are used; else, if \Co{"normalized"}, the
normalized residuals (standardized residuals pre-multiplied by the
inverse square-root factor of the estimated error correlation
matrix) are used. Partial matching of arguments is used, so only the
first character needs to be provided. Defaults to \Co{"pearson"}.
\end{Argument}
\Paragraph{VALUE}
a data frame with columns \Co{lag} and \Co{ACF} representing,
respectively, the lag between residuals within a pair and the corresponding
empirical autocorrelation. The returned value inherits from class
\Co{ACF}.
\Paragraph{REFERENCES}
Box, G.E.P., Jenkins, G.M., and Reinsel G.C. (1994) "Time Series
Analysis: Forecasting and Control", 3rd Edition, Holden-Day.
\Paragraph{SEE ALSO}
\Co{ACF.gls}, \Co{plot.ACF}
\need 15pt
\Paragraph{EXAMPLE}
\vspace{-16pt}
\begin{Example}
fm1 <- lme(follicles {\Twiddle} sin(2*pi*Time) + cos(2*pi*Time), Ovary,
           random = {\Twiddle} sin(2*pi*Time) | Mare)
ACF(fm1, maxLag = 11)
\end{Example}
\end{Helpfile}
\begin{Helpfile}{AIC}{Akaike Information Criterion}
  This generic function calculates the Akaike information criterion
  for one or several fitted model objects for which a log-likelihood
  value can be obtained, according to the formula $-2logLik +
  2n_{par}$, where $n_{par}$ represents the number of parameters in
  the fitted model. When comparing fitted objects, the smaller the
  AIC, the better the fit.
\begin{Example}
AIC(object, ...)
\end{Example}
\begin{Argument}{ARGUMENTS}
\item[\Co{object:}]
a fitted model object, for which there exists a
\Co{logLik} method to extract the corresponding log-likelihood, or
an object inheriting from class \Co{logLik}.
\item[\Co{...:}]
optional fitted model objects.
\end{Argument}
\Paragraph{VALUE}
if just one object is provided, returns a numeric value
with the corresponding AIC; if more than one object are provided,
returns a \Co{data.frame} with rows corresponding to the objects and
columns representing the number of parameters in the model
(\Co{df}) and the AIC.
\Paragraph{REFERENCES}
Sakamoto, Y., Ishiguro, M., and Kitagawa G. (1986) "Akaike
Information Criterion Statistics", D. Reidel Publishing Company.
\Paragraph{SEE ALSO}
\Co{logLik}, \Co{BIC}, \Co{AIC.logLik}
\need 15pt
\Paragraph{EXAMPLE}
\vspace{-16pt} 
\begin{Example}
fm1 <- lm(distance \Twiddle age, data = Orthodont) # no random effects
fm2 <- lme(distance \Twiddle age, data = Orthodont) # random is \Twiddle age
AIC(fm1, fm2)
\end{Example}
\end{Helpfile}
\begin{Helpfile}{AIC.logLik}{AIC of a logLik Object}
  This function calculates the Akaike information criterion for an
  object inheriting from class \Co{logLik}, according to the formula
  $-2logLik + 2n_{par}$, where $n_{par}$ represents the number of
  parameters in the fitted model. When comparing fitted objects, the
  smaller the AIC, the better the fit.
\begin{Example}
AIC(object)
\end{Example}
\begin{Argument}{ARGUMENTS}
\item[\Co{object:}]
an object inheriting from class \Co{logLik}, usually
resulting from applying a \Co{logLik} method to a fitted model
object.
\end{Argument}
\Paragraph{VALUE}
a numeric value with the corresponding AIC.
\Paragraph{REFERENCES}
Sakamoto, Y., Ishiguro, M., and Kitagawa G. (1986) "Akaike
Information Criterion Statistics", D. Reidel Publishing Company.
\Paragraph{SEE ALSO}
\Co{AIC}, \Co{logLik}, \Co{BIC}
\need 15pt
\Paragraph{EXAMPLE}
\vspace{-16pt} 
\begin{Example}
fm1 <- lm(distance \Twiddle age, data = Orthodont) 
AIC(logLik(fm1))
\end{Example}
\end{Helpfile}
\begin{Helpfile}{allCoef}{Extract Coefficients from a Set of Objects}
The extractor function is applied to each object in \Co{...}, with
the result being converted to a vector. A \Co{map} attribute is
included to indicate which pieces of the returned vector correspond to
the original objects in \Co{...}.
\begin{Example}
allCoef(..., extract)
\end{Example}
\begin{Argument}{ARGUMENTS}
\item[\Co{...:}]
objects to which \Co{extract} will be applied. Generally
these will be model components, such as \Co{corStruct} and
\Co{varFunc} objects.
\item[\Co{extract:}]
an optional extractor function. Defaults to \Co{coef}.
\end{Argument}
\Paragraph{VALUE}
a vector with all elements, generally coefficients, obtained by
applying \Co{extract} to the objects in \Co{...}.
\Paragraph{SEE ALSO}
\Co{modelStruct}
\need 15pt
\Paragraph{EXAMPLE}
\vspace{-16pt} 
\begin{Example}
cs1 <- corAR1(0.1)
vf1 <- varPower(0.5)
allCoef(cs1, vf1)
\end{Example}
\end{Helpfile}
\begin{Helpfile}{anova.gls}{Compare Likelihoods of Fitted Objects}
When only one fitted model object is present, a data frame with the
sums of squares, numerator degrees of freedom, F-values, and P-values
for Wald tests for the terms in the model (when \Co{terms} and
\Co{L} are \Co{NULL}), a combination of model terms (when
\Co{terms} in not \Co{NULL}), or linear combinations of the model
coefficients (when \Co{L} is not \Co{NULL}). Otherwise, when
multiple fitted objects are being compared, a data frame with
the degrees of freedom, the (restricted) log-likelihood, the 
Akaike Information Criterion (AIC), and the Bayesian Information
Criterion (BIC) of each object is returned. If \Co{test=TRUE},
whenever two consecutive  objects have different number of degrees of
freedom, a likelihood ratio statistic, with the associated p-value is
included in the returned data frame.
\begin{Example}
anova(object, ..., test, type, adjustSigma, terms, L, verbose)
\end{Example}
\begin{Argument}{ARGUMENTS}
\item[\Co{object:}]
a fitted model object inheriting from class \Co{gls},
representing a generalized least squares fit.
\item[\Co{...:}]
other optional fitted model objects inheriting from
classes \Co{gls}, \Co{gnls}, \Co{lm}, \Co{lme},
\Co{lmList}, \Co{nlme}, \Co{nlsList}, or \Co{nls}.
\item[\Co{test:}]
an optional logical value controlling whether likelihood
ratio tests should be used to compare the fitted models represented
by \Co{object} and the objects in \Co{...}. Defaults to
\Co{TRUE}.
\item[\Co{type:}]
an optional character string specifying the type of sum of
squares to be used in F-tests for the terms in the model. If 
\Co{"sequential"}, the sequential sum of squares obtained by
including the terms in the order they appear in the model is used;
else, if \Co{"marginal"}, the marginal sum of squares
obtained by deleting a term from the model at a time is used. This
argument is only used when a single fitted object is passed to the
function. Partial matching of arguments is used, so only the first
character needs to be provided. Defaults to \Co{"sequential"}.
\item[\Co{adjustSigma:}] an optional logical value. If \Co{TRUE} and
  the estimation method used to obtain \Co{object} was maximum
  likelihood, the residual standard error is multiplied by
  $\sqrt{n_{obs}/(n_{obs} - n_{par})}$, where $n_{par}$ represents the
  number of coefficients and $n_{obs}$ the number of observations in
  the fitted model, converting it to a REML-like estimate. This
  argument is only used when a single fitted object is passed to the
  function. Default is \Co{TRUE}.
\item[\Co{terms:}]
an optional integer of character vector specifying which
terms in the model should be jointly tested to be zero using a Wald
F-test. If given as a character vector, its elements must correspond
to term names; else, if given as an integer vector, its elements must
correspond to the order in which terms are included in the
model. This argument is only used when a single fitted object is
passed to the function. Default is \Co{NULL}.
\item[\Co{L:}]
an optional numeric vector or array specifying linear
combinations of the coefficients in the model that should be tested
to be zero. If given as an array, its rows define the linear
combinations to be tested. If names are assigned to the vector
elements (array columns), they must correspond to coefficients
names and will be used to map the linear combination(s) to the
coefficients; else, if no names are available, the vector elements
(array columns) are assumed in the same order as the coefficients
appear in the model. This argument is only used when a single fitted
object is passed to the function. Default is \Co{NULL}.
\item[\Co{verbose:}]
an optional logical value. If \Co{TRUE}, the calling
sequences for each fitted model object are printed with the rest of
the output, being omitted if \Co{verbose = FALSE}. Defaults to
\Co{FALSE}.
\end{Argument}
\Paragraph{VALUE}
a data frame inheriting from class \Co{anova.lme}.
\Paragraph{NOTE} Likelihood comparisons are not meaningful for objects fit using 
restricted maximum likelihood and with different fixed effects.
\Paragraph{SEE ALSO}
\Co{gls}, \Co{gnls}, \Co{lme},
\Co{AIC}, \Co{BIC}, \Co{print.anova.lme}
\need 15pt
\Paragraph{EXAMPLE}
\vspace{-16pt}
\begin{Example}
# AR(1) errors within each Mare
fm1 <- gls(follicles {\Twiddle} sin(2*pi*Time) + cos(2*pi*Time), Ovary,
           correlation = corAR1(form = {\Twiddle} 1 | Mare))
anova(fm1)
# variance changes with a power of the absolute fitted values?
fm2 <- update(fm1, weights = varPower())
anova(fm1, fm2)
\end{Example}
\end{Helpfile}
\begin{Helpfile}{anova.lme}{Compare Likelihoods of Fitted Objects}
When only one fitted model object is present, a data frame with the
sums of squares, numerator degrees of freedom, denominator degrees of
freedom, F-values, and P-values for Wald tests for the terms in the
model (when \Co{terms} and \Co{L} are \Co{NULL}), a combination
of model terms (when \Co{terms} in not \Co{NULL}), or linear
combinations of the model coefficients (when \Co{L} is not
\Co{NULL}). Otherwise, when multiple fitted objects are being
compared, a data frame with the degrees of freedom, the (restricted)
log-likelihood, the Akaike Information Criterion (AIC), and the
Bayesian Information Criterion (BIC) of each object is returned. If
\Co{test=TRUE}, whenever two consecutive  objects have different
number of degrees of freedom, a likelihood ratio statistic, with the
associated p-value is included in the returned data frame.
\begin{Example}
anova(object, ..., test, type, adjustSigma, terms, L, verbose)
\end{Example}
\begin{Argument}{ARGUMENTS}
\item[\Co{object:}]
a fitted model object inheriting from class \Co{lme},
representing a mixed-effects model.
\item[\Co{...:}]
other optional fitted model objects inheriting from
classes \Co{gls}, \Co{gnls}, \Co{lm}, \Co{lme},
\Co{lmList}, \Co{nlme}, \Co{nlsList}, or \Co{nls}.
\item[\Co{test:}]
an optional logical value controlling whether likelihood
ratio tests should be used to compare the fitted models represented
by \Co{object} and the objects in \Co{...}. Defaults to
\Co{TRUE}.
\item[\Co{type:}]
an optional character string specifying the type of sum of
squares to be used in F-tests for the terms in the model. If 
\Co{"sequential"}, the sequential sum of squares obtained by
including the terms in the order they appear in the model is used;
else, if \Co{"marginal"}, the marginal sum of squares
obtained by deleting a term from the model at a time is used. This
argument is only used when a single fitted object is passed to the
function. Partial matching of arguments is used, so only the first
character needs to be provided. Defaults to \Co{"sequential"}.
\item[\Co{adjustSigma:}] an optional logical value. If \Co{TRUE} and
  the estimation method used to obtain \Co{object} was maximum
  likelihood, the residual standard error is multiplied by
  $\sqrt{n_{obs}/(n_{obs} - n_{par})}$, where $n_{par}$ represents the
  number of coefficients and $n_{obs}$ the number of observations in
  the fitted model, converting it to a REML-like estimate. This
  argument is only used when a single fitted object is passed to the
  function. Default is \Co{TRUE}.
\item[\Co{terms:}]
an optional integer of character vector specifying which
terms in the model should be jointly tested to be zero using a Wald
F-test. If given as a character vector, its elements must correspond
to term names; else, if given as an integer vector, its elements must
correspond to the order in which terms are included in the
model. This argument is only used when a single fitted object is
passed to the function. Default is \Co{NULL}.
\item[\Co{L:}]
an optional numeric vector or array specifying linear
combinations of the coefficients in the model that should be tested
to be zero. If given as an array, its rows define the linear
combinations to be tested. If names are assigned to the vector
elements (array columns), they must correspond to coefficients
names and will be used to map the linear combination(s) to the
coefficients; else, if no names are available, the vector elements
(array columns) are assumed in the same order as the coefficients
appear in the model. This argument is only used when a single fitted
object is passed to the function. Default is \Co{NULL}.
\item[\Co{verbose:}]
an optional logical value. If \Co{TRUE}, the calling
sequences for each fitted model object are printed with the rest of
the output, being omitted if \Co{verbose = FALSE}. Defaults to
\Co{FALSE}.
\end{Argument}
\Paragraph{VALUE}
a data frame inheriting from class \Co{anova.lme}.
\Paragraph{NOTE} Likelihood comparisons are not meaningful for objects fit using
restricted maximum likelihood and with different fixed effects.
\Paragraph{SEE ALSO}
\Co{gls}, \Co{gnls}, \Co{nlme},
\Co{lme}, \Co{AIC}, \Co{BIC},
\Co{print.anova.lme}
\need 15pt
\Paragraph{EXAMPLE}
\vspace{-16pt}
\begin{Example}
fm1 <- lme(distance {\Twiddle} age, Orthodont, random = {\Twiddle} age | Subject)
anova(fm1)
fm2 <- update(fm1, random = pdDiag({\Twiddle}age))
anova(fm1, fm2)
\end{Example}
\end{Helpfile}
\begin{Helpfile}{as.matrix.corStruct}{Matrix of a corStruct Object}
This method function extracts the correlation matrix, or list of
correlation matrices, associated with \Co{object}.
\begin{Example}
as.matrix(x)
\end{Example}
\begin{Argument}{ARGUMENTS}
\item[\Co{x:}]
an object inheriting from class \Co{corStruct},
representing a correlation structure.
\end{Argument}
\Paragraph{VALUE}
If the correlation structure includes a grouping factor, the returned
value will be a list with components given by the correlation
matrices for each group. Otherwise, the returned value will be a
matrix representing the correlation structure associated with
\Co{object}.
\Paragraph{SEE ALSO}
\Co{corClasses}, \Co{corMatrix}
\need 15pt
\Paragraph{EXAMPLE}
\vspace{-16pt} 
\begin{Example}
cst1 <- corAR1(form = \Twiddle 1|Subject)
cst1 <- initialize(cst1, data = Orthodont)
as.matrix(cst1)
\end{Example}
\end{Helpfile}
\begin{Helpfile}{as.matrix.pdMat}{Matrix of a pdMat Object}
This method function extracts the positive-definite matrix represented
by \Co{x}.
\begin{Example}
as.matrix(x)
\end{Example}
\begin{Argument}{ARGUMENTS}
\item[\Co{x:}]
an object inheriting from class \Co{pdMat}, representing a
positive-definite matrix.
\end{Argument}
\Paragraph{VALUE}
a matrix corresponding to the positive-definite matrix represented by
\Co{x}.
\Paragraph{SEE ALSO}
\Co{pdMat}, \Co{corMatrix}
\need 15pt
\Paragraph{EXAMPLE}
\vspace{-16pt} 
\begin{Example}
as.matrix(pdSymm(diag(4)))
\end{Example}
\end{Helpfile}
\begin{Helpfile}{as.matrix.reStruct}{Matrices of an reStruct Object}
This method function extracts the positive-definite matrices
corresponding to the \Co{pdMat} elements of \Co{object}.
\begin{Example}
as.matrix(object)
\end{Example}
\begin{Argument}{ARGUMENTS}
\item[\Co{object:}]
an object inheriting from class \Co{reStruct},
representing a random effects structure and consisting of a list of
\Co{pdMat} objects.
\end{Argument}
\Paragraph{VALUE}
a list with components given by the positive-definite matrices
corresponding to the elements of \Co{object}.
\Paragraph{SEE ALSO}
\Co{as.matrix.pdMat}, \Co{reStruct},
\Co{pdMat}
\need 15pt
\Paragraph{EXAMPLE}
\vspace{-16pt} 
\begin{Example}
rs1 <- reStruct(pdSymm(diag(3), form=\Twiddle Sex+age, data=Orthodont))
as.matrix(rs1)
\end{Example}
\end{Helpfile}
\begin{Helpfile}{asOneFormula}{Combine Formulas of a Set of Objects}
The names of all variables used in the formulas extracted from the
objects defined in \Co{...} are converted into a single linear
formula, with the variables names separated by \Co{+}.
\begin{Example}
asOneFormula(..., omit)
\end{Example}
\begin{Argument}{ARGUMENTS}
\item[\Co{...:}]
objects, or lists of objects, from which a formula can be
extracted.
\item[\Co{omit:}]
an optional character vector with the names of variables to
be omitted from the returned formula. Defaults to \Co{c(".", "pi")}.
\end{Argument}
\Paragraph{VALUE}
a one-sided linear formula with all variables named in the formulas
extracted from the objects in \Co{...}, except the ones listed in
\Co{omit}.
\Paragraph{SEE ALSO}
\Co{formula}, \Co{all.vars}
\need 15pt
\Paragraph{EXAMPLE}
\vspace{-16pt} 
\begin{Example}
asOneFormula(y \Twiddle x + z | g, list(\Twiddle w, \Twiddle t * sin(2 * pi)))
\end{Example}
\end{Helpfile}
\begin{Helpfile}{asOneSidedFormula}{Convert to One-Sided Formula}
Names, expressions, and strings are converted to one-sided
formulas. If \Co{object} is a formula, it must be one-sided, in
which case it is returned unaltered.
\begin{Example}
asOneSidedFormula(object)
\end{Example}
\begin{Argument}{ARGUMENTS}
\item[\Co{object:}]
a one-sided formula, an expression, a numeric value, or a character
string.
\end{Argument}
\Paragraph{VALUE}
a one-sided formula representing \Co{object}
\Paragraph{SEE ALSO}
\Co{formula}
\need 15pt
\Paragraph{EXAMPLE}
\vspace{-16pt} 
\begin{Example}
asOneSidedFormula("age")
asOneSidedFormula(\Twiddle age)
\end{Example}
\end{Helpfile}
\begin{Helpfile}{asTable}{Convert groupedData to a matrix}
Create a tabular representation of the response in a balanced
groupedData object.
\begin{Example}
asTable(object)
\end{Example}
\begin{Argument}{ARGUMENTS}
\item[\Co{object:}]
A balanced \Co{groupedData} object
\end{Argument}
\Paragraph{VALUE}
A matrix.  The data in the matrix are the values of the response.  The
columns correspond to the distinct values of the primary covariate and
are labelled as such.  The rows correspond to the distinct levels of
the grouping factor and are labelled as such.
\Paragraph{SEE ALSO}
\Co{groupedData}, \Co{isBalanced},
\Co{balancedGrouped}
\need 15pt
\Paragraph{EXAMPLE}
\vspace{-16pt} 
\begin{Example}
asTable( Orthodont )
\end{Example}
\end{Helpfile}
\begin{Helpfile}{augPred}{Augmented Predictions}
Predicted values are obtained at the specified values of
\Co{primary}. If \Co{object} has a grouping structure
(i.e.\ \Co{getGroups(object)} is not \Co{NULL}), predicted values
are obtained for each group. If \Co{level} has more than one
element, predictions are obtained for each level of the
\Co{max(level)} grouping factor. If other covariates besides
\Co{primary} are used in the prediction model, their average
(numeric covariates) or most frequent value (categorical covariates)
are used to obtain the predicted values. The original observations are
also included in the returned object.
\begin{Example}
augPred(object, primary, minimum, maximum, length.out, level, ...)
\end{Example}
\begin{Argument}{ARGUMENTS}
\item[\Co{object:}]
a fitted model object from which predictions can be
extracted, using a \Co{predict} method.
\item[\Co{primary:}]
an optional one-sided formula specifying the primary
covariate to be used to generate the augmented predictions. By
default, if a  covariate can be extracted from the data used to generate
\Co{object} (using \Co{getCovariate}), it will be used as
\Co{primary}.
\item[\Co{minimum:}]
an optional lower limit for the primary
covariate. Defaults to \Co{min(primary)}.
\item[\Co{maximum:}]
an optional upper limit for the primary
covariate. Defaults to \Co{max(primary)}.
\item[\Co{length.out:}]
an optional integer with the number of primary
covariate values at which to evaluate the predictions. Defaults to
51.
\item[\Co{level:}]
an optional integer vector specifying the desired
prediction levels. Levels increase from outermost to innermost
grouping, with level 0 representing the population (fixed effects)
predictions. Defaults to the innermost level.
\item[\Co{...:}]
some methods for the generic may require additional
arguments.
\end{Argument}
\Paragraph{VALUE}
a data frame with four columns representing, respectively, the values
of the primary covariate, the groups (if \Co{object} does not have a
grouping structure, all elements will be \Co{1}), the predicted or
observed values, and the type of value in the third column:
\Co{original} for the observed values and \Co{predicted} (single
or no grouping factor) or \Co{predict.groupVar} (multiple levels of
grouping), with \Co{groupVar} replaced by the actual grouping
variable name (\Co{fixed} is used for population predictions). The
returned object inherits from class \Co{augPred}.
\Paragraph{NOTE} This function is generic; method functions can be written to handle
specific classes of objects. Classes which already have methods for
this function include: \Co{gls}, \Co{lme}, and \Co{lmList}.
\Paragraph{SEE ALSO}
\Co{plot.augPred}, \Co{getGroups},
\Co{predict}
\need 15pt
\Paragraph{EXAMPLE}
\vspace{-16pt} 
\begin{Example}
fm1 <- lme(Orthodont)
augPred(fm1, length.out = 2, level = c(0,1))
\end{Example}
\end{Helpfile}
\begin{Helpfile}{balancedGrouped}{Create a groupedData object from a matrix}
Create a \Co{groupedData} object from a data matrix.  This function
can only be used with balanced grouped data that will be representable
as a matrix.  The opposite conversion (\Co{groupedData} to
\Co{matrix}) is performed by \Co{asTable}.
\begin{Example}
balancedGrouped(form, data, labels=NULL, units=NULL)
\end{Example}
\begin{Argument}{ARGUMENTS}
\item[\Co{form:}]
A formula of the form \Co{y {\Twiddle} x | g} giving the name of
the response, the primary covariate, and the grouping factor.
\item[\Co{data:}]
A matrix or data frame containing the values of the
response grouped according to the levels of the grouping factor
(rows) and the distinct levels of the primary covariate (columns).
The \Co{dimnames} of the matrix are used to construct the levels of
the grouping factor and the primary covariate.
\item[\Co{labels:}]
an optional list of character strings giving labels for
the response and the primary covariate.  The label for the primary
covariate is named \Co{x} and that for the response is named
\Co{y}.  Either label can be omitted.
\item[\Co{units:}]
an optional list of character strings giving the units for
the response and the primary covariate.  The units string for the
primary covariate is named \Co{x} and that for the response is
named \Co{y}.  Either units string can be omitted.
\end{Argument}
\Paragraph{VALUE}
A balanced \Co{groupedData} object.
\Paragraph{SEE ALSO}
\Co{groupedData},\Co{isBalanced},\Co{asTable}
\need 15pt
\Paragraph{EXAMPLE}
\vspace{-16pt} 
\begin{Example}
OrthoMat <- asTable( Orthodont )
Orth2 <- balancedGrouped(distance ~ age | Subject, data = OrthoMat,
    labels = list(x = "Age",
       y = "Distance from pituitary to pterygomaxillary fissure"),
    units = list(x = "(yr)", y = "(mm)"))
Orth2[ 1:10, ]        ## check the first few entries
\end{Example}
\end{Helpfile}
\begin{Helpfile}{BIC}{Bayesian Information Criterion}
  This generic function calculates the Bayesian information criterion,
  also known as Schwarz's Bayesian criterion (SBC), for one or several
  fitted model objects for which a log-likelihood value can be
  obtained, according to the formula $-2logLik + n_{par}\log(n_{obs})$,
  where $n_{par}$ represents the number of parameters and $n_{obs}$
  the number of observations in the fitted model.
\begin{Example}
BIC(object, ...)
\end{Example}
\begin{Argument}{ARGUMENTS}
\item[\Co{object:}]
a fitted model object, for which there exists a
\Co{logLik} method to extract the corresponding log-likelihood, or
an object inheriting from class \Co{logLik}.
\item[\Co{...:}]
optional fitted model objects.
\end{Argument}
\Paragraph{VALUE}
if just one object is provided, returns a numeric value with the
corresponding BIC; if more than one object are provided, returns a
\Co{data.frame} with rows corresponding to the objects and columns
representing the number of parameters in the model (\Co{df}) and the
BIC.
\Paragraph{REFERENCES}
Schwarz, G. (1978) "Estimating the Dimension of a Model", Annals of
Statistics, 6, 461-464.
\Paragraph{SEE ALSO}
\Co{logLik}, \Co{AIC}, \Co{BIC.logLik}
\need 15pt
\Paragraph{EXAMPLE}
\vspace{-16pt} 
\begin{Example}
fm1 <- lm(distance \Twiddle age, data = Orthodont) # no random effects
fm2 <- lme(distance \Twiddle age, data = Orthodont) # random is \Twiddle age
BIC(fm1, fm2)
\end{Example}
\end{Helpfile}
\begin{Helpfile}{BIC.logLik}{BIC of a logLik Object}
  This function calculates the Bayesian information criterion, also
  known as Schwarz's Bayesian criterion (SBC) for an object inheriting
  from class \Co{logLik}, according to the formula $-2logLik +
  n_{par}\log(n_{obs})$, where $n_{par}$ represents the number of
  parameters and $n_{obs}$ the number of observations in the fitted
  model. When comparing fitted objects, the smaller the BIC, the
  better the fit.
\begin{Example}
BIC(object)
\end{Example}
\begin{Argument}{ARGUMENTS}
\item[\Co{object:}]
an object inheriting from class \Co{logLik}, usually
resulting from applying a \Co{logLik} method to a fitted model
object.
\end{Argument}
\Paragraph{VALUE}
a numeric value with the corresponding BIC.
\Paragraph{REFERENCES}
Schwarz, G. (1978) "Estimating the Dimension of a Model", Annals of
Statistics, 6, 461-464.
\Paragraph{SEE ALSO}
\Co{BIC}, \Co{logLik}, \Co{AIC}
\need 15pt
\Paragraph{EXAMPLE}
\vspace{-16pt} 
\begin{Example}
fm1 <- lm(distance \Twiddle age, data = Orthodont) 
BIC(logLik(fm1))
\end{Example}
\end{Helpfile}
\begin{Helpfile}{coef.corStruct}{Coefficients of a corStruct Object}
This method function extracts the coefficients associated with the
correlation structure represented by \Co{object}.
\begin{Example}
coef(object, unconstrained)
coef(object) <- value
\end{Example}
\begin{Argument}{ARGUMENTS}
\item[\Co{object:}]
an object inheriting from class \Co{corStruct},
representing a correlation structure.
\item[\Co{unconstrained:}]
a logical value. If \Co{TRUE} the coefficients
are returned in unconstrained form (the same used in the optimization
algorithm). If \Co{FALSE} the coefficients are returned in
"natural", possibly constrained, form. Defaults to \Co{TRUE}.
\item[\Co{value:}]
a vector with the replacement values for the coefficients
associated with \Co{object}. It must be a vector with the same length
of \Co{coef(object)} and must be given in unconstrained form.
\end{Argument}
\Paragraph{VALUE}
a vector with the coefficients corresponding to \Co{object}.
\Paragraph{SIDE EFFECTS}
On the left side of an assignment, sets the values of the coefficients
of \Co{object} to \Co{value}. \Co{Object} must be initialized (using
\Co{initialize}) before new values can be assigned to its
coefficients.
\Paragraph{SEE ALSO}
\Co{corClasses}, \Co{initialize}
\need 15pt
\Paragraph{EXAMPLE}
\vspace{-16pt} 
\begin{Example}
cst1 <- corARMA(p = 1, q = 1)
coef(cst1)
\end{Example}
\end{Helpfile}
\begin{Helpfile}{coef.gnls}{Extract gnls Coefficients}
The estimated coefficients for the nonlinear model represented by
\Co{object} are extracted.
\begin{Example}
coef(object)
\end{Example}
\begin{Argument}{ARGUMENTS}
\item[\Co{object:}]
an object inheriting from class \Co{gnls}, representing
a generalized nonlinear least squares fitted model.
\end{Argument}
\Paragraph{VALUE}
a vector with the estimated coefficients for the nonlinear model
represented by \Co{object}.
\Paragraph{SEE ALSO}
\Co{gnls}
\need 15pt
\Paragraph{EXAMPLE}
\vspace{-16pt}
\begin{Example}
fm1 <- gnls(weight {\Twiddle} SSlogis(Time, Asym, xmid, scal), Soybean,
            weights = varPower())
coef(fm1)
\end{Example}
\end{Helpfile}
\begin{Helpfile}{coef.lmList}{Extract lmList Coefficients}
The coefficients of each \Co{lm} object in the \Co{object} list are
extracted and organized into a data frame, with rows corresponding to
the \Co{lm} components and columns corresponding to the
coefficients.  Optionally, the returned data frame may be augmented
with covariates summarized over the groups associated with the
\Co{lm} components.
\begin{Example}
coef(object, augFrame, which, FUN, omitGroupingFactor)
\end{Example}
\begin{Argument}{ARGUMENTS}
\item[\Co{object:}]
an object inheriting from class \Co{lmList}, representing
a list of \Co{lm} objects with a common model.
\item[\Co{augFrame:}]
an optional logical value. If \Co{TRUE}, the returned
data frame is augmented with variables defined in data frame used to
produce \Co{object}; else, if \Co{FALSE}, only the coefficients
are returned. Defaults to \Co{FALSE}.
\item[\Co{which:}]
an optional positive integer or character vector specifying which
columns of the data frame used to produce \Co{object} should be used in
the augmentation of the returned data frame. Defaults to all variables
in the data.
\item[\Co{FUN:}]
an optional summary function or a list of summary functions
to be applied to group-varying variables, when collapsing the data
by groups.  Group-invariant variables are always summarized by the
unique value that they assume within that group. If \Co{FUN} is a
single function it will be applied to each non-invariant variable by
group to produce the summary for that variable.  If \Co{FUN} is a
list of functions, the names in the list should designate classes of
variables in the frame such as \Co{ordered}, \Co{factor}, or
\Co{numeric}.  The indicated function will be applied to any
group-varying variables of that class.  The default functions to be
used are \Co{mean} for numeric factors, and \Co{Mode} for both
\Co{factor} and \Co{ordered}.  The \Co{Mode} function, defined
internally in \Co{gsummary}, returns the modal or most popular
value of the variable.  It is different from the \Co{mode} function
that returns the S-language mode of the variable.
\item[\Co{omitGroupingFactor:}]
an optional logical value.  When \Co{TRUE}
the grouping factor itself will be omitted from the group-wise
summary of \Co{data} but the levels of the grouping factor will
continue to be used as the row names for the returned data frame.
Defaults to \Co{FALSE}.
\end{Argument}
\Paragraph{VALUE}
a data frame inheriting from class \Co{coef.lmList} with the estimated
coefficients for each \Co{lm} component of \Co{object} and,
optionally, other covariates summarized over the groups corresponding
to the \Co{lm} components. The returned object also inherits from
classes \Co{ranef.lmList} and \Co{data.frame}.
\Paragraph{SEE ALSO}
\Co{lmList}, \Co{fixef.lmList},
\Co{ranef.lmList},
\Co{plot.ranef.lmList}, \Co{gsummary}
\need 15pt
\Paragraph{EXAMPLE}
\vspace{-16pt}
\begin{Example}
fm1 <- lmList(distance {\Twiddle} age|Subject, data = Orthodont)
coef(fm1)
coef(fm1, augFrame = TRUE)
\end{Example}
\end{Helpfile}
\begin{Helpfile}{coef.lme}{Extract lme Coefficients}
The estimated coefficients at level i are obtained by adding
together the fixed effects estimates and the corresponding random
effects estimates at grouping levels less or equal to i. The
resulting estimates are returned as a data frame, with rows
corresponding to groups and columns to coefficients. Optionally, the
returned data frame may be augmented with covariates summarized over
groups.
\begin{Example}
coef(object, augFrame, level, data, which, FUN, omitGroupingFactor)
\end{Example}
\begin{Argument}{ARGUMENTS}
\item[\Co{object:}]
an object inheriting from class \Co{lme}, representing
a fitted linear mixed-effects model.
\item[\Co{augFrame:}]
an optional logical value. If \Co{TRUE}, the returned
data frame is augmented with variables defined in \Co{data}; else,
if \Co{FALSE}, only the coefficients are returned. Defaults to
\Co{FALSE}.
\item[\Co{level:}]
an optional positive integer giving the level of grouping
to be used in extracting the coefficients from an object with
multiple nested grouping levels. Defaults to the highest or innermost
level of grouping.
\item[\Co{data:}]
an optional data frame with the variables to be used for
augmenting the returned data frame when \Co{augFrame = TRUE}. Defaults
to the data frame used to fit \Co{object}. 
\item[\Co{which:}]
an optional positive integer or character vector specifying which
columns of \Co{data} should be used in the augmentation of the
returned data frame. Defaults to all columns in \Co{data}.
\item[\Co{FUN:}]
an optional summary function or a list of summary functions
to be applied to group-varying variables, when collapsing \Co{data}
by groups.  Group-invariant variables are always summarized by the
unique value that they assume within that group. If \Co{FUN} is a
single function it will be applied to each non-invariant variable by
group to produce the summary for that variable.  If \Co{FUN} is a
list of functions, the names in the list should designate classes of
variables in the frame such as \Co{ordered}, \Co{factor}, or
\Co{numeric}.  The indicated function will be applied to any
group-varying variables of that class.  The default functions to be
used are \Co{mean} for numeric factors, and \Co{Mode} for both
\Co{factor} and \Co{ordered}.  The \Co{Mode} function, defined
internally in \Co{gsummary}, returns the modal or most popular
value of the variable.  It is different from the \Co{mode} function
that returns the S-language mode of the variable.
\item[\Co{omitGroupingFactor:}]
an optional logical value.  When \Co{TRUE}
the grouping factor itself will be omitted from the group-wise
summary of \Co{data} but the levels of the grouping factor will
continue to be used as the row names for the returned data frame.
Defaults to \Co{FALSE}.
\end{Argument}
\Paragraph{VALUE}
a data frame inheriting from class \Co{coef.lme} with the estimated
coefficients at level \Co{level} and, optionally, other covariates
summarized over groups. The returned object also inherits from classes
\Co{ranef.lme} and \Co{data.frame}.
\Paragraph{SEE ALSO}
\Co{lme}, \Co{fixef.lme},
\Co{ranef.lme},
\Co{plot.ranef.lme}, \Co{gsummary}
\need 15pt
\Paragraph{EXAMPLE}
\vspace{-16pt}
\begin{Example}
fm1 <- lme(distance {\Twiddle} age, Orthodont, random = {\Twiddle} age | Subject)
coef(fm1)
coef(fm1, augFrame = TRUE)
\end{Example}
\end{Helpfile}
\begin{Helpfile}{coef.modelStruct}{Extract modelStruct Coefficients}
This method function extracts the coefficients associated with each
component of the \Co{modelStruct} list.
\begin{Example}
coef(object, unconstrained)
coef(object) <- value
\end{Example}
\begin{Argument}{ARGUMENTS}
\item[\Co{object:}]
an object inheriting from class \Co{modelStruct},
representing a list of model components, such as \Co{corStruct} and
\Co{varFunc} objects.
\item[\Co{unconstrained:}]
a logical value. If \Co{TRUE} the coefficients
are returned in unconstrained form (the same used in the optimization
algorithm). If \Co{FALSE} the coefficients are returned in
"natural", possibly constrained, form. Defaults to \Co{TRUE}.
\item[\Co{value:}]
a vector with the replacement values for the coefficients
associated with \Co{object}. It must be a vector with the same length
of \Co{coef(object)} and must be given in unconstrained form.
\end{Argument}
\Paragraph{VALUE}
a vector with all coefficients corresponding to the components of
\Co{object}.
\Paragraph{SIDE EFFECTS}
On the left side of an assignment, sets the values of the coefficients
of \Co{object} to \Co{value}. \Co{Object} must be initialized (using
\Co{initialize}) before new values can be assigned to its
coefficients.
\Paragraph{SEE ALSO}
\Co{initialize}
\need 15pt
\Paragraph{EXAMPLE}
\vspace{-16pt} 
\begin{Example}
lms1 <- lmeStruct(reStruct = reStruct(pdDiag(diag(2), \Twiddle age)),
   corStruct = corAR1(0.3))
coef(lms1)
\end{Example}
\end{Helpfile}
\begin{Helpfile}{coef.pdCompSymm}{pdCompSymm Coefficients}
This method function extracts the coefficients associated with the
positive-definite matrix represented by \Co{object}.
\begin{Example}
coef(object, unconstrained)
coef(object) <- value
\end{Example}
\begin{Argument}{ARGUMENTS}
\item[\Co{object:}]
an object inheriting from class \Co{pdCompSymm},
representing a positive-definite matrix with compound symmetry
structure.
\item[\Co{unconstrained:}]
a logical value. If \Co{TRUE} the coefficients
are returned in unconstrained form (the same used in the optimization
algorithm). If \Co{FALSE} the standard deviation and the
correlation coefficient of the compound symmetry  of
positive-definite matrix represented by \Co{object} are
returned. Defaults to \Co{TRUE}.
\item[\Co{value:}]
a vector with the replacement values for the coefficients
associated with \Co{object}. It must be a vector of length two
and must be given in unconstrained form.
\end{Argument}
\Paragraph{VALUE}
a vector with the coefficients corresponding to \Co{object}.
\Paragraph{SIDE EFFECTS}
On the left side of an assignment, sets the values of the coefficients
of \Co{object} to \Co{value}. \Co{Object} must be initialized (using
\Co{initialize}) before new values can be assigned to its
coefficients.
\Paragraph{SEE ALSO}
\Co{coef.pdMat}, \Co{pdMat}
\need 15pt
\Paragraph{EXAMPLE}
\vspace{-16pt} 
\begin{Example}
coef(pdCompSymm(diag(3)), F)
\end{Example}
\end{Helpfile}
\begin{Helpfile}{coef.pdDiag}{pdDiag Coefficients}
This method function extracts the coefficients associated with the
positive-definite matrix represented by \Co{object}.
\begin{Example}
coef(object, unconstrained)
coef(object) <- value
\end{Example}
\begin{Argument}{ARGUMENTS}
\item[\Co{object:}]
an object inheriting from class \Co{pdDiag},
representing a positive-definite matrix with diagonal structure.
\item[\Co{unconstrained:}]
a logical value. If \Co{TRUE} the logarithm of
the standard deviations corresponding to the variance-covariance
matrix represented by \Co{object} are returned. If \Co{FALSE} the
standard deviations are returned. Defaults to \Co{TRUE}.
\item[\Co{value:}]
a vector with the replacement values for the coefficients
associated with \Co{object}. It must be a vector with the same length
of \Co{coef(object)} and must be given in unconstrained form.
\end{Argument}
\Paragraph{VALUE}
a vector with the coefficients corresponding to \Co{object}.
\Paragraph{SIDE EFFECTS}
On the left side of an assignment, sets the values of the coefficients
of \Co{object} to \Co{value}. \Co{Object} must be initialized (using
\Co{initialize}) before new values can be assigned to its
coefficients.
\Paragraph{SEE ALSO}
\Co{coef.pdMat}, \Co{pdMat}
\need 15pt
\Paragraph{EXAMPLE}
\vspace{-16pt} 
\begin{Example}
coef(pdDiag(diag(3)))
\end{Example}
\end{Helpfile}
\begin{Helpfile}{coef.pdIdent}{pdIdent Coefficients}
This method function extracts the coefficients associated with the
positive-definite matrix represented by \Co{object}.
\begin{Example}
coef(object, unconstrained)
coef(object) <- value
\end{Example}
\begin{Argument}{ARGUMENTS}
\item[\Co{object:}]
an object inheriting from class \Co{pdIdent},
representing a multiple of the identity positive-definite matrix.
\item[\Co{unconstrained:}]
a logical value. If \Co{TRUE} the logarithm of
the standard deviation corresponding to the variance-covariance
matrix represented by \Co{object} is returned. If \Co{FALSE} the
standard deviation is returned. Defaults to \Co{TRUE}.
\item[\Co{value:}]
a vector with the replacement values for the coefficients
associated with \Co{object}. It must be a numeric value
given in unconstrained form.
\end{Argument}
\Paragraph{VALUE}
a vector with the coefficients corresponding to \Co{object}.
\Paragraph{SIDE EFFECTS}
On the left side of an assignment, sets the values of the coefficients
of \Co{object} to \Co{value}. \Co{Object} must be initialized (using
\Co{initialize}) before new values can be assigned to its
coefficients.
\Paragraph{SEE ALSO}
\Co{coef.pdMat}, \Co{pdMat}
\need 15pt
\Paragraph{EXAMPLE}
\vspace{-16pt} 
\begin{Example}
coef(pdIdent(diag(3)))
\end{Example}
\end{Helpfile}
\begin{Helpfile}{coef.pdMat}{pdMat Coefficients}
This method function extracts the coefficients associated with the
positive-definite matrix represented by \Co{object}.
\begin{Example}
coef(object, unconstrained)
coef(object) <- value
\end{Example}
\begin{Argument}{ARGUMENTS}
\item[\Co{object:}]
an object inheriting from class \Co{pdMat},
representing a positive-definite matrix.
\item[\Co{unconstrained:}]
a logical value. If \Co{TRUE} the coefficients
are returned in unconstrained form (the same used in the optimization
algorithm). If \Co{FALSE} the upper triangular elements of the
positive-definite matrix represented by \Co{object} are
returned. Defaults to \Co{TRUE}.
\item[\Co{value:}]
a vector with the replacement values for the coefficients
associated with \Co{object}. It must be a vector with the same length
of \Co{coef(object)} and must be given in unconstrained form.
\end{Argument}
\Paragraph{VALUE}
a vector with the coefficients corresponding to \Co{object}.
\Paragraph{SIDE EFFECTS}
On the left side of an assignment, sets the values of the coefficients
of \Co{object} to \Co{value}.
\Paragraph{REFERENCES}
Pinheiro, J.C. and Bates., D.M.  (1996) "Unconstrained Parametrizations
for Variance-Covariance Matrices", Statistics and Computing, 6, 289-296.
\Paragraph{SEE ALSO}
\Co{pdMat}
\need 15pt
\Paragraph{EXAMPLE}
\vspace{-16pt} 
\begin{Example}
coef(pdSymm(diag(3)))
\end{Example}
\end{Helpfile}
\begin{Helpfile}{coef.reStruct}{reStruct Coefficients}
This method function extracts the coefficients associated with the
positive-definite matrix represented by \Co{object}.
\begin{Example}
coef(object, unconstrained)
coef(object) <- value
\end{Example}
\begin{Argument}{ARGUMENTS}
\item[\Co{object:}]
an object inheriting from class \Co{reStruct},
representing a random effects structure and consisting of a list of
\Co{pdMat} objects.
\item[\Co{unconstrained:}]
a logical value. If \Co{TRUE} the coefficients
are returned in unconstrained form (the same used in the optimization
algorithm). If \Co{FALSE} the coefficients are returned in
"natural", possibly constrained, form. Defaults to \Co{TRUE}.
\item[\Co{value:}]
a vector with the replacement values for the coefficients
associated with \Co{object}. It must be a vector with the same length
of \Co{coef(object)} and must be given in unconstrained form.
\end{Argument}
\Paragraph{VALUE}
a vector with the coefficients corresponding to \Co{object}.
\Paragraph{SIDE EFFECTS}
On the left side of an assignment, sets the values of the coefficients
of \Co{object} to \Co{value}.
\Paragraph{SEE ALSO}
\Co{coef.pdMat}, \Co{reStruct},
\Co{pdMat}
\need 15pt
\Paragraph{EXAMPLE}
\vspace{-16pt} 
\begin{Example}
rs1 <- reStruct(list(A = pdSymm(diag(1:3), form = \Twiddle Score),
  B = pdDiag(2 * diag(4), form = \Twiddle Educ)))
coef(rs1)
\end{Example}
\end{Helpfile}
\begin{Helpfile}{coef.varFunc}{varFunc Coefficients}
This method function extracts the coefficients associated with the
variance function structure represented by \Co{object}.
\begin{Example}
coef(object, unconstrained, allCoef)
coef(object) <- value
\end{Example}
\begin{Argument}{ARGUMENTS}
\item[\Co{object:}]
an object inheriting from class \Co{varFunc}
representing a variance function structure.
\item[\Co{unconstrained:}]
a logical value. If \Co{TRUE} the coefficients
are returned in unconstrained form (the same used in the optimization
algorithm). If \Co{FALSE} the coefficients are returned in
"natural", generally constrained form. Defaults to \Co{TRUE}.
\item[\Co{allCoef:}]
a logical value. If \Co{FALSE} only the coefficients
which may vary during the optimization are returned. If \Co{TRUE}
all coefficients are returned. Defaults to \Co{FALSE}.
\item[\Co{value:}]
a vector with the replacement values for the coefficients
associated with \Co{object}. It must be have the same length of
\Co{coef(object)} and must be given in unconstrained
form. \Co{Object} must be initialized before new values can be
assigned to its coefficients.
\end{Argument}
\Paragraph{VALUE}
a vector with the coefficients corresponding to \Co{object}.
\Paragraph{SIDE EFFECTS}
On the left side of an assignment, sets the values of the coefficients
of \Co{object} to \Co{value}.
\Paragraph{SEE ALSO}
\Co{varFunc}
\need 15pt
\Paragraph{EXAMPLE}
\vspace{-16pt} 
\begin{Example}
vf1 <- varPower(1)
coef(vf1)
coef(vf1) <- 2
\end{Example}
\end{Helpfile}
\begin{Helpfile}{coef\Co{<-}}{Assign Values to Coefficients}
This function is generic; method functions can be written to handle
specific classes of objects. Classes which already have methods for
this function include all \Co{pdMat}, \Co{corStruct}, and 
\Co{varFunc} classes, \Co{reStruct}, and \Co{modelStruct}.
\begin{Example}
coef(object, ...) <-  value
\end{Example}
\begin{Argument}{ARGUMENTS}
\item[\Co{object:}]
any object representing a fitted model, or, by default,
any object with a \Co{coef} component.
\item[\Co{...:}]
some methods for this generic function may require
additional arguments.
\item[\Co{value:}]
a value to be assigned to the coefficients associated with
\Co{object}.
\end{Argument}
\Paragraph{VALUE}
will depend on the method function; see the appropriate documentation.
\Paragraph{SEE ALSO}
\Co{coef}
\need 15pt
\Paragraph{EXAMPLE}
\vspace{-16pt} 
\begin{Example}
## see the method function documentation
\end{Example}
\end{Helpfile}
\begin{Helpfile}{collapse}{Collapse According to Groups}
This function is generic; method functions can be written to handle
specific classes of objects. Currently, only a \Co{groupedData}
method is available.
\begin{Example}
collapse(object, ...)
\end{Example}
\begin{Argument}{ARGUMENTS}
\item[\Co{object:}]
an object to be collapsed, usually a data frame.
\item[\Co{...:}]
some methods for the generic may require additional
arguments.
\end{Argument}
\Paragraph{VALUE}
will depend on the method function used; see the appropriate documentation.
\Paragraph{SEE ALSO}
\Co{collapse.groupedData}
\need 15pt
\Paragraph{EXAMPLE}
\vspace{-16pt} 
\begin{Example}
## see the method function documentation
\end{Example}
\end{Helpfile}
\begin{Helpfile}{collapse.groupedData}{Collapse groupedData}
If \Co{object} has a single grouping factor, it is returned
unchanged. Else, it is summarized by the values of the
\Co{displayLevel} grouping factor (or the combination of its values
and the values of the covariate indicated in \Co{preserve}, if any is
present). The collapsed data is used to produce a new
\Co{groupedData} object, with grouping factor given by the
\Co{displayLevel} factor.
\begin{Example}
collapse(object, collapseLevel, displayLevel, outer, inner,
         preserve, FUN, subset)
\end{Example}
\begin{Argument}{ARGUMENTS}
\item[\Co{object:}]
an object inheriting from class \Co{groupedData},
generally with multiple grouping factors.
\item[\Co{collapseLevel:}]
an optional positive integer or character string
indicating the grouping level to use when collapsing the data. Level
values increase from outermost to innermost grouping. Default is the
highest or innermost level of grouping.
\item[\Co{displayLevel:}]
an optional positive integer or character string
indicating the grouping level to use as the grouping factor for the
collapsed data. Default is \Co{collapseLevel}.
\item[\Co{outer:}]
an optional logical value or one-sided formula,
indicating covariates that are outer to the \Co{displayLevel}
grouping factor. If equal to \Co{TRUE}, the \Co{displayLevel}
element \Co{attr(object, "outer")} is used to indicate the 
outer covariates. An outer covariate is invariant within the sets
of rows defined by the grouping factor.  Ordering of the groups is
done in such a way as to preserve adjacency of groups with the same
value of the outer variables. Defaults to \Co{NULL}, meaning that
no outer covariates are to be used.
\item[\Co{inner:}]
an optional logical value or one-sided formula, indicating
a covariate that is inner to the \Co{displayLevel} grouping
factor. If equal to \Co{TRUE}, \Co{attr(object, "outer")} is used
to indicate the inner covariate. An inner covariate can change within
the sets of rows defined by the grouping  factor. Defaults to
\Co{NULL}, meaning that no inner covariate is present.  
\item[\Co{preserve:}]
an optional one-sided formula indicating a covariate
whose levels should be preserved when collapsing the data according
to the \Co{collapseLevel} grouping factor. The collapsing factor is
obtained by pasting together the levels of the \Co{collapseLevel}
grouping factor and the values of the covariate to be
preserved. Default is \Co{NULL}, meaning that no covariates need to
be preserved.
\item[\Co{FUN:}]
an optional summary function or a list of summary functions
to be used for collapsing the data.  The function or functions are
applied only to variables in \Co{object} that vary within the
groups defined by \Co{collapseLevel}.  Invariant variables are 
always summarized by group using the unique value that they assume
within that group.  If \Co{FUN} is a single
function it will be applied to each non-invariant variable by group
to produce the summary for that variable.  If \Co{FUN} is a list of
functions, the names in the list should designate classes of
variables in the data such as \Co{ordered}, \Co{factor}, or
\Co{numeric}.  The indicated function will be applied to any
non-invariant variables of that class.  The default functions to be
used are \Co{mean} for numeric factors, and \Co{Mode} for both
\Co{factor} and \Co{ordered}.  The \Co{Mode} function, defined
internally in \Co{gsummary}, returns the modal or most popular
value of the variable.  It is different from the \Co{mode} function
that returns the S-language mode of the variable.
\item[\Co{subset:}]
an optional named list. Names can be either positive
integers representing grouping levels, or names of grouping
factors. Each element in the list is a vector indicating the levels
of the corresponding grouping factor to be preserved in the collapsed
data. Default is \Co{NULL}, meaning that all levels are
used.
\end{Argument}
\Paragraph{VALUE}
a \Co{groupedData} object with a single grouping factor given by the
\Co{displayLevel} grouping factor, resulting from collapsing
\Co{object} over the levels of the \Co{collapseLevel} grouping
factor.
\Paragraph{REFERENCES}
Pinheiro, J.C. and Bates, D.M. (1997) "Future Directions in
Mixed-Effects Software: Design of NLME 3.0" available at
http://nlme.stat.wisc.edu.
\Paragraph{SEE ALSO}
\Co{groupedData}, \Co{plot.nmGroupedData}
\need 15pt
\Paragraph{EXAMPLE}
\vspace{-16pt} 
\begin{Example}
# collapsing by Dog
# same as collapse(Pixel, collapse = "Dog")
collapse(Pixel, collapse = 1)  
\end{Example}
\end{Helpfile}
\begin{Helpfile}{compareFits}{Compare Fitted Objects}
The columns in \Co{object1} and \Co{object2} are put together in
matrices which allow direct comparison of the individual elements for
each object. Missing columns in either object are replaced by
\Co{NA}s.
\begin{Example}
compareFits(object1, object2, which)
\end{Example}
\begin{Argument}{ARGUMENTS}
\item[\Co{object1,object2:}]
data frames, or matrices, with the same
row names, but possibly different column names. These will usually
correspond to coefficients from fitted objects with a grouping
structure (e.g. \Co{lme} and \Co{lmList} objects).
\item[\Co{which:}]
an optional integer of character vector indicating which
columns in \Co{object1} and \Co{object2} are to be used in the
returned object. Defaults to all columns.
\end{Argument}
\Paragraph{VALUE}
a three-dimensional array, with the third dimension given by the number
of unique column names in either \Co{object1} or \Co{object2}. To
each column name there corresponds a matrix with as many rows as the
rows in \Co{object1} and two columns, corresponding to \Co{object1}
and \Co{object2}. The returned object inherits from class
\Co{compareFits}.
\Paragraph{SEE ALSO}
\Co{plot.compareFits},
\Co{pairs.compareFits}, \Co{comparePred},
\Co{coef}, \Co{ranef}
\need 15pt
\Paragraph{EXAMPLE}
\vspace{-16pt} 
\begin{Example}
fm1 <- lmList(Orthodont)
fm2 <- lme(fm1)
compareFits(coef(fm1), coef(fm2))
\end{Example}
\end{Helpfile}
\begin{Helpfile}{comparePred}{Compare Predictions}
Predicted values are obtained at the specified values of
\Co{primary} for each object. If either \Co{object1} or
\Co{object2} have a grouping structure
(i.e.\ \Co{getGroups(object)} is not \Co{NULL}), predicted values
are obtained for each group. When both objects determine groups, the
group levels must be the same. If other covariates besides
\Co{primary} are used in the prediction model, their average
(numeric covariates) or most frequent value (categorical covariates)
are used to obtain the predicted values. The original observations are
also included in the returned object.
\begin{Example}
comparePred(object1, object2, primary, minimum, maximum, 
            length.out, level, ...) 
\end{Example}
\begin{Argument}{ARGUMENTS}
\item[\Co{object1,object2:}]
fitted model objects, from which predictions can
be extracted using the \Co{predict} method.
\item[\Co{primary:}]
an optional one-sided formula specifying the primary
covariate to be used to generate the augmented predictions. By
default, if a  covariate can be extracted from the data used to generate
the objects (using \Co{getCovariate}), it will be used as
\Co{primary}.
\item[\Co{minimum:}]
an optional lower limit for the primary
covariate. Defaults to \Co{min(primary)}.
\item[\Co{maximum:}]
an optional upper limit for the primary
covariate. Defaults to \Co{max(primary)}.
\item[\Co{length.out:}]
an optional integer with the number of primary
covariate values at which to evaluate the predictions. Defaults to
51.
\item[\Co{level:}]
an optional integer specifying the desired
prediction level. Levels increase from outermost to innermost
grouping, with level 0 representing the population (fixed effects)
predictions. Only one level can be specified. Defaults to the
innermost level.
\item[\Co{...:}]
some methods for the generic may require additional
arguments.
\end{Argument}
\Paragraph{VALUE}
a data frame with four columns representing, respectively, the values
of the primary covariate, the groups (if \Co{object} does not have a
grouping structure, all elements will be \Co{1}), the predicted or
observed values, and the type of value in the third column: the
objects' names are used to classify the predicted values and
\Co{original} is used for the observed values. The returned object
inherits from classes \Co{comparePred} and \Co{augPred}.
\Paragraph{NOTE} This function is generic; method functions can be written to handle
specific classes of objects. Classes which already have methods for
this function include: \Co{gls}, \Co{lme}, and \Co{lmList}.
\Paragraph{SEE ALSO}
\Co{augPred}, \Co{getGroups}
\need 15pt
\Paragraph{EXAMPLE}
\vspace{-16pt} 
\begin{Example}
fm1 <- lme(distance \Twiddle age * Sex, data = Orthodont, random = \Twiddle age)
fm2 <- update(fm1, distance \Twiddle age)
comparePred(fm1, fm2, length.out = 2)
\end{Example}
\end{Helpfile}
\begin{Helpfile}{corAR1}{AR(1) Correlation Structure}
This function is a constructor for the \Co{corAR1} class,
representing an autocorrelation structure of order 1. Objects
created using this constructor need to be later initialized using the
appropriate \Co{initialize} method.
\begin{Example}
corAR1(value, form, fixed)
\end{Example}
\begin{Argument}{ARGUMENTS}
\item[\Co{value:}]
the value of the lag 1 autocorrelation, which must be
between -1 and 1. Defaults to 0 (no autocorrelation).
\item[\Co{form:}]
a one sided formula of the form \Co{\Twiddle t}, or \Co{\Twiddle t |
g}, specifying a time covariate \Co{t} and,  optionally, a
grouping factor \Co{g}. A covariate for this correlation structure
must be integer valued. When a grouping factor is present in
\Co{form}, the correlation structure is assumed to apply only
to observations within the same grouping level; observations with
different grouping levels are assumed to be uncorrelated. Defaults to
\Co{\Twiddle 1}, which corresponds to using the order of the observations
in the data as a covariate, and no groups.
\item[\Co{fixed:}]
an optional logical value indicating whether the
coefficient should be allowed to vary in the optimization, or kept
fixed at its initial value. Defaults to \Co{FALSE}, in which case
the coefficient is allowed to vary.
\end{Argument}
\Paragraph{VALUE}
an object of class \Co{corAR1}, representing an autocorrelation
structure of order 1.
\Paragraph{REFERENCES}
Box, G.E.P., Jenkins, G.M., and Reinsel G.C. (1994) "Time Series
Analysis: Forecasting and Control", 3rd Edition, Holden-Day.
\Paragraph{SEE ALSO}
\Co{initialize.corStruct}
\need 15pt
\Paragraph{EXAMPLE}
\vspace{-16pt} 
\begin{Example}
## covariate is observation order and grouping factor is Mare
cs1 <- corAR1(0.2, form = \Twiddle 1 | Mare)
\end{Example}
\end{Helpfile}
\begin{Helpfile}{corARMA}{ARMA(p,q) Correlation Structure}
This function is a constructor for the \Co{corARMA} class,
representing an autocorrelation-moving average correlation structure
of order (p, q). Objects created using this constructor need to be
later initialized using the appropriate \Co{initialize} method.
\begin{Example}
corARMA(value, form, p, q, fixed)
\end{Example}
\begin{Argument}{ARGUMENTS}
\item[\Co{value:}]
a vector with the values of the autoregressive and moving
average parameters, which must have length \Co{p + q} and all
elements between -1 and 1. Defaults to a vector of zeros,
corresponding to uncorrelated observations.
\item[\Co{form:}]
a one sided formula of the form \Co{\Twiddle t}, or \Co{\Twiddle t |
g}, specifying a time covariate \Co{t} and,  optionally, a
grouping factor \Co{g}. A covariate for this correlation structure
must be integer valued. When a grouping factor is present in
\Co{form}, the correlation structure is assumed to apply only
to observations within the same grouping level; observations with
different grouping levels are assumed to be uncorrelated. Defaults to
\Co{\Twiddle 1}, which corresponds to using the order of the observations
in the data as a covariate, and no groups.
\item[\Co{p, q:}]
non-negative integers specifying respectively the
autoregressive order and the moving average order of the \Co{ARMA}
structure. Both default to 0.
\item[\Co{fixed:}]
an optional logical value indicating whether the
coefficients should be allowed to vary in the optimization, or kept
fixed at their initial values. Defaults to \Co{FALSE}, in which case
the coefficients are allowed to vary.
\end{Argument}
\Paragraph{VALUE}
an object of class \Co{corARMA}, representing an
autocorrelation-moving average correlation structure.
\Paragraph{REFERENCES}
Box, G.E.P., Jenkins, G.M., and Reinsel G.C. (1994) "Time Series
Analysis: Forecasting and Control", 3rd Edition, Holden-Day.
\Paragraph{SEE ALSO}
\Co{initialize.corStruct}
\need 15pt
\Paragraph{EXAMPLE}
\vspace{-16pt} 
\begin{Example}
## ARMA(1,2) structure, with observation order as a covariate and
## Mare as grouping factor
cs1 <- corARMA(c(0.2, 0.3, -0.1), form = \Twiddle 1 | Mare, p = 1, q = 2)
\end{Example}
\end{Helpfile}
\begin{Helpfile}{corCAR1}{Continuous AR(1) Correlation Structure}
This function is a constructor for the \Co{corCAR1} class,
representing an autocorrelation structure of order 1, with a
continuous time covariate. Objects created using this constructor need
to be later initialized using the appropriate \Co{initialize}
method.
\begin{Example}
corCAR1(value, form, fixed)
\end{Example}
\begin{Argument}{ARGUMENTS}
\item[\Co{value:}]
the correlation between two observations one unit of time
apart. Must be between 0 and 1. Defaults to 0.2.
\item[\Co{form:}]
a one sided formula of the form \Co{\Twiddle t}, or \Co{\Twiddle t |
g}, specifying a time covariate \Co{t} and,  optionally, a
grouping factor \Co{g}. Covariates for this correlation structure
need not be integer valued.  When a grouping factor is present in
\Co{form}, the correlation structure is assumed to apply only
to observations within the same grouping level; observations with
different grouping levels are assumed to be uncorrelated. Defaults to
\Co{\Twiddle 1}, which corresponds to using the order of the observations
in the data as a covariate, and no groups.
\item[\Co{fixed:}]
an optional logical value indicating whether the
coefficient should be allowed to vary in the optimization, or kept
fixed at its initial value. Defaults to \Co{FALSE}, in which case
the coefficient is allowed to vary.
\end{Argument}
\Paragraph{VALUE}
an object of class \Co{corCAR1}, representing an autocorrelation
structure of order 1, with a continuous time covariate.
\Paragraph{REFERENCES}
Box, G.E.P., Jenkins, G.M., and Reinsel G.C. (1994) "Time Series
Analysis: Forecasting and Control", 3rd Edition, Holden-Day.\\

Jones, R.H. (1993) "Longitudinal Data with Serial Correlation: A
State-space Approach", Chapman and Hall

\Paragraph{SEE ALSO}
\Co{initialize.corStruct}
\need 15pt
\Paragraph{EXAMPLE}
\vspace{-16pt} 
\begin{Example}
## covariate is Time and grouping factor is Mare
cs1 <- corCAR1(0.2, form = \Twiddle Time | Mare)
\end{Example}
\end{Helpfile}
\begin{Helpfile}{corClasses}{Correlation Structure Classes}
Standard classes of correlation structures (\Co{corStruct})
available in the \Co{nlme} library.
\begin{Argument}{STANDARD CLASSES}
\item[\Co{corAR1:}]
autoregressive process of order 1.
\item[\Co{corARMA:}]
autoregressive moving average process, with arbitrary
orders for the autoregressive and moving average components.
\item[\Co{corCAR1:}]
continuous autoregressive process (AR(1) process for a
continuous time covariate).
\item[\Co{corCompSymm:}]
compound symmetry structure corresponding to a
constant correlation.
\item[\Co{corExp:}]
exponential spatial correlation.
\item[\Co{corGaus:}]
Gaussian spatial correlation.
\item[\Co{corLin:}]
linear spatial correlation.
\item[\Co{corRatio:}]
Rational Quadratic spatial correlation.
\item[\Co{corSpher:}]
spherical spatial correlation.
\item[\Co{corSymm:}]
general correlation matrix, with no additional
structure.
\Paragraph{NOTE} Users may define their own \Co{corStruct} classes by specifying a
\Co{constructor} function and, at a minimum, methods for the
functions \Co{corMatrix} and \Co{coef}. For
examples of these functions, see the methods for classes \Co{corSymm}
and \Co{corAR1}.
\end{Argument}
\Paragraph{SEE ALSO}
\Co{corAR1}, \Co{corARMA},
\Co{corCAR1}, \Co{corCompSymm},
\Co{corExp}, \Co{corGaus}, 
\Co{corLin}, \Co{corRatio}, \Co{corSpher}, \Co{corSymm}
\end{Helpfile}
\begin{Helpfile}{corCompSymm}{Compound Symmetry Correlation Structure}
This function is a constructor for the \Co{corCompSymm} class,
representing a compound symmetry structure corresponding to uniform
correlation. Objects created using this constructor need to be later
initialized using the appropriate \Co{initialize} method.
\begin{Example}
corCompSymm(value, form, fixed)
\end{Example}
\begin{Argument}{ARGUMENTS}
\item[\Co{value:}]
the correlation between any two correlated
observations. Defaults to 0.
\item[\Co{form:}]
a one sided formula of the form \Co{\Twiddle t}, or \Co{\Twiddle t |
g}, specifying a time covariate \Co{t} and,  optionally, a
grouping factor \Co{g}. When a grouping factor is present in
\Co{form}, the correlation structure is assumed to apply only
to observations within the same grouping level; observations with
different grouping levels are assumed to be uncorrelated. Defaults to
\Co{\Twiddle 1}, which corresponds to using the order of the observations
in the data as a covariate, and no groups.
\item[\Co{fixed:}]
an optional logical value indicating whether the
coefficient should be allowed to vary in the optimization, or kept
fixed at its initial value. Defaults to \Co{FALSE}, in which case
the coefficient is allowed to vary.
\end{Argument}
\Paragraph{VALUE}
an object of class \Co{corCompSymm}, representing a compound
symmetry correlation structure.
\Paragraph{REFERENCES}
Milliken, G. A. and Johnson, D. E. (1992) "Analysis of Messy Data,
Volume I: Designed Experiments", Van Nostrand Reinhold.
\Paragraph{SEE ALSO}
\Co{initialize.corStruct}
\need 15pt
\Paragraph{EXAMPLE}
\vspace{-16pt} 
\begin{Example}
## covariate is observation order and grouping factor is Subject
cs1 <- corCompSymm(0.5, form = \Twiddle 1 | Subject)
\end{Example}
\end{Helpfile}
\begin{Helpfile}{corExp}{Exponential Correlation Structure}
This function is a constructor for the \Co{corExp} class,
representing an exponential spatial correlation structure. Letting
$d$ denote the range and $n$ denote the nugget
effect, the correlation between two observations a distance
$r$ apart is $\exp(-r/d)$ when no nugget effect
is present and $n\exp(-r/d)$ when a nugget effect is
assumed. Objects created using this constructor need to be later
initialized using the appropriate \Co{initialize} method.
\begin{Example}
corExp(value, form, nugget, metric, fixed)
\end{Example}
\begin{Argument}{ARGUMENTS}
\item[\Co{value:}] an optional vector with the parameter values in
  constrained form. If \Co{nugget} is \Co{FALSE}, \Co{value} can have
  only one element, corresponding to the "range" of the Exponential
  correlation structure, which must be greater than zero. If
  \Co{nugget} is \Co{TRUE}, meaning that a nugget effect is present,
  \Co{value} can contain one or two elements, the first being the
  "range" and the second the "nugget effect" (one minus the
  correlation between observations taken arbitrarily close together);
  the first must be greater than zero and the second must be between
  zero and one. Defaults to \Co{numeric(0)}, which results in a range
  of 90\% of the minimum distance and a nugget ratio of 0.9 being
  assigned to the parameters when \Co{object} is initialized.
\item[\Co{form:}] a one sided formula of the form \Co{{\Twiddle}
    S1+...+Sp}, or \Co{{\Twiddle} S1+...+Sp | g}, specifying spatial
  covariates \Co{S1} through \Co{Sp} and, optionally, a grouping
  factor \Co{g}.  When a grouping factor is present in \Co{form}, the
  correlation structure is assumed to apply only to observations
  within the same grouping level; observations with different grouping
  levels are assumed to be uncorrelated. Defaults to \Co{{\Twiddle}
    1}, which corresponds to using the order of the observations in
  the data as a covariate, and no groups.
\item[\Co{nugget:}] an optional logical value indicating whether a
  nugget effect is present. Defaults to \Co{FALSE}.
\item[\Co{metric:}] an optional character string specifying the
  distance metric to be used. The currently available options are
  \Co{"euclidian"} for the root sum-of-squares of distances;
  \Co{"maximum"} for the maximum difference; and \Co{"manhattan"} for
  the sum of the absolute differences. Partial matching of arguments
  is used, so only the first three characters need to be provided.
  Defaults to \Co{"euclidian"}.
\item[\Co{fixed:}] an optional logical value indicating whether the
  coefficients should be allowed to vary in the optimization, or kept
  fixed at their initial values. Defaults to \Co{FALSE}, in which case
  the coefficients are allowed to vary.
\end{Argument}
\Paragraph{VALUE}
an object of class \Co{corExp}, also inheriting from class
\Co{corSpatial}, representing an exponential spatial correlation
structure.
\Paragraph{REFERENCES}
  Cressie, N.A.C. (1993), "Statistics for Spatial Data", J. Wiley \& Sons.\\
Venables, W.N. and Ripley, B.D. (1997) "Modern Applied Statistics with
S-plus", 2nd Edition, Springer-Verlag.\\
Littel, Milliken, Stroup, and Wolfinger (1997) "SAS Systems for Mixed
Models", SAS Institute.
\Paragraph{SEE ALSO}
\Co{initialize.corStruct}, \Co{dist}
\need 15pt
\Paragraph{EXAMPLE}
\vspace{-16pt}
\begin{Example}
sp1 <- corExp(form = {\Twiddle} x + y + z)
\end{Example}
\end{Helpfile}
\begin{Helpfile}{corFactor}{Factor of a Correlation Matrix}
This function is generic; method functions can be written to handle
specific classes of objects. Classes which already have methods for
this function include all \Co{corStruct} classes.
\begin{Example}
corFactor(object, ...)
\end{Example}
\begin{Argument}{ARGUMENTS}
\item[\Co{object:}]
an object from which a correlation matrix can be
extracted.
\item[\Co{...:}]
some methods for this generic function require additional
arguments.
\end{Argument}
\Paragraph{VALUE}
will depend on the method function used; see the appropriate
documentation.
\need 15pt
\Paragraph{EXAMPLE}
\vspace{-16pt} 
\begin{Example}
## see the method function documentation
\end{Example}
\end{Helpfile}
\begin{Helpfile}{corFactor.corStruct}{Factor of a corStruct Matrix}
  This method function extracts a transpose inverse square-root
  factor, or a series of transpose inverse square-root factors, of the
  correlation matrix, or list of correlation matrices, represented by
  \Co{object}. Letting $\bS$ denote a correlation matrix, a
  square-root factor of $\bS$ is any square matrix $\bL$ such that
  $\bS = \bL^t\bL$. This method extracts $\bL^{-t}$.
\begin{Example}
corFactor(object)
\end{Example}
\begin{Argument}{ARGUMENTS}
\item[\Co{object:}]
an object inheriting from class \Co{corStruct}
representing a correlation structure, which must have been
initialized (using \Co{initialize}).
\end{Argument}
\Paragraph{VALUE}
If the correlation structure does not include a grouping factor, the
returned value will be a vector with a transpose inverse square-root
factor of the correlation matrix associated with \Co{object} stacked
column-wise.  If the correlation structure includes a grouping factor,
the returned value will be a vector with transpose inverse
square-root factors of the correlation matrices for each group, stacked
by group and stacked column-wise within each group.
\Paragraph{NOTE} This method function is used intensively in optimization
algorithms and its value is returned as a vector for efficiency
reasons. The \Co{corMatrix} method function can be used to obtain
transpose inverse square-root factors in matrix form.
\Paragraph{SEE ALSO}
\Co{corMatrix.corStruct},
\Co{recalc.corStruct}, \Co{initialize.corStruct}
\need 15pt
\Paragraph{EXAMPLE}
\vspace{-16pt} 
\begin{Example}
cs1 <- corAR1(form = \Twiddle 1 | Subject)
cs1 <- initialize(cs1, data = Orthodont)
corFactor(cs1)
\end{Example}
\end{Helpfile}
\begin{Helpfile}{corGaus}{Gaussian Correlation Structure}
This function is a constructor for the \Co{corGaus} class,
representing a Gaussian spatial correlation structure. Letting
$d$ denote the range and $n$ denote the nugget
effect, the correlation between two observations a distance
$r$ apart is $\exp(-(r/d)^2)$ when no nugget
effect is present and $n\exp(-(r/d)^2)$ when a
nugget effect is assumed. Objects created using this constructor need
to be later initialized using the appropriate \Co{initialize} method.
\begin{Example}
corGaus(value, form, nugget, metric, fixed)
\end{Example}
\begin{Argument}{ARGUMENTS}
\item[\Co{value:}] an optional vector with the parameter values in
  constrained form. If \Co{nugget} is \Co{FALSE}, \Co{value} can have
  only one element, corresponding to the "range" of the Gaussian
  correlation structure, which must be greater than zero. If
  \Co{nugget} is \Co{TRUE}, meaning that a nugget effect is present,
  \Co{value} can contain one or two elements, the first being the
  "range" and the second the "nugget effect" (one minus the
  correlation between observations taken arbitrarily close together);
  the first must be greater than zero and the second must be between
  zero and one. Defaults to \Co{numeric(0)}, which results in a range
  of 90\% of the minimum distance and a nugget ratio of 0.9 being
  assigned to the parameters when \Co{object} is initialized.
\item[\Co{form:}]
a one sided formula of the form \Co{{\Twiddle} S1+...+Sp}, or
\Co{{\Twiddle} S1+...+Sp | g}, specifying spatial covariates \Co{S1}
through \Co{Sp} and,  optionally, a grouping factor \Co{g}. 
When a grouping factor is present in \Co{form}, the correlation
structure is assumed to apply only to observations within the same
grouping level; observations with different grouping levels are
assumed to be uncorrelated. Defaults to \Co{{\Twiddle} 1}, which corresponds
to using the order of the observations in the data as a covariate,
and no groups.
\item[\Co{nugget:}]
an optional logical value indicating whether a nugget
effect is present. Defaults to \Co{FALSE}.
\item[\Co{metric:}]
an optional character string specifying the distance
metric to be used. The currently available options are
\Co{"euclidian"} for the root sum-of-squares of distances;
\Co{"maximum"} for the maximum difference; and \Co{"manhattan"}
for the sum of the absolute differences. Partial matching of
arguments is used, so only the first three characters need to be
provided. Defaults to \Co{"euclidian"}.
\item[\Co{fixed:}]
an optional logical value indicating whether the
coefficients should be allowed to vary in the optimization, or kept
fixed at their initial values. Defaults to \Co{FALSE}, in which case
the coefficients are allowed to vary.
\end{Argument}
\Paragraph{VALUE}
an object of class \Co{corGaus}, also inheriting from class
\Co{corSpatial}, representing a Gaussian spatial correlation
structure.
\Paragraph{REFERENCES}
  Cressie, N.A.C. (1993), "Statistics for Spatial Data", J. Wiley \& Sons.\\
Venables, W.N. and Ripley, B.D. (1997) "Modern Applied Statistics with
S-plus", 2nd Edition, Springer-Verlag.\\
Littel, Milliken, Stroup, and Wolfinger (1997) "SAS Systems for Mixed
Models", SAS Institute.
\Paragraph{SEE ALSO}
\Co{initialize.corStruct}, \Co{dist}
\need 15pt
\Paragraph{EXAMPLE}
\vspace{-16pt}
\begin{Example}
sp1 <- corGaus(form = {\Twiddle} x + y + z)
\end{Example}
\end{Helpfile}
\begin{Helpfile}{corLin}{Linear Correlation Structure}
  This function is a constructor for the \Co{corLin} class,
  representing a linear spatial correlation structure. Letting $d$
  denote the range and $n$ denote the nugget effect, the correlation
  between two observations a distance $r < d$ apart is $1-(r/d)$ when
  no nugget effect is present and $n(1-(r/d))$ when a nugget effect is
  assumed. If $r \geq d$ the correlation is zero. Objects created
  using this constructor need to be later initialized using the
  appropriate \Co{initialize} method.
\begin{Example}
corLin(value, form, nugget, metric, fixed)
\end{Example}
\begin{Argument}{ARGUMENTS}
\item[\Co{value:}] an optional vector with the parameter values in
  constrained form. If \Co{nugget} is \Co{FALSE}, \Co{value} can have
  only one element, corresponding to the "range" of the Linear
  correlation structure, which must be greater than zero. If
  \Co{nugget} is \Co{TRUE}, meaning that a nugget effect is present,
  \Co{value} can contain one or two elements, the first being the
  "range" and the second the "nugget effect" (one minus the
  correlation between observations taken arbitrarily close together);
  the first must be greater than zero and the second must be between
  zero and one. Defaults to \Co{numeric(0)}, which results in a range
  of 90\% of the minimum distance and a nugget ratio of 0.9 being
  assigned to the parameters when \Co{object} is initialized.
\item[\Co{form:}]
a one sided formula of the form \Co{{\Twiddle} S1+...+Sp}, or
\Co{{\Twiddle} S1+...+Sp | g}, specifying spatial covariates \Co{S1}
through \Co{Sp} and,  optionally, a grouping factor \Co{g}. 
When a grouping factor is present in \Co{form}, the correlation
structure is assumed to apply only to observations within the same
grouping level; observations with different grouping levels are
assumed to be uncorrelated. Defaults to \Co{{\Twiddle} 1}, which corresponds
to using the order of the observations in the data as a covariate,
and no groups.
\item[\Co{nugget:}]
an optional logical value indicating whether a nugget
effect is present. Defaults to \Co{FALSE}.
\item[\Co{metric:}]
an optional character string specifying the distance
metric to be used. The currently available options are
\Co{"euclidian"} for the root sum-of-squares of distances;
\Co{"maximum"} for the maximum difference; and \Co{"manhattan"}
for the sum of the absolute differences. Partial matching of
arguments is used, so only the first three characters need to be
provided. Defaults to \Co{"euclidian"}.
\item[\Co{fixed:}]
an optional logical value indicating whether the
coefficients should be allowed to vary in the optimization, or kept
fixed at their initial values. Defaults to \Co{FALSE}, in which case
the coefficients are allowed to vary.
\end{Argument}
\Paragraph{VALUE}
an object of class \Co{corLin}, also inheriting from class
\Co{corSpatial}, representing a linear spatial correlation
structure.
\Paragraph{REFERENCES}
Venables, W.N. and Ripley, B.D. (1997) "Modern Applied Statistics with
S-plus", 2nd Edition, Springer-Verlag.\\
Littel, Milliken, Stroup, and Wolfinger (1997) "SAS Systems for Mixed
Models", SAS Institute.
\Paragraph{SEE ALSO}
\Co{initialize.corStruct}, \Co{dist}
\need 15pt
\Paragraph{EXAMPLE}
\vspace{-16pt}
\begin{Example}
sp1 <- corLin(form = {\Twiddle} x + y)
\end{Example}
\end{Helpfile}
\begin{Helpfile}{corMatrix}{Extract Correlation Matrix}
This function is generic; method functions can be written to handle
specific classes of objects. Classes which already have methods for
this function include all \Co{corStruct} classes.
\begin{Example}
corMatrix(object, ...)
\end{Example}
\begin{Argument}{ARGUMENTS}
\item[\Co{object:}]
an object for which a correlation matrix can be
extracted.
\item[\Co{...:}]
some methods for this generic function require additional
arguments.
\end{Argument}
\Paragraph{VALUE}
will depend on the method function used; see the appropriate
documentation.
\need 15pt
\Paragraph{EXAMPLE}
\vspace{-16pt} 
\begin{Example}
## see the method function documentation
\end{Example}
\end{Helpfile}
\begin{Helpfile}{corMatrix.corStruct}{corStruct Correlation Matrix}
  This method function extracts the correlation matrix (or its
  transpose inverse square-root factor), or list of correlation
  matrices (or their transpose inverse square-root factors)
  corresponding to \Co{covariate} and \Co{object}. Letting $\bS$
  denote a correlation matrix, a square-root factor of $\bS$ is any
  square matrix $\bL$ such that $\bS = \bL^t\bL$.  When \Co{corr =
    FALSE}, this method extracts $\bL^{-t}$.
\begin{Example}
corMatrix(object, covariate, corr)
\end{Example}
\begin{Argument}{ARGUMENTS}
\item[\Co{object:}]
an object inheriting from class \Co{corStruct}
representing a correlation structure.
\item[\Co{covariate:}]
an optional covariate vector (matrix), or list of
covariate vectors (matrices), at which values the correlation matrix,
or list of correlation  matrices, are to be evaluated. Defaults to
\Co{getCovariate(object)}.
\item[\Co{corr:}]
a logical value. If \Co{TRUE} the function returns the
correlation matrix, or list of correlation matrices, represented by
\Co{object}. If \Co{FALSE} the function returns a transpose
inverse square-root of the correlation matrix, or a list of transpose
inverse square-root factors of the correlation matrices.
\end{Argument}
\Paragraph{VALUE}
If \Co{covariate} is a vector (matrix), the returned value will be
an array with the corresponding correlation matrix (or its transpose
inverse square-root factor). If the \Co{covariate} is a list of
vectors (matrices), the returned value will be a list with the
correlation matrices (or their transpose inverse square-root factors)
corresponding to each component of \Co{covariate}.
\Paragraph{SEE ALSO}
\Co{corFactor.corStruct}, \Co{initialize.corStruct}
\need 15pt
\Paragraph{EXAMPLE}
\vspace{-16pt} 
\begin{Example}
cs1 <- corAR1(0.3)
corMatrix(cs1, covariate = 1:4)
corMatrix(cs1, covariate = 1:4, corr = F)
\end{Example}
\end{Helpfile}
\begin{Helpfile}{corMatrix.pdMat}{pdMat Correlation Matrix}
The correlation matrix corresponding to the positive-definite matrix
represented by \Co{object} is obtained.
\begin{Example}
corMatrix(object)
\end{Example}
\begin{Argument}{ARGUMENTS}
\item[\Co{object:}]
an object inheriting from class \Co{pdMat}, representing
a positive definite matrix.
\end{Argument}
\Paragraph{VALUE}
the correlation matrix corresponding to the positive-definite matrix
represented by \Co{object}.
\Paragraph{SEE ALSO}
\Co{as.matrix.pdMat}, \Co{pdMatrix}
\need 15pt
\Paragraph{EXAMPLE}
\vspace{-16pt} 
\begin{Example}
pd1 <- pdSymm(diag(1:4))
corMatrix(pd1)
\end{Example}
\end{Helpfile}
\begin{Helpfile}{corMatrix.reStruct}{reStruct Correlation Matrix}
This method function extracts the correlation matrices
corresponding to the \Co{pdMat} elements of \Co{object}.
\begin{Example}
corMatrix(object)
\end{Example}
\begin{Argument}{ARGUMENTS}
\item[\Co{object:}]
an object inheriting from class \Co{reStruct},
representing a random effects structure and consisting of a list of
\Co{pdMat} objects.
\end{Argument}
\Paragraph{VALUE}
a list with components given by the correlation matrices
corresponding to the elements of \Co{object}.
\Paragraph{SEE ALSO}
\Co{as.matrix.reStruct}, \Co{reStruct},
\Co{pdMat}
\need 15pt
\Paragraph{EXAMPLE}
\vspace{-16pt} 
\begin{Example}
rs1 <- reStruct(pdSymm(diag(3), form=\Twiddle Sex+age, data=Orthodont))
corMatrix(rs1)
\end{Example}
\end{Helpfile}
\begin{Helpfile}{corRatio}{Rational Quadratic Correlation Structure}
  This function is a constructor for the \Co{corRatio} class,
  representing a Rational Quadratic spatial correlation structure.
  Letting $d$ denote the range and $n$ denote the nugget effect, the
  correlation between two observations a distance r apart is
  $(r/d)^2/\left(1+(r/d)^2\right)$ when no nugget effect is present
  and $(1-n)(r/d)^2/\left(1+(r/d)^2\right)$ when a nugget effect is
  assumed. Objects created using this constructor need to be later
  initialized using the appropriate \Co{initialize} method.
\begin{Example}
corRatio(value, form, nugget, metric, fixed)
\end{Example}
\begin{Argument}{ARGUMENTS}
\item[\Co{value:}]
an optional vector with the parameter values in
constrained form. If \Co{nugget} is \Co{FALSE}, \Co{value} can
have only one element, corresponding to the "range" of the
Rational Quadratic correlation structure, which must be greater than
zero. If \Co{nugget} is \Co{TRUE}, meaning that a nugget effect
is present, \Co{value} can contain one or two elements, the first
being the "range" and the second the "nugget effect" (one minus the
correlation between two observations taken arbitrarily close
together); the first must be greater than zero and the second must be
between zero and one. Defaults to \Co{numeric(0)}, which results in
a range of 90% of the minimum distance and a nugget effect of 0.1
being assigned to the parameters when \Co{object} is initialized.
\item[\Co{form:}]
a one sided formula of the form \Co{{\Twiddle} S1+...+Sp}, or
\Co{{\Twiddle} S1+...+Sp | g}, specifying spatial covariates \Co{S1}
through \Co{Sp} and,  optionally, a grouping factor \Co{g}. 
When a grouping factor is present in \Co{form}, the correlation
structure is assumed to apply only to observations within the same
grouping level; observations with different grouping levels are
assumed to be uncorrelated. Defaults to \Co{{\Twiddle} 1}, which corresponds
to using the order of the observations in the data as a covariate,
and no groups.
\item[\Co{nugget:}]
an optional logical value indicating whether a nugget
effect is present. Defaults to \Co{FALSE}.
\item[\Co{metric:}]
an optional character string specifying the distance
metric to be used. The currently available options are
\Co{"euclidian"} for the root sum-of-squares of distances;
\Co{"maximum"} for the maximum difference; and \Co{"manhattan"}
for the sum of the absolute differences. Partial matching of
arguments is used, so only the first three characters need to be
provided. Defaults to \Co{"euclidian"}.
\item[\Co{fixed:}]
an optional logical value indicating whether the
coefficients should be allowed to vary in the optimization, or kept
fixed at their initial value. Defaults to \Co{FALSE}, in which case
the coefficients are allowed to vary.
\end{Argument}
\Paragraph{VALUE}
an object of class \Co{corRatio}, also inheriting from class
\Co{corSpatial}, representing a rational quadratic spatial correlation
structure.
\Paragraph{REFERENCES}
Cressie, N.A.C. (1993), "Statistics for Spatial Data", J. Wiley \& Sons.\\
Venables, W.N. and Ripley, B.D. (1997) "Modern Applied Statistics with
S-plus", 2nd Edition, Springer-Verlag.\\
Littel, Milliken, Stroup, and Wolfinger (1997) "SAS Systems for Mixed
Models", SAS Institute.
\Paragraph{SEE ALSO}
\Co{initialize.corStruct}, \Co{dist}
\need 15pt
\Paragraph{EXAMPLE}
\vspace{-16pt}
\begin{Example}
sp1 <- corRatio(form = {\Twiddle} x + y + z)
\end{Example}
\end{Helpfile}
\begin{Helpfile}{corSpatial}{Spatial Correlation Structure}
  This function is a constructor for the \Co{corSpatial} class,
  representing a spatial correlation structure. This class is
  "virtual", having four "real" classes, corresponding to specific
  spatial correlation structures, associated with it: \Co{corExp},
  \Co{corGaus}, \Co{corLin}, \Co{corRatio}, and \Co{corSpher}. The
  returned object will inherit from one of these "real" classes,
  determined by the \Co{type} argument, and from the "virtual"
  \Co{corSpatial} class. Objects created using this constructor need
  to be later initialized using the appropriate \Co{initialize}
  method.
\begin{Example}
corSpatial(value, form, nugget, type, metric, fixed)
\end{Example}
\begin{Argument}{ARGUMENTS}
\item[\Co{value:}] an optional vector with the parameter values in
  constrained form. If \Co{nugget} is \Co{FALSE}, \Co{value} can have
  only one element, corresponding to the "range" of the spatial
  correlation structure, which must be greater than zero. If
  \Co{nugget} is \Co{TRUE}, meaning that a nugget effect is present,
  \Co{value} can contain one or two elements, the first being the
  "range" and the second the "nugget effect" (one minus the
  correlation between observations taken arbitrarily close together);
  the first must be greater than zero and the second must be between
  zero and one. Defaults to \Co{numeric(0)}, which results in a range
  of 90\% of the minimum distance and a nugget ratio of 0.9 being
  assigned to the parameters when \Co{object} is initialized.
\item[\Co{form:}]
a one sided formula of the form \Co{{\Twiddle} S1+...+Sp}, or
\Co{{\Twiddle} S1+...+Sp | g}, specifying spatial covariates \Co{S1}
through \Co{Sp} and,  optionally, a grouping factor \Co{g}. 
When a grouping factor is present in \Co{form}, the correlation
structure is assumed to apply only to observations within the same
grouping level; observations with different grouping levels are
assumed to be uncorrelated. Defaults to \Co{{\Twiddle} 1}, which corresponds
to using the order of the observations in the data as a covariate,
and no groups.
\item[\Co{nugget:}]
an optional logical value indicating whether a nugget
effect is present. Defaults to \Co{FALSE}.
\item[\Co{type:}]
an optional character string specifying the desired type of
correlation structure. Available types include \Co{"spherical"},
\Co{"exponential"}, \Co{"gaussian"}, and \Co{"linear"}. See the
documentation on the functions \Co{corSpher}, \Co{corExp},
\Co{corGaus}, and \Co{corLin} for a description of these
correlation structures. Partial matching of arguments is used, so
only the first character needs to be provided. Defaults to
\Co{"spherical"}.
\item[\Co{metric:}]
an optional character string specifying the distance
metric to be used. The currently available options are
\Co{"euclidian"} for the root sum-of-squares of distances;
\Co{"maximum"} for the maximum difference; and \Co{"manhattan"}
for the sum of the absolute differences. Partial matching of
arguments is used, so only the first three characters need to be
provided. Defaults to \Co{"euclidian"}.
\item[\Co{fixed:}]
an optional logical value indicating whether the
coefficients should be allowed to vary in the optimization, or kept
fixed at their initial values. Defaults to \Co{FALSE}, in which case
the coefficients are allowed to vary.
\end{Argument}
\Paragraph{VALUE}
an object of class determined by the \Co{type} argument and also
inheriting from class \Co{corSpatial}, representing a spatial
correlation structure.
\Paragraph{REFERENCES}
  Cressie, N.A.C. (1993), "Statistics for Spatial Data", J. Wiley \& Sons.\\
Venables, W.N. and Ripley, B.D. (1997) "Modern Applied Statistics with
S-plus", 2nd Edition, Springer-Verlag.\\
Littel, Milliken, Stroup, and Wolfinger (1997) "SAS Systems for Mixed
Models", SAS Institute.
\Paragraph{SEE ALSO}
\Co{corExp}, \Co{corGaus},
\Co{corLin}, \Co{corSpher},
\Co{initialize.corStruct}, \Co{dist}
\need 15pt
\Paragraph{EXAMPLE}
\vspace{-16pt}
\begin{Example}
sp1 <- corSpatial(form = {\Twiddle} x + y + z, type = "g", metric = "man")
\end{Example}
\end{Helpfile}
\begin{Helpfile}{corSpher}{Spherical Correlation Structure}
  This function is a constructor for the \Co{corSpher} class,
  representing a spherical spatial correlation structure. Letting $d$
  denote the range and $n$ denote the nugget effect, the correlation
  between two observations a distance $r < d$ apart is
  $1-1.5(r/d)+0.5(r/d)^3$ when no nugget effect is present and
  $n(1-1.5(r/d)+0.5(r/d)^3)$ when a nugget effect is assumed. If $r
  \geq d$ the correlation is zero. Objects created using this
  constructor need to be later initialized using the appropriate
  \Co{initialize} method.
\begin{Example}
corSpher(value, form, nugget, metric, fixed)
\end{Example}
\begin{Argument}{ARGUMENTS}
\item[\Co{value:}] an optional vector with the parameter values in
  constrained form. If \Co{nugget} is \Co{FALSE}, \Co{value} can have
  only one element, corresponding to the "range" of the Spherical
  correlation structure, which must be greater than zero. If
  \Co{nugget} is \Co{TRUE}, meaning that a nugget effect is present,
  \Co{value} can contain one or two elements, the first being the
  "range" and the second the "nugget effect" (one minus the
  correlation between observations taken arbitrarily close together);
  the first must be greater than zero and the second must be between
  zero and one. Defaults to \Co{numeric(0)}, which results in a range
  of 90\% of the minimum distance and a nugget ratio of 0.9 being
  assigned to the parameters when \Co{object} is initialized.
\item[\Co{form:}]
a one sided formula of the form \Co{{\Twiddle} S1+...+Sp}, or
\Co{{\Twiddle} S1+...+Sp | g}, specifying spatial covariates \Co{S1}
through \Co{Sp} and,  optionally, a grouping factor \Co{g}. 
When a grouping factor is present in \Co{form}, the correlation
structure is assumed to apply only to observations within the same
grouping level; observations with different grouping levels are
assumed to be uncorrelated. Defaults to \Co{{\Twiddle} 1}, which corresponds
to using the order of the observations in the data as a covariate,
and no groups.
\item[\Co{nugget:}]
an optional logical value indicating whether a nugget
effect is present. Defaults to \Co{FALSE}.
\item[\Co{metric:}]
an optional character string specifying the distance
metric to be used. The currently available options are
\Co{"euclidian"} for the root sum-of-squares of distances;
\Co{"maximum"} for the maximum difference; and \Co{"manhattan"}
for the sum of the absolute differences. Partial matching of
arguments is used, so only the first three characters need to be
provided. Defaults to \Co{"euclidian"}.
\item[\Co{fixed:}]
an optional logical value indicating whether the
coefficients should be allowed to vary in the optimization, or kept
fixed at their initial values. Defaults to \Co{FALSE}, in which case
the coefficients are allowed to vary.
\end{Argument}
\Paragraph{VALUE}
an object of class \Co{corSpher}, also inheriting from class
\Co{corSpatial}, representing a spherical spatial correlation
structure.
\Paragraph{REFERENCES}
  Cressie, N.A.C. (1993), "Statistics for Spatial Data", J. Wiley \& Sons.\\
Venables, W.N. and Ripley, B.D. (1997) "Modern Applied Statistics with
S-plus", 2nd Edition, Springer-Verlag.\\
Littel, Milliken, Stroup, and Wolfinger (1997) "SAS Systems for Mixed
Models", SAS Institute.
\Paragraph{SEE ALSO}
\Co{initialize.corStruct}, \Co{dist}
\need 15pt
\Paragraph{EXAMPLE}
\vspace{-16pt}
\begin{Example}
sp1 <- corSpher(form = {\Twiddle} x + y)
\end{Example}
\end{Helpfile}
\begin{Helpfile}{corSymm}{General Correlation Structure}
This function is a constructor for the \Co{corSymm} class,
representing a general correlation structure. The internal
representation of this structure, in terms of unconstrained
parameters, uses the spherical parametrization defined in Pinheiro and
Bates (1996).  Objects created using this constructor need to be later
initialized using the  appropriate \Co{initialize} method.
\begin{Example}
corSymm(value, form, fixed)
\end{Example}
\begin{Argument}{ARGUMENTS}
\item[\Co{value:}]
an optional vector with the parameter values. Default is
\Co{numeric(0)}, which results in a vector of zeros of appropriate
dimension being assigned to the parameters when \Co{object} is
initialized (corresponding to an identity correlation structure).
\item[\Co{form:}]
a one sided formula of the form \Co{\Twiddle t}, or \Co{\Twiddle t |
g}, specifying a time covariate \Co{t} and,  optionally, a
grouping factor \Co{g}. A covariate for this correlation structure
must be integer valued. When a grouping factor is present in
\Co{form}, the correlation structure is assumed to apply only
to observations within the same grouping level; observations with
different grouping levels are assumed to be uncorrelated. Defaults to
\Co{\Twiddle 1}, which corresponds to using the order of the observations
in the data as a covariate, and no groups.
\item[\Co{fixed:}]
an optional logical value indicating whether the
coefficients should be allowed to vary in the optimization, or kept
fixed at their initial values. Defaults to \Co{FALSE}, in which case
the coefficients are allowed to vary.
\end{Argument}
\Paragraph{VALUE}
an object of class \Co{corSymm} representing a general correlation
structure.
\Paragraph{REFERENCES}
Pinheiro, J.C. and Bates., D.M.  (1996) "Unconstrained
Parametrizations for Variance-Covariance Matrices", Statistics and
Computing, 6, 289-296.
\Paragraph{SEE ALSO}
\Co{initialize.corSymm}
\need 15pt
\Paragraph{EXAMPLE}
\vspace{-16pt} 
\begin{Example}
## covariate is observation order and grouping factor is Subject
cs1 <- corSymm(form = \Twiddle 1 | Subject)
\end{Example}
\end{Helpfile}
\begin{Helpfile}{covariate\Co{<-}}{Assign Covariate Values}
This function is generic; method functions can be written to handle
specific classes of objects. Classes which already have methods for
this function include all \Co{varFunc} classes.
\begin{Example}
covariate(object) <- value
\end{Example}
\begin{Argument}{ARGUMENTS}
\item[\Co{object:}]
any object with a \Co{covariate} component.
\item[\Co{value:}]
a value to be assigned to the covariate associated with
\Co{object}.
\end{Argument}
\Paragraph{VALUE}
will depend on the method function; see the appropriate documentation.
\Paragraph{SEE ALSO}
\Co{getCovariate}
\need 15pt
\Paragraph{EXAMPLE}
\vspace{-16pt} 
\begin{Example}
## see the method function documentation
\end{Example}
\end{Helpfile}
\begin{Helpfile}{covariate\Co{<-}.varFunc}{Assign varFunc Covariate}
The covariate(s) used in the calculation of the weights of the
variance function represented by \Co{object} is (are) replaced by
\Co{value}. If \Co{object} has been initialized, \Co{value} must
have the same dimensions as \Co{getCovariate(object)}.
\begin{Example}
covariate(object) <- value
\end{Example}
\begin{Argument}{ARGUMENTS}
\item[\Co{object:}]
an object inheriting from class \Co{varFunc},
representing a variance function structure.
\item[\Co{value:}]
a value to be assigned to the covariate associated with
\Co{object}.
\end{Argument}
\Paragraph{VALUE}
a \Co{varFunc} object similar to \Co{object}, but with its
\Co{covariate} attribute replaced by \Co{value}.
\Paragraph{SEE ALSO}
\Co{getCovariate.varFunc}
\need 15pt
\Paragraph{EXAMPLE}
\vspace{-16pt} 
\begin{Example}
vf1 <- varPower(1.1, form = \Twiddle age)
covariate(vf1) <- Orthodont[["age"]]
\end{Example}
\end{Helpfile}
\begin{Helpfile}{Dim}{Extract Dimensions from an Object}
This function is generic; method functions can be written to handle
specific classes of objects. Classes which already have methods for
this function include: \Co{corSpatial}, \Co{corStruct},
\Co{pdCompSymm}, \Co{pdDiag}, \Co{pdIdent}, \Co{pdMat},
and \Co{pdSymm}.
\begin{Example}
Dim(object, ...)
\end{Example}
\begin{Argument}{ARGUMENTS}
\item[\Co{object:}]
any object for which dimensions can be extracted.
\item[\Co{...:}]
some methods for this generic function require additional
arguments.
\end{Argument}
\Paragraph{VALUE}
will depend on the method function used; see the appropriate documentation.
\Paragraph{NOTE} If \Co{dim} allowed more than one argument, there would be no
need for this generic function.
\Paragraph{SEE ALSO}
\Co{Dim.pdMat}, \Co{Dim.corStruct}
\need 15pt
\Paragraph{EXAMPLE}
\vspace{-16pt} 
\begin{Example}
## see the method function documentation
\end{Example}
\end{Helpfile}
\begin{Helpfile}{Dim.corSpatial}{Dimensions of a corSpatial Object}
if \Co{groups} is missing, it returns the \Co{Dim} attribute of
\Co{object}; otherwise, calculates the dimensions associated with
the grouping factor.
\begin{Example}
Dim(object, groups)
\end{Example}
\begin{Argument}{ARGUMENTS}
\item[\Co{object:}]
an object inheriting from class \Co{corSpatial},
representing a spatial correlation structure.
\item[\Co{groups:}]
an optional factor defining the grouping of the
observations; observations within a group are correlated and
observations in different groups are uncorrelated.
\end{Argument}
\Paragraph{VALUE}
a list with components:
\begin{Argument}{}
\vspace{-16pt} 
\item[\Co{N:}]
length of \Co{groups}
\item[\Co{M:}]
number of groups
\item[\Co{spClass:}]
an integer representing the spatial correlation class;
0 = user defined class, 1 = \Co{corSpher}, 2 = \Co{corExp}, 3 =
\Co{corGaus}, 4 = \Co{corLin}
\item[\Co{sumLenSq:}]
sum of the squares of the number of observations per
group
\item[\Co{len:}]
an integer vector with the number of observations per
group
\item[\Co{start:}]
an integer vector with the starting position for the
distance vectors in each group, beginning from zero
\end{Argument}
\Paragraph{SEE ALSO}
\Co{Dim}, \Co{Dim.corStruct}
\need 15pt
\Paragraph{EXAMPLE}
\vspace{-16pt} 
\begin{Example}
Dim(corGaus(), getGroups(Orthodont))
\end{Example}
\end{Helpfile}
\begin{Helpfile}{Dim.corStruct}{Dimensions of a corStruct Object}
if \Co{groups} is missing, it returns the \Co{Dim} attribute of
\Co{object}; otherwise, calculates the dimensions associated with
the grouping factor.
\begin{Example}
Dim(object, groups)
\end{Example}
\begin{Argument}{ARGUMENTS}
\item[\Co{object:}]
an object inheriting from class \Co{corStruct},
representing a correlation structure.
\item[\Co{groups:}]
an optional factor defining the grouping of the
observations; observations within a group are correlated and
observations in different groups are uncorrelated.
\end{Argument}
\Paragraph{VALUE}
a list with components:
\begin{Argument}{}
\vspace{-16pt} 
\item[\Co{N:}]
length of \Co{groups}
\item[\Co{M:}]
number of groups
\item[\Co{maxLen:}]
maximum number of observations in a group
\item[\Co{sumLenSq:}]
sum of the squares of the number of observations per
group
\item[\Co{len:}]
an integer vector with the number of observations per
group
\item[\Co{start:}]
an integer vector with the starting position for the
observations in each group, beginning from zero
\end{Argument}
\Paragraph{SEE ALSO}
\Co{Dim}, \Co{Dim.corSpatial}
\need 15pt
\Paragraph{EXAMPLE}
\vspace{-16pt} 
\begin{Example}
Dim(corAR1(), getGroups(Orthodont))
\end{Example}
\end{Helpfile}
\begin{Helpfile}{Dim.pdMat}{Dimensions of a pdMat Object}
This method function returns the dimensions of the matrix represented
by \Co{object}.
\begin{Example}
Dim(object)
\end{Example}
\begin{Argument}{ARGUMENTS}
\item[\Co{object:}]
an object inheriting from class \Co{pdMat},
representing a positive-definite matrix.
\end{Argument}
\Paragraph{VALUE}
an integer vector with the number of rows and columns of the
matrix represented by \Co{object}.
\Paragraph{SEE ALSO}
\Co{Dim}
\need 15pt
\Paragraph{EXAMPLE}
\vspace{-16pt} 
\begin{Example}
Dim(pdSymm(diag(3)))
\end{Example}
\end{Helpfile}
\begin{Helpfile}{fitted.gls}{Extract gls Fitted Values}
The fitted values for the linear model represented by \Co{object}
are extracted.
\begin{Example}
fitted(object)
\end{Example}
\begin{Argument}{ARGUMENTS}
\item[\Co{object:}]
an object inheriting from class \Co{gls}, representing
a generalized least squares fitted linear model.
\end{Argument}
\Paragraph{VALUE}
a vector with the fitted values for the linear model represented by
\Co{object}.
\Paragraph{SEE ALSO}
\Co{gls}, \Co{residuals.gls}
\need 15pt
\Paragraph{EXAMPLE}
\vspace{-16pt} 
\begin{Example}
fm1 <- gls(follicles \Twiddle sin(2*pi*Time) + cos(2*pi*Time), Ovary,
           correlation = corAR1(form = \Twiddle 1 | Mare))
fitted(fm1)
\end{Example}
\end{Helpfile}
\begin{Helpfile}{fitted.glsStruct}{Calculate glsStruct Fitted Values}
The fitted values for the linear model represented by \Co{object}
are extracted.
\begin{Example}
fitted(object, glsFit)
\end{Example}
\begin{Argument}{ARGUMENTS}
\item[\Co{object:}]
an object inheriting from class \Co{glsStruct},
representing a list of linear model components, such as
\Co{corStruct} and \Co{varFunc} objects.
\item[\Co{glsFit:}]
an optional list with components \Co{logLik}
(log-likelihood), \Co{beta} (coefficients), \Co{sigma} (standard
deviation for error term), \Co{varBeta} (coefficients' covariance
matrix), \Co{fitted} (fitted values), and \Co{residuals}
(residuals). Defaults to \Co{attr(object, "glsFit")}.
\end{Argument}
\Paragraph{VALUE}
a vector with the fitted values for the linear model represented by
\Co{object}.
\Paragraph{NOTE} This method function is generally only used inside \Co{gls} and 
\Co{fitted.gls}.
\Paragraph{SEE ALSO}
\Co{gls}, \Co{fitted.gls},
\Co{residuals.glsStruct}
\end{Helpfile}
\begin{Helpfile}{fitted.gnls}{Extract gnls Fitted Values}
The fitted values for the nonlinear model represented by \Co{object}
are extracted.
\begin{Example}
fitted(object)
\end{Example}
\begin{Argument}{ARGUMENTS}
\item[\Co{object:}]
an object inheriting from class \Co{gnls}, representing
a generalized nonlinear least squares fitted model.
\end{Argument}
\Paragraph{VALUE}
a vector with the fitted values for the nonlinear model represented by
\Co{object}.
\Paragraph{SEE ALSO}
\Co{gnls}, \Co{residuals.gnls}
\need 15pt
\Paragraph{EXAMPLE}
\vspace{-16pt}
\begin{Example}
fm1 <- gnls(weight {\Twiddle} SSlogis(Time, Asym, xmid, scal), Soybean,
            weights = varPower())
fitted(fm1)
\end{Example}
\end{Helpfile}
\begin{Helpfile}{fitted.gnlsStruct}{Calculate gnlsStruct Fitted Values}
The fitted values for the nonlinear model represented by \Co{object}
are extracted.
\begin{Example}
fitted(object)
\end{Example}
\begin{Argument}{ARGUMENTS}
\item[\Co{object:}]
an object inheriting from class \Co{gnlsStruct},
representing a list of model components, such as
\Co{corStruct} and \Co{varFunc} objects, and attributes
specifying the underlying nonlinear model and the response variable.
\end{Argument}
\Paragraph{VALUE}
a vector with the fitted values for the nonlinear model represented by
\Co{object}.
\Paragraph{NOTE} This method function is generally only used inside \Co{gnls} and 
\Co{fitted.gnls}.
\Paragraph{SEE ALSO}
\Co{gnls}, \Co{fitted.gnls},
\Co{residuals.gnlsStruct}
\end{Helpfile}
\begin{Helpfile}{fitted.lmList}{Extract lmList Fitted Values}
The fitted values are extracted from each \Co{lm} component of
\Co{object} and arranged into a list with as many components as
\Co{object}, or combined into a single vector.
\begin{Example}
fitted(object, subset, asList)
\end{Example}
\begin{Argument}{ARGUMENTS}
\item[\Co{object:}]
an object inheriting from class \Co{lmList}, representing
a list of \Co{lm} objects with a common model.
\item[\Co{subset:}]
an optional character or integer vector naming the
\Co{lm} components of \Co{object} from which the fitted values
are to be extracted. Default is \Co{NULL}, in which case all
components are used.
\item[\Co{asList:}]
an optional logical value. If \Co{TRUE}, the returned
object is a list with the fitted values split by groups; else the
returned value is a vector. Defaults to \Co{FALSE}.
\end{Argument}
\Paragraph{VALUE}
a list with components given by the fitted values of each \Co{lm}
component of \Co{object}, or a vector with the fitted values for all
\Co{lm} components of \Co{object}.
\Paragraph{SEE ALSO}
\Co{lmList}, \Co{residuals.lmList}
\need 15pt
\Paragraph{EXAMPLE}
\vspace{-16pt}
\begin{Example}
fm1 <- lmList(distance {\Twiddle} age | Subject, Orthodont)
fitted(fm1)
\end{Example}
\end{Helpfile}
\begin{Helpfile}{fitted.lmList}{Extract lmList Fitted Values}
The fitted values are extracted from each \Co{lm} component of
\Co{object} and arranged into a list with as many components as
\Co{object}, or combined into a single vector.
\begin{Example}
fitted(object, subset, asList)
\end{Example}
\begin{Argument}{ARGUMENTS}
\item[\Co{object:}]
an object inheriting from class \Co{lmList}, representing
a list of \Co{lm} objects with a common model.
\item[\Co{subset:}]
an optional character or integer vector naming the
\Co{lm} components of \Co{object} from which the fitted values
are to be extracted. Default is \Co{NULL}, in which case all
components are used.
\item[\Co{asList:}]
an optional logical value. If \Co{TRUE}, the returned
object is a list with the fitted values split by groups; else the
returned value is a vector. Defaults to \Co{FALSE}.
\end{Argument}
\Paragraph{VALUE}
a list with components given by the fitted values of each \Co{lm}
component of \Co{object}, or a vector with the fitted values for all
\Co{lm} components of \Co{object}.
\Paragraph{SEE ALSO}
\Co{lmList}, \Co{residuals.lmList}
\need 15pt
\Paragraph{EXAMPLE}
\vspace{-16pt} 
\begin{Example}
fm1 <- lmList(distance \Twiddle age, Orthodont, groups = \Twiddle Subject)
fitted(fm1)
\end{Example}
\end{Helpfile}
\begin{Helpfile}{fitted.lme}{Extract lme Fitted Values}
The fitted values at level i are obtained by adding together the
population fitted values (based only on the fixed effects estimates)
and the estimated contributions of the random effects to the fitted
values at grouping levels less or equal to i. The resulting
values estimate the best linear unbiased predictions (BLUPs) at level
i.
\begin{Example}
fitted(object, level, asList)
\end{Example}
\begin{Argument}{ARGUMENTS}
\item[\Co{object:}]
an object inheriting from class \Co{lme}, representing
a fitted linear mixed-effects model.
\item[\Co{level:}]
an optional integer vector giving the level(s) of grouping
to be used in extracting the fitted values from \Co{object}. Level
values increase from outermost to innermost grouping, with
level zero corresponding to the population fitted values. Defaults to
the highest or innermost level of grouping.
\item[\Co{asList:}]
an optional logical value. If \Co{TRUE} and a single
value is given in \Co{level}, the returned object is a list with
the fitted values split by groups; else the returned value is
either a vector or a data frame, according to the length of
\Co{level}. Defaults to \Co{FALSE}.
\end{Argument}
\Paragraph{VALUE}
if a single level of grouping is specified in \Co{level}, the
returned value is either a list with the fitted values split by groups
(\Co{asList = TRUE}) or a vector with the fitted values
(\Co{asList = FALSE}); else, when multiple grouping levels are
specified in \Co{level}, the returned object is a data frame with
columns given by the fitted values at different levels and the
grouping factors.
\Paragraph{REFERENCES}
Bates, D.M. and Pinheiro, J.C. (1998) "Computational methods for
multilevel models" available in PostScript or PDF formats at \\
http://nlme.stat.wisc.edu
\Paragraph{SEE ALSO}
\Co{lme}, \Co{residuals.lme}
\need 15pt
\Paragraph{EXAMPLE}
\vspace{-16pt} 
\begin{Example}
fm1 <- lme(distance \Twiddle age + Sex, data = Orthodont, random = \Twiddle 1)
fitted(fm1, level = 0:1)
\end{Example}
\end{Helpfile}
\begin{Helpfile}{fitted.lmeStruct}{Calculate lmeStruct Fitted Values}
The fitted values at level i are obtained by adding together the
population fitted values (based only on the fixed effects estimates)
and the estimated contributions of the random effects to the fitted
values at grouping levels less or equal to i. The resulting
values estimate the best linear unbiased predictions (BLUPs) at level
i.
\begin{Example}
fitted(object, levels, lmeFit, conLin)
\end{Example}
\begin{Argument}{ARGUMENTS}
\item[\Co{object:}]
an object inheriting from class \Co{lmeStruct},
representing a list of linear mixed-effects model components, such as
\Co{reStruct}, \Co{corStruct}, and \Co{varFunc} objects.
\item[\Co{level:}]
an optional integer vector giving the level(s) of grouping
to be used in extracting the fitted values from \Co{object}. Level
values increase from outermost to innermost grouping, with
level zero corresponding to the population fitted values. Defaults to
the highest or innermost level of grouping.
\item[\Co{lmeFit:}]
an optional list with components \Co{beta} and \Co{b}
containing respectively the fixed effects estimates and the random
effects estimates to be used to calculate the fitted values. Defaults
to \Co{attr(object, "lmeFit")}.
\item[\Co{conLin:}]
an optional condensed linear model object, consisting of
a list with components \Co{"Xy"}, corresponding to a regression
matrix (\Co{X}) combined with a response vector (\Co{y}), and 
\Co{"logLik"}, corresponding to the log-likelihood of the
underlying lme model. Defaults to \Co{attr(object, "conLin")}.
\end{Argument}
\Paragraph{VALUE}
if a single level of grouping is specified in \Co{level},
the returned value is a vector with the fitted values at the desired
level; else, when multiple grouping levels are specified in
\Co{level}, the returned object is a matrix with 
columns given by the fitted values at different levels.
\Paragraph{NOTE} This method function is generally only used inside \Co{lme} and 
\Co{fitted.lme}.
\Paragraph{REFERENCES}
Bates, D.M. and Pinheiro, J.C. (1998) "Computational methods for
multilevel models" available in PostScript or PDF formats at
http://nlme.stat.wisc.edu
\Paragraph{SEE ALSO}
\Co{lme}, \Co{fitted.lme},
\Co{residuals.lmeStruct}
\end{Helpfile}
\begin{Helpfile}{fitted.nlmeStruct}{Calculate nlmeStruct Fitted Values}
The fitted values at level i are obtained by adding together the
contributions from the estimated fixed effects and the estimated
random effects at levels less or equal to i and evaluating the
model function at the resulting estimated parameters. The resulting
values estimate the predictions at level i.
\begin{Example}
fitted(object, levels, nlmeFit, conLin)
\end{Example}
\begin{Argument}{ARGUMENTS}
\item[\Co{object:}]
an object inheriting from class \Co{nlmeStruct},
representing a list of mixed-effects model components, such as
\Co{reStruct}, \Co{corStruct}, and \Co{varFunc} objects, plus
attributes  specifying the underlying nonlinear model and the
response variable.
\item[\Co{level:}]
an optional integer vector giving the level(s) of grouping
to be used in extracting the fitted values from \Co{object}. Level
values increase from outermost to innermost grouping, with
level zero corresponding to the population fitted values. Defaults to
the highest or innermost level of grouping.
\item[\Co{conLin:}]
an optional condensed linear model object, consisting of
a list with components \Co{"Xy"}, corresponding to a regression
matrix (\Co{X}) combined with a response vector (\Co{y}), and 
\Co{"logLik"}, corresponding to the log-likelihood of the
underlying nlme model. Defaults to \Co{attr(object, "conLin")}.
\end{Argument}
\Paragraph{VALUE}
if a single level of grouping is specified in \Co{level},
the returned value is a vector with the fitted values at the desired
level; else, when multiple grouping levels are specified in
\Co{level}, the returned object is a matrix with 
columns given by the fitted values at different levels.
\Paragraph{NOTE} This method function is generally only used inside \Co{nlme} and 
\Co{fitted.nlme}
\Paragraph{REFERENCES}
Bates, D.M. and Pinheiro, J.C. (1998) "Computational methods for
multilevel models" available in PostScript or PDF formats at
http://nlme.stat.wisc.edu
\Paragraph{SEE ALSO}
\Co{nlme}, \Co{fitted.nlme},
\Co{residuals.nlmeStruct}
\end{Helpfile}
\begin{Helpfile}{fixed.effects}{Extract Fixed Effects}
This function is generic; method functions can be written to handle
specific classes of objects. Classes which already have methods for
this function include \Co{lmList} and \Co{lme}.
\begin{Example}
fixed.effects(object, ...)
fixef(object, ...)
\end{Example}
\begin{Argument}{ARGUMENTS}
\item[\Co{object:}]
any fitted model object from which fixed effects
estimates can be extracted.
\item[\Co{...:}]
some methods for this generic function require additional
arguments.
\end{Argument}
\Paragraph{VALUE}
will depend on the method function used; see the appropriate documentation.
\Paragraph{SEE ALSO}
\Co{fixef.lmList},\Co{fixef.lme}
\need 15pt
\Paragraph{EXAMPLE}
\vspace{-16pt}
\begin{Example}
## see the method function documentation
\end{Example}
\end{Helpfile}
\begin{Helpfile}{fixef}{Extract Fixed Effects}
This function is generic; method functions can be written to handle
specific classes of objects. Classes which already have methods for
this function include \Co{lmList} and \Co{lme}.
\begin{Example}
fixef(object, ...)
fixed.effects(object, ...)
\end{Example}
\begin{Argument}{ARGUMENTS}
\item[\Co{object:}]
any fitted model object from which fixed effects
estimates can be extracted.
\item[\Co{...:}]
some methods for this generic function require additional
arguments.
\end{Argument}
\Paragraph{VALUE}
will depend on the method function used; see the appropriate documentation.
\Paragraph{SEE ALSO}
\Co{fixef.lmList},\Co{fixef.lme}
\need 15pt
\Paragraph{EXAMPLE}
\vspace{-16pt}
\begin{Example}
## see the method function documentation
\end{Example}
\end{Helpfile}
\begin{Helpfile}{fixef.lmList}{Extract lmList Fixed Effects}
The average of the coefficients corresponding to the \Co{lm}
components of \Co{object} is calculated.
\begin{Example}
fixef(object)
\end{Example}
\begin{Argument}{ARGUMENTS}
\item[\Co{object:}]
an object inheriting from class \Co{lmList}, representing
a list of \Co{lm} objects with a common model.
\end{Argument}
\Paragraph{VALUE}
a vector with the average of the individual \Co{lm} coefficients in
\Co{object}.
\Paragraph{SEE ALSO}
\Co{lmList}, \Co{ranef.lmList}
\need 15pt
\Paragraph{EXAMPLE}
\vspace{-16pt}
\begin{Example}
fm1 <- lmList(distance {\Twiddle} age | Subject, Orthodont)
fixef(fm1)
\end{Example}
\end{Helpfile}
\begin{Helpfile}{fixef.lme}{Extract lme Fixed Effects}
The fixed effects estimates corresponding to the linear mixed-effects
model represented by \Co{object} are returned.
\begin{Example}
fixef(object)
\end{Example}
\begin{Argument}{ARGUMENTS}
\item[\Co{object:}]
an object inheriting from class \Co{lme}, representing
a fitted linear mixed-effects model.
\end{Argument}
\Paragraph{VALUE}
a vector with the fixed effects estimates corresponding to
\Co{object}.
\Paragraph{SEE ALSO}
\Co{coef.lme}, \Co{ranef.lme}
\need 15pt
\Paragraph{EXAMPLE}
\vspace{-16pt}
\begin{Example}
fm1 <- lme(distance {\Twiddle} age, Orthodont, random = {\Twiddle} age | Subject)
fixef(fm1)
\end{Example}
\end{Helpfile}
\begin{Helpfile}{formula.corStruct}{Extract corStruct Formula}
This method function extracts the formula associated with a
\Co{corStruct} object, in which the covariate and the grouping
factor, if any is present, are defined.
\begin{Example}
formula(object)
\end{Example}
\begin{Argument}{ARGUMENTS}
\item[\Co{object:}]
an object inheriting from class \Co{corStruct}
representing a correlation structure.
\end{Argument}
\Paragraph{VALUE}
an object of class \Co{formula} specifying the covariate and the
grouping factor, if any is present, associated with \Co{object}.
\Paragraph{SEE ALSO}
\Co{formula}
\need 15pt
\Paragraph{EXAMPLE}
\vspace{-16pt} 
\begin{Example}
cs1 <- corCAR1(form = \Twiddle Time | Mare)
formula(cs1)
\end{Example}
\end{Helpfile}
\begin{Helpfile}{formula.gls}{Extract gls Formula}
This method function extracts the linear model formula
associated with \Co{object}.
\begin{Example}
formula(object)
\end{Example}
\begin{Argument}{ARGUMENTS}
\item[\Co{object:}]
an object inheriting from class \Co{gls}, representing
a generalized least squares fitted linear model.
\end{Argument}
\Paragraph{VALUE}
a two-sided linear formula specifying the linear model used to
obtain \Co{object}.
\Paragraph{SEE ALSO}
\Co{gls}
\need 15pt
\Paragraph{EXAMPLE}
\vspace{-16pt} 
\begin{Example}
fm1 <- gls(follicles \Twiddle sin(2*pi*Time) + cos(2*pi*Time), Ovary,
           correlation = corAR1(form = \Twiddle 1 | Mare))
formula(fm1)
\end{Example}
\end{Helpfile}
\begin{Helpfile}{formula.gnls}{Extract gnls Object Formula}
This method function extracts the nonlinear model formula
associated with \Co{object}.
\begin{Example}
formula(object)
\end{Example}
\begin{Argument}{ARGUMENTS}
\item[\Co{object:}]
an object inheriting from class \Co{gnls}, representing
a generalized nonlinear least squares fitted model.
\end{Argument}
\Paragraph{VALUE}
a two-sided formula specifying the nonlinear model used to
obtain \Co{object}.
\Paragraph{SEE ALSO}
\Co{gnls}
\need 15pt
\Paragraph{EXAMPLE}
\vspace{-16pt}
\begin{Example}
fm1 <- gnls(weight {\Twiddle} SSlogis(Time, Asym, xmid, scal), Soybean,
            weights = varPower())
formula(fm1)
\end{Example}
\end{Helpfile}
\begin{Helpfile}{formula.groupedData}{Extract groupedData Formula}
This method function extracts the display formula associated with a
\Co{groupedData} object. This is a two-sided formula of the form
\Co{resp \Twiddle cov | group}, with \Co{resp} is the response,
\Co{cov} is the primary covariate, and \Co{group} is the grouping
structure.
\begin{Example}
formula(object)
\end{Example}
\begin{Argument}{ARGUMENTS}
\item[\Co{object:}]
an object inheriting from class \Co{groupedData}.
\end{Argument}
\Paragraph{VALUE}
a two-sided formula with a conditioning expression, representing the
display formula for \Co{object}.
\Paragraph{SEE ALSO}
\Co{groupedData}
\need 15pt
\Paragraph{EXAMPLE}
\vspace{-16pt} 
\begin{Example}
formula(Orthodont)
\end{Example}
\end{Helpfile}
\begin{Helpfile}{formula.lmList}{Extract lmList Object Formula}
This method function extracts the common linear model formula
associated with each \Co{lm} component of \Co{object}.
\begin{Example}
formula(object)
\end{Example}
\begin{Argument}{ARGUMENTS}
\item[\Co{object:}]
an object inheriting from class \Co{lmList}, representing
a list of \Co{lm} objects with a common model.
\end{Argument}
\Paragraph{VALUE}
a two-sided linear formula specifying the linear model used to
obtain the \Co{lm} components of \Co{object}.
\Paragraph{SEE ALSO}
\Co{lmList}
\need 15pt
\Paragraph{EXAMPLE}
\vspace{-16pt}
\begin{Example}
fm1 <- lmList(distance {\Twiddle} age | Subject, Orthodont)
formula(fm1)
\end{Example}
\end{Helpfile}
\begin{Helpfile}{formula.lme}{Extract lme Formula}
This method function extracts the fixed effects model formula
associated with \Co{object}.
\begin{Example}
formula(object)
\end{Example}
\begin{Argument}{ARGUMENTS}
\item[\Co{object:}]
an object inheriting from class \Co{lme}, representing
a fitted linear mixed-effects model.
\end{Argument}
\Paragraph{VALUE}
a two-sided linear formula specifying the fixed effects model used to
obtain \Co{object}.
\Paragraph{SEE ALSO}
\Co{lme}
\need 15pt
\Paragraph{EXAMPLE}
\vspace{-16pt} 
\begin{Example}
fm1 <- lme(distance \Twiddle age, Orthodont, random = \Twiddle age | Subject)
formula(fm1)
\end{Example}
\end{Helpfile}
\begin{Helpfile}{formula.modelStruct}{Extract modelStructFormula}
This method function extracts a formula from each of the 
components of \Co{object}, returning a list of formulas.
\begin{Example}
formula(object)
\end{Example}
\begin{Argument}{ARGUMENTS}
\item[\Co{object:}]
an object inheriting from class \Co{modelStruct},
representing a list of model components, such as \Co{corStruct} and
\Co{varFunc} objects.
\end{Argument}
\Paragraph{VALUE}
a list with the formulas of each component of \Co{object}.
\Paragraph{SEE ALSO}
\Co{formula}
\need 15pt
\Paragraph{EXAMPLE}
\vspace{-16pt} 
\begin{Example}
lms1 <- lmeStruct(reStruct = reStruct(pdDiag(diag(2), \Twiddle age)),
   corStruct = corAR1(0.3))
formula(lms1)
\end{Example}
\end{Helpfile}
\begin{Helpfile}{formula.nlme}{Extract nlme Object Formula}
This method function extracts the nonlinear model formula
associated with \Co{object}.
\begin{Example}
formula(object)
\end{Example}
\begin{Argument}{ARGUMENTS}
\item[\Co{object:}]
an object inheriting from class \Co{nlme}, representing
a fitted nonlinear mixed-effects model.
\end{Argument}
\Paragraph{VALUE}
a two-sided nonlinear formula specifying the model used to
obtain \Co{object}.
\Paragraph{SEE ALSO}
\Co{nlme}
\need 15pt
\Paragraph{EXAMPLE}
\vspace{-16pt}
\begin{Example}
fm1 <- nlme(weight {\Twiddle} SSlogis(Time, Asym, xmid, scal), 
            data = Soybean, fixed = Asym + xmid + scal {\Twiddle} 1, 
            start = c(18, 52, 7.5))
formula(fm1)
\end{Example}
\end{Helpfile}
\begin{Helpfile}{formula.nlsList}{Extract nlsList Object Formula}
This method function extracts the common nonlinear model formula
associated with each \Co{nls} component of \Co{object}.
\begin{Example}
formula(object)
\end{Example}
\begin{Argument}{ARGUMENTS}
\item[\Co{object:}]
an object inheriting from class \Co{nlsList}, representing
a list of \Co{nls} objects with a common model.
\end{Argument}
\Paragraph{VALUE}
a two-sided nonlinear formula specifying the model used to
obtain the \Co{nls} components of \Co{object}.
\Paragraph{SEE ALSO}
\Co{nlsList}
\need 15pt
\Paragraph{EXAMPLE}
\vspace{-16pt}
\begin{Example}
fm1 <- nlsList(weight {\Twiddle} SSlogis(Time, Asym, xmid, scal)|Plot, 
               data=Soybean)
formula(fm1)
\end{Example}
\end{Helpfile}
\begin{Helpfile}{formula.nls}{Extract Model Formula from nls Object}
Returns the model used to fit \Co{object}.
\begin{Example}
formula(object)
\end{Example}
\begin{Argument}{ARGUMENTS}
\item[\Co{object:}]
an object inheriting from class \Co{nls}, representing
a non-linear least squares fit.
\end{Argument}
\Paragraph{VALUE}
a formula representing the model used to obtain \Co{object}.
\Paragraph{SEE ALSO}
\Co{nls}, \Co{formula}
\need 15pt
\Paragraph{EXAMPLE}
\vspace{-16pt} 
\begin{Example}
fm1 <- nls(circumference \Twiddle A/(1+exp((B-age)/C)), Orange,
  start = list(A=160, B=700, C = 350))
formula(fm1)
\end{Example}
\end{Helpfile}
\begin{Helpfile}{formula.pdBlocked}{Extract pdBlocked Formula}
The \Co{formula} attributes of the \Co{pdMat} elements of
\Co{object} are extracted and returned as a list, in case
\Co{asList=TRUE}, or converted to a single one-sided formula when
\Co{asList=FALSE}. If the \Co{pdMat} elements do not have a
\Co{formula} attribute, a \Co{NULL} value is returned.
\begin{Example}
formula(object, asList)
\end{Example}
\begin{Argument}{ARGUMENTS}
\item[\Co{object:}]
an object inheriting from class \Co{pdBlocked},
representing a positive definite block diagonal matrix.
\item[\Co{asList:}]
an optional logical value. If \Co{TRUE}, a list with
the formulas for the individual block diagonal elements of
\Co{object} is returned; else, if \Co{FALSE}, a one-sided formula
combining all individual formulas is returned. Defaults to
\Co{FALSE}.
\end{Argument}
\Paragraph{VALUE}
a list of one-sided formulas, or a single one-sided formula, or
\Co{NULL}.
\Paragraph{SEE ALSO}
\Co{pdBlocked}, \Co{pdMat}
\need 15pt
\Paragraph{EXAMPLE}
\vspace{-16pt} 
\begin{Example}
pd1 <- pdBlocked(list(\Twiddle age, \Twiddle Sex - 1))
formula(pd1)
formula(pd1, asList = TRUE)
\end{Example}
\end{Helpfile}
\begin{Helpfile}{formula.pdMat}{Extract pdMat Formula}
This method function extracts the formula associated with a
\Co{pdMat} object, in which the column and row names are specified.
\begin{Example}
formula(object)
\end{Example}
\begin{Argument}{ARGUMENTS}
\item[\Co{object:}]
an object inheriting from class \Co{pdMat}, representing
a positive definite matrix.
\end{Argument}
\Paragraph{VALUE}
if \Co{object} has a \Co{formula} attribute, its value is
returned, else \Co{NULL} is returned.
\Paragraph{NOTE} Because factors may be present in \Co{formula(object)}, the
\Co{pdMat} object needs to have access to a data frame where the
variables named in the formula can be evaluated, before it can resolve
its row and column names from the formula.
\Paragraph{SEE ALSO}
\Co{pdMat}
\need 15pt
\Paragraph{EXAMPLE}
\vspace{-16pt} 
\begin{Example}
pd1 <- pdSymm(\Twiddle Sex*age)
formula(pd1)
\end{Example}
\end{Helpfile}
\begin{Helpfile}{formula.reStruct}{Extract reStruct Formula}
This method function extracts a formula from each of the 
components of \Co{object}, returning a list of formulas.
\begin{Example}
formula(object)
\end{Example}
\begin{Argument}{ARGUMENTS}
\item[\Co{object:}]
an object inheriting from class \Co{reStruct},
representing a random effects structure and consisting of a list of
\Co{pdMat} objects.
\end{Argument}
\Paragraph{VALUE}
a list with the formulas of each component of \Co{object}.
\Paragraph{SEE ALSO}
\Co{formula}
\need 15pt
\Paragraph{EXAMPLE}
\vspace{-16pt} 
\begin{Example}
rs1 <- reStruct(list(A = pdDiag(diag(2), \Twiddle age), B = \Twiddle 1))
formula(rs1)
\end{Example}
\end{Helpfile}
\begin{Helpfile}{formula.varFunc}{Extract varFunc Formula}
This method function extracts the formula associated with a
\Co{varFunc} object, in which covariates and grouping factors are
specified.
\begin{Example}
formula(object)
\end{Example}
\begin{Argument}{ARGUMENTS}
\item[\Co{object:}]
an object inheriting from class \Co{varFunc},
representing a variance function structure.
\end{Argument}
\Paragraph{VALUE}
if \Co{object} has a \Co{formula} attribute, its value is
returned; else \Co{NULL} is returned.
\need 15pt
\Paragraph{EXAMPLE}
\vspace{-16pt} 
\begin{Example}
formula(varPower(form = \Twiddle fitted(.) | Sex))
\end{Example}
\end{Helpfile}
\begin{Helpfile}{gapply}{Apply a Function by Groups}
Applies the function to the distinct sets of rows of the data frame
defined by \Co{groups}.
\begin{Example}
gapply(object, which, FUN, form, level, groups, ...)
\end{Example}
\begin{Argument}{ARGUMENTS}
\item[\Co{object:}]
an object to which the function will be applied - usually
a \Co{groupedData} object or a \Co{data.frame}.
\item[\Co{which:}]
an optional character or positive integer vector specifying which
columns of \Co{object} should be used with \Co{FUN}. Defaults to all columns
in \Co{object}. 
\item[\Co{FUN:}]
function to apply to the distinct sets of rows
of the data frame \Co{object} defined by the values of
\Co{groups}.
\item[\Co{form:}]
an optional one-sided formula that defines the groups.
When this formula is given the right-hand side is evaluated in
\Co{object}, converted to a factor if necessary, and the unique
levels are used to define the groups.  Defaults to
\Co{formula(object)}.
\item[\Co{level:}]
an optional positive integer giving the level of grouping
to be used in an object with multiple nested grouping levels.
Defaults to the highest or innermost level of grouping.
\item[\Co{groups:}]
an optional factor that will be used to split the 
rows into groups.  Defaults to \Co{getGroups(object, form, level)}.
\item[\Co{...:}]
optional additional arguments to the summary function
\Co{FUN}.  Often it is helpful to specify \Co{na.rm = TRUE}.
\end{Argument}
\Paragraph{VALUE}
Returns a data frame with as many rows as there are levels in the
\Co{groups} argument.
\Paragraph{SEE ALSO}
\Co{gsummary}
\need 15pt
\Paragraph{EXAMPLE}
\vspace{-16pt} 
\begin{Example}
## Find number of non-missing "conc" observations for each Subject
gapply( Quinidine, FUN = function(x) sum(!is.na(x\$conc)) )
\end{Example}
\end{Helpfile}
\begin{Helpfile}{getCovariate}{Extract Covariate from an Object}
This function is generic; method functions can be written to handle
specific classes of objects. Classes which already have methods for
this function include \Co{corStruct}, \Co{corSpatial},
\Co{data.frame}, and \Co{varFunc}.
\begin{Example}
getCovariate(object, form, data)
\end{Example}
\begin{Argument}{ARGUMENTS}
\item[\Co{object:}]
any object with a \Co{covariate} component
\item[\Co{form:}]
an optional one-sided formula specifying the covariate(s)
to be extracted. Defaults to \Co{formula(object)}.
\item[\Co{data:}]
a data frame in which to evaluate the variables defined in
\Co{form}. 
\end{Argument}
\Paragraph{VALUE}
will depend on the method function used; see the appropriate documentation.
\Paragraph{SEE ALSO}
\Co{getCovariateFormula}
\need 15pt
\Paragraph{EXAMPLE}
\vspace{-16pt} 
\begin{Example}
## see the method function documentation
\end{Example}
\end{Helpfile}
\begin{Helpfile}{getCovariate.corStruct}{corStruct Covariate}
This method function extracts the covariate(s) associated with
\Co{object}.
\begin{Example}
getCovariate(object, form, data)
\end{Example}
\begin{Argument}{ARGUMENTS}
\item[\Co{object:}]
an object inheriting from class \Co{corStruct}
representing a correlation structure.
\item[\Co{form:}]
this argument is included to make the method function
compatible with the generic. It will be assigned the value of
\Co{formula(object)} and should not be modified.
\item[\Co{data:}]
an optional data frame in which to evaluate the variables
defined in \Co{form}, in case \Co{object} is not initialized and
the covariate needs to be evaluated.
\end{Argument}
\Paragraph{VALUE}
when the correlation structure does not include a grouping factor, the
returned value will be a vector or a matrix with the covariate(s)
associated with \Co{object}. If a grouping factor is present, the
returned value will be a list of vectors or matrices with the
covariate(s) corresponding to each grouping level.
\Paragraph{SEE ALSO}
\Co{getCovariate}
\need 15pt
\Paragraph{EXAMPLE}
\vspace{-16pt} 
\begin{Example}
cs1 <- corAR1(form = \Twiddle 1 | Subject)
getCovariate(cs1, data = Orthodont)
\end{Example}
\end{Helpfile}
\begin{Helpfile}{getCovariate.data.frame}{Data Frame Covariate}
The right hand side of \Co{form}, stripped of any conditioning
expression (i.e.\ an expression following a \Co{|} operator), is
evaluated in \Co{object}.
\begin{Example}
getCovariate(object, form)
\end{Example}
\begin{Argument}{ARGUMENTS}
\item[\Co{object:}]
an object inheriting from class \Co{data.frame}.
\item[\Co{form:}]
an optional formula specifying the covariate to be
evaluated in \Co{object}. Defaults to \Co{formula(object)}.
\end{Argument}
\Paragraph{VALUE}
the value of the right hand side of \Co{form}, stripped of
any conditional expression, evaluated in \Co{object}.
\Paragraph{SEE ALSO}
\Co{getCovariateFormula}
\need 15pt
\Paragraph{EXAMPLE}
\vspace{-16pt} 
\begin{Example}
getCovariate(Orthodont)
\end{Example}
\end{Helpfile}
\begin{Helpfile}{getCovariate.varFunc}{Extract varFunc Covariate}
This method function extracts the covariate(s) associated with the
variance function represented by \Co{object}, if any is present.
\begin{Example}
getCovariate(object)
\end{Example}
\begin{Argument}{ARGUMENTS}
\item[\Co{object:}]
an object inheriting from class \Co{varFunc},
representing a variance function structure.
\end{Argument}
\Paragraph{VALUE}
if \Co{object} has a \Co{covariate} attribute, its value is
returned; else \Co{NULL} is returned.
\Paragraph{SEE ALSO}
\Co{covariate<-.varFunc}
\need 15pt
\Paragraph{EXAMPLE}
\vspace{-16pt} 
\begin{Example}
vf1 <- varPower(1.1, form = \Twiddle age)
covariate(vf1) <- Orthodont[["age"]]
getCovariate(vf1)
\end{Example}
\end{Helpfile}
\begin{Helpfile}{getCovariateFormula}{Extract Covariates Formula}
The right hand side of \Co{formula(object)}, without any
conditioning expressions (i.e.\ any expressions after a \Co{|}
operator) is returned as a one-sided formula.
\begin{Example}
getCovariateFormula(object)
\end{Example}
\begin{Argument}{ARGUMENTS}
\item[\Co{object:}]
any object from which a formula can be extracted.
\end{Argument}
\Paragraph{VALUE}
a one-sided formula describing the covariates associated with
\Co{formula(object)}.
\Paragraph{SEE ALSO}
\Co{getCovariate}
\need 15pt
\Paragraph{EXAMPLE}
\vspace{-16pt} 
\begin{Example}
getCovariateFormula(y \Twiddle x | g)
getCovariateFormula(y \Twiddle x)
\end{Example}
\end{Helpfile}
\begin{Helpfile}{getData}{Extract Data from an Object}
This function is generic; method functions can be written to handle
specific classes of objects. Classes which already have methods for
this function include \Co{gls}, \Co{lme},
and \Co{lmList}.
\begin{Example}
getData(object)
\end{Example}
\begin{Argument}{ARGUMENTS}
\item[\Co{object:}]
an object from which a data.frame can be extracted,
generally a fitted model object.
\end{Argument}
\Paragraph{VALUE}
will depend on the method function used; see the appropriate documentation.
\need 15pt
\Paragraph{EXAMPLE}
\vspace{-16pt} 
\begin{Example}
## see the method function documentation
\end{Example}
\end{Helpfile}
\begin{Helpfile}{getData.gls}{Extract gls Object Data}
If present in the calling sequence used to produce \Co{object}, the
data frame used to fit the model is obtained.
\begin{Example}
getData(object)
\end{Example}
\begin{Argument}{ARGUMENTS}
\item[\Co{object:}]
an object inheriting from class \Co{gls}, representing
a generalized least squares fitted linear model.
\end{Argument}
\Paragraph{VALUE}
if a \Co{data} argument is present in the calling sequence that
produced \Co{object}, the corresponding data frame (with
\Co{na.action} and \Co{subset} applied to it, if also present in
the call that produced \Co{object}) is returned;
else, \Co{NULL} is returned.
\Paragraph{SEE ALSO}
\Co{gls}
\need 15pt
\Paragraph{EXAMPLE}
\vspace{-16pt} 
\begin{Example}
fm1 <- gls(follicles {\Twiddle} sin(2*pi*Time) + cos(2*pi*Time), data = Ovary,
           correlation = corAR1(form = {\Twiddle} 1 | Mare))
getData(fm1)
\end{Example}
\end{Helpfile}
\begin{Helpfile}{getData.lmList}{Extract lmList Object Data}
If present in the calling sequence used to produce \Co{object}, the
data frame used to fit the model is obtained.
\begin{Example}
getData(object)
\end{Example}
\begin{Argument}{ARGUMENTS}
\item[\Co{object:}]
an object inheriting from class \Co{lmList}, representing
a list of \Co{lm} objects with a common model.
\end{Argument}
\Paragraph{VALUE}
if a \Co{data} argument is present in the calling sequence that
produced \Co{object}, the corresponding data frame (with
\Co{na.action} and \Co{subset} applied to it, if also present in
the call that produced \Co{object}) is returned;
else, \Co{NULL} is returned.
\Paragraph{SEE ALSO}
\Co{lmList}
\need 15pt
\Paragraph{EXAMPLE}
\vspace{-16pt} 
\begin{Example}
fm1 <- lmList(distance {\Twiddle} age | Subject, Orthodont)
getData(fm1)
\end{Example}
\end{Helpfile}
\begin{Helpfile}{getData.lme}{Extract lme Object Data}
If present in the calling sequence used to produce \Co{object}, the
data frame used to fit the model is obtained.
\begin{Example}
getData(object)
\end{Example}
\begin{Argument}{ARGUMENTS}
\item[\Co{object:}]
an object inheriting from class \Co{lme}, representing
a linear mixed-effects fitted model.
\end{Argument}
\Paragraph{VALUE}
if a \Co{data} argument is present in the calling sequence that
produced \Co{object}, the corresponding data frame (with
\Co{na.action} and \Co{subset} applied to it, if also present in
the call that produced \Co{object}) is returned;
else, \Co{NULL} is returned.
\Paragraph{SEE ALSO}
\Co{lme}
\need 15pt
\Paragraph{EXAMPLE}
\vspace{-16pt} 
\begin{Example}
fm1 <- lme(follicles {\Twiddle} sin(2*pi*Time) + cos(2*pi*Time), data = Ovary,
           random = {\Twiddle} sin(2*pi*Time))
getData(fm1)
\end{Example}
\end{Helpfile}
\begin{Helpfile}{getGroups}{Extract Grouping Factors from an Object}
This function is generic; method functions can be written to handle
specific classes of objects. Classes which already have methods for
this function include \Co{corStruct}, \Co{data.frame},
\Co{gls}, \Co{lme}, \Co{lmList}, and \Co{varFunc}.
\begin{Example}
getGroups(object, form, level, data)
\end{Example}
\begin{Argument}{ARGUMENTS}
\item[\Co{object:}]
any object
\item[\Co{form:}]
an optional formula with a conditioning expression on its
right hand side (i.e.\ an expression involving the \Co{|}
operator). Defaults to \Co{formula(object)}.
\item[\Co{level:}]
a positive integer vector with the level(s) of grouping to
be used when multiple nested levels of grouping are present. This
argument is optional for most methods of this generic function and
defaults to all levels of nesting.
\item[\Co{data:}]
a data frame in which to interpret the variables named in
\Co{form}. Optional for most methods.
\end{Argument}
\Paragraph{VALUE}
will depend on the method function used; see the appropriate documentation.
\Paragraph{SEE ALSO}
\Co{getGroupsFormula}
\need 15pt
\Paragraph{EXAMPLE}
\vspace{-16pt} 
\begin{Example}
## see the method function documentation
\end{Example}
\end{Helpfile}
\begin{Helpfile}{getGroups.corStruct}{Extract corStruct Groups}
This method function extracts the grouping factor associated with
\Co{object}, if any is present.
\begin{Example}
getGroups(object, form, data, level)
\end{Example}
\begin{Argument}{ARGUMENTS}
\item[\Co{object:}]
an object inheriting from class \Co{corStruct}
representing a correlation structure.
\item[\Co{form:}]
this argument is included to make the method function
compatible with the generic. It will be assigned the value of
\Co{formula(object)} and should not be modified.
\item[\Co{data:}]
an optional data frame in which to evaluate the variables
defined in \Co{form}, in case \Co{object} is not initialized and
the grouping factor needs to be evaluated.
\item[\Co{level:}]
this argument is included to make the method function
compatible with the generic and is not used.
\end{Argument}
\Paragraph{VALUE}
if a grouping factor is present in the correlation structure
represented by \Co{object}, the function returns the corresponding
factor vector; else the function returns \Co{NULL}.
\Paragraph{SEE ALSO}
\Co{getGroups}
\need 15pt
\Paragraph{EXAMPLE}
\vspace{-16pt} 
\begin{Example}
cs1 <- corAR1(form = \Twiddle 1 | Subject)
getGroups(cs1, data = Orthodont)
\end{Example}
\end{Helpfile}
\begin{Helpfile}{getGroups.data.frame}{Groups from Data Frame}
Each variable named in the expression after the \Co{|} operator on
the right hand side of \Co{form} is evaluated in \Co{object}. If
more than one variable is indicated in \Co{level} they are combined
into a data frame; else the selected variable is returned as a vector.
When multiple grouping levels are defined in \Co{form} and
\Co{level > 1}, the levels of the returned factor are obtained by
pasting together the levels of the grouping factors of level greater
or equal to \Co{level}, to ensure their uniqueness.
\begin{Example}
getGroups(object, form, level)
\end{Example}
\begin{Argument}{ARGUMENTS}
\item[\Co{object:}]
an object inheriting from class \Co{data.frame}.
\item[\Co{form:}]
an optional formula with a conditioning expression on its
right hand side (i.e.\ an expression involving the \Co{|}
operator). Defaults to \Co{formula(object)}.
\item[\Co{level:}]
a positive integer vector with the level(s) of grouping to
be used when multiple nested levels of grouping are present. Defaults
to all levels of nesting.
\end{Argument}
\Paragraph{VALUE}
either a data frame with columns given by the grouping factors
indicated in \Co{level}, from outer to inner, or, when a single
level is requested, a factor representing the selected grouping
factor.
\Paragraph{SEE ALSO}
\Co{getGroupsFormula}
\need 15pt
\Paragraph{EXAMPLE}
\vspace{-16pt} 
\begin{Example}
getGroups(Pixel)
getGroups(Pixel, level = 2)
\end{Example}
\end{Helpfile}
\begin{Helpfile}{getGroups.gls}{Extract gls Object Groups}
If present, the grouping factor associated to the correlation
structure for the linear model represented by \Co{object} is extracted.
\begin{Example}
getGroups(object)
\end{Example}
\begin{Argument}{ARGUMENTS}
\item[\Co{object:}]
an object inheriting from class \Co{gls}, representing
a generalized least squares fitted linear model.
\end{Argument}
\Paragraph{VALUE}
if the linear model represented by \Co{object} incorporates a
correlation structure and the corresponding \Co{corStruct} object
has a grouping factor, a vector with the group values is returned;
else, \Co{NULL} is returned.
\Paragraph{SEE ALSO}
\Co{gls}, \Co{corClasses}
\need 15pt
\Paragraph{EXAMPLE}
\vspace{-16pt} 
\begin{Example}
fm1 <- gls(follicles \Twiddle sin(2*pi*Time) + cos(2*pi*Time), Ovary,
           correlation = corAR1(form = \Twiddle 1 | Mare))
getGroups(fm1)
\end{Example}
\end{Helpfile}
\begin{Helpfile}{getGroups.lmList}{Extract lmList Object Groups}
The grouping factor determining the partitioning of the observations
used to produce the \Co{lm} components of \Co{object} is
extracted.
\begin{Example}
getGroups(object)
\end{Example}
\begin{Argument}{ARGUMENTS}
\item[\Co{object:}]
an object inheriting from class \Co{lmList}, representing
a list of \Co{lm} objects with a common model.
\end{Argument}
\Paragraph{VALUE}
a vector with the grouping factor corresponding to the \Co{lm}
components of \Co{object}.
\Paragraph{SEE ALSO}
\Co{lmList}
\need 15pt
\Paragraph{EXAMPLE}
\vspace{-16pt}
\begin{Example}
fm1 <- lmList(distance {\Twiddle} age | Subject, Orthodont)
getGroups(fm1)
\end{Example}
\end{Helpfile}
\begin{Helpfile}{getGroups.lme}{Extract lme Object Groups}
The grouping factors corresponding to the linear mixed-effects model
represented by \Co{object} are extracted. If more than one level is
indicated in \Co{level}, the corresponding grouping factors are
combined  into a data frame; else the selected grouping factor is
returned as a vector.
\begin{Example}
getGroups(object, form, level)
\end{Example}
\begin{Argument}{ARGUMENTS}
\item[\Co{object:}]
an object inheriting from class \Co{lme}, representing
a fitted linear mixed-effects model.
\item[\Co{form:}]
this argument is included to make the method function
compatible with the generic and is ignored in this method.
\item[\Co{level:}]
an optional integer vector giving the level(s) of grouping
to be extracted from \Co{object}. Defaults to the highest or
innermost level of grouping.
\end{Argument}
\Paragraph{VALUE}
either a data frame with columns given by the grouping factors
indicated in \Co{level}, or, when a single level is requested, a
factor representing the selected grouping factor.
\Paragraph{SEE ALSO}
\Co{lme}
\need 15pt
\Paragraph{EXAMPLE}
\vspace{-16pt} 
\begin{Example}
fm1 <- lme(pixel \Twiddle day + day^2, Pixel,
  random = list(Dog = \Twiddle day, Side = \Twiddle 1))
getGroups(fm1, level = 1:2)
\end{Example}
\end{Helpfile}
\begin{Helpfile}{getGroups.varFunc}{Extract varFunc Groups}
This method function extracts the grouping factor associated with the
variance function represented by \Co{object}, if any is present.
\begin{Example}
getGroups(object)
\end{Example}
\begin{Argument}{ARGUMENTS}
\item[\Co{object:}]
an object inheriting from class \Co{varFunc},
representing a variance function structure.
\end{Argument}
\Paragraph{VALUE}
if \Co{object} has a \Co{groups} attribute, its value is
returned; else \Co{NULL} is returned.
\need 15pt
\Paragraph{EXAMPLE}
\vspace{-16pt} 
\begin{Example}
vf1 <- varPower(form = \Twiddle age | Sex)
vf1 <- initialize(vf1, Orthodont)
getGroups(vf1)
\end{Example}
\end{Helpfile}
\begin{Helpfile}{getGroupsFormula}{Extract Grouping Formula}
The conditioning expression associated with \Co{formula(object)}
(i.e.\ an expression after a \Co{|} operator) is returned either as
a named list of one-sided formulas, or a single one-sided formula,
depending  on the value of \Co{asList}. The components of the
returned list are ordered from outermost to innermost level and are
named after the grouping factor expression.
\begin{Example}
getGroupsFormula(object, asList)
\end{Example}
\begin{Argument}{ARGUMENTS}
\item[\Co{object:}]
any object from which a formula can be extracted.
\item[\Co{asList:}]
an optional logical value. If \Co{TRUE} the returned
value with be a list of formulas; else, if \Co{FALSE} the returned
value will be a one-sided formula. Defaults to \Co{FALSE}.
\end{Argument}
\Paragraph{VALUE}
a one-sided formula, or a list of one-sided formulas, with the
grouping structure associated with \Co{formula(object)}. If no
conditioning expression is present in \Co{formula(object)} a
\Co{NULL} value is returned.
\Paragraph{SEE ALSO}
\Co{getGroups}
\need 15pt
\Paragraph{EXAMPLE}
\vspace{-16pt} 
\begin{Example}
getGroupsFormula(y \Twiddle x | g1/g2)
\end{Example}
\end{Helpfile}
\begin{Helpfile}{getGroupsFormula.gls}{gls Grouping Formula}
If present, the grouping formula associated with the correlation
structure (\Co{corStruct}) of \Co{object} is returned either as
a named list with a single one-sided formula, or a single one-sided
formula, depending on the value of \Co{asList}. If \Co{object}
does not include a correlation structure, or if the correlation
structure does not include groups, \Co{NULL} is returned.
\begin{Example}
getGroupsFormula(object, asList)
\end{Example}
\begin{Argument}{ARGUMENTS}
\item[\Co{object:}]
an object inheriting from class \Co{gls}, representing
a generalized least squares fitted linear model.
\item[\Co{asList:}]
an optional logical value. If \Co{TRUE} the returned
value with be a list of formulas; else, if \Co{FALSE} the returned
value will be a one-sided formula. Defaults to \Co{FALSE}.
\end{Argument}
\Paragraph{VALUE}
if a correlation structure with groups is included in \Co{object}, a
one-sided formula, or a list with a single one-sided formula, with the
corresponding grouping structure, else \Co{NULL}.
\Paragraph{SEE ALSO}
\Co{corClasses}, \Co{getGroups.gls}
\need 15pt
\Paragraph{EXAMPLE}
\vspace{-16pt} 
\begin{Example}
fm1 <- gls(follicles \Twiddle sin(2*pi*Time) + cos(2*pi*Time), Ovary,
           correlation = corAR1(form = \Twiddle 1 | Mare))
getGroupsFormula(fm1)
\end{Example}
\end{Helpfile}
\begin{Helpfile}{getGroupsFormula.lmList}{lmList Grouping Formula}
A formula representing the grouping factor determining the
partitioning of the observations used to produce the \Co{lm}
components of \Co{object} is obtained and returned as a list with a
single component, or as a one-sided formula.
\begin{Example}
getGroupsFormula(object, asList)
\end{Example}
\begin{Argument}{ARGUMENTS}
\item[\Co{object:}]
an object inheriting from class \Co{lmList}, representing
a list of \Co{lm} objects with a common model.
\item[\Co{asList:}]
an optional logical value. If \Co{TRUE} the returned
value with be a list of formulas; else, if \Co{FALSE} the returned
value will be a one-sided formula. Defaults to \Co{FALSE}.
\end{Argument}
\Paragraph{VALUE}
a one-sided formula, or a list with a single one-sided formula,
representing the grouping factor corresponding to the \Co{lm}
components of \Co{object}.
\Paragraph{SEE ALSO}
\Co{lmList}, \Co{getGroups.lmList}
\need 15pt
\Paragraph{EXAMPLE}
\vspace{-16pt}
\begin{Example}
fm1 <- lmList(distance {\Twiddle} age | Subject, Orthodont)
getGroupsFormula(fm1)
\end{Example}
\end{Helpfile}
\begin{Helpfile}{getGroupsFormula.lme}{lme Grouping Formula}
The grouping formula associated with the random effects structure
(\Co{reStruct}) of \Co{object} is returned either as
a named list of one-sided formulas, or a single one-sided formula,
depending  on the value of \Co{asList}. The components of the
returned list are ordered from outermost to innermost level and are
named after the grouping factor expression.
\begin{Example}
getGroupsFormula(object, asList)
\end{Example}
\begin{Argument}{ARGUMENTS}
\item[\Co{object:}]
an object inheriting from class \Co{lme}, representing
a fitted linear mixed-effects model.
\item[\Co{asList:}]
an optional logical value. If \Co{TRUE} the returned
value with be a list of formulas; else, if \Co{FALSE} the returned
value will be a one-sided formula. Defaults to \Co{FALSE}.
\end{Argument}
\Paragraph{VALUE}
a one-sided formula, or a list of one-sided formulas, with the
grouping structure associated with the random effects structure of
\Co{object}.
\Paragraph{SEE ALSO}
\Co{reStruct}, \Co{getGroups.lme}
\need 15pt
\Paragraph{EXAMPLE}
\vspace{-16pt} 
\begin{Example}
fm1 <- lme(distance \Twiddle age + Sex, data = Orthodont, random = \Twiddle  1)
getGroupsFormula(fm1)
\end{Example}
\end{Helpfile}
\begin{Helpfile}{getGroupsFormula.reStruct}{reStruct Grouping Formula}
The names of the \Co{object} components are used to construct a
one-sided formula, or a named list of formulas, depending on the value
of \Co{asList}.  The components of the returned list are ordered
from outermost to innermost level.
\begin{Example}
getGroupsFormula(object, asList)
\end{Example}
\begin{Argument}{ARGUMENTS}
\item[\Co{object:}]
an object inheriting from class \Co{reStruct},
representing a random effects structure and consisting of a list of
\Co{pdMat} objects.
\item[\Co{asList:}]
an optional logical value. If \Co{TRUE} the returned
value with be a list of formulas; else, if \Co{FALSE} the returned
value will be a one-sided formula. Defaults to \Co{FALSE}.
\end{Argument}
\Paragraph{VALUE}
a one-sided formula, or a list of one-sided formulas, with the
grouping structure associated with \Co{object}.
\Paragraph{SEE ALSO}
\Co{reStruct}, \Co{getGroups}
\need 15pt
\Paragraph{EXAMPLE}
\vspace{-16pt} 
\begin{Example}
rs1 <- reStruct(list(A = pdDiag(diag(2), \Twiddle age), B = \Twiddle 1))
getGroupsFormula(rs1)
\end{Example}
\end{Helpfile}
\begin{Helpfile}{getResponse}{Extract Response Variable from an Object}
This function is generic; method functions can be written to handle
specific classes of objects. Classes which already have methods for
this function include \Co{data.frame}, \Co{gls}, \Co{lme},
and \Co{lmList}.
\begin{Example}
getResponse(object, form, data)
\end{Example}
\begin{Argument}{ARGUMENTS}
\item[\Co{object:}]
any object
\item[\Co{form:}]
an optional two-sided formula. Defaults to
\Co{formula(object)}.
\item[\Co{data:}]
a data frame in which to interpret the variables named in
\Co{form}. Optional for most methods.
\end{Argument}
\Paragraph{VALUE}
will depend on the method function used; see the appropriate documentation.
\Paragraph{SEE ALSO}
\Co{getResponseFormula}
\need 15pt
\Paragraph{EXAMPLE}
\vspace{-16pt} 
\begin{Example}
## see the method function documentation
\end{Example}
\end{Helpfile}
\begin{Helpfile}{getResponse.data.frame}{Response from Data Frame}
The left hand side of \Co{form} is evaluated in \Co{object}.
\begin{Example}
getResponse(object, form)
\end{Example}
\begin{Argument}{ARGUMENTS}
\item[\Co{object:}]
an object inheriting from class \Co{data.frame}.
\item[\Co{form:}]
an optional formula specifying the response to be
evaluated in \Co{object}. Defaults to \Co{formula(object)}.
\end{Argument}
\Paragraph{VALUE}
the value of the left hand side of \Co{form} evaluated in
\Co{object}.
\Paragraph{SEE ALSO}
\Co{getResponseFormula}
\need 15pt
\Paragraph{EXAMPLE}
\vspace{-16pt} 
\begin{Example}
getResponse(Orthodont)
\end{Example}
\end{Helpfile}
\begin{Helpfile}{getResponse.gls}{Extract gls Object Response}
This method function extracts the response variable used in fitting
the linear model corresponding to \Co{object}.
\begin{Example}
getResponse(object)
\end{Example}
\begin{Argument}{ARGUMENTS}
\item[\Co{object:}]
an object inheriting from class \Co{gls}, representing
a generalized least squares fitted linear model.
\end{Argument}
\Paragraph{VALUE}
a vector with the response variable corresponding to the linear
model represented by \Co{object}.
\Paragraph{SEE ALSO}
\Co{gls}
\need 15pt
\Paragraph{EXAMPLE}
\vspace{-16pt} 
\begin{Example}
fm1 <- gls(follicles \Twiddle sin(2*pi*Time) + cos(2*pi*Time), Ovary,
           correlation = corAR1(form = \Twiddle 1 | Mare))
getResponse(fm1)
\end{Example}
\end{Helpfile}
\begin{Helpfile}{getResponse.lmList}{Extract lmList Object Response}
The response vectors from each of the \Co{lm} components of
\Co{object} are extracted and combined into a single vector.
\begin{Example}
getResponse(object)
\end{Example}
\begin{Argument}{ARGUMENTS}
\item[\Co{object:}]
an object inheriting from class \Co{lmList}, representing
a list of \Co{lm} objects with a common model.
\end{Argument}
\Paragraph{VALUE}
a vector with the response vectors corresponding to the \Co{lm}
components of \Co{object}.
\Paragraph{SEE ALSO}
\Co{lmList}
\need 15pt
\Paragraph{EXAMPLE}
\vspace{-16pt}
\begin{Example}
fm1 <- lmList(distance {\Twiddle} age | Subject, Orthodont)
getResponse(fm1)
\end{Example}
\end{Helpfile}
\begin{Helpfile}{getResponse.lme}{Extract lme Object Response}
This method function extracts the response variable used in fitting
the linear mixed-effects model corresponding to \Co{object}.
\begin{Example}
getResponse(object)
\end{Example}
\begin{Argument}{ARGUMENTS}
\item[\Co{object:}]
an object inheriting from class \Co{lme}, representing
a fitted linear mixed-effects model.
\end{Argument}
\Paragraph{VALUE}
a vector with the response variable corresponding to the linear
mixed-effects model represented by \Co{object}.
\Paragraph{SEE ALSO}
\Co{lme}
\need 15pt
\Paragraph{EXAMPLE}
\vspace{-16pt} 
\begin{Example}
fm1 <- lme(distance \Twiddle age, Orthodont, random = \Twiddle age | Subject)
getResponse(fm1)
\end{Example}
\end{Helpfile}
\begin{Helpfile}{getResponseFormula}{Response Formula}
The left hand side of \Co{formula(object)} is returned as a
one-sided formula.
\begin{Example}
getResponseFormula(object)
\end{Example}
\begin{Argument}{ARGUMENTS}
\item[\Co{object:}]
any object from which a formula can be extracted.
\end{Argument}
\Paragraph{VALUE}
a one-sided formula with the response variable associated with
\Co{formula(object)}.
\Paragraph{SEE ALSO}
\Co{getResponse}
\need 15pt
\Paragraph{EXAMPLE}
\vspace{-16pt} 
\begin{Example}
getResponseFormula(y \Twiddle x | g)
\end{Example}
\end{Helpfile}
\begin{Helpfile}{gls}{Fit Linear Model Using Generalized Least Squares}
This function fits a linear model using generalized least
squares. The errors are allowed to be correlated and/or have unequal
variances.
\begin{Example}
gls(model, data, correlation, weights, subset, method, na.action, 
    control, verbose)
\end{Example}
\begin{Argument}{ARGUMENTS}
\item[\Co{model:}]
a two-sided linear formula object describing the
model, with the response on the left of a \Co{{\Twiddle}} operator and the
terms, separated by \Co{+} operators, on the right.
\item[\Co{data:}]
an optional data frame containing the variables named in
\Co{model}, \Co{correlation}, \Co{weights}, and
\Co{subset}. By default the variables are taken from the
environment from which \Co{gls} is called.
\item[\Co{correlation:}]
an optional \Co{corStruct} object describing the
within-group correlation structure. See the documentation of
\Co{corClasses} for a description of the available \Co{corStruct}
classes. If a grouping variable is to be used, it must be specified
in the \Co{form} argument to the the \Co{corStruct}
constructor. Defaults to \Co{NULL}, corresponding to uncorrelated
errors.
\item[\Co{weights:}]
an optional \Co{varFunc} object or one-sided formula
describing the within-group heteroscedasticity structure. If given as
a formula, it is used as the argument to \Co{varFixed},
corresponding to fixed variance weights. See the documentation on
\Co{varClasses} for a description of the available \Co{varFunc}
classes. Defaults to \Co{NULL}, corresponding to homocesdatic
errors.
\item[\Co{subset:}]
an optional expression saying which subset of the rows of
\Co{data} should  be  used in the fit. This can be a logical
vector, or a numeric vector indicating which observation numbers are
to be included, or a  character  vector of the row names to be
included.  All observations are included by default.
\item[\Co{method:}]
a character string.  If \Co{"REML"} the model is fit by
maximizing the restricted log-likelihood.  If \Co{"ML"} the
log-likelihood is maximized.  Defaults to \Co{"REML"}.
\item[\Co{na.action:}]
a function that indicates what should happen when the
data contain \Co{NA}s.  The default action (\Co{na.fail}) causes
\Co{gls} to print an error message and terminate if there are any
incomplete observations.
\item[\Co{control:}]
a list of control values for the estimation algorithm to
replace the default values returned by the function \Co{glsControl}.
Defaults to an empty list.
\item[\Co{verbose:}]
an optional logical value. If \Co{TRUE} information on
the evolution of the iterative algorithm is printed. Default is
\Co{FALSE}.
\end{Argument}
\Paragraph{VALUE}
an object of class \Co{gls} representing the linear model
fit. Generic functions such as \Co{print}, \Co{plot} and 
\Co{summary} have methods to show the results of the fit. See
\Co{glsObject} for the components of the fit. The functions
\Co{resid}, \Co{coef}, and \Co{fitted} can be used to extract
some of its components.
\Paragraph{REFERENCES}
The different correlation structures available for the
\Co{correlation} argument are described in Box, G.E.P., Jenkins,
G.M., and Reinsel G.C. (1994), Littel, R.C., Milliken, G.A., Stroup,
W.W., and Wolfinger, R.D. (1997), and Venables, W.N. and Ripley,
B.D. (1997). The use of variance functions for linear 
and nonlinear models is presented in detail in Carrol, R.J. and Rupert,
D. (1988) and Davidian, M. and Giltinan, D.M. (1995). \\
Box, G.E.P., Jenkins, G.M., and Reinsel G.C. (1994) "Time Series
Analysis: Forecasting and Control", 3rd Edition, Holden-Day. \\
Carrol, R.J. and Rupert, D. (1988) "Transformation and Weighting in
Regression", Chapman and Hall.\\
Davidian, M. and Giltinan, D.M. (1995) "Nonlinear Mixed Effects Models
for Repeated Measurement Data", Chapman and Hall.\\
Littel, R.C., Milliken, G.A., Stroup, W.W., and Wolfinger, R.D. (1997)
"SAS Systems for Mixed Models", SAS Institute.\\
Venables, W.N. and Ripley, B.D. (1997) "Modern Applied Statistics with
S-plus", 2nd Edition, Springer-Verlag.
\Paragraph{SEE ALSO}
\Co{glsControl}, \Co{glsObject},
\Co{corClasses}, \Co{varClasses},
\Co{corClasses}, \Co{varClasses}
\need 15pt
\Paragraph{EXAMPLE}
\vspace{-16pt}
\begin{Example}
# AR(1) errors within each Mare
fm1 <- gls(follicles {\Twiddle} sin(2*pi*Time) + cos(2*pi*Time), Ovary,
           correlation = corAR1(form = {\Twiddle} 1 | Mare))
# variance increases with a power of the absolute fitted values
fm1 <- gls(follicles {\Twiddle} sin(2*pi*Time) + cos(2*pi*Time), Ovary,
           weights = varPower())
\end{Example}
\end{Helpfile}
\begin{Helpfile}{glsControl}{Control Values for gls Fit}
The values supplied in the function call replace the defaults and a
list with all possible arguments is returned. The returned list is
used as the \Co{control} argument to the \Co{gls} function.
\begin{Example}
glsControl(maxIter, msMaxIter, tolerance, msTol, msScale, 
           msVerbose, singular.ok, qrTol, returnObject, 
           apVar, .relStep) 
\end{Example}
\begin{Argument}{ARGUMENTS}
\item[\Co{maxIter:}]
maximum number of iterations for the \Co{gls}
optimization algorithm. Default is 50.
\item[\Co{msMaxIter:}]
maximum number of iterations
for the \Co{ms} optimization step inside the \Co{gls}
optimization. Default is 50.
\item[\Co{tolerance:}]
tolerance for the convergence criterion in the
\Co{gls} algorithm. Default is 1e-6.
\item[\Co{msTol:}]
tolerance for the convergence criterion in \Co{ms},
passed as the \Co{rel.tolerance} argument to the function (see
documentation on \Co{ms}). Default is 1e-7.
\item[\Co{msScale:}]
scale function passed as the \Co{scale} argument to
the \Co{ms} function (see documentation on that function). Default
is \Co{lmeScale}.
\item[\Co{msVerbose:}]
a logical value passed as the \Co{trace} argument to
\Co{ms} (see documentation on that function). Default is
\Co{FALSE}.
\item[\Co{singular.ok:}]
a logical value indicating whether non-estimable
coefficients (resulting from linear dependencies among the columns of
the regression matrix) should be allowed. Default is \Co{FALSE}.
\item[\Co{qrTol:}]
a tolerance for detecting linear dependencies among the
columns of the regression matrix in its QR decomposition. Default is
\Co{.Machine\$single.eps}.
\item[\Co{returnObject:}]
a logical value indicating whether the fitted
object should be returned when the maximum number of iterations is
reached without convergence of the algorithm. Default is
\Co{FALSE}.
\item[\Co{apVar:}]
a logical value indicating whether the approximate
covariance matrix of the variance-covariance parameters should be
calculated. Default is \Co{TRUE}.
\item[\Co{.relStep:}]
relative step for numerical derivatives
calculations. Default is \\ \Co{.Machine\$double.eps}$^{1/3}$.
\end{Argument}
\Paragraph{VALUE}
a list with components for each of the possible arguments.
\Paragraph{SEE ALSO}
\Co{gls}, \Co{ms}, \Co{lmeScale}
\need 15pt
\Paragraph{EXAMPLE}
\vspace{-16pt} 
\begin{Example}
# decrease the maximum number iterations in the ms call and
# request that information on the evolution of the ms iterations 
# be printed
glsControl(msMaxIter = 20, msVerbose = TRUE)
\end{Example}
\end{Helpfile}
\begin{Helpfile}{glsObject}{Fitted gls Object}
An object returned by the \Co{gls} function, inheriting from class
\Co{gls} and representing a generalized least squares fitted linear 
model. Objects of this class have methods for the generic functions 
\Co{anova}, \Co{coef}, \Co{fitted}, \Co{formula},
\Co{getGroups}, \Co{getResponse}, \Co{intervals}, \Co{logLik},
\Co{plot}, \Co{predict}, \Co{print}, \Co{residuals},
\Co{summary}, and \Co{update}.
\Paragraph{VALUE}
The following components must be included in a legitimate \Co{gls}
object. 
\begin{Argument}{COMPONENTS}
\item[\Co{apVar:}]
an approximate covariance matrix for the
variance-covariance coefficients. If \Co{apVar = FALSE} in the list
of control values used in the call to \Co{gls}, this
component is equal to \Co{NULL}.
\item[\Co{call:}]
a list containing an image of the \Co{gls} call that
produced the object.
\item[\Co{coefficients:}]
a vector with the estimated linear model
coefficients.
\item[\Co{contrasts:}]
a list with the contrasts used to represent factors
in the model formula. This information is important for making
predictions from a new data frame in which not all levels of the
original factors are observed. If no factors are used in the model,
this component will be an empty list.
\item[\Co{dims:}]
a list with basic dimensions used in the model fit,
including the components \Co{N} - the number of observations in
the data and \Co{p} - the number of coefficients in the linear
model.
\item[\Co{fitted:}]
a vector with the fitted values..
\item[\Co{glsStruct:}]
an object inheriting from class \Co{glsStruct},
representing a list of linear model components, such as
\Co{corStruct} and \Co{varFunc} objects.
\item[\Co{groups:}]
a vector with the correlation structure grouping factor,
if any is present.
\item[\Co{logLik:}]
the log-likelihood at convergence.
\item[\Co{method:}]
the estimation method: either \Co{"ML"} for maximum
likelihood, or \Co{"REML"} for restricted maximum likelihood.
\item[\Co{numIter:}]
the number of iterations used in the iterative
algorithm.
\item[\Co{residuals:}]
a vector with the residuals.
\item[\Co{sigma:}]
the estimated residual standard error.
\item[\Co{varBeta:}]
an approximate covariance matrix of the
coefficients estimates.
\end{Argument}
\Paragraph{SEE ALSO}
\Co{gls}, \Co{glsStruct}
\end{Helpfile}
\begin{Helpfile}{glsStruct}{Generalized Least Squares Structure}
A generalized least squares structure is a list of model components
representing different sets of parameters in the linear 
model. A \Co{glsStruct}  may contain \Co{corStruct} and
\Co{varFunc} objects. \Co{NULL} arguments are not included in the
\Co{glsStruct} list.
\begin{Example}
glsStruct(corStruct, varStruct)
\end{Example}
\begin{Argument}{ARGUMENTS}
\item[\Co{corStruct:}]
an optional \Co{corStruct} object, representing a
correlation structure. Default is \Co{NULL}.
\item[\Co{varStruct:}]
an optional \Co{varFunc} object, representing a
variance function structure. Default is \Co{NULL}.
\end{Argument}
\Paragraph{VALUE}
a list of model variance-covariance components determining the
parameters to be estimated for the associated linear model.
\Paragraph{SEE ALSO}
\Co{gls}, \Co{corClasses},
\Co{varClasses}
\need 15pt
\Paragraph{EXAMPLE}
\vspace{-16pt} 
\begin{Example}
gls1 <- glsStruct(corAR1(), varPower())
\end{Example}
\end{Helpfile}
\begin{Helpfile}{gnls}{Fit Nonlinear Model Using Generalized Least Squares}
This function fits a nonlinear model using generalized least
squares. The errors are allowed to be correlated and/or have unequal
variances.
\begin{Example}
gnls(model, data, params, start, correlation, weights, subset,
     na.action, naPattern, control, verbose)
\end{Example}
\begin{Argument}{ARGUMENTS}
\item[\Co{model:}]
a two-sided formula object describing the
model, with the response on the left of a \Co{{\Twiddle}} operator and 
a nonlinear expression involving parameters and covariates on the
right. If \Co{data} is given, all names used in the formula should
be defined as parameters or variables in the data frame.
\item[\Co{data:}]
an optional data frame containing the variables named in
\Co{model}, \Co{correlation}, \Co{weights}, 
\Co{subset}, and \Co{naPattern}. By default the variables are 
taken from the environment from which \Co{gnls} is called.
\item[\Co{params:}]
an optional two-sided linear formula of the form
\Co{p1+...+pn{\Twiddle}x1+...+xm}, or list of two-sided formulas of the form
\Co{p1{\Twiddle}x1+...+xm}, with possibly different models for each
parameter. The \Co{p1,...,pn} represent parameters included on the
right hand side of \Co{model} and \Co{x1+...+xm} define a linear
model for the parameters (when the left hand side of the formula
contains several parameters, they are all assumed to follow the same
linear model described by the right hand side expression). A \Co{1}
on the right hand side of the formula(s) indicates a single fixed
effects for the corresponding parameter(s). By default, the
parameters are obtained from the names of \Co{start}.
\item[\Co{start:}]
an optional named list, or numeric vector, with the
initial values for the parameters in \Co{model}. It can be omitted
when a \Co{selfStarting} function is used in \Co{model}, in which
case the starting estimates will be obtained from a single call to the
\Co{nls} function.
\item[\Co{correlation:}]
an optional \Co{corStruct} object describing the
within-group correlation structure. See the documentation of
\Co{corClasses} for a description of the available \Co{corStruct}
classes. If a grouping variable is to be used, it must be specified
in the \Co{form} argument to the the \Co{corStruct}
constructor. Defaults to \Co{NULL}, corresponding to uncorrelated
errors.
\item[\Co{weights:}]
an optional \Co{varFunc} object or one-sided formula
describing the within-group heteroscedasticity structure. If given as
a formula, it is used as the argument to \Co{varFixed},
corresponding to fixed variance weights. See the documentation on
\Co{varClasses} for a description of the available \Co{varFunc}
classes. Defaults to \Co{NULL}, corresponding to homocesdatic
errors.
\item[\Co{subset:}]
an optional expression saying which subset of the rows of
\Co{data} should  be  used in the fit. This can be a logical
vector, or a numeric vector indicating which observation numbers are
to be included, or a  character  vector of the row names to be
included.  All observations are included by default.
\item[\Co{na.action:}]
a function that indicates what should happen when the
data contain \Co{NA}s.  The default action (\Co{na.fail}) causes
\Co{gnls} to print an error message and terminate if there are any
incomplete observations.
\item[\Co{naPattern:}]
an expression or formula object, specifying which returned
values are to be regarded as missing.
\item[\Co{control:}]
a list of control values for the estimation algorithm to
replace the default values returned by the function \Co{gnlsControl}.
Defaults to an empty list.
\item[\Co{verbose:}]
an optional logical value. If \Co{TRUE} information on
the evolution of the iterative algorithm is printed. Default is
\Co{FALSE}.
\end{Argument}
\Paragraph{VALUE}
an object of class \Co{gnls}, also inheriting from class \Co{gls},
representing the nonlinear model fit. Generic functions such as
\Co{print}, \Co{plot} and  \Co{summary} have methods to show the
results of the fit. See \Co{gnlsObject} for the components of the
fit. The functions \Co{resid}, \Co{coef}, and \Co{fitted} can be
used to extract some of its components.
\Paragraph{REFERENCES}
The different correlation structures available for the
\Co{correlation} argument are described in Box, G.E.P., Jenkins,
G.M., and Reinsel G.C. (1994), Littel, R.C., Milliken, G.A., Stroup,
W.W., and Wolfinger, R.D. (1997), and Venables, W.N. and Ripley,
B.D. (1997). The use of variance functions for linear 
and nonlinear models is presented in detail in Carrol, R.J. and Rupert,
D. (1988) and Davidian, M. and Giltinan, D.M. (1995).  \\
Box, G.E.P., Jenkins, G.M., and Reinsel G.C. (1994) "Time Series
Analysis: Forecasting and Control", 3rd Edition, Holden-Day. \\
Carrol, R.J. and Rupert, D. (1988) "Transformation and Weighting in
Regression", Chapman and Hall.\\
Davidian, M. and Giltinan, D.M. (1995) "Nonlinear Mixed Effects Models
for Repeated Measurement Data", Chapman and Hall.\\
Littel, R.C., Milliken, G.A., Stroup, W.W., and Wolfinger, R.D. (1997)
"SAS Systems for Mixed Models", SAS Institute.\\
Venables, W.N. and Ripley, B.D. (1997) "Modern Applied Statistics with
S-plus", 2nd Edition, Springer-Verlag.
\Paragraph{SEE ALSO}
\Co{gnlsControl}, \Co{gnlsObject},
\Co{corClasses}, \Co{varClasses},
\Co{corClasses}, \Co{varClasses}
\need 15pt
\Paragraph{EXAMPLE}
\vspace{-16pt}
\begin{Example}
# variance increases with a power of the absolute fitted values
fm1 <- gnls(weight {\Twiddle} SSlogis(Time, Asym, xmid, scal), Soybean,
            weights = varPower())
# errors follow an auto-regressive process of order 1 
fm2 <- gnls(weight {\Twiddle} SSlogis(Time, Asym, xmid, scal), Soybean,
            correlation = corAR1())
\end{Example}
\end{Helpfile}
\begin{Helpfile}{gnlsControl}{Control Values for gnls Fit}
The values supplied in the function call replace the defaults and a
list with all possible arguments is returned. The returned list is
used as the \Co{control} argument to the \Co{gnls} function.
\begin{Example}
gnlsControl(maxIter,nlsMaxIter,msMaxIter,minScale,tolerance, 
            nlsTol,msTol,msScale,returnObject,msVerbose, 
            apVar,.relStep) 
\end{Example}
\begin{Argument}{ARGUMENTS}
\item[\Co{maxIter:}]
maximum number of iterations for the \Co{gnls}
optimization algorithm. Default is 50.
\item[\Co{nlsMaxIter:}]
maximum number of iterations
for the \Co{nls} optimization step inside the \Co{gnls}
optimization. Default is 7.
\item[\Co{msMaxIter:}]
maximum number of iterations
for the \Co{ms} optimization step inside the \Co{gnls}
optimization. Default is 50.
\item[\Co{minScale:}]
minimum factor by which to shrink the default step size
in an attempt to decrease the sum of squares in the \Co{nls} step.
Default 0.001.
\item[\Co{tolerance:}]
tolerance for the convergence criterion in the
\Co{gnls} algorithm. Default is 1e~-~6.
\item[\Co{nlsTol:}]
tolerance for the convergence criterion in \Co{nls}
step. Default is 1e-3.
\item[\Co{msTol:}]
tolerance for the convergence criterion in \Co{ms},
passed as the \Co{rel.tolerance} argument to the function (see
documentation on \Co{ms}). Default is 1e-7.
\item[\Co{msScale:}]
scale function passed as the \Co{scale} argument to
the \Co{ms} function (see documentation on that function). Default
is \Co{lmeScale}.
\item[\Co{returnObject:}]
a logical value indicating whether the fitted
object should be returned when the maximum number of iterations is
reached without convergence of the algorithm. Default is
\Co{FALSE}.
\item[\Co{msVerbose:}]
a logical value passed as the \Co{trace} argument to
\Co{ms} (see documentation on that function). Default is
\Co{FALSE}.
\item[\Co{apVar:}]
a logical value indicating whether the approximate
covariance matrix of the variance-covariance parameters should be
calculated. Default is \Co{TRUE}.
\item[\Co{.relStep:}]
relative step for numerical derivatives
calculations. Default is \\
\Co{.Machine\$double.eps}$^{1/3}$.
\end{Argument}
\Paragraph{VALUE}
a list with components for each of the possible arguments.
\Paragraph{SEE ALSO}
\Co{gnls}, \Co{ms}, \Co{lmeScale}
\need 15pt
\Paragraph{EXAMPLE}
\vspace{-16pt}
\begin{Example}
# decrease the maximum number iterations in the ms call and
# request that information on the evolution of the ms iterations 
# be printed
gnlsControl(msMaxIter = 20, msVerbose = TRUE)
\end{Example}
\end{Helpfile}
\begin{Helpfile}{gnlsObject}{Fitted gnls Object}
An object returned by the \Co{gnls} function, inheriting from class
\Co{gnls} and also from class \Co{gls}, and representing a
generalized nonlinear least squares fitted model. Objects of this
class have methods for the generic functions  \Co{anova},
\Co{coef}, \Co{fitted}, \Co{formula}, \Co{getGroups},
\Co{getResponse}, \Co{intervals}, \Co{logLik}, \Co{plot},
\Co{predict}, \Co{print}, \Co{residuals}, \Co{summary}, and
\Co{update}.
\Paragraph{VALUE}
The following components must be included in a legitimate \Co{gnls}
object. 
\begin{Argument}{COMPONENTS}
\item[\Co{apVar:}]
an approximate covariance matrix for the
variance-covariance coefficients. If \Co{apVar = FALSE} in the list
of control values used in the call to \Co{gnls}, this
component is equal to \Co{NULL}.
\item[\Co{call:}]
a list containing an image of the \Co{gnls} call that
produced the object.
\item[\Co{coefficients:}]
a vector with the estimated nonlinear model
coefficients.
\item[\Co{contrasts:}]
a list with the contrasts used to represent factors
in the model formula. This information is important for making
predictions from a new data frame in which not all levels of the
original factors are observed. If no factors are used in the model,
this component will be an empty list.
\item[\Co{dims:}]
a list with basic dimensions used in the model fit,
including the components \Co{N} - the number of observations used in
the fit and \Co{p} - the number of coefficients in the nonlinear
model.
\item[\Co{fitted:}]
a vector with the fitted values.
\item[\Co{modelStruct:}]
an object inheriting from class \Co{gnlsStruct},
representing a list of model components, such as \Co{corStruct} and
\Co{varFunc} objects.
\item[\Co{groups:}]
a vector with the correlation structure grouping factor,
if any is present.
\item[\Co{logLik:}]
the log-likelihood at convergence.
\item[\Co{numIter:}]
the number of iterations used in the iterative
algorithm.
\item[\Co{residuals:}]
a vector with the residuals.
\item[\Co{sigma:}]
the estimated residual standard error.
\item[\Co{varBeta:}]
an approximate covariance matrix of the
coefficients estimates.
\end{Argument}
\Paragraph{SEE ALSO}
\Co{gnls}, \Co{gnlsStruct}
\end{Helpfile}
\begin{Helpfile}{gnlsStruct}{Generalized Nonlinear Least Squares Structure}
A generalized nonlinear least squares structure is a list of model
components representing different sets of parameters in the nonlinear 
model. A \Co{gnlsStruct}  may contain \Co{corStruct} and
\Co{varFunc} objects. \Co{NULL} arguments are not included in the
\Co{gnlsStruct} list.
\begin{Example}
gnlsStruct(corStruct, varStruct)
\end{Example}
\begin{Argument}{ARGUMENTS}
\item[\Co{corStruct:}]
an optional \Co{corStruct} object, representing a
correlation structure. Default is \Co{NULL}.
\item[\Co{varStruct:}]
an optional \Co{varFunc} object, representing a
variance function structure. Default is \Co{NULL}.
\end{Argument}
\Paragraph{VALUE}
a list of model variance-covariance components determining the
parameters to be estimated for the associated nonlinear model.
\Paragraph{SEE ALSO}
\Co{gnls}, \Co{corClasses},
\Co{varClasses}
\need 15pt
\Paragraph{EXAMPLE}
\vspace{-16pt}
\begin{Example}
gnls1 <- gnlsStruct(corAR1(), varPower())
\end{Example}
\end{Helpfile}
\begin{Helpfile}{groupedData}{Construct a groupedData Object}
An object of the \Co{groupedData} class is constructed from the
\Co{formula} and \Co{data} by attaching the \Co{formula} as an
attribute of the data, along with any of \Co{outer}, \Co{inner},
\Co{labels}, and \Co{units} that are given.  If
\Co{order.groups} is \Co{TRUE} the grouping factor is converted to
an ordered factor with the ordering determined by
\Co{FUN}. Depending on the number of grouping levels and the type of
primary covariate, the returned object will be of one of three
classes: \Co{nfnGroupedData} - numeric covariate, single level of
nesting; \Co{nffGroupedData} - factor covariate, single level of
nesting; and \Co{nmGroupedData} - multiple levels of
nesting. Several modelling and plotting functions can use the formula
stored with a \Co{groupedData} object to construct default plots and
models.
\begin{Example}
groupedData(formula, data, order.groups, FUN, outer, inner,
 labels, units)
\end{Example}
\begin{Argument}{ARGUMENTS}
\item[\Co{formula:}]
a formula of the form \Co{resp \Twiddle cov | group} where
\Co{resp} is the response, \Co{cov} is the primary covariate, and
\Co{group} is the grouping factor.  The expression \Co{1} can be
used for the primary covariate when there is no other suitable
candidate.  Multiple nested grouping factors can be listed separated
by the \Co{/} symbol as in \Co{fact1/fact2}.  In an expression
like this the \Co{fact2} factor is nested within the \Co{fact1}
factor.
\item[\Co{data:}]
a data frame in which the expressions in \Co{formula} can
be evaluated.  The resulting \Co{groupedData} object will consist
of the same data values in the same order but with additional
attributes.
\item[\Co{order.groups:}]
an optional logical value, or list of logical
values, indicating if the grouping factors should be converted to
ordered factors according to the function \Co{FUN} applied to the
response from each group. If multiple levels of grouping are present,
this argument can be either a single logical value (which will be
repeated for all grouping levels) or a list of logical values. If no
names are assigned to the list elements, they are assumed in the same
order as the group levels (outermost to innermost grouping). Ordering
within a level of grouping is done within the levels of the grouping
factors which are outer to it. Changing the grouping factor to an
ordered factor does not affect the ordering of the rows in the data
frame but it does affect the order of the panels in a trellis display
of the data or models fitted to the data.  Defaults to \Co{TRUE}.
\item[\Co{FUN:}]
an optional summary function that will be applied to the
values of the response for each level of the grouping factor, when
\Co{order.groups = TRUE}, to determine the ordering.  Defaults to
the \Co{max} function.
\item[\Co{outer:}]
an optional one-sided formula, or list of one-sided
formulas, indicating covariates that are outer to the grouping
factor(s).  If multiple levels of grouping are present,
this argument can be either a single one-sided formula, or a list of
one-sided formulas. If no names are assigned to the list elements,
they are assumed in the same order as the group levels (outermost to
innermost grouping). An outer covariate is invariant within the sets
of rows defined by the grouping factor.  Ordering of the groups is
done in such a way as to preserve adjacency of groups with the same
value of the outer variables.  When plotting a  groupedData object,
the argument \Co{outer = TRUE} causes the panels to be determined
by the \Co{outer} formula.  The points within the panels are 
associated by level of the grouping factor. Defaults to \Co{NULL},
meaning that no outer covariates are present.
\item[\Co{inner:}]
an optional one-sided formula, or list of one-sided
formulas, indicating covariates that are inner to the grouping
factor(s). If multiple levels of grouping are present,
this argument can be either a single one-sided formula, or a list of
one-sided formulas. If no names are assigned to the list elements,
they are assumed in the same order as the group levels (outermost to
innermost grouping). An inner covariate can change 
within the sets of rows defined by the grouping factor.  An inner
formula can be used to associate points in a plot of a groupedData
object.  Defaults to \Co{NULL}, meaning that no inner covariates
are present.
\item[\Co{labels:}]
an optional list of character strings giving labels for
the response and the primary covariate.  The label for the primary
covariate is named \Co{x} and that for the response is named
\Co{y}.  Either label can be omitted.
\item[\Co{units:}]
an optional list of character strings giving the units for
the response and the primary covariate.  The units string for the
primary covariate is named \Co{x} and that for the response is
named \Co{y}.  Either units string can be omitted.
\end{Argument}
\Paragraph{VALUE}
an object inheriting from one of the classes \Co{nfnGroupedData},
\Co{nffGroupedData}, or \Co{nmGroupedData}, and also inheriting
from  classes \Co{groupedData} and \Co{data.frame}.
\Paragraph{REFERENCES}
Bates, D.M. and Pinheiro, J.C. (1997), "Software Design for Longitudinal
Data", in "Modelling Longitudinal and Spatially Correlated Data:
Methods, Applications and Future Directions", T.G. Gregoire (ed.),
Springer-Verlag, New York.
Pinheiro, J.C. and Bates, D.M. (1997) "Future Directions in
Mixed-Effects Software: Design of NLME 3.0" available at
http://nlme.stat.wisc.edu.
\Paragraph{SEE ALSO}
\Co{formula}, \Co{gapply},
\Co{gsummary}, \Co{lme}
\need 15pt
\Paragraph{EXAMPLE}
\vspace{-16pt} 
\begin{Example}
Orth.new <-  # create a new copy of the groupedData object
  groupedData(distance \Twiddle age | Subject,
     data = as.data.frame(Orthodont),
     FUN = mean, outer = \Twiddle Sex,
     labels = list( x = "Age",
     y="Distance from pituitary to pterygomaxillary fissure"),
     units = list(x = "(yr)", y = "(mm)"))
plot( Orth.new )         # trellis plot by Subject
formula( Orth.new )      # extractor for the formula
gsummary( Orth.new )     # apply summary by Subject
fm1 <- lme( Orth.new )   # fixed and groups formulae extracted 
                         # from object
\end{Example}
\end{Helpfile}
\begin{Helpfile}{gsummary}{Summarize by Groups}
Provide a summary of the variables in a data frame by groups of rows.
This is most useful with a \Co{groupedData} object to examine the
variables by group.
\begin{Example}
gsummary(object, FUN, omitGroupingFactor, form, level,
   groups, invariantsOnly, ...)
\end{Example}
\begin{Argument}{ARGUMENTS}
\item[\Co{object:}]
an object to be summarized - usually a \Co{groupedData}
object or a \Co{data.frame}.
\item[\Co{FUN:}]
an optional summary function or a list of summary functions
to be applied to each variable in the frame.  The function or
functions are applied only to variables in \Co{object} that vary
within the groups defined by \Co{groups}.  Invariant variables are
always summarized by group using the unique value that they assume
within that group.  If \Co{FUN} is a single
function it will be applied to each non-invariant variable by group
to produce the summary for that variable.  If \Co{FUN} is a list of
functions, the names in the list should designate classes of
variables in the frame such as \Co{ordered}, \Co{factor}, or
\Co{numeric}.  The indicated function will be applied to any
non-invariant variables of that class.  The default functions to be
used are \Co{mean} for numeric factors, and \Co{Mode} for both
\Co{factor} and \Co{ordered}.  The \Co{Mode} function, defined
internally in \Co{gsummary}, returns the modal or most popular
value of the variable.  It is different from the \Co{mode} function
that returns the S-language mode of the variable.
\item[\Co{omitGroupingFactor:}]
an optional logical value.  When \Co{TRUE}
the grouping factor itself will be omitted from the group-wise
summary but the levels of the grouping factor will continue to be
used as the row names for the data frame that is produced by the
summary. Defaults to \Co{FALSE}.
\item[\Co{form:}]
an optional one-sided formula that defines the groups.
When this formula is given the right-hand side is evaluated in
\Co{object}, converted to a factor if necessary, and the unique
levels are used to define the groups.  Defaults to
\Co{formula(object)}.
\item[\Co{level:}]
an optional positive integer giving the level of grouping
to be used in an object with multiple nested grouping levels.
Defaults to the highest or innermost level of grouping.
\item[\Co{groups:}]
an optional factor that will be used to split the 
rows into groups.  Defaults to \Co{getGroups(object, form, level)}.
\item[\Co{invariantsOnly:}]
an optional logical value.  When \Co{TRUE} only 
those covariates that are invariant within each group will be
summarized.  The summary value for the group is always the unique
value taken on by that covariate within the group.  The columns in
the summary are of the same class as the corresponding columns in
\Co{object}. By definition, the grouping factor itself must be an
invariant.   When combined with \Co{omitGroupingFactor = TRUE},
this option can be used to discover is there are invariant covariates 
in the data frame.  Defaults to \Co{FALSE}.
\item[\Co{...:}]
optional additional arguments to the summary functions
that are invoked on the variables by group.  Often it is helpful to
specify \Co{na.rm = TRUE}.
\end{Argument}
\Paragraph{VALUE}
A \Co{data.frame} with one row for each level of the grouping
factor.  The number of columns is at most the number of columns in
\Co{object}.
\Paragraph{SEE ALSO}
\Co{summary}, \Co{groupedData},
\Co{getGroups}
\need 15pt
\Paragraph{EXAMPLE}
\vspace{-16pt} 
\begin{Example}
gsummary( Orthodont )  # default summary by Subject
## gsummary with invariantsOnly = TRUE and 
## omitGroupingFactor = TRUE determines whether there 
## are covariates like Sex that are invariant
## within the repeated observations on the same Subject.
gsummary( Orthodont, inv = TRUE, omit = TRUE )
\end{Example}
\end{Helpfile}
\begin{Helpfile}{initialize}{Initialize Object}
This function is generic; method functions can be written to handle
specific classes of objects. Classes which already have methods for
this function include: \Co{corStruct}, \Co{lmeStruct},
\Co{reStruct}, and \Co{varFunc}.
\begin{Example}
initialize(object, data, ...)
\end{Example}
\begin{Argument}{ARGUMENTS}
\item[\Co{object:}]
any object requiring initialization, e.g. "plug-in"
structures such as \Co{corStruct} and \Co{varFunc} objects. 
\item[\Co{data:}]
a data frame to be used in the initialization procedure.
\item[\Co{...:}]
some methods for this generic function require additional
arguments.
\end{Argument}
\Paragraph{VALUE}
an initialized object with the same class as \Co{object}. Changes
introduced by the initialization procedure will depend on the method
function used; see the appropriate documentation.
\need 15pt
\Paragraph{EXAMPLE}
\vspace{-16pt} 
\begin{Example}
## see the method function documentation
\end{Example}
\end{Helpfile}
\begin{Helpfile}{initialize.corStruct}{Initialize corStruct Object}
This method initializes \Co{object} by evaluating its associated
covariate(s) and grouping factor, if any is present, in \Co{data},
calculating various dimensions and constants used by optimization
algorithms involving \Co{corStruct} objects (see the appropriate
\Co{Dim} method documentation), and assigning initial values for
the coefficients in \Co{object}, if none were present.
\begin{Example}
initialize(object, data, ...)
\end{Example}
\begin{Argument}{ARGUMENTS}
\item[\Co{object:}]
an object inheriting from class \Co{corStruct}
representing a correlation structure.
\item[\Co{data:}]
a data frame in which to evaluate the variables defined in
\Co{formula(object)}.
\item[\Co{...:}]
this argument is included to make this method compatible
with the generic.
\end{Argument}
\Paragraph{VALUE}
an initialized object with the same class as \Co{object}
representing a correlation structure.
\Paragraph{SEE ALSO}
\Co{Dim.corStruct}
\need 15pt
\Paragraph{EXAMPLE}
\vspace{-16pt} 
\begin{Example}
cs1 <- corAR1(form = \Twiddle 1 | Subject)
cs1 <- initialize(cs1, data = Orthodont)
\end{Example}
\end{Helpfile}
\begin{Helpfile}{initialize.glsStruct}{Initialize a glsStruct Object}
The individual linear model components of the \Co{glsStruct} list
are initialized.
\begin{Example}
initialize(object, data, control)
\end{Example}
\begin{Argument}{ARGUMENTS}
\item[\Co{object:}]
an object inheriting from class \Co{glsStruct},
representing a list of linear model components, such as
\Co{corStruct} and \Co{varFunc} objects.
\item[\Co{data:}]
a data frame in which to evaluate the variables defined in
\Co{formula(object)}.
\item[\Co{control:}]
an optional list with control parameters for the
initialization and optimization algorithms used in
\Co{gls}. Defaults to `list(singular.ok = FALSE, qrTol =
     .Machine\$single.eps)', implying that linear dependencies are not
allowed in the model and that the tolerance for detecting linear
dependencies among the columns of the regression matrix is
\Co{.Machine\$single.eps}.
\end{Argument}
\Paragraph{VALUE}
a \Co{glsStruct} object similar to \Co{object}, but with
initialized model components.
\Paragraph{SEE ALSO}
\Co{gls}, \Co{initialize.corStruct} ,
\Co{initialize.varFunc}
\end{Helpfile}
\begin{Helpfile}{initialize.lmeStruct}{Initialize an lmeStruct Object}
The individual linear mixed-effects model components of the
\Co{lmeStruct} list are initialized.
\begin{Example}
initialize(object, data, groups, conLin, control)
\end{Example}
\begin{Argument}{ARGUMENTS}
\item[\Co{object:}]
an object inheriting from class \Co{lmeStruct},
representing a list of linear mixed-effects model components, such as
\Co{reStruct}, \Co{corStruct}, and \Co{varFunc} objects.
\item[\Co{data:}]
a data frame in which to evaluate the variables defined in
\Co{formula(object)}.
\item[\Co{groups:}]
a data frame with the grouping factors corresponding to
the lme model associated with \Co{object} as columns, sorted from
innermost to outermost grouping level.
\item[\Co{conLin:}]
an optional condensed linear model object, consisting of
a list with components \Co{"Xy"}, corresponding to a regression
matrix (\Co{X}) combined with a response vector (\Co{y}), and 
\Co{"logLik"}, corresponding to the log-likelihood of the
underlying lme model. Defaults to \Co{attr(object, "conLin")}.
\item[\Co{control:}]
an optional list with control parameters for the
initialization and optimization algorithms used in
\Co{lme}. Defaults to \Co{list(niterEM=20, gradHess=TRUE)},
implying that 20 EM iterations are to be used in the derivation of
initial estimates for the coefficients of the \Co{reStruct}
component of \Co{object} and, if possible, numerical gradient
vectors and Hessian matrices for the log-likelihood function are to
be used in the optimization algorithm.
\end{Argument}
\Paragraph{VALUE}
an \Co{lmeStruct} object similar to \Co{object}, but with
initialized model components.
\Paragraph{SEE ALSO}
\Co{lme}, \Co{initialize.reStruct},
\Co{initialize.corStruct} , \Co{initialize.varFunc}
\end{Helpfile}
\begin{Helpfile}{initialize.reStruct}{Initialize reStruct Object}
Initial estimates for the parameters in the \Co{pdMat} objects
forming \Co{object}, which have not yet been initialized, are
obtained using the methodology described in Bates and Pinheiro
(1998). These estimates may be refined using a series of EM
iterations, as described in Bates and Pinheiro (1998). The number of
EM iterations to be used is defined in \Co{control}.
\begin{Example}
initialize(object, data, conLin, control)
\end{Example}
\begin{Argument}{ARGUMENTS}
\item[\Co{object:}]
an object inheriting from class \Co{reStruct},
representing a random effects structure and consisting of a list of
\Co{pdMat} objects.
\item[\Co{data:}]
a data frame in which to evaluate the variables defined in
\Co{formula(object)}.
\item[\Co{conLin:}]
a condensed linear model object, consisting of a list
with components \Co{"Xy"}, corresponding to a regression matrix
(\Co{X}) combined with a response vector (\Co{y}), and
\Co{"logLik"}, corresponding to the log-likelihood of the
underlying model.
\item[\Co{control:}]
an optional list with a single component \Co{niterEM}
controlling the number of iterations for the EM algorithm used to
refine initial parameter estimates. It is given as a list for
compatibility with other \Co{initialize} methods. Defaults to
\Co{list(niterEM = 20)}.
\end{Argument}
\Paragraph{VALUE}
an \Co{reStruct} object similar to \Co{object}, but with all
\Co{pdMat} components initialized.
\Paragraph{REFERENCES}
Bates, D.M. and Pinheiro, J.C. (1998) "Computational methods for
multilevel models" available in PostScript or PDF formats at \\
http://nlme.stat.wisc.edu
\Paragraph{SEE ALSO}
\Co{reStruct}, \Co{pdMat}
\end{Helpfile}
\begin{Helpfile}{initialize.varFunc}{Initialize varFunc Object}
This method initializes \Co{object} by evaluating its associated
covariate(s) and grouping factor, if any is present, in \Co{data};
determining if the covariate(s) need to be updated when the
values of the coefficients associated with \Co{object} change;
initializing the log-likelihood and the weights associated with
\Co{object}; and assigning initial values for the coefficients in
\Co{object}, if none were present. The covariate(s) will only be
initialized if no update is needed when \Co{coef(object)} changes.
\begin{Example}
initialize(object, data, ...)
\end{Example}
\begin{Argument}{ARGUMENTS}
\item[\Co{object:}]
an object inheriting from class \Co{varFunc},
representing a variance function structure.
\item[\Co{data:}]
a data frame in which to evaluate the variables named in
\Co{formula(object)}. 
\item[\Co{...:}]
this argument is included to make this method compatible
with the generic.
\end{Argument}
\Paragraph{VALUE}
an initialized object with the same class as \Co{object}
representing a variance function structure.
\Paragraph{SEE ALSO}
\need 15pt
\Paragraph{EXAMPLE}
\vspace{-16pt} 
\begin{Example}
vf1 <- varPower(form = \Twiddle age|Sex)
vf1 <- initialize(vf1, Orthodont)
\end{Example}
\end{Helpfile}
\begin{Helpfile}{intervals}{Confidence Intervals on Coefficients}
Confidence intervals on the parameters associated with the model
represented by \Co{object} are obtained. This function is generic;
method functions can be written to handle specific classes of
objects. Classes which already have methods for this function include:
\Co{gls}, \Co{lme}, and \Co{lmList}.
\begin{Example}
intervals(object, level, ...)
\end{Example}
\begin{Argument}{ARGUMENTS}
\item[\Co{object:}]
a fitted model object from which parameter estimates can
be extracted.
\item[\Co{level:}]
an optional numeric value for the interval confidence
level. Defaults to 0.95.
\item[\Co{...:}]
some methods for the generic may require additional
arguments.
\end{Argument}
\Paragraph{VALUE}
will depend on the method function used; see the appropriate documentation.
\need 15pt
\Paragraph{EXAMPLE}
\vspace{-16pt} 
\begin{Example}
## see the method documentation
\end{Example}
\end{Helpfile}
\begin{Helpfile}{intervals.gls}{Confidence Intervals on gls Parameters}
Approximate confidence intervals for the parameters in the linear
model represented by \Co{object} are obtained, using
a normal approximation to the distribution of the (restricted)
maximum likelihood estimators (the estimators are assumed to have a
normal distribution centered at the true parameter values and with
covariance matrix equal to the negative inverse Hessian matrix of the
(restricted) log-likelihood evaluated at the estimated parameters).
Confidence intervals are obtained in an unconstrained scale first,
using the normal approximation, and, if necessary, transformed to the
constrained scale.
\begin{Example}
intervals(object, which)
\end{Example}
\begin{Argument}{ARGUMENTS}
\item[\Co{object:}]
an object inheriting from class \Co{gls}, representing
a generalized least squares fitted linear model.
\item[\Co{which:}]
an optional character string with specifying the  subset
of parameters for which to construct the confidence
intervals. Options include \Co{"all"} for all parameters,
\Co{"var-cov"} for the variance-covariance parameters only, and
\Co{"coef"} for the linear model coefficients  only. Defaults to
\Co{"all"}.
\end{Argument}
\Paragraph{VALUE}
a list with components given by data frames with rows corresponding to
parameters and columns \Co{lower}, \Co{est.}, and \Co{upper}
representing respectively lower confidence limits, the estimated
values, and upper confidence limits for the parameters. Possible
components are:
\begin{Argument}{ARGUMENTS}
\item[\Co{coef:}]
linear model coefficients, only present when \Co{which}
is not equal to \Co{"var-cov"}.
\item[\Co{corStruct:}]
correlation parameters, only present when
\Co{which} is not equal to \Co{"coef"} and a 
correlation structure is used in \Co{object}.
\item[\Co{varFunc:}]
variance function parameters, only present when
\Co{which} is not equal to \Co{"coef"} and a variance function
structure is used in \Co{object}.
\item[\Co{sigma:}]
residual standard error.
\end{Argument}
\Paragraph{SEE ALSO}
\Co{gls}, \Co{print.intervals.gls}
\need 15pt
\Paragraph{EXAMPLE}
\vspace{-16pt} 
\begin{Example}
fm1 <- gls(follicles \Twiddle sin(2*pi*Time) + cos(2*pi*Time), Ovary,
           correlation = corAR1(form = \Twiddle  1 | Mare))
intervals(fm1)
\end{Example}
\end{Helpfile}
\begin{Helpfile}{intervals.lmList}{Confidence Intervals on lmList Coefficients}
Confidence intervals on the linear model coefficients are obtained for
each \Co{lm} component of \Co{object} and organized into a three
dimensional array. The first dimension corresponding to the names
of the \Co{object} components. The second dimension is given by
\Co{lower}, \Co{est.}, and \Co{upper} corresponding,
respectively, to the lower confidence limit, estimated coefficient,
and upper confidence limit. The third dimension is given by the
coefficients names.
\begin{Example}
intervals(object, level, pool)
\end{Example}
\begin{Argument}{ARGUMENTS}
\item[\Co{object:}]
an object inheriting from class \Co{lmList}, representing
a list of \Co{lm} objects with a common model.
\item[\Co{level:}]
an optional numeric value with the confidence level for
the intervals. Defaults to 0.95.
\item[\Co{pool:}]
an optional logical value indicating whether a pooled
estimate of the residual standard error should be used. Default is
\Co{attr(object, "pool")}.
\end{Argument}
\Paragraph{VALUE}
a three dimensional array with the confidence intervals and estimates
for the coefficients of each \Co{lm} component of \Co{object}.
\Paragraph{SEE ALSO}
\Co{lmList}, \Co{plot.intervals.lmList}
\need 15pt
\Paragraph{EXAMPLE}
\vspace{-16pt}
\begin{Example}
fm1 <- lmList(distance {\Twiddle} age | Subject, Orthodont)
intervals(fm1)
\end{Example}
\end{Helpfile}
\begin{Helpfile}{intervals.lme}{Confidence Intervals on lme Parameters}
Approximate confidence intervals for the parameters in the linear
mixed-effects model represented by \Co{object} are obtained, using
a normal approximation to the distribution of the (restricted)
maximum likelihood estimators (the estimators are assumed to have a
normal distribution centered at the true parameter values and with
covariance matrix equal to the negative inverse Hessian matrix of the
(restricted) log-likelihood evaluated at the estimated parameters).
Confidence intervals are obtained in an unconstrained scale first,
using the normal approximation, and, if necessary, transformed to the
constrained scale. The \Co{pdNatural} parametrization is used for
general positive-definite matrices.
\begin{Example}
intervals(object, level, which)
\end{Example}
\begin{Argument}{ARGUMENTS}
\item[\Co{object:}]
an object inheriting from class \Co{lme}, representing
a fitted linear mixed-effects model.
\item[\Co{level:}]
an optional numeric value with the confidence level for
the intervals. Defaults to 0.95.
\item[\Co{which:}]
an optional character string with specifying the  subset
of parameters for which to construct the confidence
intervals. Options include \Co{"all"} for all parameters,
\Co{"var-cov"} for the variance-covariance parameters only, and
\Co{"fixed"} for the fixed effects only. Defaults to \Co{"all"}.
\end{Argument}
\Paragraph{VALUE}
a list with components given by data frames with rows corresponding to
parameters and columns \Co{lower}, \Co{est.}, and \Co{upper}
representing respectively lower confidence limits, the estimated
values, and upper confidence limits for the parameters. Possible
components are:
\begin{Argument}{ARGUMENTS}
\item[\Co{fixed:}]
fixed effects, only present when \Co{which} is not
equal to \Co{"var-cov"}.
\item[\Co{reStruct:}]
random effects variance-covariance parameters, only
present when \Co{which} is not equal to \Co{"fixed"}.
\item[\Co{corStruct:}]
within-group correlation parameters, only
present when \Co{which} is not equal to \Co{"fixed"} and a
correlation structure is used in \Co{object}.
\item[\Co{varFunc:}]
within-group variance function parameters, only
present when \Co{which} is not equal to \Co{"fixed"} and a
variance function structure is used in \Co{object}.
\item[\Co{sigma:}]
within-group standard deviation.
\end{Argument}
\Paragraph{SEE ALSO}
\Co{lme}, \Co{print.intervals.lme},
\Co{pdNatural}
\need 15pt
\Paragraph{EXAMPLE}
\vspace{-16pt} 
\begin{Example}
fm1 <- lme(distance \Twiddle age, Orthodont, random = \Twiddle age | Subject)
intervals(fm1)
\end{Example}
\end{Helpfile}
\begin{Helpfile}{isInitialized}{Check if Object is Initialized}
Checks if \Co{object} has been initialized (generally through a call
to \Co{initialize}), by searching for components and attributes
which are modified during initialization.
\begin{Example}
isInitialized(object)
\end{Example}
\begin{Argument}{ARGUMENTS}
\item[\Co{object:}]
any object requiring initialization.
\end{Argument}
\Paragraph{VALUE}
a logical value indicating whether \Co{object} has been
initialized.
\Paragraph{SEE ALSO}
\Co{initialize}
\need 15pt
\Paragraph{EXAMPLE}
\vspace{-16pt} 
\begin{Example}
pd1 <- pdDiag(\Twiddle age)
isInitialized(pd1)
\end{Example}
\end{Helpfile}
\begin{Helpfile}{isBalanced}{Check a Design for Balance}
Check the design of the experiment or study for balance.
\begin{Example}
isBalanced(object, countOnly, level)
\end{Example}
\begin{Argument}{ARGUMENTS}
\item[\Co{object:}]
A \Co{groupedData} object containing a data frame and a
formula that describes the roles of variables in the data frame.  The
object will have one or more nested grouping factors and a primary
covariate.
\item[\Co{countOnly:}]
A logical value indicating if the check for balance
should only consider the number of observations at each level of the
grouping factor(s).  Defaults to \Co{FALSE}.
\item[\Co{level:}]
an optional positive integer giving the level of grouping
to be used with multilevel data. Defaults to the highest or
innermost level of grouping.
\end{Argument}
\Paragraph{VALUE}
\Co{TRUE} or \Co{FALSE} according to whether the data are balanced
or not
\Paragraph{SEE ALSO}
\Co{table}, \Co{groupedData}
\need 15pt
\Paragraph{EXAMPLE}
\vspace{-16pt} 
\begin{Example}
isBalanced(Orthodont)                    # should return TRUE
isBalanced(Orthodont, countOnly = TRUE)  # should return TRUE
isBalanced(Pixel)                        # should return FALSE
isBalanced(Pixel, level = 1)             # should return FALSE
\end{Example}
\end{Helpfile}
\begin{Helpfile}{isInitialized.reStruct}{Check reStruct Initialization}
Checks if all \Co{pdMat} components of \Co{object} have been
initialized.
\begin{Example}
isInitialized(object)
\end{Example}
\begin{Argument}{ARGUMENTS}
\item[\Co{object:}]
an object inheriting from class \Co{reStruct},
representing a random effects structure and consisting of a list of
\Co{pdMat} objects.
\end{Argument}
\Paragraph{VALUE}
a logical value indicating whether all components of \Co{object} have
been initialized.
\Paragraph{SEE ALSO}
\Co{initialize}, \Co{reStruct}
\need 15pt
\Paragraph{EXAMPLE}
\vspace{-16pt} 
\begin{Example}
rs1 <- reStruct(\Twiddle age|Subject)
isInitialized(rs1)
\end{Example}
\end{Helpfile}
\begin{Helpfile}{lmList}{List of lm Objects with a Common Model}
\Co{Data} is partitioned according to the levels of the grouping
factor \Co{g} and individual \Co{lm} fits are obtained for each
\Co{data} partition, using the model defined in \Co{object}.
\begin{Example}
lmList(object, data, level, na.action, pool)
\end{Example}
\begin{Argument}{ARGUMENTS}
\item[\Co{object:}]
either a linear formula object of the form \Co{y {\Twiddle} x1+...+xn | g}
or a \Co{groupedData} object. In the formula object, \Co{y}
represents the response, \Co{x1,...,xn} the covariates, and
\Co{g} the grouping factor specifying the partitioning of the data
according to which different \Co{lm} fits should be performed. The
grouping factor \Co{g} may be omitted from the formula, in which
case the grouping structure will be obtained from \Co{data}, which
must inherit from class \Co{groupedData}. The method function 
\Co{lmList.groupedData} is documented separately.
\item[\Co{data:}]
an data frame in which to interpret the variables named in
\Co{object}. 
\item[\Co{level:}]
an optional integer specifying the level of grouping to be used when 
multiple nested levels of grouping are present.
\item[\Co{na.action:}]
a function that indicates what should happen when the
data contain \Co{NA}s.  The default action (\Co{na.fail}) causes
\Co{lmList} to print an error message and terminate if there are any
incomplete observations.
\item[\Co{pool:}]
an optional logical value that is preserved as an attribute of the
returned value.  This will be used as the default for \Co{pool} in
calculations of standard deviations or standard errors for summaries.
\end{Argument}
\Paragraph{VALUE}
a list of \Co{lm} objects with as many components as the number of
groups defined by the grouping factor. Generic functions such as
\Co{coef}, \Co{fixef}, \Co{lme}, \Co{pairs},
\Co{plot}, \Co{predict}, \Co{ranef}, \Co{summary},
and \Co{update} have methods that can be applied to an \Co{lmList}
object.
\Paragraph{SEE ALSO}
\Co{lm}, \Co{lme.lmList}.
\need 15pt
\Paragraph{EXAMPLE}
\vspace{-16pt}
\begin{Example}
fm1 <- lmList(distance {\Twiddle} age | Subject, Orthodont)
\end{Example}
\end{Helpfile}
\begin{Helpfile}{lmList.groupedData}{lmList Fit from a groupedData Object}
The response variable and primary covariate in \Co{formula(object)}
are used to construct the linear model formula. This formula
and the \Co{groupedData} object are passed as the \Co{object} and
\Co{data} arguments to \Co{lmList.formula}, together with any other
additional arguments in the function call. See the documentation on
\Co{lmList.formula} for a description of that function.
\begin{Example}
lmList(object, data, level, na.action, pool)
\end{Example}
\begin{Argument}{ARGUMENTS}
\item[\Co{object:}]
a data frame inheriting from class \Co{groupedData}.
\item[\Co{data:}]
this argument is included for consistency with the generic
function. It is ignored in this method function.
\item[\Co{other arguments:}]
identical to the arguments in the generic
function call. See the documentation on \Co{lmList}.
\end{Argument}
\Paragraph{VALUE}
a list of \Co{lm} objects with as many components as the number of
groups defined by the grouping factor. Generic functions such as
\Co{coef}, \Co{fixef}, \Co{lme}, \Co{pairs},
\Co{plot}, \Co{predict}, \Co{ranef}, \Co{summary},
and \Co{update} have methods that can be applied to an \Co{lmList}
object.
\Paragraph{SEE ALSO}
\Co{groupedData}, \Co{lm}, \Co{lme.lmList},
\Co{lmList.formula}
\need 15pt
\Paragraph{EXAMPLE}
\vspace{-16pt}
\begin{Example}
fm1 <- lmList(Orthodont)
\end{Example}
\end{Helpfile}
\begin{Helpfile}{lme}{Linear Mixed-Effects Models}
This generic function fits a linear mixed-effects model in the
formulation described in Laird and Ware (1982) but allowing for nested
random effects. The within-group errors are allowed to be correlated
and/or have unequal variances.
\begin{Example}
lme(fixed, data, random, correlation, weights, subset, method,
    na.action, control)
\end{Example}
\begin{Argument}{ARGUMENTS}
\item[\Co{fixed:}]
a two-sided linear formula object describing the
fixed-effects part of the model, with the response on the left of a
\Co{{\Twiddle}} operator and the terms, separated by \Co{+} operators, on
the right, an \Co{lmList} object, or a \Co{groupedData}
object. The method functions \Co{lme.lmList} and
\Co{lme.groupedData} are documented separately.
\item[\Co{data:}]
an optional data frame containing the variables named in
\Co{fixed}, \Co{random}, \Co{correlation}, \Co{weights}, and
\Co{subset}.  By default the variables are taken from the
environment from which \Co{lme} is called.
\item[\Co{random:}]
optionally, any of the following: (i) a one-sided formula
of the form \Co{{\Twiddle}x1+...+xn | g1/.../gm}, with \Co{x1+...+xn}
specifying the model for the random effects and \Co{g1/.../gm} the
grouping structure (\Co{m} may be equal to 1, in which case no
\Co{/} is required). The random effects formula will be repeated
for all levels of grouping, in the case of multiple levels of
grouping; (ii) a list of one-sided formulas of the form
\Co{{\Twiddle}x1+...+xn | g}, with possibly different random effects models
for each grouping level. The order of nesting will be assumed the
same as the order of the elements in the list; (iii) a one-sided
formula of the form \Co{{\Twiddle}x1+...+xn}, or a \Co{pdMat} object with
a formula (i.e. a non-\Co{NULL} value for \Co{formula(object)}),
or a list of such formulas or \Co{pdMat} objects. In this case, the
grouping structure formula will be derived from the data used to to
fit the linear mixed-effects model, which should inherit from class
\Co{groupedData}; (iv) a named list of formulas or \Co{pdMat}
objects as in (iii), with the grouping factors as names. The order of
nesting will be assumed the same as the order of the order of the
elements in the list; (v) an \Co{reStruct} object. See the
documentation on \Co{pdClasses} for a description of the available
\Co{pdMat} classes. Defaults to a formula consisting of the right
hand side of \Co{fixed}.
\item[\Co{correlation:}]
an optional \Co{corStruct} object describing the
within-group correlation structure. See the documentation of
\Co{corClasses} for a description of the available \Co{corStruct}
classes. Defaults to \Co{NULL},
corresponding to no within-group correlations.
\item[\Co{weights:}]
an optional \Co{varFunc} object or one-sided formula
describing the within-group heteroscedasticity structure. If given as
a formula, it is used as the argument to \Co{varFixed},
corresponding to fixed variance weights. See the documentation on
\Co{varClasses} for a description of the available \Co{varFunc}
classes. Defaults to \Co{NULL}, corresponding to homocesdatic
within-group errors.
\item[\Co{subset:}]
an optional expression saying which subset of the rows of
\Co{data} should  be  used in the fit. This can be a logical
vector, or a numeric vector indicating which observation numbers are
to be included, or a  character  vector of the row names to be
included.  All observations are included by default.
\item[\Co{method:}]
a character string.  If \Co{"REML"} the model is fit by
maximizing the restricted log-likelihood.  If \Co{"ML"} the
log-likelihood is maximized.  Defaults to \Co{"REML"}.
\item[\Co{na.action:}]
a function that indicates what should happen when the
data contain \Co{NA}s.  The default action (\Co{na.fail}) causes
\Co{lme} to print an error message and terminate if there are any
incomplete observations.
\item[\Co{control:}]
a list of control values for the estimation algorithm to
replace the default values returned by the function \Co{lmeControl}.
Defaults to an empty list.
\end{Argument}
\Paragraph{VALUE}
an object of class \Co{lme} representing the linear mixed-effects
model fit. Generic functions such as \Co{print}, \Co{plot} and
\Co{summary} have methods to show the results of the fit. See
\Co{lmeObject} for the components of the fit. The functions
\Co{resid}, \Co{coef}, \Co{fitted}, \Co{fixef}, and
\Co{ranef}  can be used to extract some of its components.
\Paragraph{REFERENCES}
The computational methods are described in Bates, D.M. and Pinheiro, J.C.
(1998) and follow on the general framework of Lindstrom, M.J. and Bates,
D.M. (1988). The model formulation is described in Laird, N.M. and Ware,
J.H. (1982).  The variance-covariance parametrizations are described in
Pinheiro, J.C. and Bates., D.M.  (1996).   The different correlation
structures available for the \Co{correlation} argument are described
in Box, G.E.P., Jenkins, G.M., and Reinsel G.C. (1994), Littel, R.C.,
Milliken, G.A., Stroup, W.W., and Wolfinger, R.D. (1997), and Venables,
W.N. and Ripley, B.D. (1997). The use of variance functions for linear
and nonlinear mixed effects models is presented in detail in Davidian,
M. and Giltinan, D.M. (1995). \\
Bates, D.M. and Pinheiro, J.C. (1998) "Computational methods for
multilevel models" available in PostScript or PDF formats at
http://nlme.stat.wisc.edu \\
Box, G.E.P., Jenkins, G.M., and Reinsel G.C. (1994) "Time Series
Analysis: Forecasting and Control", 3rd Edition, Holden-Day. \\
Davidian, M. and Giltinan, D.M. (1995) "Nonlinear Mixed Effects Models
for Repeated Measurement Data", Chapman and Hall.\\
Laird, N.M. and Ware, J.H. (1982) "Random-Effects Models for
Longitudinal Data", Biometrics, 38, 963-974.  \\
Lindstrom, M.J. and Bates, D.M. (1988) "Newton-Raphson and EM
Algorithms for Linear Mixed-Effects Models for Repeated-Measures
Data", Journal of the American Statistical Association, 83,
1014-1022. \\
Littel, R.C., Milliken, G.A., Stroup, W.W., and Wolfinger, R.D. (1997)
"SAS Systems for Mixed Models", SAS Institute.\\
Pinheiro, J.C. and Bates., D.M.  (1996) "Unconstrained
Parametrizations for Variance-Covariance Matrices", Statistics and
Computing, 6, 289-296.\\
Venables, W.N. and Ripley, B.D. (1997) "Modern Applied Statistics with
S-plus", 2nd Edition, Springer-Verlag.
\Paragraph{SEE ALSO}
\Co{lmeControl}, \Co{lme.lmList},
\Co{lme.groupedData}, \Co{lmeObject},
\Co{lmList}, \Co{reStruct}, \Co{reStruct},
\Co{pdClasses}, \Co{corClasses}, \Co{varClasses}
\need 15pt
\Paragraph{EXAMPLE}
\vspace{-16pt}
\begin{Example}
fm1 <- lme(distance {\Twiddle} age, data = Orthodont) # random is {\Twiddle} age
fm2 <- lme(distance {\Twiddle} age + Sex, data = Orthodont, random = {\Twiddle} 1)
\end{Example}
\end{Helpfile}
\begin{Helpfile}{lme.groupedData}{{\sc lme} fit from groupedData Object}
The response variable and primary covariate in \Co{formula(fixed)}
are used to construct the fixed effects model formula. This formula
and the   \Co{groupedData} object are passed as the \Co{fixed} and
\Co{data} arguments to \Co{lme.formula}, together with any other
additional arguments in the function call. See the documentation on
\Co{lme.formula} for a description of that function.
\begin{Example}
lme(fixed, data, random, correlation, weights, subset, method,
    na.action, control)
\end{Example}
\begin{Argument}{ARGUMENTS}
\item[\Co{fixed:}]
a data frame inheriting from class \Co{groupedData}.
\item[\Co{data:}]
this argument is included for consistency with the generic
function. It is ignored in this method function.
\item[\Co{other arguments:}]
identical to the arguments in the generic
function call. See the documentation on \Co{lme}.
\end{Argument}
\Paragraph{VALUE}
an object of class \Co{lme} representing the linear mixed-effects
model fit. Generic functions such as \Co{print}, \Co{plot} and
\Co{summary} have methods to show the results of the fit. See
\Co{lmeObject} for the components of the fit. The functions
\Co{resid}, \Co{coef}, \Co{fitted}, \Co{fixef}, and
\Co{ranef}  can be used to extract some of its components.
\Paragraph{REFERENCES}
The computational methods are described in Bates, D.M. and Pinheiro, J.C.
(1998) and follow on the general framework of Lindstrom, M.J. and Bates,
D.M. (1988). The model formulation is described in Laird, N.M. and Ware,
J.H. (1982).  The variance-covariance parametrizations are described in
Pinheiro, J.C. and Bates., D.M.  (1996).   The different correlation
structures available for the \Co{correlation} argument are described
in Box, G.E.P., Jenkins, G.M., and Reinsel G.C. (1994), Littel, R.C.,
Milliken, G.A., Stroup, W.W., and Wolfinger, R.D. (1997), and Venables,
W.N. and Ripley, B.D. (1997). The use of variance functions for linear
and nonlinear mixed effects models is presented in detail in Davidian,
M. and Giltinan, D.M. (1995). \\
Bates, D.M. and Pinheiro, J.C. (1998) "Computational methods for
multilevel models" available in PostScript or PDF formats at
http://nlme.stat.wisc.edu\\
Box, G.E.P., Jenkins, G.M., and Reinsel G.C. (1994) "Time Series
Analysis: Forecasting and Control", 3rd Edition, Holden-Day. \\
Davidian, M. and Giltinan, D.M. (1995) "Nonlinear Mixed Effects Models
for Repeated Measurement Data", Chapman and Hall.\\
Laird, N.M. and Ware, J.H. (1982) "Random-Effects Models for
Longitudinal Data", Biometrics, 38, 963-974.  \\
Lindstrom, M.J. and Bates, D.M. (1988) "Newton-Raphson and EM
Algorithms for Linear Mixed-Effects Models for Repeated-Measures
Data", Journal of the American Statistical Association, 83,
1014-1022. \\
Littel, R.C., Milliken, G.A., Stroup, W.W., and Wolfinger, R.D. (1997)
"SAS Systems for Mixed Models", SAS Institute.\\
Pinheiro, J.C. and Bates., D.M.  (1996) "Unconstrained
Parametrizations for Variance-Covariance Matrices", Statistics and
Computing, 6, 289-296.\\
Venables, W.N. and Ripley, B.D. (1997) "Modern Applied Statistics with
S-plus", 2nd Edition, Springer-Verlag.
\Paragraph{SEE ALSO}
\Co{lme}, \Co{groupedData},
\Co{lmeObject}
\need 15pt
\Paragraph{EXAMPLE}
\vspace{-16pt}
\begin{Example}
fm1 <- lme(Orthodont)
\end{Example}
\end{Helpfile}
\begin{Helpfile}{lme.lmList}{{\sc lme} fit from lmList Object}
If the random effects names defined in \Co{random} are a subset of
the \Co{lmList} object coefficient names, initial estimates for the
covariance matrix of the random effects are obtained (overwriting any
values given in \Co{random}). \Co{formula(fixed)} and the
\Co{data} argument in the calling sequence used to obtain
\Co{fixed} are passed as the \Co{fixed} and \Co{data} arguments
to \Co{lme.formula}, together with any other additional arguments in
the function call. See the documentation on \Co{lme.formula} for a
description of that function.
\begin{Example}
lme(fixed, data, random, correlation, weights, subset, method,
    na.action, control)
\end{Example}
\begin{Argument}{ARGUMENTS}
\item[\Co{fixed:}]
an object inheriting from class \Co{lmList},
representing a list of \Co{lm} fits with a common model.
\item[\Co{data:}]
this argument is included for consistency with the generic
function. It is ignored in this method function.
\item[\Co{random:}]
an optional one-sided linear formula with no conditioning
expression, or a \Co{pdMat} object with a \Co{formula}
attribute. Multiple levels of grouping are not allowed with this
method function.  Defaults to a formula consisting of the right hand
side of \Co{formula(fixed)}.
\item[\Co{other arguments:}]
identical to the arguments in the generic
function call. See the documentation on \Co{lme}.
\end{Argument}
\Paragraph{VALUE}
an object of class \Co{lme} representing the linear mixed-effects
model fit. Generic functions such as \Co{print}, \Co{plot} and
\Co{summary} have methods to show the results of the fit. See
\Co{lmeObject} for the components of the fit. The functions
\Co{resid}, \Co{coef}, \Co{fitted}, \Co{fixef}, and
\Co{ranef}  can be used to extract some of its components.
\Paragraph{REFERENCES}
The computational methods are described in Bates, D.M. and Pinheiro, J.C.
(1998) and follow on the general framework of Lindstrom, M.J. and Bates,
D.M. (1988). The model formulation is described in Laird, N.M. and Ware,
J.H. (1982).  The variance-covariance parametrizations are described in
Pinheiro, J.C. and Bates., D.M.  (1996).   The different correlation
structures available for the \Co{correlation} argument are described
in Box, G.E.P., Jenkins, G.M., and Reinsel G.C. (1994), Littel, R.C.,
Milliken, G.A., Stroup, W.W., and Wolfinger, R.D. (1997), and Venables,
W.N. and Ripley, B.D. (1997). The use of variance functions for linear
and nonlinear mixed effects models is presented in detail in Davidian,
M. and Giltinan, D.M. (1995). \\
Bates, D.M. and Pinheiro, J.C. (1998) "Computational methods for
multilevel models" available in PostScript or PDF formats at
http://nlme.stat.wisc.edu\\
Box, G.E.P., Jenkins, G.M., and Reinsel G.C. (1994) "Time Series
Analysis: Forecasting and Control", 3rd Edition, Holden-Day. \\
Davidian, M. and Giltinan, D.M. (1995) "Nonlinear Mixed Effects Models
for Repeated Measurement Data", Chapman and Hall.\\
Laird, N.M. and Ware, J.H. (1982) "Random-Effects Models for
Longitudinal Data", Biometrics, 38, 963-974.  \\
Lindstrom, M.J. and Bates, D.M. (1988) "Newton-Raphson and EM
Algorithms for Linear Mixed-Effects Models for Repeated-Measures
Data", Journal of the American Statistical Association, 83,
1014-1022. \\
Littel, R.C., Milliken, G.A., Stroup, W.W., and Wolfinger, R.D. (1997)
"SAS Systems for Mixed Models", SAS Institute.\\
Pinheiro, J.C. and Bates., D.M.  (1996) "Unconstrained
Parametrizations for Variance-Covariance Matrices", Statistics and
Computing, 6, 289-296.\\
Venables, W.N. and Ripley, B.D. (1997) "Modern Applied Statistics with
S-plus", 2nd Edition, Springer-Verlag.
\Paragraph{SEE ALSO}
\Co{lme}, \Co{lmList},
\Co{lmeObject}
\need 15pt
\Paragraph{EXAMPLE}
\vspace{-16pt}
\begin{Example}
fm1 <- lmList(Orthodont)
fm2 <- lme(fm1)
\end{Example}
\end{Helpfile}
\begin{Helpfile}{lmeControl}{Control Values for lme Fit}
The values supplied in the function call replace the defaults and a
list with all possible arguments is returned. The returned list is
used as the \Co{control} argument to the \Co{lme} function.
\begin{Example}
lmeControl(maxIter, msMaxIter, tolerance, niterEM, msTol,
           msScale, msVerbose, returnObject, gradHess, apVar,
           .relStep, natural)
\end{Example}
\begin{Argument}{ARGUMENTS}
\item[\Co{maxIter:}]
maximum number of iterations for the \Co{lme}
optimization algorithm. Default is 50.
\item[\Co{msMaxIter:}]
maximum number of iterations
for the \Co{ms} optimization step inside the \Co{lme}
optimization. Default is 50.
\item[\Co{tolerance:}]
tolerance for the convergence criterion in the
\Co{lme} algorithm. Default is 1e-6.
\item[\Co{niterEM:}]
number of iterations for the EM algorithm used to refine
the initial estimates of the random effects variance-covariance
coefficients. Default is 25.
\item[\Co{msTol:}]
tolerance for the convergence criterion in \Co{ms},
passed as the \Co{rel.tolerance} argument to the function (see
documentation on \Co{ms}). Default is 1e-7. 
\item[\Co{msScale:}]
scale function passed as the \Co{scale} argument to
the \Co{ms} function (see documentation on that function). Default
is \Co{lmeScale}.
\item[\Co{msVerbose:}]
a logical value passed as the \Co{trace} argument to
\Co{ms} (see documentation on that function). Default is
\Co{FALSE}.
\item[\Co{returnObject:}]
a logical value indicating whether the fitted
object should be returned when the maximum number of iterations is
reached without convergence of the algorithm. Default is
\Co{FALSE}.
\item[\Co{gradHess:}]
a logical value indicating whether numerical gradient
vectors and Hessian matrices of the log-likelihood function should
be used in the \Co{ms} optimization. This option is only available
when the correlation structure (\Co{corStruct}) and the variance
function structure (\Co{varFunc}) have no "varying" parameters and
the \Co{pdMat} classes used in the random effects structure are
\Co{pdSymm} (general positive-definite), \Co{pdDiag} (diagonal),
\Co{pdIdent} (multiple of the identity),  or
\Co{pdCompSymm} (compound symmetry). Default is \Co{TRUE}.
\item[\Co{apVar:}]
a logical value indicating whether the approximate
covariance matrix of the variance-covariance parameters should be
calculated. Default is \Co{TRUE}.
\item[\Co{.relStep:}]
relative step for numerical derivatives
calculations. Default is \\ \Co{.Machine\$double.eps}$^{1/3}$.
\item[\Co{natural:}]
a logical value indicating whether the \Co{pdNatural}
parametrization should be used for general positive-definite matrices
(\Co{pdSymm}) in \Co{reStruct}, when the approximate covariance
matrix of the estimators is calculated. Default is \Co{TRUE}.
\end{Argument}
\Paragraph{VALUE}
a list with components for each of the possible arguments.
\Paragraph{SEE ALSO}
\Co{lme}, \Co{ms}, \Co{lmeScale}
\need 15pt
\Paragraph{EXAMPLE}
\vspace{-16pt} 
\begin{Example}
# decrease the maximum number iterations in the ms call and
# request that information on the evolution of the ms iterations be printed
lmeControl(msMaxIter = 20, msVerbose = TRUE)
\end{Example}
\end{Helpfile}
\begin{Helpfile}{lmeObject}{Fitted lme Object}
An object returned by the \Co{lme} function, inheriting from class
\Co{lme} and representing a fitted linear mixed-effects
model. Objects of this class have methods for the generic functions 
\Co{anova}, \Co{coef}, \Co{fitted}, \Co{fixef},
\Co{formula}, \Co{getGroups}, \Co{getResponse},
\Co{intervals}, \Co{logLik}, \Co{pairs}, \Co{plot},
\Co{predict}, \Co{print}, \Co{ranef}, \Co{residuals},
\Co{summary}, and \Co{update}.
\Paragraph{VALUE}
The following components must be included in a legitimate \Co{lme}
object. 
\begin{Argument}{COMPONENTS}
\item[\Co{apVar:}]
an approximate covariance matrix for the
variance-covariance coefficients. If \Co{apVar = FALSE} in the list
of control values used in the call to \Co{lme}, this
component is equal to \Co{NULL}.
\item[\Co{call:}]
a list containing an image of the \Co{lme} call that
produced the object.
\item[\Co{coefficients:}]
a list with two components, \Co{fixed} and
\Co{random}, where the first is a vector containing the estimated
fixed effects and the second is a list of matrices with the estimated
random effects for each level of grouping. For each matrix in the
\Co{random} list, the columns refer to the random effects and the
rows to the groups.
\item[\Co{contrasts:}]
a list with the contrasts used to represent factors
in the fixed effects formula  and/or random effects formula. This
information is important for making predictions from a new data
frame in which not all levels of the original factors are
observed. If no factors are used in the lme model, this component
will be an empty list.
\item[\Co{dims:}]
a list with basic dimensions used in the lme fit,
including the components \Co{N} - the number of observations in
the data, \Co{Q} - the number of grouping levels, \Co{qvec} -
the number of random effects at each level from innermost to
outermost (last two values are equal to zero and correspond to the
fixed effects and the response), \Co{ngrps} - the number of groups
at each level from innermost to outermost (last two values are one
and correspond to the fixed effects and the response), and
\Co{ncol} - the number of columns in the model matrix for each
level of grouping from innermost to outermost (last two values are
equal to the number of fixed effects and one).
\item[\Co{fitted:}]
a data frame with the fitted values as columns. The
leftmost column corresponds to the population fixed effects
(corresponding to the fixed effects only) and successive columns
from left to right correspond to increasing levels of grouping.
\item[\Co{fixDF:}]
a list with components \Co{X} and \Co{terms}
specifying the denominator degrees of freedom for, respectively,
t-tests for the individual fixed effects and F-tests for the
fixed-effects terms in the models.
\item[\Co{groups:}]
a data frame with the grouping factors as
columns. The grouping level increases from left to right.
\item[\Co{logLik:}]
the (restricted) log-likelihood at convergence.
\item[\Co{method:}]
the estimation method: either \Co{"ML"} for maximum
likelihood, or \Co{"REML"} for restricted maximum likelihood.
\item[\Co{modelStruct:}]
an object inheriting from class \Co{lmeStruct},
representing a list of mixed-effects model components, such
as \Co{reStruct}, \Co{corStruct}, and \Co{varFunc} objects.
\item[\Co{numIter:}]
the number of iterations used in the iterative
algorithm.
\item[\Co{residuals:}]
a data frame with the residuals as columns. The
leftmost column corresponds to the population residuals
and successive columns from left to right correspond to increasing
levels of grouping.
\item[\Co{sigma:}]
the estimated within-group error standard deviation.
\item[\Co{varFix:}]
an approximate covariance matrix of the
fixed effects estimates.
\end{Argument}
\Paragraph{SEE ALSO}
\Co{lme}, \Co{lmeStruct}
\end{Helpfile}
\begin{Helpfile}{lmeScale}{Scale for lme Optimization}
This function calculates the scales to be used for each coefficient
estimated through an \Co{ms} optimization in the \Co{lme}
function. If all initial values are zero, the scale is set to one for
all coefficients; else, the scale for a coefficient with non-zero
initial value is equal to the inverse of its initial value and the
scale for the coefficients with initial value equal to zero is set to
the median of the non-zero initial value coefficients.
\begin{Example}
lmeScale(start)
\end{Example}
\begin{Argument}{ARGUMENTS}
\item[\Co{start:}]
the starting values for the coefficients to be estimated.
\end{Argument}
\Paragraph{VALUE}
a vector with the scales to be used in \Co{ms} for estimating the
coefficients.
\Paragraph{SEE ALSO}
\Co{ms}
\end{Helpfile}
\begin{Helpfile}{lmeStruct}{Linear Mixed-Effects Structure}
A linear mixed-effects structure is a list of model components
representing different sets of parameters in the linear mixed-effects
model. An \Co{lmeStruct} list must contain at least a
\Co{reStruct} object, but may also contain \Co{corStruct} and
\Co{varFunc} objects. \Co{NULL} arguments are not included in the
\Co{lmeStruct} list.
\begin{Example}
lmeStruct(reStruct, corStruct, varStruct)
\end{Example}
\begin{Argument}{ARGUMENTS}
\item[\Co{reStruct:}]
a \Co{reStruct} representing a random effects
structure.
\item[\Co{corStruct:}]
an optional \Co{corStruct} object, representing a
correlation structure. Default is \Co{NULL}.
\item[\Co{varStruct:}]
an optional \Co{varFunc} object, representing a
variance function structure. Default is \Co{NULL}.
\end{Argument}
\Paragraph{VALUE}
a list of model components determining the parameters to be estimated
for the associated linear mixed-effects model.
\Paragraph{SEE ALSO}
\Co{lme}, \Co{reStruct},
\Co{corClasses}, \Co{varClasses}
\need 15pt
\Paragraph{EXAMPLE}
\vspace{-16pt} 
\begin{Example}
lms1 <- lmeStruct(reStruct(\Twiddle age), corAR1(), varPower())
\end{Example}
\end{Helpfile}
\begin{Helpfile}{logDet}{Extract the Logarithm of the Determinant}
This function is generic; method functions can be written to handle
specific classes of objects. Classes which already have methods for
this function include: \Co{corStruct}, several \Co{pdMat} classes,
and \Co{reStruct}.
\begin{Example}
logDet(object, ...)
\end{Example}
\begin{Argument}{ARGUMENTS}
\item[\Co{object:}]
any object from which a matrix, or list of matrices, can
be extracted
\item[\Co{...:}]
some methods for this generic function require additional
arguments.
\end{Argument}
\Paragraph{VALUE}
will depend on the method function used; see the appropriate
documentation.
\Paragraph{SEE ALSO}
\Co{logLik}
\need 15pt
\Paragraph{EXAMPLE}
\vspace{-16pt} 
\begin{Example}
## see the method function documentation
\end{Example}
\end{Helpfile}
\begin{Helpfile}{logDet.corStruct}{Extract corStruct Log-Determinant}
This method function extracts the logarithm of the determinant of a
square-root factor of the correlation matrix associated with
\Co{object}, or the sum of the log-determinants of square-root
factors of the list of correlation matrices associated with
\Co{object}.
\begin{Example}
logDet(object, covariate)
\end{Example}
\begin{Argument}{ARGUMENTS}
\item[\Co{object:}]
an object inheriting from class \Co{corStruct},
representing a correlation structure.
\item[\Co{covariate:}]
an optional covariate vector (matrix), or list of
covariate vectors (matrices), at which values the correlation matrix,
or list of correlation  matrices, are to be evaluated. Defaults to
\Co{getCovariate(object)}.
\end{Argument}
\Paragraph{VALUE}
the log-determinant of a square-root factor of the correlation matrix
associated with \Co{object}, or the sum of the log-determinants of
square-root factors of the list of correlation matrices associated
with \Co{object}.
\Paragraph{SEE ALSO}
\Co{logLik.corStruct}, \Co{corMatrix.corStruct}
\need 15pt
\Paragraph{EXAMPLE}
\vspace{-16pt} 
\begin{Example}
cs1 <- corAR1(0.3)
logDet(cs1, covariate = 1:4)
\end{Example}
\end{Helpfile}
\begin{Helpfile}{logDet.pdMat}{pdMat Log-Determinant}
This method function extracts the logarithm of the determinant of a
square-root factor of the positive-definite matrix represented by
\Co{object}.
\begin{Example}
logDet(object)
\end{Example}
\begin{Argument}{ARGUMENTS}
\item[\Co{object:}]
an object inheriting from class \Co{pdMat}, representing
a positive definite matrix.
\end{Argument}
\Paragraph{VALUE}
the log-determinant of a square-root factor of the positive-definite
matrix represented by \Co{object}.
\Paragraph{SEE ALSO}
\Co{pdMat}
\need 15pt
\Paragraph{EXAMPLE}
\vspace{-16pt} 
\begin{Example}
pd1 <- pdSymm(diag(1:3))
logDet(pd1)
\end{Example}
\end{Helpfile}
\begin{Helpfile}{logDet.reStruct}{Extract reStruct Log-Determinants}
Calculates, for each of the \Co{pdMat} components of \Co{object},
the logarithm of the determinant of a square-root factor.
\begin{Example}
logDet(object)
\end{Example}
\begin{Argument}{ARGUMENTS}
\item[\Co{object:}]
an object inheriting from class \Co{reStruct},
representing a random effects structure and consisting of a list of
\Co{pdMat} objects.
\end{Argument}
\Paragraph{VALUE}
a vector with the log-determinants of square-root factors of the
\Co{pdMat} components of \Co{object}.
\Paragraph{SEE ALSO}
\Co{reStruct}, \Co{pdMat}
\need 15pt
\Paragraph{EXAMPLE}
\vspace{-16pt} 
\begin{Example}
rs1 <- reStruct(list(A = pdSymm(diag(1:3), form = \Twiddle Score),
  B = pdDiag(2 * diag(4), form = \Twiddle Educ)))
logDet(rs1)
\end{Example}
\end{Helpfile}
\begin{Helpfile}{logLik}{Extract Log-Likelihood}
This function is generic; method functions can be written to handle
specific classes of objects. Classes which already have methods for
this function include: \Co{corStruct}, \Co{gls}, \Co{lm},
\Co{lme}, \Co{lmList}, \Co{lmeStruct}, \Co{reStruct}, and
\Co{varFunc}.
\begin{Example}
logLik(object, ...)
\end{Example}
\begin{Argument}{ARGUMENTS}
\item[\Co{object:}]
any object from which a log-likelihood value, or a
contribution to a log-likelihood value, can be extracted.
\item[\Co{...:}]
some methods for this generic function require additional
arguments.
\end{Argument}
\Paragraph{VALUE}
will depend on the method function used; see the appropriate
documentation.
\need 15pt
\Paragraph{EXAMPLE}
\vspace{-16pt} 
\begin{Example}
## see the method function documentation
\end{Example}
\end{Helpfile}
\begin{Helpfile}{logLik.corStruct}{corStruct Log-Likelihood}
This method function extracts the component of a Gaussian
log-likelihood associated with the correlation structure, which is
equal to the negative of the logarithm of the determinant (or sum of
the logarithms of the determinants) of the matrix (or matrices)
represented by \Co{object}.
\begin{Example}
logLik(object, data)
\end{Example}
\begin{Argument}{ARGUMENTS}
\item[\Co{object:}]
an object inheriting from class \Co{corStruct},
representing a correlation structure.
\item[\Co{data:}]
this argument is included to make this method function
compatible with other \Co{logLik} methods and will be ignored.
\end{Argument}
\Paragraph{VALUE}
the negative of the logarithm of the determinant (or sum of
the logarithms of the determinants) of the correlation matrix (or
matrices) represented by \Co{object}.
\Paragraph{SEE ALSO}
\Co{logDet.corStruct}
\need 15pt
\Paragraph{EXAMPLE}
\vspace{-16pt} 
\begin{Example}
cs1 <- corAR1(0.2)
cs1 <- initialize(cs1, data = Orthodont)
logLik(cs1)
\end{Example}
\end{Helpfile}
\begin{Helpfile}{logLik.gls}{Log-Likelihood of a gls Object}
If \Co{REML=FALSE}, returns the log-likelihood value of the linear
model represented by \Co{object} evaluated at the estimated
coefficients; else, the restricted log-likelihood evaluated at the
estimated coefficients is returned.
\begin{Example}
logLik(object, REML)
\end{Example}
\begin{Argument}{ARGUMENTS}
\item[\Co{object:}]
an object inheriting from class \Co{gls}, representing
a generalized least squares fitted linear model.
\item[\Co{REML:}]
an optional logical value. If \Co{TRUE} the restricted
log-likelihood is returned, else, if \Co{FALSE}, the log-likelihood
is returned. Defaults to \Co{FALSE}. 
\end{Argument}
\Paragraph{VALUE}
the (restricted) log-likelihood of the linear model represented by
\Co{object} evaluated at the estimated coefficients.
\Paragraph{REFERENCES}
Harville, D.A. (1974) "Bayesian Inference for Variance Components
Using Only Error Contrasts", Biometrika, 61, 383-385.
\Paragraph{SEE ALSO}
\Co{gls}
\need 15pt
\Paragraph{EXAMPLE}
\vspace{-16pt}
\begin{Example}
fm1 <- gls(follicles {\Twiddle} sin(2*pi*Time) + cos(2*pi*Time), Ovary,
           correlation = corAR1(form = {\Twiddle} 1 | Mare))
logLik(fm1)
logLik(fm1, REML = FALSE)
\end{Example}
\end{Helpfile}
\begin{Helpfile}{logLik.glsStruct}{Log-Likelihood of a glsStruct Object}
\Co{Pars} is used to update the coefficients of the model components
of \Co{object} and the individual (restricted) log-likelihood
contributions of each component are added together. The type of
log-likelihood (restricted or not) is determined by the
\Co{settings} attribute of \Co{object}.
\begin{Example}
logLik(object, Pars, conLin)
\end{Example}
\begin{Argument}{ARGUMENTS}
\item[\Co{object:}]
an object inheriting from class \Co{glsStruct},
representing a list of linear model components, such as
\Co{corStruct} and \Co{varFunc} objects.
\item[\Co{Pars:}]
the parameter values at which the (restricted)
log-likelihood is to be evaluated.
\item[\Co{conLin:}]
an optional condensed linear model object, consisting of
a list with components \Co{"Xy"}, corresponding to a regression
matrix (\Co{X}) combined with a response vector (\Co{y}), and 
\Co{"logLik"}, corresponding to the log-likelihood of the
underlying linear model. Defaults to \Co{attr(object, "conLin")}.
\end{Argument}
\Paragraph{VALUE}
the (restricted) log-likelihood for the linear model described by
\Co{object}, evaluated at \Co{Pars}.
\Paragraph{SEE ALSO}
\Co{gls}, \Co{glsStruct}
\end{Helpfile}
\begin{Helpfile}{logLik.gnls}{Log-Likelihood of a gnls Object}
Returns the log-likelihood value of the nonlinear model represented by
\Co{object} evaluated at the estimated coefficients.
\begin{Example}
logLik(object)
\end{Example}
\begin{Argument}{ARGUMENTS}
\item[\Co{object:}]
an object inheriting from class \Co{gnls}, representing
a generalized nonlinear least squares fitted model.
\end{Argument}
\Paragraph{VALUE}
the log-likelihood of the linear model represented by
\Co{object} evaluated at the estimated coefficients.
\Paragraph{SEE ALSO}
\Co{gnls}
\need 15pt
\Paragraph{EXAMPLE}
\vspace{-16pt}
\begin{Example}
fm1 <- gnls(weight {\Twiddle} SSlogis(Time, Asym, xmid, scal), Soybean,
            weights = varPower())
logLik(fm1)
\end{Example}
\end{Helpfile}
\begin{Helpfile}{logLik.gnlsStruct}{Log-Likelihood of a gnlsStruct Object}
\Co{Pars} is used to update the coefficients of the model components
of \Co{object} and the individual log-likelihood
contributions of each component are added together.
\begin{Example}
logLik(object, Pars, conLin)
\end{Example}
\begin{Argument}{ARGUMENTS}
\item[\Co{object:}]
an object inheriting from class \Co{gnlsStruct},
representing a list of model components, such as
\Co{corStruct} and \Co{varFunc} objects, and attributes
specifying the underlying nonlinear model and the response variable.
\item[\Co{Pars:}]
the parameter values at which the log-likelihood is to be
evaluated.
\item[\Co{conLin:}]
an optional condensed linear model object, consisting of
a list with components \Co{"Xy"}, corresponding to a regression
matrix (\Co{X}) combined with a response vector (\Co{y}), and 
\Co{"logLik"}, corresponding to the log-likelihood of the
underlying nonlinear model. Defaults to \Co{attr(object,"conLin")}.
\end{Argument}
\Paragraph{VALUE}
the log-likelihood for the linear model described by \Co{object},
evaluated at \Co{Pars}.
\Paragraph{SEE ALSO}
\Co{gnls}, \Co{gnlsStruct}
\end{Helpfile}
\begin{Helpfile}{logLik.lm}{lm Log-Likelihood}
If \Co{REML=FALSE}, returns the log-likelihood value of the linear
model represented by \Co{object} evaluated at the estimated
coefficients; else, the restricted log-likelihood evaluated at the
estimated coefficients is returned.
\begin{Example}
logLik(object, REML)
\end{Example}
\begin{Argument}{ARGUMENTS}
\item[\Co{object:}]
an object inheriting from class \Co{lm}.
\item[\Co{REML:}]
an optional logical value. If \Co{TRUE} the restricted
log-likelihood is returned, else, if \Co{FALSE}, the log-likelihood
is returned. Defaults to \Co{FALSE}.
\end{Argument}
\Paragraph{VALUE}
the (restricted) log-likelihood of the linear model represented by
\Co{object} evaluated at the estimated coefficients.
\Paragraph{REFERENCES}
Harville, D.A. (1974) "Bayesian Inference for Variance Components
Using Only Error Contrasts", Biometrika, 61, 383-385.
\Paragraph{SEE ALSO}
\Co{lm}
\need 15pt
\Paragraph{EXAMPLE}
\vspace{-16pt} 
\begin{Example}
fm1 <- lm(distance \Twiddle Sex * age, Orthodont)
logLik(fm1)
logLik(fm1, REML = TRUE)
\end{Example}
\end{Helpfile}
\begin{Helpfile}{logLik.lmList}{Log-Likelihood of an lmList Object}
If \Co{pool=FALSE}, the (restricted) log-likelihoods of the \Co{lm}
components of \Co{object} are summed together. Else, the (restricted)
log-likelihood of the \Co{lm} fit with different coefficients for
each level of the grouping factor associated with the partitioning of
the \Co{object} components is obtained.
\begin{Example}
logLik(object, REML, pool)
\end{Example}
\begin{Argument}{ARGUMENTS}
\item[\Co{object:}]
an object inheriting from class \Co{lmList}, representing
a list of \Co{lm} objects with a common model.
\item[\Co{REML:}]
an optional logical value. If \Co{TRUE} the restricted
log-likelihood is returned, else, if \Co{FALSE}, the log-likelihood
is returned. Defaults to \Co{FALSE}.
\item[\Co{pool:}]
an optional logical value indicating whether all \Co{lm}
components of \Co{object} may be assumed to have the same error
variance. Default is \Co{attr(object, "pool")}.
\end{Argument}
\Paragraph{VALUE}
either the sum of the (restricted) log-likelihoods of each \Co{lm}
component in \Co{object}, or the (restricted) log-likelihood for the
\Co{lm} fit with separate coefficients for each component of
\Co{object}.
\Paragraph{SEE ALSO}
\Co{lmList}
\need 15pt
\Paragraph{EXAMPLE}
\vspace{-16pt}
\begin{Example}
fm1 <- lmList(distance {\Twiddle} age | Subject, Orthodont)
logLik(fm1)
\end{Example}
\end{Helpfile}
\begin{Helpfile}{logLik.lme}{lme Log-Likelihood}
If \Co{REML=FALSE}, returns the log-likelihood value of the linear
mixed-effects model represented by \Co{object} evaluated at the
estimated  coefficients; else, the restricted log-likelihood evaluated
at the estimated coefficients is returned.
\begin{Example}
logLik(object, REML)
\end{Example}
\begin{Argument}{ARGUMENTS}
\item[\Co{object:}]
an object inheriting from class \Co{lme}, representing
a fitted linear mixed-effects model.
\item[\Co{REML:}]
an optional logical value. If \Co{TRUE} the restricted
log-likelihood is returned, else, if \Co{FALSE}, the log-likelihood
is returned. Defaults to \Co{FALSE}. 
\end{Argument}
\Paragraph{VALUE}
the (restricted) log-likelihood of the linear mixed-effects model
represented by \Co{object} evaluated at the estimated coefficients.
\Paragraph{REFERENCES}
Harville, D.A. (1974) "Bayesian Inference for Variance Components
Using Only Error Contrasts", Biometrika, 61, 383-385.
\Paragraph{SEE ALSO}
\Co{lme}
\need 15pt
\Paragraph{EXAMPLE}
\vspace{-16pt} 
\begin{Example}
fm1 <- lme(distance \Twiddle  Sex * age, Orthodont, random = \Twiddle age)
logLik(fm1)
logLik(fm1, REML = TRUE)
\end{Example}
\end{Helpfile}
\begin{Helpfile}{logLik.lmeStruct}{lmeStruct Log-Likelihood}
\Co{Pars} is used to update the coefficients of the model components
of \Co{object} and the individual (restricted) log-likelihood
contributions of each component are added together. The type of
log-likelihood (restricted or not) is determined by the
\Co{settings} attribute of \Co{object}.
\begin{Example}
logLik(object, Pars, conLin)
\end{Example}
\begin{Argument}{ARGUMENTS}
\item[\Co{object:}]
an object inheriting from class \Co{lmeStruct},
representing a list of linear mixed-effects model components, such as
\Co{reStruct}, \Co{corStruct}, and \Co{varFunc} objects.
\item[\Co{Pars:}]
the parameter values at which the (restricted)
log-likelihood is to be evaluated.
\item[\Co{conLin:}]
an optional condensed linear model object, consisting of
a list with components \Co{"Xy"}, corresponding to a regression
matrix (\Co{X}) combined with a response vector (\Co{y}), and 
\Co{"logLik"}, corresponding to the log-likelihood of the
underlying lme model. Defaults to \Co{attr(object, "conLin")}.
\end{Argument}
\Paragraph{VALUE}
the (restricted) log-likelihood for the linear mixed-effects model
described by \Co{object}, evaluated at \Co{Pars}.
\Paragraph{SEE ALSO}
\Co{lme}, \Co{lmeStruct}
\end{Helpfile}
\begin{Helpfile}{logLik.reStruct}{Calculate reStruct Log-Likelihood}
Calculates the log-likelihood, or restricted log-likelihood, of the
Gaussian linear mixed-effects model represented by \Co{object} and
\Co{conLin} (assuming spherical within-group covariance structure),
evaluated at \Co{coef(object)}. The \Co{settings} attribute of
\Co{object} determines whether the log-likelihood, or the restricted
log-likelihood, is to be calculated. The computational methods are 
described in Bates and Pinheiro (1998).
\begin{Example}
logLik(object, conLin)
\end{Example}
\begin{Argument}{ARGUMENTS}
\item[\Co{object:}]
an object inheriting from class \Co{reStruct},
representing a random effects structure and consisting of a list of
\Co{pdMat} objects.
\item[\Co{conLin:}]
a condensed linear model object, consisting of a list
with components \Co{"Xy"}, corresponding to a regression matrix
(\Co{X}) combined with a response vector (\Co{y}), and
\Co{"logLik"}, corresponding to the log-likelihood of the
underlying model.
\end{Argument}
\Paragraph{VALUE}
the log-likelihood, or restricted log-likelihood, of linear
mixed-effects model represented by \Co{object} and \Co{conLin},
evaluated at \Co{coef(object)}.
\Paragraph{REFERENCES}
Bates, D.M. and Pinheiro, J.C. (1998) "Computational methods for
multilevel models" available in PostScript or PDF formats at
http://nlme.stat.wisc.edu
\Paragraph{SEE ALSO}
\Co{reStruct}, \Co{pdMat}
\end{Helpfile}
\begin{Helpfile}{logLik.varFunc}{varFunc Log-Likelihood}
This method function extracts the component of a Gaussian
log-likelihood associated with the variance function structure
represented by \Co{object}, which is equal to the sum of the
logarithms of the corresponding weights.
\begin{Example}
logLik(object, data)
\end{Example}
\begin{Argument}{ARGUMENTS}
\item[\Co{object:}]
an object inheriting from class \Co{varFunc},
representing a variance function structure.
\item[\Co{data:}]
this argument is included to make this method function
compatible with other \Co{logLik} methods and will be ignored.
\end{Argument}
\Paragraph{VALUE}
the sum of the logarithms of the weights corresponding to the variance
function structure represented by \Co{object}.
\need 15pt
\Paragraph{EXAMPLE}
\vspace{-16pt} 
\begin{Example}
vf1 <- varPower(form = \Twiddle age)
vf1 <- initialize(vf1, Orthodont)
coef(vf1) <- 0.1
logLik(vf1)
\end{Example}
\end{Helpfile}
\begin{Helpfile}{matrix\Co{<-}}{Assign Matrix Values}
This function is generic; method functions can be written to handle
specific classes of objects. Classes which already have methods for
this function include \Co{pdMat}, \Co{pdBlocked}, and 
\Co{reStruct}.
\begin{Example}
matrix(object) <- value
\end{Example}
\begin{Argument}{ARGUMENTS}
\item[\Co{object:}]
any object to which \Co{as.matrix} can be applied.
\item[\Co{value:}]
a matrix, or list of matrices, with the same dimensions as
\Co{as.matrix(object)} with the new values to be assigned to the
matrix associated with \Co{object}.
\end{Argument}
\Paragraph{VALUE}
will depend on the method function; see the appropriate documentation.
\Paragraph{SEE ALSO}
\Co{as.matrix}
\need 15pt
\Paragraph{EXAMPLE}
\vspace{-16pt} 
\begin{Example}
## see the method function documentation
\end{Example}
\end{Helpfile}
\begin{Helpfile}{matrix\Co{<-}.pdMat}{Assign Matrix to a pdMat Object}
The positive-definite matrix represented by \Co{object} is replaced
by \Co{value}. If the original matrix had row and/or column names,
the corresponding names for \Co{value} can either be \Co{NULL}, or
a permutation of the original names.
\begin{Example}
matrix(object) <- value
\end{Example}
\begin{Argument}{ARGUMENTS}
\item[\Co{object:}]
an object inheriting from class \Co{pdMat}, representing
a positive definite matrix.
\item[\Co{value:}]
a matrix with the new values to be assigned to the
positive-definite matrix represented by \Co{object}. Must have the
same dimensions as \Co{as.matrix(object)}.
\end{Argument}
\Paragraph{VALUE}
a \Co{pdMat} object similar to \Co{object}, but with its
coefficients modified to  produce the matrix in \Co{value}.
\Paragraph{SEE ALSO}
\Co{pdMat}
\need 15pt
\Paragraph{EXAMPLE}
\vspace{-16pt} 
\begin{Example}
pd1 <- pdSymm(diag(3))
matrix(pd1) <- diag(1:3)
pd1
\end{Example}
\end{Helpfile}
\begin{Helpfile}{matrix\Co{<-}.reStruct}{Assign reStruct Matrices}
The individual matrices in \Co{value} are assigned to each
\Co{pdMat} component of \Co{object}, in the they are listed. The
new matrices must have the same dimensions as the matrices they are
meant to replace.
\begin{Example}
matrix(object) <-  value
\end{Example}
\begin{Argument}{ARGUMENTS}
\item[\Co{object:}]
an object inheriting from class \Co{reStruct},
representing a random effects structure and consisting of a list of
\Co{pdMat} objects.
\item[\Co{value:}]
a matrix, or list of matrices, with the new values to be
assigned to the matrices associated with the \Co{pdMat} components
of \Co{object}.
\end{Argument}
\Paragraph{VALUE}
an \Co{reStruct} object similar to \Co{object}, but with the
coefficients of the individual \Co{pdMat} components modified to
produce the matrices listed in \Co{value}.
\Paragraph{SEE ALSO}
\Co{reStruct}, \Co{pdMat}
\need 15pt
\Paragraph{EXAMPLE}
\vspace{-16pt} 
\begin{Example}
rs1 <- reStruct(list(Dog = \Twiddle day, Side = \Twiddle 1), data = Pixel)
matrix(rs1) <- list(diag(2), 3)
\end{Example}
\end{Helpfile}
\begin{Helpfile}{model.matrix.reStruct}{reStruct Model Matrix}
The model matrices for each element of \Co{formula(object)},
calculated using \Co{data}, are bound together column-wise. When
multiple grouping levels are present (i.e.\ when \Co{length(object)>1}),
 the individual model matrices are combined from innermost (at
the leftmost position) to outermost (at the rightmost position).
\begin{Example}
model.matrix(object, data, contr)
\end{Example}
\begin{Argument}{ARGUMENTS}
\item[\Co{object:}]
an object inheriting from class \Co{reStruct},
representing a random effects structure and consisting of a list of
\Co{pdMat} objects.
\item[\Co{data:}]
a data frame in which to evaluate the variables defined in
\Co{formula(object)}.
\item[\Co{contr:}]
an optional named list specifying the contrasts to be used
for representing the \Co{factor} variables in \Co{data}. The
components names should match the names of the variables in
\Co{data} for which the contrasts are to be specified. The
components of this list will be used as the \Co{contrasts}
attribute of the corresponding factor. If missing, the default
contrast specification is used.
\end{Argument}
\Paragraph{VALUE}
a matrix obtained by binding together, column-wise, the model matrices
for each element of \Co{formula(object)}.
\Paragraph{SEE ALSO}
\Co{model.matrix}, \Co{contrasts},
\Co{reStruct}, \Co{formula.reStruct}
\need 15pt
\Paragraph{EXAMPLE}
\vspace{-16pt} 
\begin{Example}
rs1 <- reStruct(list(Dog = \Twiddle day, Side = \Twiddle 1), data = Pixel)
model.matrix(rs1, Pixel)
\end{Example}
\end{Helpfile}
\begin{Helpfile}{Names}{Names Associated with an Object}
This function is generic; method functions can be written to handle
specific classes of objects. Classes which already have methods for
this function include: \Co{formula}, \Co{modelStruct},
\Co{pdBlocked}, \Co{pdMat}, and \Co{reStruct}.
\begin{Example}
Names(object, ...)
Names(object, ...) <- value
\end{Example}
\begin{Argument}{ARGUMENTS}
\item[\Co{object:}]
any object for which names can be extracted and/or assigned.
\item[\Co{...:}]
some methods for this generic function require additional
arguments.
\end{Argument}
\Paragraph{VALUE}
will depend on the method function used; see the appropriate documentation.
\Paragraph{SIDE EFFECTS}
On the left side of an assignment, sets the names associated with
\Co{object} to \Co{value}, which must have an appropriate length.
\Paragraph{NOTE} If \Co{names} were generic, there would be no need for this generic
function.
\Paragraph{SEE ALSO}
\Co{Names.formula}, \Co{Names.pdMat}
\need 15pt
\Paragraph{EXAMPLE}
\vspace{-16pt} 
\begin{Example}
## see the method function documentation
\end{Example}
\end{Helpfile}
\begin{Helpfile}{Names.formula}{Extract Names from a formula}
This method function returns the names of the terms corresponding to
the right hand side of \Co{object} (treated as a linear formula),
obtained as the column names of the corresponding
\Co{model.matrix}.
\begin{Example}
Names(object, data, exclude)
\end{Example}
\begin{Argument}{ARGUMENTS}
\item[\Co{object:}]
an object inheriting from class \Co{formula}.
\item[\Co{data:}]
an optional data frame containing the variables specified
in \Co{object}. By default the variables are taken from the
environment from which \Co{Names.formula} is called.
\item[\Co{exclude:}]
an optional character vector with names to be excluded
from the returned value. Default is \Co{c("pi",".")}.
\end{Argument}
\Paragraph{VALUE}
a character vector with the column names of the \Co{model.matrix}
corresponding to the right hand side of \Co{object} which are not
listed in \Co{excluded}.
\Paragraph{SEE ALSO}
\Co{model.matrix}, \Co{terms},
\Co{Names}
\need 15pt
\Paragraph{EXAMPLE}
\vspace{-16pt} 
\begin{Example}
Names(distance \Twiddle Sex * age, data = Orthodont)
\end{Example}
\end{Helpfile}
\begin{Helpfile}{Names.pdBlocked}{Names of a pdBlocked Object}
This method function extracts the first element of the \Co{Dimnames}
attribute, which contains the column names, for each block diagonal
element in the matrix represented by \Co{object}.
\begin{Example}
Names(object, asList)
\end{Example}
\begin{Argument}{ARGUMENTS}
\item[\Co{object:}]
an object inheriting from class \Co{pdBlocked}
representing a positive-definite matrix with block diagonal
structure
\item[\Co{asList:}]
a logical value. If \Co{TRUE} a \Co{list} with the
names for each block diagonal element is returned. If \Co{FALSE}
a character vector with all column names is returned. Defaults to
\Co{FALSE}.
\end{Argument}
\Paragraph{VALUE}
if \Co{asList} is \Co{FALSE}, a character vector with column names
of the matrix represented by \Co{object}; otherwise, if
\Co{asList} is \Co{TRUE}, a list with components given by the
column names of the individual block diagonal elements in the matrix
represented by \Co{object}.
\Paragraph{SEE ALSO}
\Co{Names}, \Co{Names.pdMat}
\need 15pt
\Paragraph{EXAMPLE}
\vspace{-16pt} 
\begin{Example}
pd1 <- pdBlocked(list(\Twiddle Sex - 1, \Twiddle age - 1), data = Orthodont)
Names(pd1)
\end{Example}
\end{Helpfile}
\begin{Helpfile}{Names.pdMat}{Names of a pdMat Object}
This method function returns the fist element of the \Co{Dimnames}
attribute of \Co{object}, which contains the column names of the
matrix represented by \Co{object}.
\begin{Example}
Names(object)
Names(object) <- value
\end{Example}
\begin{Argument}{ARGUMENTS}
\item[\Co{object:}]
an object inheriting from class \Co{pdMat},
representing a positive-definite matrix.
\item[\Co{value:}]
a character vector with the replacement values for the
column and row names of the matrix represented by \Co{object}. It
must have length equal to the dimension of the matrix
represented by \Co{object} and, if names have been previously
assigned to \Co{object}, it must correspond to a permutation of the
original names.
\end{Argument}
\Paragraph{VALUE}
if \Co{object} has a \Co{Dimnames} attribute then the first
element of this attribute is returned; otherwise \Co{NULL}.
\Paragraph{SIDE EFFECTS}
On the left side of an assignment, sets the \Co{Dimnames} attribute
of \Co{object} to \Co{list(value, value)}.
\Paragraph{SEE ALSO}
\Co{Names}, \Co{Names.pdBlocked}
\need 15pt
\Paragraph{EXAMPLE}
\vspace{-16pt} 
\begin{Example}
pd1 <- pdSymm(\Twiddle age, data = Orthodont)
Names(pd1)
\end{Example}
\end{Helpfile}
\begin{Helpfile}{Names.reStruct}{Names of an reStruct Object}
This method function extracts the column names of each of the
positive-definite matrices represented the \Co{pdMat}
elements of \Co{object}.
\begin{Example}
Names(object)
Names(object) <- value
\end{Example}
\begin{Argument}{ARGUMENTS}
\item[\Co{object:}]
an object inheriting from class \Co{reStruct},
representing a random effects structure and consisting of a list of
\Co{pdMat} objects.
\item[\Co{value:}]
a list of character vectors with the replacement values
for the names of the individual \Co{pdMat} objects that form
\Co{object}. It must have the same length as \Co{object}.
\end{Argument}
\Paragraph{VALUE}
a list containing the column names of each of the positive-definite
matrices represented by the \Co{pdMat} elements of \Co{object}.
\Paragraph{SIDE EFFECTS}
On the left side of an assignment, sets the \Co{Names} of the
\Co{pdMat} elements of \Co{object} to the corresponding element of
\Co{value}.
\Paragraph{SEE ALSO}
\Co{reStruct}, \Co{pdMat},
\Co{Names.pdMat}
\need 15pt
\Paragraph{EXAMPLE}
\vspace{-16pt} 
\begin{Example}
rs1 <- reStruct(list(Dog = \Twiddle day, Side = \Twiddle 1), data = Pixel)
Names(rs1)
\end{Example}
\end{Helpfile}
\begin{Helpfile}{needUpdate}{Check if Update is Needed}
This function is generic; method functions can be written to handle
specific classes of objects. By default, it tries to extract a
\Co{needUpdate} attribute of \Co{object}. If this is \Co{NULL}
or \Co{FALSE} it returns \Co{FALSE}; else it returns \Co{TRUE}.
Updating of objects usually takes place in iterative algorithms in
which auxiliary quantities associated with the object, and not being
optimized over, may change.
\begin{Example}
needUpdate(object)
\end{Example}
\begin{Argument}{ARGUMENTS}
\item[\Co{object:}]
any object
\end{Argument}
\Paragraph{VALUE}
a logical value indicating whether \Co{object} needs to be updated.
\need 15pt
\Paragraph{EXAMPLE}
\vspace{-16pt} 
\begin{Example}
vf1 <- varExp()
vf1 <- initialize(vf1, data = Orthodont)
needUpdate(vf1)
\end{Example}
\end{Helpfile}
\begin{Helpfile}{needUpdate.modelStruct}{Check modelStruct Updating}
This method function checks if any of the elements of \Co{object}
needs to be updated. Updating of objects usually takes place in
iterative algorithms in which auxiliary quantities associated with the
object, and not being optimized over, may change.
\begin{Example}
needUpdate(object)
\end{Example}
\begin{Argument}{ARGUMENTS}
\item[\Co{object:}]
an object inheriting from class \Co{modelStruct},
representing a list of model components, such as \Co{corStruct} and
\Co{varFunc} objects.
\end{Argument}
\Paragraph{VALUE}
a logical value indicating whether any element of \Co{object} needs
to be updated.
\need 15pt
\Paragraph{EXAMPLE}
\vspace{-16pt} 
\begin{Example}
lms1 <- lmeStruct(reStruct = reStruct(pdDiag(diag(2), \Twiddle age)),
   varStruct = varPower(form = \Twiddle age))
needUpdate(lms1)
\end{Example}
\end{Helpfile}
\begin{Helpfile}{nlme}{Nonlinear Mixed-Effects Models}
This generic function fits a nonlinear mixed-effects model in the
formulation described in Lindstrom and Bates (1990) but allowing for nested
random effects. The within-group errors are allowed to be correlated
and/or have unequal variances.
\begin{Example}
nlme(model, data, fixed, random, groups, start, correlation, 
     weights, subset, method, na.action, naPattern, control,
     verbose)
\end{Example}
\begin{Argument}{ARGUMENTS}
\item[\Co{model:}]
a nonlinear model formula, with the response on the left
of a \Co{{\Twiddle}} operator and an expression involving parameters and
covariates on the right, or an \Co{nlsList} object.  If
\Co{data} is given, all names used in the formula should be
defined as parameters or variables in the data frame. The method
function \Co{nlme.nlsList} is documented separately.
\item[\Co{fixed:}]
a two-sided linear formula of the form
\Co{f1+...+fn{\Twiddle}x1+...+xm}, or a list of two-sided formulas of the form
\Co{f1{\Twiddle}x1+...+xm}, with possibly different models for each fixed
effect. The \Co{f1,...,fn} represent fixed effects included on the
right hand side of \Co{model} and \Co{x1+...+xm} define a linear
model for these parameters (when the left hand side of the formula
contains several parameters, they all are assumed to follow the same
linear model, described by the right hand side expression). A
\Co{1} on the right hand side of the formula(s) indicates a single
fixed effects for the corresponding parameter(s).
\item[\Co{data:}]
an optional data frame containing the variables named in
\Co{model}, \Co{fixed}, \Co{random}, \Co{correlation},
\Co{weights}, \Co{subset}, and \Co{naPattern}.  By default the
variables are taken from the environment from which \Co{nlme} is
called.
\item[\Co{random:}]
optionally, any of the following: (i) a two-sided formula
of the form \Co{r1+...+rn{\Twiddle}x1+...+xm | g1/.../gQ}, with
\Co{r1,...,rn} representing random effects included on the right
hand side of \Co{model}, \Co{x1+...+xm} specifying the model for
these random effects and \Co{g1/.../gQ} the grouping structure
(\Co{Q} may be equal to 1, in which case no \Co{/} is
required). The random effects formula will be repeated 
for all levels of grouping, in the case of multiple levels of
grouping; (ii) a two-sided formula of the form
\Co{r1+...+rn{\Twiddle}x1+..+xm}, a list of two-sided formulas of the form
\Co{r1{\Twiddle}x1+...+xm}, with possibly different models for each random
effect, a \Co{pdMat} object with a two-sided formula, or list of
two-sided formulas (i.e. a non-\Co{NULL} value for
\Co{formula(random)}), or a list of pdMat objects with two-sided
formulas, or lists of two-sided formulas. In this case, the grouping
structure formula will be given in \Co{groups}, or derived from the
data used to to fit the nonlinear mixed-effects model, which should
inherit from class  \Co{groupedData}; (iii) a named list
of formulas, lists of formulas, or \Co{pdMat} objects as in (ii),
with the grouping factors as names. The order of nesting will be
assumed the same as the order of the order of the elements in the
list; (iv) an \Co{reStruct} object. See the documentation on
\Co{pdClasses} for a description of the available \Co{pdMat}
classes. Defaults to \Co{fixed}, 
resulting in all fixed effects having also random effects.
\item[\Co{groups:}]
an optional one-sided formula of the form \Co{{\Twiddle}g1}
(single level of nesting) or \Co{{\Twiddle}g1/.../gQ} (multiple levels of
nesting), specifying the partitions of the data over which the random
effects vary. \Co{g1,...,gQ} must evaluate to factors in
\Co{data}. The order of nesting, when multiple levels are present,
is taken from left to right (i.e. \Co{g1} is the first level,
\Co{g2} the second, etc.).
\item[\Co{start:}]
an optional numeric vector, or list of initial estimates
for the fixed effects and random effects. If declared as a numeric
vector, it is converted internally to a list with a single component
\Co{fixed}, given by the vector. The \Co{fixed} component
is required, unless the model function inherits from class
\Co{selfStart}, in which case initial values will be derived from a
call to \Co{nlsList}. The \Co{random} is optionally used to specify
initial values for the random effects and should consist of a matrix,
or a list of matrices with length equal to the number of grouping
levels. Each matrix should have as many rows as the number of groups
at the corresponding level and as many columns as the number of
random effects in that level.
\item[\Co{correlation:}]
an optional \Co{corStruct} object describing the
within-group correlation structure. See the documentation of
\Co{corClasses} for a description of the available \Co{corStruct}
classes. Defaults to \Co{NULL}, corresponding to no within-group
correlations.
\item[\Co{weights:}]
an optional \Co{varFunc} object or one-sided formula
describing the within-group heteroscedasticity structure. If given as
a formula, it is used as the argument to \Co{varFixed},
corresponding to fixed variance weights. See the documentation on
\Co{varClasses} for a description of the available \Co{varFunc}
classes. Defaults to \Co{NULL}, corresponding to homoscesdatic
within-group errors.
\item[\Co{subset:}]
an optional expression saying which subset of the rows of
\Co{data} should  be  used in the fit. This can be a logical
vector, or a numeric vector indicating which observation numbers are
to be included, or a  character  vector of the row names to be
included.  All observations are included by default.
\item[\Co{method:}]
a character string.  If \Co{"REML"} the model is fit by
maximizing the restricted log-likelihood.  If \Co{"ML"} the
log-likelihood is maximized.  Defaults to \Co{"REML"}.
\item[\Co{na.action:}]
a function that indicates what should happen when the
data contain \Co{NA}s.  The default action (\Co{na.fail}) causes
\Co{nlme} to print an error message and terminate if there are any
incomplete observations.
\item[\Co{naPattern:}]
an expression or formula object, specifying which returned
values are to be regarded as missing.
\item[\Co{control:}]
a list of control values for the estimation algorithm to
replace the default values returned by the function \Co{nlmeControl}.
Defaults to an empty list.
\item[\Co{verbose:}]
an optional logical value. If \Co{TRUE} information on
the evolution of the iterative algorithm is printed. Default is
\Co{FALSE}.
\end{Argument}
\Paragraph{VALUE}
an object of class \Co{nlme} representing the nonlinear
mixed-effects model fit. Generic functions such as \Co{print},
\Co{plot} and \Co{summary} have methods to show the results of the
fit. See \Co{nlmeObject} for the components of the fit. The functions
\Co{resid}, \Co{coef}, \Co{fitted}, \Co{fixef}, and
\Co{ranef}  can be used to extract some of its components.
\Paragraph{REFERENCES}
The model formulation and computational methods are described in
Lindstrom, M.J. and Bates, D.M. (1990). The variance-covariance
parametrizations are described in Pinheiro, J.C. and Bates., D.M.
(1996).   The different correlation structures available for the
\Co{correlation} argument are described in Box, G.E.P., Jenkins,
G.M., and Reinsel G.C. (1994), Littel, R.C., Milliken, G.A., Stroup,
W.W., and Wolfinger, R.D. (1997), and Venables, W.N. and Ripley,
B.D. (1997). The use of variance functions for linear and nonlinear
mixed effects models is presented in detail in Davidian, M. and
Giltinan, D.M. (1995).  \\
Box, G.E.P., Jenkins, G.M., and Reinsel G.C. (1994) "Time Series
Analysis: Forecasting and Control", 3rd Edition, Holden-Day. \\
Davidian, M. and Giltinan, D.M. (1995) "Nonlinear Mixed Effects Models
for Repeated Measurement Data", Chapman and Hall.\\
Laird, N.M. and Ware, J.H. (1982) "Random-Effects Models for
Longitudinal Data", Biometrics, 38, 963-974.  \\
Littel, R.C., Milliken, G.A., Stroup, W.W., and Wolfinger, R.D. (1997)
"SAS Systems for Mixed Models", SAS Institute.\\
Lindstrom, M.J. and Bates, D.M. (1990) "Nonlinear Mixed Effects Models
for Repeated Measures Data", Biometrics, 46, 673-687.\\
Pinheiro, J.C. and Bates., D.M.  (1996) "Unconstrained
Parametrizations for Variance-Covariance Matrices", Statistics and
Computing, 6, 289-296.\\
Venables, W.N. and Ripley, B.D. (1997) "Modern Applied Statistics with
S-plus", 2nd Edition, Springer-Verlag.
\Paragraph{SEE ALSO}
\Co{nlmeControl}, \Co{nlme.nlsList},
\Co{nlmeObject}, \Co{nlsList},
\Co{reStruct}, \Co{pdClasses}, \Co{corClasses}, \Co{varClasses}
\need 15pt
\Paragraph{EXAMPLE}
\vspace{-16pt}
\begin{Example}
## all parameters as fixed and random effects
fm1 <- nlme(weight {\Twiddle} SSlogis(Time, Asym, xmid, scal), 
            data = Soybean, fixed = Asym + xmid + scal {\Twiddle} 1, 
            start = c(18, 52, 7.5))
## only Asym and xmid as random, with a diagonal covariance 
fm2 <- nlme(weight {\Twiddle} SSlogis(Time, Asym, xmid, scal), 
            data = Soybean, fixed = Asym + xmid + scal {\Twiddle} 1, 
            random = pdDiag(Asym + xmid {\Twiddle} 1),
            start = c(18, 52, 7.5))
\end{Example}
\end{Helpfile}
\begin{Helpfile}{nlme.nlsList}{NLME fit from nlsList Object}
If the random effects names defined in \Co{random} are a subset of
the \Co{lmList} object coefficient names, initial estimates for the
covariance matrix of the random effects are obtained (overwriting any
values given in \Co{random}). \Co{formula(fixed)} and the
\Co{data} argument in the calling sequence used to obtain
\Co{fixed} are passed as the \Co{fixed} and \Co{data} arguments
to \Co{nlme.formula}, together with any other additional arguments in
the function call. See the documentation on \Co{nlme.formula} for a
description of that function.
\begin{Example}
nlme(model, data, fixed, random, groups, start, correlation, weights,
     subset, method, na.action, naPattern, control, verbose)
\end{Example}
\begin{Argument}{ARGUMENTS}
\item[\Co{model:}]
an object inheriting from class \Co{nlsList},
representing a list of \Co{nls} fits with a common model.
\item[\Co{data:}]
this argument is included for consistency with the generic
function. It is ignored in this method function.
\item[\Co{random:}]
an optional one-sided linear formula with no conditioning
expression, or a \Co{pdMat} object with a \Co{formula}
attribute. Multiple levels of grouping are not allowed with this
method function.  Defaults to a formula consisting of the right hand
side of \Co{formula(fixed)}.
\item[\Co{other arguments:}]
identical to the arguments in the generic
function call. See the documentation on \Co{nlme}.
\end{Argument}
\Paragraph{VALUE}
an object of class \Co{nlme} representing the linear mixed-effects
model fit. Generic functions such as \Co{print}, \Co{plot} and
\Co{summary} have methods to show the results of the fit. See
\Co{nlmeObject} for the components of the fit. The functions
\Co{resid}, \Co{coef}, \Co{fitted}, \Co{fixef}, and
\Co{ranef}  can be used to extract some of its components.
\Paragraph{REFERENCES}
The computational methods are described in Bates, D.M. and Pinheiro, J.C.
(1998) and follow on the general framework of Lindstrom, M.J. and Bates,
D.M. (1988). The model formulation is described in Laird, N.M. and Ware,
J.H. (1982).  The variance-covariance parametrizations are described in
Pinheiro, J.C. and Bates., D.M.  (1996).   The different correlation
structures available for the \Co{correlation} argument are described
in Box, G.E.P., Jenkins, G.M., and Reinsel G.C. (1994), Littel, R.C.,
Milliken, G.A., Stroup, W.W., and Wolfinger, R.D. (1997), and Venables,
W.N. and Ripley, B.D. (1997). The use of variance functions for linear
and nonlinear mixed effects models is presented in detail in Davidian,
M. and Giltinan, D.M. (1995). \\
Bates, D.M. and Pinheiro, J.C. (1998) "Computational methods for
multilevel models" available in PostScript or PDF formats at
http://nlme.stat.wisc.edu\\
Box, G.E.P., Jenkins, G.M., and Reinsel G.C. (1994) "Time Series
Analysis: Forecasting and Control", 3rd Edition, Holden-Day. \\
Davidian, M. and Giltinan, D.M. (1995) "Nonlinear Mixed Effects Models
for Repeated Measurement Data", Chapman and Hall.\\
Laird, N.M. and Ware, J.H. (1982) "Random-Effects Models for
Longitudinal Data", Biometrics, 38, 963-974.  \\
Lindstrom, M.J. and Bates, D.M. (1988) "Newton-Raphson and EM
Algorithms for Linear Mixed-Effects Models for Repeated-Measures
Data", Journal of the American Statistical Association, 83,
1014-1022. \\
Littel, R.C., Milliken, G.A., Stroup, W.W., and Wolfinger, R.D. (1997)
"SAS Systems for Mixed Models", SAS Institute.\\
Pinheiro, J.C. and Bates., D.M.  (1996) "Unconstrained
Parametrizations for Variance-Covariance Matrices", Statistics and
Computing, 6, 289-296.\\
Venables, W.N. and Ripley, B.D. (1997) "Modern Applied Statistics with
S-plus", 2nd Edition, Springer-Verlag.
\Paragraph{SEE ALSO}
\Co{nlme}, \Co{lmList},
\Co{nlmeObject}
\need 15pt
\Paragraph{EXAMPLE}
\vspace{-16pt}
\begin{Example}
fm1 <- nlsList(weight {\Twiddle} SSlogis(Time, Asym, xmid, scal), Soybean)
fm2 <- nlme(fm1)
\end{Example}
\end{Helpfile}
\begin{Helpfile}{nlmeControl}{Control Values for nlme Fit}
The values supplied in the function call replace the defaults and a
list with all possible arguments is returned. The returned list is
used as the \Co{control} argument to the \Co{nlme} function.
\begin{Example}
nlmeControl(maxIter,pnlsMaxIter,msMaxIter,minScale,tolerance,
            niterEM,pnlsTol,msTol,msScale,returnObject,
            msVerbose,gradHess,apVar,.relStep,natural)
\end{Example}
\begin{Argument}{ARGUMENTS}
\item[\Co{maxIter:}]
maximum number of iterations for the \Co{nlme}
optimization algorithm. Default is 50.
\item[\Co{pnlsMaxIter:}]
maximum number of iterations
for the \Co{PNLS} optimization step inside the \Co{nlme}
optimization. Default is 7.
\item[\Co{msMaxIter:}]
maximum number of iterations
for the \Co{ms} optimization step inside the \Co{nlme}
optimization. Default is 50.
\item[\Co{minScale:}]
minimum factor by which to shrink the default step size
in an attempt to decrease the sum of squares in the \Co{PNLS} step.
Default 0.001.
\item[\Co{tolerance:}]
tolerance for the convergence criterion in the
\Co{nlme} algorithm. Default is 1e-6.
\item[\Co{niterEM:}]
number of iterations for the EM algorithm used to refine
the initial estimates of the random effects variance-covariance
coefficients. Default is 25.
\item[\Co{pnlsTol:}]
tolerance for the convergence criterion in \Co{PNLS}
step. Default is 1e-3.
\item[\Co{msTol:}]
tolerance for the convergence criterion in \Co{ms},
passed as the \Co{rel.tolerance} argument to the function (see
documentation on \Co{ms}). Default is 1e-7. 
\item[\Co{msScale:}]
scale function passed as the \Co{scale} argument to
the \Co{ms} function (see documentation on that function). Default
is \Co{lmeScale}.
\item[\Co{returnObject:}]
a logical value indicating whether the fitted
object should be returned when the maximum number of iterations is
reached without convergence of the algorithm. Default is
\Co{FALSE}.
\item[\Co{msVerbose:}]
a logical value passed as the \Co{trace} argument to
\Co{ms} (see documentation on that function). Default is
\Co{FALSE}.
\item[\Co{gradHess:}]
a logical value indicating whether numerical gradient
vectors and Hessian matrices of the log-likelihood function should
be used in the \Co{ms} optimization. This option is only available
when the correlation structure (\Co{corStruct}) and the variance
function structure (\Co{varFunc}) have no "varying" parameters and
the \Co{pdMat} classes used in the random effects structure are
\Co{pdSymm} (general positive-definite), \Co{pdDiag} (diagonal),
\Co{pdIdent} (multiple of the identity),  or
\Co{pdCompSymm} (compound symmetry). Default is \Co{TRUE}.
\item[\Co{apVar:}]
a logical value indicating whether the approximate
covariance matrix of the variance-covariance parameters should be
calculated. Default is \Co{TRUE}.
\item[\Co{.relStep:}]
relative step for numerical derivatives
calculations.
Default is \\ \Co{.Machine\$double.eps}$^{1/3}$.
\item[\Co{natural:}]
a logical value indicating whether the \Co{pdNatural}
parametrization should be used for general positive-definite matrices
(\Co{pdSymm}) in \Co{reStruct}, when the approximate covariance
matrix of the estimators is calculated. Default is \Co{TRUE}.
\end{Argument}
\Paragraph{VALUE}
a list with components for each of the possible arguments.
\Paragraph{SEE ALSO}
\Co{nlme}, \Co{ms}, \Co{nlmeScale}
\need 15pt
\Paragraph{EXAMPLE}
\vspace{-16pt}
\begin{Example}
# decrease the maximum number iterations in the ms call and
# request that information on the evolution of the ms iterations be printed
nlmeControl(msMaxIter = 20, msVerbose = TRUE)
\end{Example}
\end{Helpfile}
\begin{Helpfile}{nlmeObject}{Fitted nlme Object}
An object returned by the \Co{nlme} function, inheriting from class
\Co{nlme}, also inheriting from class \Co{lme}, and representing a
fitted nonlinear mixed-effects model. Objects of this class have
methods for the generic functions  \Co{anova}, \Co{coef},
\Co{fitted}, \Co{fixef}, \Co{formula}, \Co{getGroups},
\Co{getResponse}, \Co{intervals}, \Co{logLik}, \Co{pairs},
\Co{plot}, \Co{predict}, \Co{print}, \Co{ranef},
\Co{residuals}, \Co{summary}, and \Co{update}.
\Paragraph{VALUE}
The following components must be included in a legitimate \Co{nlme}
object. 
\begin{Argument}{COMPONENTS}
\item[\Co{apVar:}]
an approximate covariance matrix for the
variance-covariance coefficients. If \Co{apVar = FALSE} in the list
of control values used in the call to \Co{nlme}, this
component is equal to \Co{NULL}.
\item[\Co{call:}]
a list containing an image of the \Co{nlme} call that
produced the object.
\item[\Co{coefficients:}]
a list with two components, \Co{fixed} and
\Co{random}, where the first is a vector containing the estimated
fixed effects and the second is a list of matrices with the estimated
random effects for each level of grouping. For each matrix in the
\Co{random} list, the columns refer to the random effects and the
rows to the groups.
\item[\Co{contrasts:}]
a list with the contrasts used to represent factors
in the fixed effects formula  and/or random effects formula. This
information is important for making predictions from a new data
frame in which not all levels of the original factors are
observed. If no factors are used in the nlme model, this component
will be an empty list.
\item[\Co{dims:}]
a list with basic dimensions used in the nlme fit,
including the components \Co{N} - the number of observations in
the data, \Co{Q} - the number of grouping levels, \Co{qvec} -
the number of random effects at each level from innermost to
outermost (last two values are equal to zero and correspond to the
fixed effects and the response), \Co{ngrps} - the number of groups
at each level from innermost to outermost (last two values are one
and correspond to the fixed effects and the response), and
\Co{ncol} - the number of columns in the model matrix for each
level of grouping from innermost to outermost (last two values are
equal to the number of fixed effects and one).
\item[\Co{fitted:}]
a data frame with the fitted values as columns. The
leftmost column corresponds to the population fixed effects
(corresponding to the fixed effects only) and successive columns
from left to right correspond to increasing levels of grouping.
\item[\Co{fixDF:}]
a list with components \Co{X} and \Co{terms}
specifying the denominator degrees of freedom for, respectively,
t-tests for the individual fixed effects and F-tests for the
fixed-effects terms in the models.
\item[\Co{groups:}]
a data frame with the grouping factors as
columns. The grouping level increases from left to right.
\item[\Co{logLik:}]
the (restricted) log-likelihood at convergence.
\item[\Co{map:}]
a list with components \Co{fmap}, \Co{rmap},
\Co{rmapRel}, and \Co{bmap}, specifying various mappings for the
fixed and random effects, used to generate predictions from the
fitted object.
\item[\Co{method:}]
the estimation method: either \Co{"ML"} for maximum
likelihood, or \Co{"REML"} for restricted maximum likelihood.
\item[\Co{modelStruct:}]
an object inheriting from class \Co{nlmeStruct},
representing a list of mixed-effects model components, such
as \Co{reStruct}, \Co{corStruct}, and \Co{varFunc} objects.
\item[\Co{numIter:}]
the number of iterations used in the iterative
algorithm.
\item[\Co{residuals:}]
a data frame with the residuals as columns. The
leftmost column corresponds to the population residuals
and successive columns from left to right correspond to increasing
levels of grouping.
\item[\Co{sigma:}]
the estimated within-group error standard deviation.
\item[\Co{varFix:}]
an approximate covariance matrix of the
fixed effects estimates.
\end{Argument}
\Paragraph{SEE ALSO}
\Co{nlme}, \Co{nlmeStruct}
\end{Helpfile}
\begin{Helpfile}{nlmeStruct}{Nonlinear Mixed-Effects Structure}
A nonlinear mixed-effects structure is a list of model components
representing different sets of parameters in the nonlinear mixed-effects
model. An \Co{nlmeStruct} list must contain at least a
\Co{reStruct} object, but may also contain \Co{corStruct} and
\Co{varFunc} objects. \Co{NULL} arguments are not included in the
\Co{nlmeStruct} list.
\begin{Example}
nlmeStruct(reStruct, corStruct, varStruct)
\end{Example}
\begin{Argument}{ARGUMENTS}
\item[\Co{reStruct:}]
a \Co{reStruct} representing a random effects
structure.
\item[\Co{corStruct:}]
an optional \Co{corStruct} object, representing a
correlation structure. Default is \Co{NULL}.
\item[\Co{varStruct:}]
an optional \Co{varFunc} object, representing a
variance function structure. Default is \Co{NULL}.
\end{Argument}
\Paragraph{VALUE}
a list of model components determining the parameters to be estimated
for the associated nonlinear mixed-effects model.
\Paragraph{SEE ALSO}
\Co{nlme}, \Co{reStruct},
\Co{corClasses}, \Co{varClasses}
\need 15pt
\Paragraph{EXAMPLE}
\vspace{-16pt}
\begin{Example}
nlms1 <- nlmeStruct(reStruct({\Twiddle}age), corAR1(), varPower())
\end{Example}
\end{Helpfile}
\begin{Helpfile}{nlsList}{List of nls Objects with a Common Model}
\Co{Data} is partitioned according to the levels of the grouping
factor defined in \Co{model} and individual \Co{nls} fits are
obtained for each \Co{data} partition, using the model defined in
\Co{model}.
\begin{Example}
nlsList(model, data, start, control, level, na.action, pool)
\end{Example}
\begin{Argument}{ARGUMENTS}
\item[\Co{model:}]
either a nonlinear model formula, with the response on
the left of a \Co{{\Twiddle}} operator and an expression involving
parameters, covariates, and a grouping factor separated by the
\Co{|} operator on the right, or a \Co{selfStart} function.  The
method function \Co{nlsList.selfStart} is documented separately.
\item[\Co{data:}]
a data frame in which to interpret the variables named in
\Co{model}.
\item[\Co{start:}]
an optional named list with initial values for the
parameters to be estimated in \Co{model}. It is passed as the
\Co{start} argument to each \Co{nls} call and is required when
the nonlinear function in \Co{model} does not inherit from class
\Co{selfStart}.
\item[\Co{control:}]
a list of control values passed as the \Co{control}
argument to \Co{nls}. Defaults to an empty list.
\item[\Co{level:}]
an optional integer specifying the level of grouping to be used when 
multiple nested levels of grouping are present.
\item[\Co{na.action:}]
a function that indicates what should happen when the
data contain \Co{NA}s.  The default action (\Co{na.fail}) causes
\Co{nlsList} to print an error message and terminate if there are any
incomplete observations.
\item[\Co{pool:}]
an optional logical value that is preserved as an attribute of the
returned value.  This will be used as the default for \Co{pool} in
calculations of standard deviations or standard errors for summaries.
\end{Argument}
\Paragraph{VALUE}
a list of \Co{nls} objects with as many components as the number of
groups defined by the grouping factor. Generic functions such as
\Co{coef}, \Co{fixef}, \Co{lme}, \Co{pairs},
\Co{plot}, \Co{predict}, \Co{ranef}, \Co{summary},
and \Co{update} have methods that can be applied to an \Co{nlsList}
object.
\Paragraph{SEE ALSO}
\Co{nls}, \Co{nlme.nlsList}.
\need 15pt
\Paragraph{EXAMPLE}
\vspace{-16pt}
\begin{Example}
fm1 <- nlsList(weight {\Twiddle} SSlogis(Time, Asym, xmid, scal) | Plot, 
               Soybean)
\end{Example}
\end{Helpfile}
\begin{Helpfile}{nlsList.selfStart}{nlsList Fit from a selfStart Function}
The response variable and primary covariate in \Co{formula(data)}
are used together with \Co{model} to construct the nonlinear model
formula. This is used in the \Co{nls} calls and, because a
selfStarting model function can calculate initial estimates for its
parameters from the data, no starting estimates need to be provided.
\begin{Example}
nlsList(model, data, start, control, level, na.action, pool)
\end{Example}
\begin{Argument}{ARGUMENTS}
\item[\Co{model:}]
a \Co{selfStart} model function, which calculates
initial estimates for the model parameters from \Co{data}.
\item[\Co{data:}]
a data frame in which to interpret the variables in
\Co{model}. Because no grouping factor can be specified in
\Co{model}, \Co{data} must inherit from class
\Co{groupedData}. 
\item[\Co{other arguments:}]
identical to the arguments in the generic
function call. See the documentation on \Co{nlsList}.
\end{Argument}
\Paragraph{VALUE}
a list of \Co{nls} objects with as many components as the number of
groups defined by the grouping factor. A \Co{NULL} value is assigned
to the components corresponding to clusters for which the \Co{nls}
algorithm failed to converge. Generic functions such as \Co{coef},
\Co{fixef}, \Co{lme}, \Co{pairs}, \Co{plot},
\Co{predict}, \Co{ranef}, \Co{summary}, and
\Co{update} have methods that can be applied to an \Co{nlsList}
object.
\Paragraph{SEE ALSO}
\Co{selfStart}, \Co{groupedData}, \Co{nls},
\Co{nlme.nlsList}, \Co{nlsList.formula}
\need 15pt
\Paragraph{EXAMPLE}
\vspace{-16pt}
\begin{Example}
fm1 <- nlsList(SSlogis, Soybean)
\end{Example}
\end{Helpfile}
\begin{Helpfile}{NLSstClosestX}{Find the x with the closest y}
This function is used to determine the \Co{x} value in a
\Co{sortedXyData} object that corresponds to the \Co{y} value that
is closest to \Co{yval}.  The function is mostly used within the
\Co{initial} functions for a self-starting nonlinear regression
models, which are in the \Co{selfStart} class.
\begin{Example}
NLSstClosestX(xy, yval)
\end{Example}
\begin{Argument}{ARGUMENTS}
\item[\Co{xy:}]
a \Co{sortedXyData} object
\item[\Co{yval:}]
the numeric value on the \Co{y} scale to get close to
\end{Argument}
\Paragraph{VALUE}
A single numeric value which is one of the values of \Co{x} in the
\Co{xy} object.
\need 15pt
\Paragraph{EXAMPLE}
\vspace{-16pt}
\begin{Example}
DNase.2 <- DNase[ DNase\$Run == "2", ]
DN.srt <- sortedXyData( expression(log(conc)), expression(density), DNase.2 )
NLSstClosestX( DN.srt, 1.0 )
\end{Example}
\end{Helpfile}
\begin{Helpfile}{NLSstLfAsymptote}{Guess the horizontal asymptote on the left side}
This function provides an initial guess at the horizontal asymptote on
the left side (smaller values of \Co{x}) of the graph of \Co{y}
versus \Co{x} from the \Co{xy} object.  It is mostly used within the
\Co{initial} functions for a self-starting nonlinear regression
models, which are in the \Co{selfStart} class.
\begin{Example}
NLSstLfAsymptote(xy)
\end{Example}
\begin{Argument}{ARGUMENTS}
\item[\Co{xy:}]
a \Co{sortedXyData} object
\end{Argument}
\Paragraph{VALUE}
A single numeric value which is a guess at the \Co{y} value that
would be the asymptote for small \Co{x}.
\need 15pt
\Paragraph{EXAMPLE}
\vspace{-16pt}
\begin{Example}
DNase.2 <- DNase[ DNase\$Run == "2", ]
DN.srt <- sortedXyData( expression(log(conc)), expression(density), DNase.2 )
NLSstLfAsymptote( DN.srt )
\end{Example}
\end{Helpfile}
\begin{Helpfile}{NLSstRtAsymptote}{Guess the horizontal asymptote on the right side}
This function provides an initial guess at the horizontal asymptote on
the right side (larger values of \Co{x}) of the graph of \Co{y}
versus \Co{x} from the \Co{xy} object.  It is mostly used within the
\Co{initial} functions for a self-starting nonlinear regression
models, which are in the \Co{selfStart} class.
\begin{Example}
NLSstRtAsymptote(xy)
\end{Example}
\begin{Argument}{ARGUMENTS}
\item[\Co{xy:}]
a \Co{sortedXyData} object
\end{Argument}
\Paragraph{VALUE}
A single numeric value which is a guess at the \Co{y} value that
would be the asymptote for large \Co{x}.
\need 15pt
\Paragraph{EXAMPLE}
\vspace{-16pt}
\begin{Example}
DNase.2 <- DNase[ DNase\$Run == "2", ]
DN.srt <- sortedXyData( expression(log(conc)), expression(density), DNase.2 )
NLSstRtAsymptote( DN.srt )
\end{Example}
\end{Helpfile}
\begin{Helpfile}{pairs.compareFits}{Pairs Plot of compareFits Object}
Scatter plots of the values being compared are generated for each pair
of coefficients in \Co{object}. Different symbols (colors) are used
for each object being compared and values corresponding to the same
group are joined by a line, to facilitate comparison of fits. If only
two coefficients are present, the \Co{trellis} function
\Co{xyplot} is used; else the \Co{trellis} function \Co{splom}
is employed.
\begin{Example}
pairs(object, subset, key, ...)
\end{Example}
\begin{Argument}{ARGUMENTS}
\item[\Co{object:}]
an object of class \Co{compareFits}.
\item[\Co{subset:}]
an optional logical or integer vector specifying which
rows of \Co{object} should be used in the plots. If missing, all
rows are used.
\item[\Co{key:}]
an optional logical value, or list. If \Co{TRUE}, a legend
is included at the top of the plot indicating which symbols (colors)
correspond to which objects being compared. If \Co{FALSE}, no legend
is included. If given as a list, \Co{key} is passed down as an
argument to the \Co{trellis} function generating the plots
(\Co{splom} or  \Co{xyplot}). Defaults to \Co{TRUE}.
\item[\Co{...:}]
optional arguments passed down to the \Co{trellis}
function generating the plots.
\end{Argument}
\Paragraph{VALUE}
Pairwise scatter plots of the values being compared, with different
symbols (colors) used for each object under comparison.
\Paragraph{SEE ALSO}
\Co{compareFits},\Co{plot.compareFits},
\Co{xyplot}, \Co{splom}
\need 15pt
\Paragraph{EXAMPLE}
\vspace{-16pt} 
\begin{Example}
fm1 <- lmList(Orthodont)
fm2 <- lme(Orthodont)
pairs(compareFits(coef(fm1), coef(fm2)))
\end{Example}
\end{Helpfile}
\begin{Helpfile}{pairs.lmList}{Pairs Plot of an lmList Object}
Diagnostic plots for the linear model fits corresponding to the
\Co{object}  components  are obtained. The \Co{form} argument
gives considerable  flexibility in the type of plot specification. A
conditioning  expression (on the right side of a \Co{|} operator)
always implies  that different panels are used for  each level of the
conditioning  factor, according to a Trellis display. The expression
on the right  hand side of the formula, before a \Co{|} operator,
must evaluate to  a data frame with at least two columns. If the data
frame has two  columns, a scatter plot of the two variables is
displayed (the Trellis  function \Co{xyplot} is used). Otherwise, if
more than two columns  are present, a scatter plot matrix with
pairwise scatter plots of the  columns in the data frame is displayed
(the Trellis function  \Co{splom} is used).
\begin{Example}
pairs(object, form, label, id, idLabels, grid, ...)
\end{Example}
\begin{Argument}{ARGUMENTS}
\item[\Co{object:}]
an object inheriting from class \Co{lmList}, representing
a list of \Co{lm} objects with a common model.
\item[\Co{form:}]
an optional one-sided formula specifying the desired type of
plot. Any variable present in the original data frame used to obtain
\Co{object} can be referenced. In addition, \Co{object} itself
can be referenced in the formula using the symbol
\Co{"."}. Conditional expressions on the right of a \Co{|}
operator can be used to define separate panels in a Trellis
display. The expression on the right hand side of \Co{form}, and to
the left of the \Co{|} operator, must evaluate to a data frame with
at least two columns. Default is \Co{{\Twiddle} coef(.)}, corresponding to
a pairs plot of the coefficients of \Co{object}.
\item[\Co{id:}]
an optional numeric value, or one-sided formula. If given as
a value, it is used as a significance level for an outlier
test based on the Mahalanobis distances of the estimated random
effects. Groups with random effects distances greater than the
1-value percentile of the appropriate chi-square distribution
are identified in the plot using \Co{idLabels}. If given as a
one-sided formula, its right hand side must evaluate to a  logical,
integer, or character vector which is used to identify points in the
plot. If missing, no points are identified.
\item[\Co{idLabels:}]
an optional vector, or one-sided formula. If given as a
vector, it is converted to character and used to label the
points identified according to \Co{id}. If given as a
one-sided formula, its right hand side must evaluate to a vector
which is converted to character and used to label the identified
points. Default is the innermost grouping factor. 
\item[\Co{grid:}]
an optional logical value indicating whether a grid should
be added to plot. Default is \Co{FALSE}.
\item[\Co{...:}]
optional arguments passed to the Trellis plot function.
\end{Argument}
\Paragraph{VALUE}
a diagnostic Trellis plot.
\Paragraph{NOTE} This function requires the \Co{trellis} library.
\Paragraph{SEE ALSO}
\Co{lmList}, \Co{xyplot}, \Co{splom}
\need 15pt
\Paragraph{EXAMPLE}
\vspace{-16pt}
\begin{Example}
fm1 <- lmList(distance {\Twiddle} age | Subject, Orthodont)
# scatter plot of coefficients by gender, identifying 
# unusual subjects
pairs(fm1, {\Twiddle}coef(.)| Sex, id = 0.1, adj = -0.5)    
# scatter plot of estimated random effects
pairs(fm1, {\Twiddle}ranef(.))
\end{Example}
\end{Helpfile}
\begin{Helpfile}{pairs.lme}{Pairs Plot of an lme Object}
Diagnostic plots for the linear mixed-effects fit are obtained. The
\Co{form} argument gives considerable flexibility in the type of
plot specification. A conditioning expression (on the right side of a
\Co{|} operator) always implies that different panels are used for
each level of the conditioning factor, according to a Trellis
display. The expression on the right hand side of the formula, before
a \Co{|} operator, must evaluate to a data frame with at least two
columns. If the data frame has two columns, a scatter plot of the two
variables is displayed (the Trellis function \Co{xyplot} is
used). Otherwise, if more than two columns are present, a scatter plot
matrix with pairwise scatter plots of the columns in the data frame is
displayed (the Trellis function \Co{splom} is used).
\begin{Example}
pairs(object, form, label, id, idLabels, grid, ...)
\end{Example}
\begin{Argument}{ARGUMENTS}
\item[\Co{object:}]
an object inheriting from class \Co{lme}, representing
a fitted linear mixed-effects model.
\item[\Co{form:}]
an optional one-sided formula specifying the desired type of
plot. Any variable present in the original data frame used to obtain
\Co{object} can be referenced. In addition, \Co{object} itself
can be referenced in the formula using the symbol
\Co{"."}. Conditional expressions on the right of a \Co{|}
operator can be used to define separate panels in a Trellis
display. The expression on the right hand side of \Co{form}, and to
the left of the \Co{|} operator, must evaluate to a data frame with
at least two columns. Default is \Co{\Twiddle coef(.) }, corresponding to
a pairs plot of the coefficients evaluated at the innermost level of
nesting.
\item[\Co{id:}]
an optional numeric value, or one-sided formula. If given as
a value, it is used as a significance level for an outlier
test based on the Mahalanobis distances of the estimated random
effects. Groups with random effects distances greater than the
1-value percentile of the appropriate chi-square distribution
are identified in the plot using \Co{idLabels}. If given as a
one-sided formula, its right hand side must evaluate to a  logical,
integer, or character vector which is used to identify points in the
plot. If missing, no points are identified.
\item[\Co{idLabels:}]
an optional vector, or one-sided formula. If given as a
vector, it is converted to character and used to label the
points identified according to \Co{id}. If given as a
one-sided formula, its right hand side must evaluate to a vector
which is converted to character and used to label the identified
points. Default is the innermost grouping factor. 
\item[\Co{grid:}]
an optional logical value indicating whether a grid should
be added to plot. Default is \Co{FALSE}.
\item[\Co{...:}]
optional arguments passed to the Trellis plot function.
\end{Argument}
\Paragraph{VALUE}
a diagnostic Trellis plot.
\Paragraph{NOTE} This function requires the \Co{trellis} library.
\Paragraph{SEE ALSO}
\Co{lme}, \Co{xyplot}, \Co{splom}
\need 15pt
\Paragraph{EXAMPLE}
\vspace{-16pt} 
\begin{Example}
fm1 <- lme(distance \Twiddle age, Orthodont, random = \Twiddle age | Subject)
# scatter plot of coefficients by gender, identifying 
unusual subjects
pairs(fm1, \Twiddle coef(.) | Sex, id = 0.1, adj = -0.5)    
# scatter plot of estimated random effects
pairs(fm1, \Twiddle ranef(.))         
\end{Example}
\end{Helpfile}
\begin{Helpfile}{pdBlocked}{Positive-Definite Block Diagonal Matrix}
This function is a constructor for the \Co{pdBlocked} class,
representing a positive-definite block-diagonal matrix. Each
block-diagonal element of the underlying  matrix is itself a
positive-definite matrix and is represented internally as an
individual \Co{pdMat} object. When \Co{value} is
\Co{numeric(0)}, a list of uninitialized \Co{pdMat} objects, a
list of one-sided formulas, or a list of vectors 
of character strings,  \Co{object} is returned
as an uninitialized \Co{pdBlocked} object (with just some of its
attributes and its class defined) and needs to have its coefficients
assigned later, generally using the \Co{coef} or \Co{matrix}
functions. If \Co{value} is a list of  initialized \Co{pdMat}
objects, \Co{object} will be constructed from the list obtained by
applying \Co{as.matrix} to each of the \Co{pdMat} elements of
\Co{value}. Finally, if \Co{value} is a list of numeric vectors,
they are assumed to represent the unrestricted coefficients
of the block-diagonal elements of the  underlying positive-definite
matrix.
\begin{Example}
pdBlocked(value, form, nam, data, pdClass)
\end{Example}
\begin{Argument}{ARGUMENTS}
\item[\Co{value:}]
an optional list with elements to be used as the
\Co{value} argument to other \Co{pdMat} constructors. These
include: \Co{pdMat} objects, positive-definite
matrices, one-sided linear formulas, vectors of character strings, or
numeric vectors. All elements in the list must be similar (e.g. all
one-sided formulas, or all numeric vectors). Defaults to
\Co{numeric(0)}, corresponding to an uninitialized object.
\item[\Co{form:}]
an optional list of one-sided linear formula specifying the
row/column names for the block-diagonal elements of the matrix
represented by \Co{object}. Because factors may be present in
\Co{form}, the formulas needs to be evaluated on a data.frame to
resolve the names they defines. This argument is ignored when
\Co{value} is a list of one-sided formulas. Defaults to \Co{NULL}.
\item[\Co{nam:}]
an optional list of vector of character strings specifying the
row/column names for the block-diagonal elements of the matrix
represented by object. Each of its components must have  
length equal to the dimension of the corresponding block-diagonal
element and unreplicated elements. This argument is ignored when 
\Co{value} is a list of vector of character strings. Defaults to 
\Co{NULL}.
\item[\Co{data:}]
an optional data frame in which to evaluate the variables
named in \Co{value} and \Co{form}. It is used to
obtain the levels for \Co{factors}, which affect the
dimensions and the row/column names of the underlying matrix. If
\Co{NULL}, no attempt is made to obtain information on 
\Co{factors} appearing the random effects model. Defaults to parent
frame from which the function was called.
\item[\Co{pdClass:}]
an optional vector of character strings naming the
\Co{pdMat} classes to be assigned to the individual blocks in the
underlying matrix. If a single class is specified, it is used for all
block-diagonal elements. This argument will only be used when
\Co{value} is missing, or its elements are not \Co{pdMat}
objects. Defaults to \Co{"pdSymm"}.
\end{Argument}
\Paragraph{VALUE}
a \Co{pdBlocked} object representing a positive-definite
block-diagonal matrix, also inheriting from class \Co{pdMat}.
\Paragraph{SEE ALSO}
\Co{as.matrix.pdMat}, \Co{coef.pdMat},
\Co{matrix<-.pdMat}
\need 15pt
\Paragraph{EXAMPLE}
\vspace{-16pt} 
\begin{Example}
pd1 <- pdBlocked(list(diag(1:2), diag(c(0.1, 0.2, 0.3))),
                 nam = list(c("A","B"), c("a1", "a2", "a3")))
\end{Example}
\end{Helpfile}
\begin{Helpfile}{pdClasses}{Positive-Definite Matrix Classes}
Standard classes of positive-definite matrices (\Co{pdMat})
structures  available in the \Co{nlme} library.
\begin{Argument}{STANDARD CLASSES}
\item[\Co{pdSymm:}]
general positive-definite matrix, with no additional
structure
\item[\Co{pdDiag:}]
diagonal
\item[\Co{pdIdent:}]
multiple of an identity
\item[\Co{pdCompSymm:}]
compound symmetry structure (constant diagonal and
constant off-diagonal elements)
\item[\Co{pdBlocked:}]
block-diagonal matrix, with diagonal blocks of any
"atomic" \Co{pdMat} class
\item[\Co{pdNatural:}]
general positive-definite matrix in natural
parametrization (i.e.\ parametrized in terms of standard deviations
and correlations). The underlying coefficients are not unrestricted,
so this class should NOT be used for optimization.
\Paragraph{NOTE} Users may define their own \Co{pdMat} classes by specifying a
\Co{constructor} function and, at a minimum, methods for the
functions \Co{pdConstruct}, \Co{pdMatrix} and \Co{coef}. For
examples of these functions, see the methods for classes \Co{pdSymm}
and \Co{pdDiag}.
\end{Argument}
\Paragraph{SEE ALSO}
\Co{pdCompSymm}, \Co{pdDiag},
\Co{pdIdent}, \Co{pdNatural}, \Co{pdSymm}
\end{Helpfile}
\begin{Helpfile}{pdCompSymm}{Compound Symmetry PD Matrix}
This function is a constructor for the \Co{pdCompSymm} class,
representing a positive-definite matrix with compound symmetry
structure (constant diagonal and constant off-diagonal elements). The
underlying matrix is represented by 2 unrestricted parameters.
When \Co{value} is \Co{numeric(0)}, an unitialized \Co{pdMat}
object, a one-sided formula, or a vector of character strings,
\Co{object} is returned  as an uninitialized \Co{pdCompSymm}
object (with just some of its  attributes and its class defined) and
needs to have its coefficients assigned later, generally using the
\Co{coef} or \Co{matrix} functions. If \Co{value} is an initialized
\Co{pdMat} object, \Co{object} will be constructed from 
\Co{as.matrix(value)}. Finally, if \Co{value} is a numeric vector
of length 2, it is assumed to represent the unrestricted coefficients
of the underlying positive-definite matrix.
\begin{Example}
pdCompSymm(value, form, nam, data)
\end{Example}
\begin{Argument}{ARGUMENTS}
\item[\Co{value:}]
an optional initialization value, which can be any of the
following: a \Co{pdMat} object, a positive-definite
matrix, a one-sided linear formula (with variables separated by
\Co{+}), a vector of character strings, or a numeric
vector of length 2. Defaults to \Co{numeric(0)}, corresponding to
an uninitialized object.
\item[\Co{form:}]
an optional one-sided linear formula specifying the
row/column names for the matrix represented by \Co{object}. Because
factors may be present in \Co{form}, the formula needs to be
evaluated on a data.frame to resolve the names it defines. This
argument is ignored when \Co{value} is a one-sided
formula. Defaults to \Co{NULL}.
\item[\Co{nam:}]
an optional vector of character strings specifying the
row/column names for the matrix represented by object. It must have 
length equal to the dimension of the underlying positive-definite
matrix and unreplicated elements. This argument is ignored when
\Co{value} is a vector of character strings. Defaults to
\Co{NULL}.
\item[\Co{data:}]
an optional data frame in which to evaluate the variables
named in \Co{value} and \Co{form}. It is used to
obtain the levels for \Co{factors}, which affect the
dimensions and the row/column names of the underlying matrix. If
\Co{NULL}, no attempt is made to obtain information on 
\Co{factors} appearing the random effects model. Defaults to parent
frame from which the function was called.
\end{Argument}
\Paragraph{VALUE}
a \Co{pdCompSymm} object representing a positive-definite
matrix with compound symmetry structure, also inheriting from class
\Co{pdMat}.
\Paragraph{SEE ALSO}
\Co{as.matrix.pdMat}, \Co{coef.pdMat},
\Co{matrix<-.pdMat}
\need 15pt
\Paragraph{EXAMPLE}
\vspace{-16pt} 
\begin{Example}
pd1 <- pdCompSymm(diag(3) + 1, nam = c("A","B","C"))
pd1
\end{Example}
\end{Helpfile}
\begin{Helpfile}{pdConstruct}{Construct pdMat Objects}
This function is an alternative constructor for the \Co{pdMat}
class associated with \Co{object} and is mostly used internally in other
functions. See the documentation on the principal constructor
function, generally with the same name as the \Co{pdMat} class of
object.
\begin{Example}
pdConstruct(object, value, form, nam, data)
\end{Example}
\begin{Argument}{ARGUMENTS}
\item[\Co{object:}]
an object inheriting from class \Co{pdMat}, representing
a positive definite matrix.
\item[\Co{value:}]
an optional initialization value, which can be any of the
following: a \Co{pdMat} object, a positive-definite
matrix, a one-sided linear formula (with variables separated by
\Co{+}), a vector of character strings, or a numeric
vector. Defaults to \Co{numeric(0)}, corresponding to an
uninitialized object.
\item[\Co{form:}]
an optional one-sided linear formula specifying the
row/column names for the matrix represented by \Co{object}. Because
factors may be present in \Co{form}, the formula needs to be
evaluated on a data.frame to resolve the names it defines. This
argument is ignored when \Co{value} is a one-sided
formula. Defaults to \Co{NULL}.
\item[\Co{nam:}]
an optional vector of character strings specifying the
row/column names for the matrix represented by object. It must have 
length equal to the dimension of the underlying positive-definite
matrix and unreplicated elements. This argument is ignored when
\Co{value} is a vector of character strings. Defaults to
\Co{NULL}.
\item[\Co{data:}]
an optional data frame in which to evaluate the variables
named in \Co{value} and \Co{form}. It is used to
obtain the levels for \Co{factors}, which affect the
dimensions and the row/column names of the underlying matrix. If
\Co{NULL}, no attempt is made to obtain information on 
\Co{factors} appearing the random effects model. Defaults to parent
frame from which the function was called.
\end{Argument}
\Paragraph{VALUE}
a \Co{pdMat} object representing a positive-definite matrix,
inheriting from the same classes as \Co{object}.
\Paragraph{SEE ALSO}
\Co{pdCompSymm}, \Co{pdDiag},
\Co{pdIdent}, \Co{pdNatural}, \Co{pdSymm}
\need 15pt
\Paragraph{EXAMPLE}
\vspace{-16pt} 
\begin{Example}
pd1 <- pdSymm()
pdConstruct(pd1, diag(1:4))
\end{Example}
\end{Helpfile}
\begin{Helpfile}{pdConstruct.pdBlocked}{Construct pdBlocked}
This function give an alternative constructor for the \Co{pdBlocked}
class, representing a positive-definite block-diagonal matrix. Each
block-diagonal element of the underlying  matrix is itself a
positive-definite matrix and is represented internally as an
individual \Co{pdMat} object. When \Co{value} is
\Co{numeric(0)}, a list of uninitialized \Co{pdMat} objects, a
list of one-sided formulas, or a list of vectors of character strings,
\Co{object} is returned as an uninitialized \Co{pdBlocked} object
(with just some of its attributes and its class defined) and needs to
have its coefficients assigned later, generally using the \Co{coef}
or \Co{matrix} functions. If \Co{value} is a list of  initialized
\Co{pdMat} objects, \Co{object} will be constructed from the list
obtained by applying \Co{as.matrix} to each of the \Co{pdMat}
elements of \Co{value}. Finally, if \Co{value} is a list of
numeric vectors, they are assumed to represent the unrestricted
coefficients of the block-diagonal elements of the  underlying
positive-definite matrix.
\begin{Example}
pdConstruct(object, value, form, nam, data, pdClass)
\end{Example}
\begin{Argument}{ARGUMENTS}
\item[\Co{object:}]
an object inheriting from class \Co{pdBlocked},
representing a positive definite block-diagonal matrix.
\item[\Co{value:}]
an optional list with elements to be used as the
\Co{value} argument to other \Co{pdMat} constructors. These
include: \Co{pdMat} objects, positive-definite
matrices, one-sided linear formulas, vectors of character strings, or
numeric vectors. All elements in the list must be similar (e.g. all
one-sided formulas, or all numeric vectors). Defaults to
\Co{numeric(0)}, corresponding to an uninitialized object.
\item[\Co{form:}]
an optional list of one-sided linear formula specifying the
row/column names for the block-diagonal elements of the matrix
represented by \Co{object}. Because factors may be present in
\Co{form}, the formulas needs to be evaluated on a data.frame to
resolve the names they defines. This argument is ignored when
\Co{value} is a list of one-sided formulas. Defaults to \Co{NULL}.
\item[\Co{nam:}]
an optional list of vector of character strings specifying the
row/column names for the block-diagonal elements of the matrix
represented by object. Each of its components must have  
length equal to the dimension of the corresponding block-diagonal
element and unreplicated elements. This argument is ignored when 
\Co{value} is a list of vector of character strings. Defaults to 
\Co{NULL}.
\item[\Co{data:}]
an optional data frame in which to evaluate the variables
named in \Co{value} and \Co{form}. It is used to
obtain the levels for \Co{factors}, which affect the
dimensions and the row/column names of the underlying matrix. If
\Co{NULL}, no attempt is made to obtain information on 
\Co{factors} appearing the random effects model. Defaults to parent
frame from which the function was called.
\item[\Co{pdClass:}]
an optional vector of character strings naming the
\Co{pdMat} classes to be assigned to the individual blocks in the
underlying matrix. If a single class is specified, it is used for all
block-diagonal elements. This argument will only be used when
\Co{value} is missing, or its elements are not \Co{pdMat}
objects. Defaults to \Co{"pdSymm"}.
\end{Argument}
\Paragraph{VALUE}
a \Co{pdBlocked} object representing a positive-definite
block-diagonal matrix, also inheriting from class \Co{pdMat}.
\Paragraph{SEE ALSO}
\Co{as.matrix.pdMat}, \Co{coef.pdMat},
\Co{matrix<-.pdMat}
\need 15pt
\Paragraph{EXAMPLE}
\vspace{-16pt} 
\begin{Example}
pd1 <- pdBlocked(list(c("A","B"), c("a1", "a2", "a3")))
pdConstruct(pd1, list(diag(1:2), diag(c(0.1, 0.2, 0.3))))
\end{Example}
\end{Helpfile}
\begin{Helpfile}{pdDiag}{Diagonal Positive-Definite Matrix}
This function is a constructor for the \Co{pdDiag} class,
representing a diagonal positive-definite matrix. If the matrix
associated with \Co{object} is of dimension n, it is
represented by n unrestricted parameters, given by the logarithm
of the square-root of the diagonal values. When \Co{value} is
\Co{numeric(0)}, an uninitialized \Co{pdMat} object, a one-sided
formula, or a vector of character strings, \Co{object} is returned
as an uninitialized \Co{pdDiag} object (with just some of its
attributes and its class defined) and needs to have its coefficients
assigned later, generally using the \Co{coef} or \Co{matrix}
functions. If \Co{value} is an initialized \Co{pdMat} object,
\Co{object} will be constructed from
\Co{as.matrix(value)}. Finally, if \Co{value} is a numeric vector,
it is assumed to represent the unrestricted coefficients of the
underlying positive-definite
matrix.
\begin{Example}
pdDiag(value, form, nam, data)
\end{Example}
\begin{Argument}{ARGUMENTS}
\item[\Co{value:}]
an optional initialization value, which can be any of the
following: a \Co{pdMat} object, a positive-definite
matrix, a one-sided linear formula (with variables separated by
\Co{+}), a vector of character strings, or a numeric
vector of length equal to the dimension of the underlying
positive-definite matrix. Defaults to \Co{numeric(0)}, corresponding
to an uninitialized object.
\item[\Co{form:}]
an optional one-sided linear formula specifying the
row/column names for the matrix represented by \Co{object}. Because
factors may be present in \Co{form}, the formula needs to be
evaluated on a data.frame to resolve the names it defines. This
argument is ignored when \Co{value} is a one-sided
formula. Defaults to \Co{NULL}.
\item[\Co{nam:}]
an optional vector of character strings specifying the
row/column names for the matrix represented by object. It must have 
length equal to the dimension of the underlying positive-definite
matrix and unreplicated elements. This argument is ignored when
\Co{value} is a vector of character strings. Defaults to
\Co{NULL}.
\item[\Co{data:}]
an optional data frame in which to evaluate the variables
named in \Co{value} and \Co{form}. It is used to
obtain the levels for \Co{factors}, which affect the
dimensions and the row/column names of the underlying matrix. If
\Co{NULL}, no attempt is made to obtain information on 
\Co{factors} appearing the random effects model. Defaults to parent
frame from which the function was called.
\end{Argument}
\Paragraph{VALUE}
a \Co{pdDiag} object representing a diagonal positive-definite
matrix, also inheriting from class \Co{pdMat}.
\Paragraph{SEE ALSO}
\Co{as.matrix.pdMat}, \Co{coef.pdMat},
\Co{matrix<-.pdMat}
\need 15pt
\Paragraph{EXAMPLE}
\vspace{-16pt} 
\begin{Example}
pd1 <- pdDiag(diag(1:3), nam = c("A","B","C"))
pd1
\end{Example}
\end{Helpfile}
\begin{Helpfile}{pdFactor}{Square-Root Factor of a Positive-Definite Matrix}
A square-root factor of the positive-definite matrix represented by
\Co{object} is obtained. Letting $\bS$ denote a
positive-definite matrix, a square-root factor of $\bS$ is
any square matrix $\bL$ such that $\bS = \bL^{t}\bL$. This function
extracts $\bL$. 
\begin{Example}
pdFactor(object, ...)
\end{Example}
\begin{Argument}{ARGUMENTS}
\item[\Co{object:}]
an object inheriting from class \Co{pdMat}, representing
a positive definite matrix, which must have been initialized
(i.e.\ \Co{length(coef(object)) > 0}).
\item[\Co{...:}]
some methods for the generic function may require additional
arguments.
\end{Argument}
\Paragraph{VALUE}
a vector with a square-root factor of the positive-definite matrix
associated with \Co{object} stacked column-wise.
\Paragraph{NOTE} This function is used intensively in optimization
algorithms and its value is returned as a vector for efficiency
reasons. The \Co{pdMatrix} function can be used to obtain
square-root factors in matrix form.
\Paragraph{SEE ALSO}
\Co{pdMatrix}
\need 15pt
\Paragraph{EXAMPLE}
\vspace{-16pt} 
\begin{Example}
pd1 <- pdCompSymm(4 * diag(3) + 1)
pdFactor(pd1)
\end{Example}
\end{Helpfile}
\begin{Helpfile}{pdFactor.reStruct}{reStruct Factors}
This method function extracts square-root factors of the
positive-definite matrices corresponding to the \Co{pdMat} elements
of \Co{object}.
\begin{Example}
pdFactor(object)
\end{Example}
\begin{Argument}{ARGUMENTS}
\item[\Co{object:}]
an object inheriting from class \Co{reStruct},
representing a random effects structure and consisting of a list of
\Co{pdMat} objects.
\end{Argument}
\Paragraph{VALUE}
a vector with square-root factors of the positive-definite matrices
corresponding to the elements of \Co{object} stacked column-wise.
\Paragraph{NOTE} This function is used intensively in optimization
algorithms and its value is returned as a vector for efficiency
reasons. The \Co{pdMatrix} function can be used to obtain
square-root factors in matrix form.
\Paragraph{SEE ALSO}
\Co{pdMatrix.reStruct}, \Co{pdFactor.pdMat}
\need 15pt
\Paragraph{EXAMPLE}
\vspace{-16pt} 
\begin{Example}
rs1 <- reStruct(pdSymm(diag(3), form=\Twiddle Sex+age, data=Orthodont))
pdFactor(rs1)
\end{Example}
\end{Helpfile}
\begin{Helpfile}{pdIdent}{Multiple of an Identity Positive-Definite Matrix}
This function is a constructor for the \Co{pdIdent} class,
representing a multiple of the identity positive-definite matrix. 
The matrix associated with \Co{object} is represented by 1
unrestricted parameter, given by the logarithm of the square-root of
the diagonal value. When \Co{value} is 
\Co{numeric(0)}, an uninitialized \Co{pdMat} object, a one-sided
formula, or a vector of character strings, \Co{object} is returned
as an uninitialized \Co{pdIdent} object (with just some of its
attributes and its class defined) and needs to have its coefficients
assigned later, generally using the \Co{coef} or \Co{matrix}
functions. If \Co{value} is an initialized \Co{pdMat} object,
\Co{object} will be constructed from
\Co{as.matrix(value)}. Finally, if \Co{value} is a numeric value,
it is assumed to represent the unrestricted coefficient of the
underlying positive-definite  matrix.
\begin{Example}
pdIdent(value, form, nam, data)
\end{Example}
\begin{Argument}{ARGUMENTS}
\item[\Co{value:}]
an optional initialization value, which can be any of the
following: a \Co{pdMat} object, a positive-definite
matrix, a one-sided linear formula (with variables separated by
\Co{+}), a vector of character strings, or a numeric
value. Defaults to \Co{numeric(0)}, corresponding
to an uninitialized object.
\item[\Co{form:}]
an optional one-sided linear formula specifying the
row/column names for the matrix represented by \Co{object}. Because
factors may be present in \Co{form}, the formula needs to be
evaluated on a data.frame to resolve the names it defines. This
argument is ignored when \Co{value} is a one-sided
formula. Defaults to \Co{NULL}.
\item[\Co{nam:}]
an optional vector of character strings specifying the
row/column names for the matrix represented by object. It must have 
length equal to the dimension of the underlying positive-definite
matrix and unreplicated elements. This argument is ignored when
\Co{value} is a vector of character strings. Defaults to
\Co{NULL}.
\item[\Co{data:}]
an optional data frame in which to evaluate the variables
named in \Co{value} and \Co{form}. It is used to
obtain the levels for \Co{factors}, which affect the
dimensions and the row/column names of the underlying matrix. If
\Co{NULL}, no attempt is made to obtain information on 
\Co{factors} appearing the random effects model. Defaults to parent
frame from which the function was called.
\end{Argument}
\Paragraph{VALUE}
a \Co{pdIdent} object representing a multiple of the identity
positive-definite matrix, also inheriting from class \Co{pdMat}.
\Paragraph{SEE ALSO}
\Co{as.matrix.pdMat}, \Co{coef.pdMat},
\Co{matrix<-.pdMat}
\need 15pt
\Paragraph{EXAMPLE}
\vspace{-16pt} 
\begin{Example}
pd1 <- pdIdent(4 * diag(3), nam = c("A","B","C"))
pd1
\end{Example}
\end{Helpfile}
\begin{Helpfile}{pdMat}{Positive-Definite Matrix}
This function gives an alternative way of constructing an object
inheriting from the \Co{pdMat} class named in \Co{pdClass}, or
from \Co{data.class(object)} if \Co{object} inherits from
\Co{pdMat},  and is mostly used internally in other functions. See
the documentation on the principal constructor function, generally
with the same name as the \Co{pdMat} class of object.
\begin{Example}
pdMat(value, form, nam, data, pdClass)
\end{Example}
\begin{Argument}{ARGUMENTS}
\item[\Co{value:}]
an optional initialization value, which can be any of the
following: a \Co{pdMat} object, a positive-definite
matrix, a one-sided linear formula (with variables separated by
\Co{+}), a vector of character strings, or a numeric
vector. Defaults to \Co{numeric(0)}, corresponding to an
uninitialized object.
\item[\Co{form:}]
an optional one-sided linear formula specifying the
row/column names for the matrix represented by \Co{object}. Because
factors may be present in \Co{form}, the formula needs to be
evaluated on a data.frame to resolve the names it defines. This
argument is ignored when \Co{value} is a one-sided
formula. Defaults to \Co{NULL}.
\item[\Co{nam:}]
an optional vector of character strings specifying the
row/column names for the matrix represented by object. It must have 
length equal to the dimension of the underlying positive-definite
matrix and unreplicated elements. This argument is ignored when
\Co{value} is a vector of character strings. Defaults to
\Co{NULL}.
\item[\Co{data:}]
an optional data frame in which to evaluate the variables
named in \Co{value} and \Co{form}. It is used to
obtain the levels for \Co{factors}, which affect the
dimensions and the row/column names of the underlying matrix. If
\Co{NULL}, no attempt is made to obtain information on 
\Co{factors} appearing the random effects model. Defaults to parent
frame from which the function was called.
\item[\Co{pdClass:}]
an optional character string naming the
\Co{pdMat} class to be assigned to the returned object. This
argument will only be used when \Co{value} is not a \Co{pdMat} 
object. Defaults to \Co{"pdSymm"}.
\end{Argument}
\Paragraph{VALUE}
a \Co{pdMat} object representing a positive-definite matrix,
inheriting from the class named in \Co{pdClass}, or from
\Co{class(object)}, if \Co{object} inherits from \Co{pdMat}.
\Paragraph{SEE ALSO}
\Co{pdCompSymm}, \Co{pdDiag},
\Co{pdIdent}, \Co{pdNatural}, \Co{pdSymm}
\need 15pt
\Paragraph{EXAMPLE}
\vspace{-16pt} 
\begin{Example}
pd1 <- pdMat(diag(1:4), pdClass = "pdDiag")
pd1
\end{Example}
\end{Helpfile}
\begin{Helpfile}{pdMatrix}{pdMat Matrix or Square-Root Factor}
The positive-definite matrix represented by \Co{object}, or a
square-root factor of it is obtained. Letting $\bS$ denote a
positive-definite matrix, a square-root factor of $\bS$ is
any square matrix $\bL$ such that $\bS = \bL^{t}\bL$. This function
extracts $\bS$ or $\bL$.
\begin{Example}
pdMatrix(object, fact, ...)
\end{Example}
\begin{Argument}{ARGUMENTS}
\item[\Co{object:}]
an object inheriting from class \Co{pdMat}, representing
a positive definite matrix.
\item[\Co{fact:}]
an optional logical value. If \Co{TRUE}, a square-root
factor of the positive-definite matrix represented by \Co{object}
is returned; else, if \Co{FALSE}, the positive-definite matrix is
returned. Defaults to \Co{FALSE}.
\item[\Co{...:}]
some methods for the generic function may require
additional arguments.
\end{Argument}
\Paragraph{VALUE}
if \Co{fact} is \Co{FALSE} the positive-definite matrix
represented by \Co{object} is returned; else a square-root of the
positive-definite matrix is returned.
\Paragraph{SEE ALSO}
\Co{as.matrix.pdMat}, \Co{pdFactor},
\Co{corMatrix}
\need 15pt
\Paragraph{EXAMPLE}
\vspace{-16pt} 
\begin{Example}
pd1 <- pdSymm(diag(1:4))
pdMatrix(pd1)
\end{Example}
\end{Helpfile}
\begin{Helpfile}{pdMatrix.reStruct}{reStruct Matrix or Square-Root Factor}
This method function extracts the positive-definite  matrices
corresponding to the \Co{pdMat} elements of \Co{object}, or
square-root factors of the positive-definite matrices.
\begin{Example}
pdMatrix(object, fact)
\end{Example}
\begin{Argument}{ARGUMENTS}
\item[\Co{object:}]
an object inheriting from class \Co{reStruct},
representing a random effects structure and consisting of a list of
\Co{pdMat} objects.
\item[\Co{fact:}]
an optional logical value. If \Co{TRUE}, square-root
factors of the positive-definite matrices represented by the elements
of \Co{object} are returned; else, if \Co{FALSE}, the
positive-definite matrices are returned. Defaults to \Co{FALSE}.
\end{Argument}
\Paragraph{VALUE}
a list with components given by the positive-definite matrices
corresponding to the elements of \Co{object}, or square-root factors
of the positive-definite matrices.
\Paragraph{SEE ALSO}
\Co{as.matrix.reStruct}, \Co{reStruct},
\Co{pdMatrix.pdMat}
\need 15pt
\Paragraph{EXAMPLE}
\vspace{-16pt} 
\begin{Example}
rs1 <- reStruct(pdSymm(diag(3), form=\Twiddle Sex+age, data=Orthodont))
pdMatrix(rs1)
\end{Example}
\end{Helpfile}
\begin{Helpfile}{pdNatural}{General PD Matrix in Natural Parametrization}
  This function is a constructor for the \Co{pdNatural} class,
  representing a general positive-definite matrix, using a natural
  parametrization . If the matrix associated with \Co{object} is of
  order $n$, it is represented by $n(n+1)/2$ parameters. Letting
  $\bS_{ij}$ denote the $ij$-th element of the underlying positive
  definite matrix and $\rho_{ij} = \bS_{ij}/\sqrt{\bS_{ii}\bS_{jj}},\;
  i \neq j$ denote the associated \emph{correlations}, the
  \emph{natural} parameters are given by
    $\log(\bS_{ii})/2,\;i=1,\ldots,n$ and
    $\log((1+\rho_{ij})/(1-\rho_{ij})),\; i \neq j$. Note that all
    natural parameters are individually unrestricted, but not jointly
    unrestricted (meaning that not all unrestricted vectors would give
    positive-definite matrices). Therefore, this parametrization
    should NOT be used for optimization. It is mostly used for
    deriving approximate confidence intervals on parameters following
    the optimization of an objective function. When \Co{value} is
    \Co{numeric(0)}, an uninitialized \Co{pdMat} object, a one-sided
    formula, or a vector of character strings, \Co{object} is returned
    as an uninitialized \Co{pdSymm} object (with just some of its
    attributes and its class defined) and needs to have its
    coefficients assigned later, generally using the \Co{coef} or
    \Co{matrix} functions. If \Co{value} is an initialized \Co{pdMat}
    object, \Co{object} will be constructed from
    \Co{as.matrix(value)}. Finally, if \Co{value} is a numeric vector,
    it is assumed to represent the natural parameters of the
    underlying positive-definite matrix.
\begin{Example}
pdNatural(value, form, nam, data)
\end{Example}
\begin{Argument}{ARGUMENTS}
\item[\Co{value:}]
an optional initialization value, which can be any of the
following: a \Co{pdMat} object, a positive-definite
matrix, a one-sided linear formula (with variables separated by
\Co{+}), a vector of character strings, or a numeric
vector. Defaults to \Co{numeric(0)}, corresponding to an
uninitialized object.
\item[\Co{form:}]
an optional one-sided linear formula specifying the
row/column names for the matrix represented by \Co{object}. Because
factors may be present in \Co{form}, the formula needs to be
evaluated on a data.frame to resolve the names it defines. This
argument is ignored when \Co{value} is a one-sided
formula. Defaults to \Co{NULL}.
\item[\Co{nam:}]
an optional vector of character strings specifying the
row/column names for the matrix represented by object. It must have 
length equal to the dimension of the underlying positive-definite
matrix and unreplicated elements. This argument is ignored when
\Co{value} is a vector of character strings. Defaults to
\Co{NULL}.
\item[\Co{data:}]
an optional data frame in which to evaluate the variables
named in \Co{value} and \Co{form}. It is used to
obtain the levels for \Co{factors}, which affect the
dimensions and the row/column names of the underlying matrix. If
\Co{NULL}, no attempt is made to obtain information on 
\Co{factors} appearing the random effects model. Defaults to parent
frame from which the function was called.
\end{Argument}
\Paragraph{VALUE}
a \Co{pdNatural} object representing a general positive-definite
matrix in natural parametrization, also inheriting from class
\Co{pdMat}.
\Paragraph{SEE ALSO}
\Co{as.matrix.pdMat}, \Co{coef.pdMat},
\Co{matrix<-.pdMat}
\need 15pt
\Paragraph{EXAMPLE}
\vspace{-16pt} 
\begin{Example}
pdNatural(diag(1:3))
\end{Example}
\end{Helpfile}
\begin{Helpfile}{pdSymm}{General Positive-Definite Matrix}
  This function is a constructor for the \Co{pdSymm} class,
  representing a general positive-definite matrix. If the matrix
  associated with \Co{object} is of order $n$, it is represented by
  $n(n+1)/2$ unrestricted parameters, using the matrix-logarithm
  parametrization described in Pinheiro and Bates (1996). When
  \Co{value} is \Co{numeric(0)}, an uninitialized \Co{pdMat} object, a
  one-sided formula, or a vector of character strings, \Co{object} is
  returned as an uninitialized \Co{pdSymm} object (with just some of
  its attributes and its class defined) and needs to have its
  coefficients assigned later, generally using the \Co{coef} or
  \Co{matrix} functions. If \Co{value} is an initialized \Co{pdMat}
  object, \Co{object} will be constructed from \Co{as.matrix(value)}.
  Finally, if \Co{value} is a numeric vector, it is assumed to
  represent the unrestricted coefficients of the matrix-logarithm
  parametrization of the underlying positive-definite matrix.
\begin{Example}
pdSymm(value, form, nam, data)
\end{Example}
\begin{Argument}{ARGUMENTS}
\item[\Co{value:}]
an optional initialization value, which can be any of the
following: a \Co{pdMat} object, a positive-definite
matrix, a one-sided linear formula (with variables separated by
\Co{+}), a vector of character strings, or a numeric
vector. Defaults to \Co{numeric(0)}, corresponding to an
uninitialized object.
\item[\Co{form:}]
an optional one-sided linear formula specifying the
row/column names for the matrix represented by \Co{object}. Because
factors may be present in \Co{form}, the formula needs to be
evaluated on a data.frame to resolve the names it defines. This
argument is ignored when \Co{value} is a one-sided
formula. Defaults to \Co{NULL}.
\item[\Co{nam:}]
an optional vector of character strings specifying the
row/column names for the matrix represented by object. It must have 
length equal to the dimension of the underlying positive-definite
matrix and unreplicated elements. This argument is ignored when
\Co{value} is a vector of character strings. Defaults to
\Co{NULL}.
\item[\Co{data:}]
an optional data frame in which to evaluate the variables
named in \Co{value} and \Co{form}. It is used to
obtain the levels for \Co{factors}, which affect the
dimensions and the row/column names of the underlying matrix. If
\Co{NULL}, no attempt is made to obtain information on 
\Co{factors} appearing the random effects model. Defaults to parent
frame from which the function was called.
\end{Argument}
\Paragraph{VALUE}
a \Co{pdSymm} object representing a general positive-definite
matrix, also inheriting from class \Co{pdMat}.
\Paragraph{REFERENCES}
Pinheiro, J.C. and Bates., D.M.  (1996) "Unconstrained
Parametrizations for Variance-Covariance Matrices", Statistics and
Computing, 6, 289-296.
\Paragraph{SEE ALSO}
\Co{as.matrix.pdMat}, \Co{coef.pdMat},
\Co{matrix<-.pdMat}
\need 15pt
\Paragraph{EXAMPLE}
\vspace{-16pt} 
\begin{Example}
pd1 <- pdSymm(diag(1:3), nam = c("A","B","C"))
pd1
\end{Example}
\end{Helpfile}
\begin{Helpfile}{plot.ACF}{Plot an ACF Object}
an \Co{xyplot} of the autocorrelations versus the lags, with
\Co{type = "h"}, is produced. If \Co{alpha > 0}, curves
representing the critical limits for a two-sided test of level
\Co{alpha} for the autocorrelations are added to the plot.
\begin{Example}
plot(object, alpha, xlab, ylab, grid, ...)
\end{Example}
\begin{Argument}{ARGUMENTS}
\item[\Co{object:}]
an object inheriting from class \Co{ACF},
consisting of a data frame with two columns named \Co{lag} and
\Co{ACF}, representing the autocorrelation values and the
corresponding lags. 
\item[\Co{alpha:}]
an optional numeric value with the significance level for
testing if the autocorrelations are zero. Lines corresponding to the
lower and upper critical values for a test of level \Co{alpha} are
added to the plot. Default is \Co{0}, in which case no lines are
plotted.
\item[\Co{xlab,ylab:}]
optional character strings with the x- and y-axis
labels. Default respectively to \Co{"Lag"} and 
\Co{"Autocorrelation"}. 
\item[\Co{grid:}]
an optional logical value indicating whether a grid should
be added to plot. Defaults to \Co{FALSE}.
\item[\Co{...:}]
optional arguments passed to the \Co{xyplot} function.
\end{Argument}
\Paragraph{VALUE}
an \Co{xyplot} Trellis plot.
\Paragraph{NOTE}
This function requires the \Co{trellis} library.
\Paragraph{SEE ALSO}
\Co{ACF}, \Co{xyplot}
\need 15pt
\Paragraph{EXAMPLE}
\vspace{-16pt}
\begin{Example}
fm1 <- lme(follicles {\Twiddle} sin(2*pi*Time) + cos(2*pi*Time), Ovary)
plot(ACF(fm1, maxLag = 10), alpha = 0.01)
\end{Example}
\end{Helpfile}
\begin{Helpfile}{plot.augPred}{Plot augPred Object}
A Trellis \Co{xyplot} of predictions versus primary  covariate is
generated, with a different panel for each value of the grouping
factor. Predicted values are joined by lines, with  
different line types (colors) being used for each level of
grouping. Original observations are represented by circles.
\begin{Example}
plot(x, key, grid, ...)
\end{Example}
\begin{Argument}{ARGUMENTS}
\item[\Co{x:}]
an object of class \Co{augPred}.
\item[\Co{key:}]
an optional logical value, or list. If \Co{TRUE}, a legend
is included at the top of the plot indicating which symbols (colors)
correspond to which prediction levels. If \Co{FALSE}, no legend
is included. If given as a list, \Co{key} is passed down as an
argument to the \Co{trellis} function generating the plots
(\Co{xyplot}). Defaults to \Co{TRUE}.
\item[\Co{grid:}]
an optional logical value indicating whether a grid should
be added to plot. Defaults to \Co{FALSE}.
\item[\Co{...:}]
optional arguments passed down to the \Co{trellis}
function generating the plots.
\end{Argument}
\Paragraph{VALUE}
A Trellis plot of predictions versus primary covariate, with panels
determined by the grouping factor.
\Paragraph{SEE ALSO}
\Co{augPred}, \Co{xyplot}
\need 15pt
\Paragraph{EXAMPLE}
\vspace{-16pt} 
\begin{Example}
fm1 <- lme(Orthodont)
plot(augPred(fm1, level = 0:1, length.out = 2))
\end{Example}
\end{Helpfile}
\begin{Helpfile}{plot.compareFits}{Plot a compareFits Object}
A Trellis \Co{dotplot} of the values being compared, with different
rows per group, is generated, with a different panel for each
coefficient. Different symbols (colors) are used for each object being
compared.
\begin{Example}
plot(object, subset, key, mark, ...)
\end{Example}
\begin{Argument}{ARGUMENTS}
\item[\Co{object:}]
an object of class \Co{compareFits}.
\item[\Co{subset:}]
an optional logical or integer vector specifying which
rows of \Co{object} should be used in the plots. If missing, all
rows are used.
\item[\Co{key:}]
an optional logical value, or list. If \Co{TRUE}, a legend
is included at the top of the plot indicating which symbols (colors)
correspond to which objects being compared. If \Co{FALSE}, no legend
is included. If given as a list, \Co{key} is passed down as an
argument to the \Co{trellis} function generating the plots
(\Co{dotplot}). Defaults to \Co{TRUE}.
\item[\Co{mark:}]
an optional numeric vector, of length equal to the number of
coefficients being compared, indicating where vertical lines should
be drawn in the plots. If missing, no lines are drawn.
\item[\Co{...:}]
optional arguments passed down to the \Co{trellis}
function generating the plots.
\end{Argument}
\Paragraph{VALUE}
A Trellis \Co{dotplot} of the values being compared, with rows
determined by the groups and panels by the coefficients.
\Paragraph{SEE ALSO}
\Co{compareFits},\Co{pairs.compareFits},
\Co{dotplot}
\need 15pt
\Paragraph{EXAMPLE}
\vspace{-16pt} 
\begin{Example}
fm1 <- lmList(Orthodont)
fm2 <- lme(Orthodont)
plot(compareFits(coef(fm1), coef(fm2)))
\end{Example}
\end{Helpfile}
\begin{Helpfile}{plot.gls}{Plot a gls Object}
Diagnostic plots for the linear model fit are obtained. The
\Co{form} argument gives considerable flexibility in the type of
plot specification. A conditioning expression (on the right side of a
\Co{|} operator) always implies that different panels are used for
each level of the conditioning factor, according to a Trellis
display. If \Co{form} is a one-sided formula, histograms of the
variable on the right hand side of the formula, before a \Co{|}
operator, are displayed (the Trellis function \Co{histogram} is
used). If \Co{form} is two-sided and both its left and
right hand side variables are numeric, scatter plots are displayed
(the Trellis function \Co{xyplot} is used). Finally, if \Co{form}
is two-sided and its left had side variable is a factor, box-plots of
the right hand side variable by the levels of the left hand side
variable are displayed (the Trellis function  \Co{bwplot} is used).
\begin{Example}
plot(object, form, abline, id, idLabels, idResType, grid, ...)
\end{Example}
\begin{Argument}{ARGUMENTS}
\item[\Co{object:}]
an object inheriting from class \Co{gls}, representing
a generalized least squares fitted linear model.
\item[\Co{form:}]
an optional formula specifying the desired type of
plot. Any variable present in the original data frame used to obtain
\Co{object} can be referenced. In addition, \Co{object} itself
can be referenced in the formula using the symbol
\Co{"."}. Conditional expressions on the right of a \Co{|}
operator can be used to define separate panels in a Trellis
display. Default is \Co{resid(., type = "p") \Twiddle fitted(.)},
corresponding to a plot of the standardized residuals versus fitted
values, both evaluated at the innermost level of nesting.
\item[\Co{abline:}]
an optional numeric value, or numeric vector of length
two. If given as a single value, a horizontal line will be added to the
plot at that coordinate; else, if given as a vector, its values are
used as the intercept and slope for a line added to the plot. If
missing, no lines are added to the plot.
\item[\Co{id:}]
an optional numeric value, or one-sided formula. If given as
a value, it is used as a significance level for a two-sided outlier
test for the standardized residuals. Observations with
absolute standardized residuals greater than the \Co{1-id/2}
quantile of the standard normal distribution are identified in the
plot using \Co{idLabels}. If given as a one-sided formula, its
right hand side must evaluate to a  logical, integer, or character
vector which is used to identify observations in the plot. If
missing, no observations are identified.
\item[\Co{idLabels:}]
an optional vector, or one-sided formula. If given as a
vector, it is converted to character and used to label the
observations identified according to \Co{id}. If given as a
one-sided formula, its right hand side must evaluate to a vector
which is converted to character and used to label the identified
observations. Default is the innermost grouping factor. 
\item[\Co{idResType:}]
an optional character string specifying the type of
residuals to be used in identifying outliers, when \Co{id} is a
numeric value. If \Co{"pearson"}, the standardized residuals (raw 
residuals divided by the corresponding standard errors) are used;
else, if \Co{"normalized"}, the normalized residuals (standardized
residuals pre-multiplied by the inverse square-root factor of the
estimated error correlation matrix) are used. Partial matching of
arguments is used, so only the first character needs to be
provided. Defaults to \Co{"pearson"}.
\item[\Co{grid:}]
an optional logical value indicating whether a grid should
be added to plot. Default depends on the type of Trellis plot used:
if \Co{xyplot} defaults to \Co{TRUE}, else defaults to
\Co{FALSE}.
\item[\Co{...:}]
optional arguments passed to the Trellis plot function.
\end{Argument}
\Paragraph{VALUE}
a diagnostic Trellis plot.
\Paragraph{NOTE} This function requires the \Co{trellis} library.
\Paragraph{SEE ALSO}
\Co{gls}, \Co{xyplot}, \Co{bwplot},
\Co{histogram}
\need 15pt
\Paragraph{EXAMPLE}
\vspace{-16pt} 
\begin{Example}
fm1 <- gls(follicles \Twiddle sin(2*pi*Time) + cos(2*pi*Time), Ovary,
           correlation = corAR1(form = \Twiddle 1 | Mare))
# standardized residuals versus fitted values by Mare
plot(fm1, resid(., type = "p") \Twiddle fitted(.) | Mare, abline = 0)
# box-plots of residuals by Mare
plot(fm1, Mare \Twiddle resid(.))
# observed versus fitted values by Mare
plot(fm1, follicles \Twiddle fitted(.) | Mare, abline = c(0,1))
\end{Example}
\end{Helpfile}
\begin{Helpfile}{plot.intervals.lmList}{Plot lmList Confidence Intervals}
A Trellis dot-plot of the confidence intervals on the linear model
coefficients is generated, with a different panel for each
coefficient. Rows in the dot-plot correspond to the names of the
\Co{lm} components of the \Co{lmList} object used to produce
\Co{object}. The lower and upper confidence limits are connected by
a line segment and the estimated coefficients are marked with a
\Co{"+"}. The Trellis function \Co{dotplot} is used in this method
function.
\begin{Example}
plot(object, ...)
\end{Example}
\begin{Argument}{ARGUMENTS}
\item[\Co{object:}]
an object inheriting from class \Co{intervals.lmList},
representing confidence intervals and estimates for the the
coefficients in the \Co{lm} components of the \Co{lmList} object
used to produce \Co{object}.
\item[\Co{...:}]
optional arguments passed to the Trellis \Co{dotplot}
function.
\end{Argument}
\Paragraph{VALUE}
a Trellis plot with the confidence intervals on the coefficients of
the individual \Co{lm} components of the \Co{lmList} that
generated \Co{object}.
\Paragraph{NOTE} This function requires the \Co{trellis} library.
\Paragraph{SEE ALSO}
\Co{intervals.lmList}, \Co{lmList},
\Co{dotplot}
\need 15pt
\Paragraph{EXAMPLE}
\vspace{-16pt}
\begin{Example}
fm1 <- lmList(distance {\Twiddle} age | Subject, Orthodont)
plot(intervals(fm1))
\end{Example}
\end{Helpfile}
\begin{Helpfile}{plot.lmList}{Plot an lmList Object}
Diagnostic plots for the linear model fits corresponding to the
\Co{object}  components are obtained. The \Co{form} argument gives
considerable flexibility in the type of  plot specification. A
conditioning expression (on the right side of a  \Co{|} operator)
always implies that different panels are used for  each level of the
conditioning factor, according to a Trellis  display. If \Co{form}
is a one-sided formula, histograms of the  variable on the right hand
side of the formula, before a \Co{|}  operator, are displayed (the
Trellis function \Co{histogram} is  used). If \Co{form} is
two-sided and both its left and  right hand side variables are
numeric, scatter plots are displayed  (the Trellis function
\Co{xyplot} is used). Finally, if \Co{form}  is two-sided and its
left had side variable is a factor, box-plots of  the right hand side
variable by the levels of the left hand side  variable are displayed
(the Trellis function  \Co{bwplot} is used).
\begin{Example}
plot(object, form, abline, id, idLabels, grid, ...)
\end{Example}
\begin{Argument}{ARGUMENTS}
\item[\Co{object:}]
an object inheriting from class \Co{lmList}, representing
a list of \Co{lm} objects with a common model.
\item[\Co{form:}]
an optional formula specifying the desired type of
plot. Any variable present in the original data frame used to obtain
\Co{object} can be referenced. In addition, \Co{object} itself
can be referenced in the formula using the symbol
\Co{"."}. Conditional expressions on the right of a \Co{|}
operator can be used to define separate panels in a Trellis
display. Default is \Co{resid(., type = "pool") {\Twiddle} fitted(.)},
corresponding to a plot of the standardized residuals (using a pooled
estimate for the residual standard error) versus fitted values.
\item[\Co{abline:}]
an optional numeric value, or numeric vector of length
two. If given as a single value, a horizontal line will be added to the
plot at that coordinate; else, if given as a vector, its values are
used as the intercept and slope for a line added to the plot. If
missing, no lines are added to the plot.
\item[\Co{id:}]
an optional numeric value, or one-sided formula. If given as
a value, it is used as a significance level for a two-sided outlier
test for the standardized residuals. Observations with
absolute standardized residuals greater than the 1 - value/2
quantile of the standard normal distribution are identified in the
plot using \Co{idLabels}. If given as a one-sided formula, its
right hand side must evaluate to a  logical, integer, or character
vector which is used to identify observations in the plot. If
missing, no observations are identified.
\item[\Co{idLabels:}]
an optional vector, or one-sided formula. If given as a
vector, it is converted to character and used to label the
observations identified according to \Co{id}. If given as a
one-sided formula, its right hand side must evaluate to a vector
which is converted to character and used to label the identified
observations. Default is \Co{getGroups(object)}. 
\item[\Co{grid:}]
an optional logical value indicating whether a grid should
be added to plot. Default depends on the type of Trellis plot used:
if \Co{xyplot} defaults to \Co{TRUE}, else defaults to
\Co{FALSE}.
\item[\Co{...:}]
optional arguments passed to the Trellis plot function.
\end{Argument}
\Paragraph{VALUE}
a diagnostic Trellis plot.
\Paragraph{NOTE} This function requires the \Co{trellis} library.
\Paragraph{SEE ALSO}
\Co{lmList}, \Co{xyplot}, \Co{bwplot},
\Co{histogram}
\need 15pt
\Paragraph{EXAMPLE}
\vspace{-16pt}
\begin{Example}
fm1 <- lmList(distance {\Twiddle} age | Subject, Orthodont)
# standardized residuals versus fitted values by gender
plot(fm1, resid(., type = "pool") {\Twiddle} fitted(.)| Sex, 
     abline = 0, id = 0.05)
# box-plots of residuals by Subject
plot(fm1, Subject {\Twiddle} resid(.))
# observed versus fitted values by Subject
plot(fm1, distance {\Twiddle} fitted(.)| Subject, abline = c(0,1))
\end{Example}
\end{Helpfile}
\begin{Helpfile}{plot.lme}{Plot an lme Object}
Diagnostic plots for the linear mixed-effects fit are obtained. The
\Co{form} argument gives considerable flexibility in the type of
plot specification. A conditioning expression (on the right side of a
\Co{|} operator) always implies that different panels are used for
each level of the conditioning factor, according to a Trellis
display. If \Co{form} is a one-sided formula, histograms of the
variable on the right hand side of the formula, before a \Co{|}
operator, are displayed (the Trellis function \Co{histogram} is
used). If \Co{form} is two-sided and both its left and
right hand side variables are numeric, scatter plots are displayed
(the Trellis function \Co{xyplot} is used). Finally, if \Co{form}
is two-sided and its left had side variable is a factor, box-plots of
the right hand side variable by the levels of the left hand side
variable are displayed (the Trellis function  \Co{bwplot} is used).
\begin{Example}
plot(object, form, abline, id, idLabels, idResType, grid, ...)
\end{Example}
\begin{Argument}{ARGUMENTS}
\item[\Co{object:}]
an object inheriting from class \Co{lme}, representing
a fitted linear mixed-effects model.
\item[\Co{form:}]
an optional formula specifying the desired type of
plot. Any variable present in the original data frame used to obtain
\Co{object} can be referenced. In addition, \Co{object} itself
can be referenced in the formula using the symbol
\Co{"."}. Conditional expressions on the right of a \Co{|}
operator can be used to define separate panels in a Trellis
display. Default is \Co{resid(., type = "p") \Twiddle fitted(.) },
corresponding to a plot of the standardized residuals versus fitted
values, both evaluated at the innermost level of nesting.
\item[\Co{abline:}]
an optional numeric value, or numeric vector of length
two. If given as a single value, a horizontal line will be added to the
plot at that coordinate; else, if given as a vector, its values are
used as the intercept and slope for a line added to the plot. If
missing, no lines are added to the plot.
\item[\Co{id:}]
an optional numeric value, or one-sided formula. If given as
a value, it is used as a significance level for a two-sided outlier
test for the standardized residuals. Observations with
absolute standardized residuals greater than the \Co{1-id/2}
quantile of the standard normal distribution are identified in the
plot using \Co{idLabels}. If given as a one-sided formula, its
right hand side must evaluate to a  logical, integer, or character
vector which is used to identify observations in the plot. If
missing, no observations are identified.
\item[\Co{idLabels:}]
an optional vector, or one-sided formula. If given as a
vector, it is converted to character and used to label the
observations identified according to \Co{id}. If given as a
one-sided formula, its right hand side must evaluate to a vector
which is converted to character and used to label the identified
observations. Default is the innermost grouping factor. 
\item[\Co{idResType:}]
an optional character string specifying the type of
residuals to be used in identifying outliers, when \Co{id} is a
numeric value. If \Co{"pearson"}, the standardized residuals (raw 
residuals divided by the corresponding standard errors) are used;
else, if \Co{"normalized"}, the normalized residuals (standardized
residuals pre-multiplied by the inverse square-root factor of the
estimated error correlation matrix) are used. Partial matching of
arguments is used, so only the first character needs to be
provided. Defaults to \Co{"pearson"}.
\item[\Co{grid:}]
an optional logical value indicating whether a grid should
be added to plot. Default depends on the type of Trellis plot used:
if \Co{xyplot} defaults to \Co{TRUE}, else defaults to
\Co{FALSE}.
\item[\Co{...:}]
optional arguments passed to the Trellis plot function.
\end{Argument}
\Paragraph{VALUE}
a diagnostic Trellis plot.
\Paragraph{NOTE} This function requires the \Co{trellis} library.
\Paragraph{SEE ALSO}
\Co{lme}, \Co{xyplot}, \Co{bwplot},
\Co{histogram}
\need 15pt
\Paragraph{EXAMPLE}
\vspace{-16pt} 
\begin{Example}
fm1 <- lme(distance \Twiddle age, Orthodont, random = \Twiddle age | Subject)
# standardized residuals versus fitted values by gender
plot(fm1, resid(., type = "p") \Twiddle fitted(.) | Sex, abline = 0)
# box-plots of residuals by Subject
plot(fm1, Subject \Twiddle resid(.))
# observed versus fitted values by Subject
plot(fm1, distance \Twiddle fitted(.) | Subject, abline = c(0,1))
\end{Example}
\end{Helpfile}
\begin{Helpfile}{plot.nffGroupedData}{Plot nffGroupedData Object}
A Trellis dot-plot of the response by group is generated. If outer
variables are specified, the combination of their levels are used to
determine the panels of the Trellis display. The Trellis function
\Co{dotplot} is used.
\begin{Example}
plot(x, outer, inner, innerGroups, xlab, ylab, strip, panel, 
     key, ...)
\end{Example}
\begin{Argument}{ARGUMENTS}
\item[\Co{x:}]
an object inheriting from class \Co{nffGroupedData},
representing a \Co{groupedData} object with a factor primary
covariate and a single grouping level.
\item[\Co{outer:}]
an optional logical value or one-sided formula,
indicating covariates that are outer to the grouping factor, which
are used to determine the panels of the Trellis plot. If
equal to \Co{TRUE}, \Co{attr(object, "outer")} is used to indicate the
outer covariates. An outer covariate is invariant within the sets
of rows defined by the grouping factor.  Ordering of the groups is
done in such a way as to preserve adjacency of groups with the same
value of the outer variables. Defaults to \Co{NULL}, meaning that
no outer covariates are to be used.
\item[\Co{inner:}]
an optional logical value or one-sided formula, indicating
a covariate that is inner to the grouping factor, which is used to
associate points within each panel of the Trellis plot. If
equal to \Co{TRUE}, \Co{attr(object, "outer")} is used to indicate the
inner covariate. An inner covariate can change within the sets of
rows defined by the grouping  factor. Defaults to \Co{NULL},
meaning that no inner covariate  is present. 
\item[\Co{innerGroups:}]
an optional one-sided formula specifying a factor
to be used for grouping the levels of the \Co{inner}
covariate. Different colors, or symbols, are used for each level
of the \Co{innerGroups} factor. Default is \Co{NULL}, meaning
that no \Co{innerGroups} covariate is present.
\item[\Co{xlab:}]
an optional character string with the label for the
horizontal axis. Default is the \Co{y} elements of `attr(object,
     "labels")' and \Co{attr(object, "units")} pasted together.
\item[\Co{ylab:}]
an optional character string with the label for the
vertical axis. Default is the grouping factor name.
\item[\Co{strip:}]
an optional function passed as the \Co{strip} argument to
the \Co{dotplot} function. Default is `strip.default(..., style
     = 1)' (see \Co{trellis.args}).
\item[\Co{panel:}]
an optional function used to generate the individual
panels in the Trellis display, passed as the \Co{panel} argument
to the \Co{dotplot} function.
\item[\Co{key:}]
an optional logical function or function. If \Co{TRUE}
and either \Co{inner} or \Co{innerGroups} are non-\Co{NULL}, a
legend for the different \Co{inner} (\Co{innerGroups}) levels is 
included at the top of the plot. If given as a function, it is passed
as the \Co{key} argument to the \Co{dotplot} function. Default is
\Co{TRUE} is either \Co{inner} or \Co{innerGroups} are
non-\Co{NULL} and \Co{FALSE} otherwise.
\item[\Co{...:}]
optional arguments passed to the \Co{dotplot} function.
\end{Argument}
\Paragraph{VALUE}
a Trellis dot-plot of the response by group.
\Paragraph{NOTE} This function requires the \Co{trellis} library.
\Paragraph{REFERENCES}
Bates, D.M. and Pinheiro, J.C. (1997), "Software Design for Longitudinal
Data", in "Modelling Longitudinal and Spatially Correlated Data:
Methods, Applications and Future Directions", T.G. Gregoire (ed.),
Springer-Verlag, New York.
Pinheiro, J.C. and Bates, D.M. (1997) "Future Directions in
Mixed-Effects Software: Design of NLME 3.0" available at
http://nlme.stat.wisc.edu.
\Paragraph{SEE ALSO}
\Co{groupedData}, \Co{dotplot}
\need 15pt
\Paragraph{EXAMPLE}
\vspace{-16pt} 
\begin{Example}
plot(Machines)
plot(Machines, inner = TRUE)
\end{Example}
\end{Helpfile}
\begin{Helpfile}{plot.nfnGroupedData}{Plot nfnGroupedData Object}
A Trellis plot of the response versus the primary covariate is
generated. If outer variables are specified, the combination of their
levels are used to determine the panels of the Trellis
display. Otherwise, the levels of the grouping variable determine the
panels. A scatter plot of the response versus the primary covariate is
displayed in each panel, with observations corresponding to same
inner group joined by line segments. The Trellis function
\Co{xyplot} is used.
\begin{Example}
plot(x, outer, inner, innerGroups, xlab, ylab, strip, aspect, 
     panel, key, grid, ...)
\end{Example}
\begin{Argument}{ARGUMENTS}
\item[\Co{x:}]
an object inheriting from class \Co{nfnGroupedData},
representing a \Co{groupedData} object with a numeric primary
covariate and a single grouping level.
\item[\Co{outer:}]
an optional logical value or one-sided formula,
indicating covariates that are outer to the grouping factor, which
are used to determine the panels of the Trellis plot. If
equal to \Co{TRUE}, \Co{attr(object, "outer")} is used to
indicate the outer covariates. An outer covariate is invariant within
the sets of rows defined by the grouping factor.  Ordering of the
groups is done in such a way as to preserve adjacency of groups with
the same value of the outer variables. Defaults to \Co{NULL},
meaning that no outer covariates are to be used.
\item[\Co{inner:}]
an optional logical value or one-sided formula, indicating
a covariate that is inner to the grouping factor, which is used to
associate points within each panel of the Trellis plot. If
equal to \Co{TRUE}, \Co{attr(object, "outer")} is used to
indicate the inner covariate. An inner covariate can change within
the sets of rows defined by the grouping  factor. Defaults to
\Co{NULL}, meaning that no inner covariate  is present. 
\item[\Co{innerGroups:}]
an optional one-sided formula specifying a factor
to be used for grouping the levels of the \Co{inner}
covariate. Different colors, or line types, are used for each level
of the \Co{innerGroups} factor. Default is \Co{NULL}, meaning
that no \Co{innerGroups} covariate is present.
\item[\Co{xlab, ylab:}]
optional character strings with the labels for the
plot. Default is the corresponding elements of `attr(object,
     "labels")' and \Co{attr(object, "units")} pasted together.
\item[\Co{strip:}]
an optional function passed as the \Co{strip} argument to
the \Co{xyplot} function. Default is `strip.default(..., style
     = 1)' (see \Co{trellis.args}).
\item[\Co{aspect:}]
an optional character string indicating the aspect ratio
for the plot passed as the \Co{aspect} argument to the
\Co{xyplot} function. Default is \Co{"xy"} (see
\Co{trellis.args}). 
\item[\Co{panel:}]
an optional function used to generate the individual
panels in the Trellis display, passed as the \Co{panel} argument
to the \Co{xyplot} function.
\item[\Co{key:}]
an optional logical function or function. If \Co{TRUE}
and \Co{innerGroups} is non-\Co{NULL}, a legend for the
different \Co{innerGroups} levels is included at the top of the
plot. If given as a function, it is passed as the \Co{key} argument
to the \Co{xyplot} function.  Default is \Co{TRUE} if
\Co{innerGroups} is non-\Co{NULL} and \Co{FALSE} otherwise.
\item[\Co{grid:}]
an optional logical value indicating whether a grid should
be added to plot. Defaults to \Co{TRUE}.
\item[\Co{...:}]
optional arguments passed to the \Co{xyplot} function.
\end{Argument}
\Paragraph{VALUE}
a Trellis plot of the response versus the primary covariate.
\Paragraph{NOTE} This function requires the \Co{trellis} library.
\Paragraph{REFERENCES}
Bates, D.M. and Pinheiro, J.C. (1997), "Software Design for Longitudinal
Data", in "Modelling Longitudinal and Spatially Correlated Data:
Methods, Applications and Future Directions", T.G. Gregoire (ed.),
Springer-Verlag, New York.
Pinheiro, J.C. and Bates, D.M. (1997) "Future Directions in
Mixed-Effects Software: Design of NLME 3.0" available at
http://nlme.stat.wisc.edu.
\Paragraph{SEE ALSO}
\Co{groupedData}, \Co{xyplot}
\need 15pt
\Paragraph{EXAMPLE}
\vspace{-16pt} 
\begin{Example}
# different panels per Subject
plot(Orthodont)
# different panels per gender
plot(Orthodont, outer = TRUE)
\end{Example}
\end{Helpfile}
\begin{Helpfile}{plot.nmGroupedData}{Plot nmGroupedData Object}
The \Co{groupedData} object is summarized by the values of the
\Co{displayLevel} grouping factor (or the combination of its values
and the values of the covariate indicated in \Co{preserve}, if any is
present). The collapsed data is used to produce a new
\Co{groupedData} object, with grouping factor given by the
\Co{displayLevel} factor, which is plotted using the
appropriate \Co{plot} method for \Co{groupedData} objects with
single level of grouping.
\begin{Example}
plot(x, collapseLevel, displayLevel, outer, inner, preserve, 
     FUN, subset, grid, ...) 
\end{Example}
\begin{Argument}{ARGUMENTS}
\item[\Co{x:}]
an object inheriting from class \Co{nmGroupedData},
representing a \Co{groupedData} object with multiple grouping
factors.
\item[\Co{collapseLevel:}]
an optional positive integer or character string
indicating the grouping level to use when collapsing the data. Level
values increase from outermost to innermost grouping. Default is the
highest or innermost level of grouping.
\item[\Co{displayLevel:}]
an optional positive integer or character string
indicating the grouping level to use for determining the panels in
the Trellis display, when \Co{outer} is missing. Default is
\Co{collapseLevel}.
\item[\Co{outer:}]
an optional logical value or one-sided formula,
indicating covariates that are outer to the \Co{displayLevel}
grouping factor, which are used to determine the panels of the
Trellis plot. If equal to \Co{TRUE}, the \Co{displayLevel}
element \Co{attr(object, "outer")} is used to indicate the 
outer covariates. An outer covariate is invariant within the sets
of rows defined by the grouping factor.  Ordering of the groups is
done in such a way as to preserve adjacency of groups with the same
value of the outer variables. Defaults to \Co{NULL}, meaning that
no outer covariates are to be used.
\item[\Co{inner:}]
an optional logical value or one-sided formula, indicating
a covariate that is inner to the \Co{displayLevel} grouping factor,
which is used to associate points within each panel of the Trellis
plot. If equal to \Co{TRUE}, \Co{attr(object, "outer")} is used
to indicate the inner covariate. An inner covariate can change within
the sets of rows defined by the grouping  factor. Defaults to
\Co{NULL}, meaning that no inner covariate  is present.  
\item[\Co{preserve:}]
an optional one-sided formula indicating a covariate
whose levels should be preserved when collapsing the data according
to the \Co{collapseLevel} grouping factor. The collapsing factor is
obtained by pasting together the levels of the \Co{collapseLevel}
grouping factor and the values of the covariate to be
preserved. Default is \Co{NULL}, meaning that no covariates need to
be preserved.
\item[\Co{FUN:}]
an optional summary function or a list of summary functions
to be used for collapsing the data.  The function or functions are
applied only to variables in \Co{object} that vary within the
groups defined by \Co{collapseLevel}.  Invariant variables are 
always summarized by group using the unique value that they assume
within that group.  If \Co{FUN} is a single
function it will be applied to each non-invariant variable by group
to produce the summary for that variable.  If \Co{FUN} is a list of
functions, the names in the list should designate classes of
variables in the data such as \Co{ordered}, \Co{factor}, or
\Co{numeric}.  The indicated function will be applied to any
non-invariant variables of that class.  The default functions to be
used are \Co{mean} for numeric factors, and \Co{Mode} for both
\Co{factor} and \Co{ordered}.  The \Co{Mode} function, defined
internally in \Co{gsummary}, returns the modal or most popular
value of the variable.  It is different from the \Co{mode} function
that returns the S-language mode of the variable.
\item[\Co{subset:}]
an optional named list. Names can be either positive
integers representing grouping levels, or names of grouping
factors. Each element in the list is a vector indicating the levels
of the corresponding grouping factor to be used for plotting the
data. Default is \Co{NULL}, meaning that all levels are
used.
\item[\Co{grid:}]
an optional logical value indicating whether a grid should
be added to plot. Defaults to \Co{TRUE}.
\item[\Co{...:}]
optional arguments passed to the Trellis plot function.
\end{Argument}
\Paragraph{VALUE}
a Trellis display of the data collapsed over the values of the
\Co{collapseLevel} grouping factor and grouped according to the
\Co{displayLevel} grouping factor.
\Paragraph{NOTE} This function requires the \Co{trellis} library.
\Paragraph{REFERENCES}
Bates, D.M. and Pinheiro, J.C. (1997), "Software Design for Longitudinal
Data", in "Modelling Longitudinal and Spatially Correlated Data:
Methods, Applications and Future Directions", T.G. Gregoire (ed.),
Springer-Verlag, New York.
Pinheiro, J.C. and Bates, D.M. (1997) "Future Directions in
Mixed-Effects Software: Design of NLME 3.0" available at
http://nlme.stat.wisc.edu.
\Paragraph{SEE ALSO}
\Co{groupedData}, \Co{collapse.groupedData},
\Co{plot.nfnGroupedData}, \Co{plot.nffGroupedData}
\need 15pt
\Paragraph{EXAMPLE}
\vspace{-16pt} 
\begin{Example}
# no collapsing, panels by Dog
plot(Pixel, display = "Dog", inner = \Twiddle Side)
# collapsing by Dog, preserving day
plot(Pixel, collapse = "Dog", preserve = \Twiddle day)
\end{Example}
\end{Helpfile}
\begin{Helpfile}{plot.ranef.lme}{Plot a ranef.lme Object}
If \Co{form} is missing, or is given as a one-sided formula, a
Trellis dot-plot of the random effects is generated, with a different
panel for each random effect (coefficient). Rows in the dot-plot are
determined by the \Co{form} argument (if not missing) or by the row
names of the random effects (coefficients). If a single factor is
specified in \Co{form}, its levels determine the dot-plot rows
(with possibly multiple dots per row); otherwise, if \Co{form}
specifies a crossing of factors, the dot-plot rows are determined by
all combinations of the levels of the individual factors in the 
formula. The Trellis function \Co{dotplot} is used in this method
function.

If \Co{form} is a two-sided formula, a Trellis display is generated,
with a different panel for each variable listed in the right hand side
of \Co{form}. Scatter plots are generated for numeric variables and
boxplots are generated for categorical (\Co{factor} or
\Co{ordered}) variables.
\begin{Example}
plot(object, form, omitFixed, level, grid, control, ...)
\end{Example}
\begin{Argument}{ARGUMENTS}
\item[\Co{object:}]
an object inheriting from class
\Co{ranef.lme}, representing the estimated coefficients or
estimated random effects for the \Co{lme} object from which it was
produced.
\item[\Co{form:}]
an optional formula specifying the desired type of plot. If
given as a one-sided formula, a \Co{dotplot} of the estimated
random effects (coefficients) grouped according to all combinations of
the levels of the factors named in \Co{form} is returned. Single
factors (\Co{{\Twiddle}g}) or crossed factors (\Co{{\Twiddle}g1*g2}) are
allowed. If given as a two-sided formula, the left hand side must
be a single random effects (coefficient) and the right hand side
is formed by covariates in \Co{object} separated by \Co{+}. A
Trellis display of the random effect (coefficient) versus the named
covariates is returned in this case. Default is \Co{NULL}, in
which case  the row names of the random effects (coefficients) are
used.
\item[\Co{omitFixed:}]
an optional logical value indicating whether
columns with values that are constant across groups should be
omitted. Default is \Co{TRUE}.
\item[\Co{level:}]
an optional integer value giving the level of grouping
to be used for \Co{object}. Only used when \Co{object} is a list
with different components for each grouping level. Defaults to the
highest or innermost level of grouping.
\item[\Co{grid:}]
an optional logical value indicating whether a grid should
be added to plot. Only applies to plots associated with two-sided
formulas in \Co{form}. Default is \Co{FALSE}.
\item[\Co{control:}]
an optional list with control values for the
plot, when \Co{form} is given as a two-sided formula. The control
values are referenced by name in the \Co{control} list and  only
the ones to be modified from the default need to be
specified. Available values include: \Co{drawLine}, a logical
value indicating whether a \Co{loess} smoother should be added to
the scatter plots and a line connecting the medians should be added
to the boxplots (default is \Co{TRUE}); \Co{span.loess}, used
as the \Co{span} argument in the call to \Co{panel.loess}
(default is \Co{2/3}); \Co{degree.loess}, used as the
\Co{degree} argument in the call to \Co{panel.loess} (default
is \Co{1}); \Co{cex.axis}, the character expansion factor for
the x-axis (default is \Co{0.8}); \Co{srt.axis}, the rotation
factor for the x-axis (default is \Co{0}); and \Co{mgp.axis}, the
margin parameters for the x-axis (default is \Co{c(2, 0.5, 0)}).
\item[\Co{...:}]
optional arguments passed to the Trellis \Co{dotplot}
function.
\end{Argument}
\Paragraph{VALUE}
a Trellis plot of the estimated random-effects (coefficients) versus
covariates, or groups.
\Paragraph{NOTE}
This function requires the \Co{trellis} library.
\Paragraph{SEE ALSO}
\Co{ranef.lme}, \Co{lme},
\Co{dotplot}
\need 15pt
\Paragraph{EXAMPLE}
\vspace{-16pt} 
\begin{Example}
fm1 <- lme(distance {\Twiddle} age, Orthodont, random = {\Twiddle} age | Subject)
plot(ranef(fm1))
fm1RE <- ranef(fm1, aug = TRUE)
plot(fm1RE, form = {\Twiddle} Sex)
plot(fm1RE, form = age {\Twiddle} Sex)
\end{Example}
\end{Helpfile}
\begin{Helpfile}{plot.ranef.lmList}{Plot a ranef.lmList Object}
If \Co{form} is missing, or is given as a one-sided formula, a
Trellis dot-plot of the random effects is generated, with a different
panel for each random effect (coefficient). Rows in the dot-plot are
determined by the \Co{form} argument (if not missing) or by the row
names of the random effects (coefficients). If a single factor is
specified in \Co{form}, its levels determine the dot-plot rows
(with possibly multiple dots per row); otherwise, if \Co{form}
specifies a crossing of factors, the dot-plot rows are determined by
all combinations of the levels of the individual factors in the 
formula. The Trellis function \Co{dotplot} is used in this method
function.

If \Co{form} is a two-sided formula, a Trellis display is generated,
with a different panel for each variable listed in the right hand side
of \Co{form}. Scatter plots are generated for numeric variables and
boxplots are generated for categorical (\Co{factor} or
\Co{ordered}) variables.
\begin{Example}
plot(object, form, grid, control, ...)
\end{Example}
\begin{Argument}{ARGUMENTS}
\item[\Co{object:}]
an object inheriting from class
\Co{ranef.lmList}, representing the estimated coefficients or
estimated random effects for the \Co{lmList} object from which it was
produced.
\item[\Co{form:}]
an optional formula specifying the desired type of plot. If
given as a one-sided formula, a \Co{dotplot} of the estimated
random effects (coefficients) grouped according to all combinations of
the levels of the factors named in \Co{form} is returned. Single
factors (\Co{{\Twiddle}g}) or crossed factors (\Co{{\Twiddle}g1*g2}) are
allowed. If given as a two-sided formula, the left hand side must
be a single random effects (coefficient) and the right hand side
is formed by covariates in \Co{object} separated by \Co{+}. A
Trellis display of the random effect (coefficient) versus the named
covariates is returned in this case. Default is \Co{NULL}, in
which case  the row names of the random effects (coefficients) are
used.
\item[\Co{grid:}]
an optional logical value indicating whether a grid should
be added to plot. Only applies to plots associated with two-sided
formulas in \Co{form}. Default is \Co{FALSE}.
\item[\Co{control:}]
an optional list with control values for the
plot, when \Co{form} is given as a two-sided formula. The control
values are referenced by name in the \Co{control} list and  only
the ones to be modified from the default need to be
specified. Available values include: \Co{drawLine}, a logical
value indicating whether a \Co{loess} smoother should be added to
the scatter plots and a line connecting the medians should be added
to the boxplots (default is \Co{TRUE}); \Co{span.loess}, used
as the \Co{span} argument in the call to \Co{panel.loess}
(default is \Co{2/3}); \Co{degree.loess}, used as the
\Co{degree} argument in the call to \Co{panel.loess} (default
is \Co{1}); \Co{cex.axis}, the character expansion factor for
the x-axis (default is \Co{0.8}); \Co{srt.axis}, the rotation
factor for the x-axis (default is \Co{0}); and \Co{mgp.axis}, the
margin parameters for the x-axis (default is \Co{c(2, 0.5, 0)}).
\item[\Co{..:}]
optional arguments passed to the Trellis \Co{dotplot}
function.
\end{Argument}
\Paragraph{VALUE}
a Trellis plot of the estimated random-effects (coefficients) versus
covariates, or groups.
\Paragraph{NOTE}
This function requires the \Co{trellis} library.
\Paragraph{SEE ALSO}
\Co{lmList}, \Co{dotplot}
\need 15pt
\Paragraph{EXAMPLE}
\vspace{-16pt} 
\begin{Example}
fm1 <- lmList(distance {\Twiddle} age | Subject, Orthodont)
plot(ranef(fm1))
fm1RE <- ranef(fm1, aug = TRUE)
plot(fm1RE, form = {\Twiddle} Sex)
plot(fm1RE, form = age {\Twiddle} Sex)
\end{Example}
\end{Helpfile}
\begin{Helpfile}{plot.Variogram}{Plot a Variogram Object}
an \Co{xyplot} of the semi-variogram versus the distances is
produced. If \Co{smooth = TRUE}, a \Co{loess} smoother is added to
the plot. If \Co{showModel = TRUE} and object includes an
\Co{"modelVariog"} attribute, the corresponding semi-variogram
is added to the plot.
\begin{Example}
plot(object, smooth, showModel, sigma, span, xlab, ylab, type, ylim, ...)
\end{Example}
\begin{Argument}{ARGUMENTS}
\item[\Co{object:}]
an object inheriting from class \Co{Variogram},
consisting of a data frame with two columns named \Co{variog} and
\Co{dist}, representing the semi-variogram values and the corresponding
distances.
\item[\Co{smooth:}]
an optional logical value controlling whether a
\Co{loess} smoother should be added to the plot. Defaults to
\Co{TRUE}, when \Co{showModel} is \Co{FALSE}.
\item[\Co{showModel:}]
an optional logical value controlling whether the
semi-variogram corresponding to an \Co{"modelVariog"} attribute of
\Co{object}, if any is present, should be added to the
plot. Defaults to \Co{TRUE}, when the \Co{"modelVariog"}
attribute is present.
\item[\Co{sigma:}]
an optional numeric value used as the height of a
horizontal line displayed in the plot. Can be used to represent the
process standard deviation Default is \Co{NULL}, implying that no
horizontal line is drawn. 
\item[\Co{span:}]
an optional numeric value with the smoothing parameter for
the \Co{loess} fit. Default is 0.6.
\item[\Co{xlab,ylab:}]
optional character strings with the x- and y-axis
labels. Default respectively to \Co{"Distance"} and
\Co{"Semivariogram"}. 
\item[\Co{type:}]
an optional character with the type of plot. Defaults to
"p".
\item[\Co{ylim:}]
an optional numeric vector with the limits for the
y-axis. Defaults to \Co{c(0, max(object\$variog))}.
\item[\Co{...:}]
optional arguments passed to the Trellis plot function.
\end{Argument}
\Paragraph{VALUE}
an \Co{xyplot} Trellis plot.
\Paragraph{NOTE}
This function requires the \Co{trellis} library.
\Paragraph{SEE ALSO}
\Co{Variogram}, \Co{xyplot}, \Co{loess}
\need 15pt
\Paragraph{EXAMPLE}
\vspace{-16pt}
\begin{Example}
fm1 <- lme(follicles {\Twiddle} sin(2*pi*Time) + cos(2*pi*Time), Ovary)
plot(Variogram(fm1, form = {\Twiddle} Time | Mare, maxDist = 0.7))
\end{Example}
\end{Helpfile}
\begin{Helpfile}{pooledSD}{Extract Pooled Standard Deviation}
The pooled estimated standard deviation is obtained by adding together
the residual sum of squares for each non-null element of
\Co{object}, dividing by the sum of the corresponding residual
degrees-of-freedom, and taking the square-root.
\begin{Example}
pooledSD(object)
\end{Example}
\begin{Argument}{ARGUMENTS}
\item[\Co{object:}]
an object inheriting from class \Co{lmList}.
\end{Argument}
\Paragraph{VALUE}
the pooled standard deviation for the non-null elements of
\Co{object}, with an attribute \Co{df} with the number of
degrees-of-freedom used in the estimation.
\Paragraph{SEE ALSO}
\Co{lmList}, \Co{lm}
\need 15pt
\Paragraph{EXAMPLE}
\vspace{-16pt} 
\begin{Example}
fm1 <- lmList(Orthodont)
pooledSD(fm1)
\end{Example}
\end{Helpfile}
\begin{Helpfile}{predict.gls}{Predictions from a gls Object}
The predictions for the linear model represented by \Co{object} are
obtained at the covariate values defined in \Co{newdata}.
\begin{Example}
predict(object, newdata, na.action)
\end{Example}
\begin{Argument}{ARGUMENTS}
\item[\Co{object:}]
an object inheriting from class \Co{gls}, representing
a generalized least squares fitted linear model.
\item[\Co{newdata:}]
an optional data frame to be used for obtaining the
predictions. All variables used in the linear model must be present
in the data frame. If missing, the fitted values are returned.
\item[\Co{na.action:}]
a function that indicates what should happen when
\Co{newdata} contains \Co{NA}s.  The default action
(\Co{na.fail}) causes the function to print an error message and
terminate if there are any incomplete observations.
\end{Argument}
\Paragraph{VALUE}
a vector with the predicted values.
\Paragraph{SEE ALSO}
\Co{gls}, \Co{fitted.gls}
\need 15pt
\Paragraph{EXAMPLE}
\vspace{-16pt} 
\begin{Example}
fm1 <- gls(follicles \Twiddle sin(2*pi*Time) + cos(2*pi*Time), Ovary,
           correlation = corAR1(form = \Twiddle 1 | Mare))
newOvary <- data.frame(Time = c(-0.75, -0.5, 0, 0.5, 0.75))
predict(fm1, newOvary)
\end{Example}
\end{Helpfile}
\begin{Helpfile}{predict.gnls}{Predictions from a gnls Object}
The predictions for the nonlinear model represented by \Co{object} are
obtained at the covariate values defined in \Co{newdata}.
\begin{Example}
predict(object, newdata, na.action, naPattern)
\end{Example}
\begin{Argument}{ARGUMENTS}
\item[\Co{object:}]
an object inheriting from class \Co{gnls}, representing
a generalized nonlinear least squares fitted model.
\item[\Co{newdata:}]
an optional data frame to be used for obtaining the
predictions. All variables used in the nonlinear model must be present
in the data frame. If missing, the fitted values are returned.
\item[\Co{na.action:}]
a function that indicates what should happen when
\Co{newdata} contains \Co{NA}s.  The default action
(\Co{na.fail}) causes the function to print an error message and
terminate if there are any incomplete observations.
\item[\Co{naPattern:}]
an expression or formula object, specifying which returned
values are to be regarded as missing.
\end{Argument}
\Paragraph{VALUE}
a vector with the predicted values.
\Paragraph{SEE ALSO}
\Co{gnls}, \Co{fitted.gnls}
\need 15pt
\Paragraph{EXAMPLE}
\vspace{-16pt}
\begin{Example}
fm1 <- gnls(weight {\Twiddle} SSlogis(Time, Asym, xmid, scal), Soybean,
            weights = varPower())
newSoybean <- data.frame(Time = c(10,30,50,80,100))
predict(fm1, newSoybean)
\end{Example}
\end{Helpfile}
\begin{Helpfile}{predict.lmList}{Predictions from an lmList Object}
If the grouping factor corresponding to \Co{object} is included
in \Co{newdata}, the data frame is partitioned according to the
grouping factor levels; else, \Co{newdata} is repeated for all
\Co{lm} components. The predictions and, optionally, the standard
errors for the predictions, are obtained for each \Co{lm}
component of \Co{object}, using the corresponding element of the
partitioned \Co{newdata}, and arranged into a list with as many
components as \Co{object}, or combined into a single vector or data
frame (if \Co{se.fit=TRUE}).
\begin{Example}
predict(object, newdata, subset, pool, asList, se.fit)
\end{Example}
\begin{Argument}{ARGUMENTS}
\item[\Co{object:}]
an object inheriting from class \Co{lmList}, representing
a list of \Co{lm} objects with a common model.
\item[\Co{newdata:}]
an optional data frame to be used for obtaining the
predictions. All variables used in the \Co{object} model formula
must be present in the data frame. If missing, the same data frame
used to produce \Co{object} is used.
\item[\Co{subset:}]
an optional character or integer vector naming the
\Co{lm} components of \Co{object} from which the predictions
are to be extracted. Default is \Co{NULL}, in which case all
components are used.
\item[\Co{asList:}]
an optional logical value. If \Co{TRUE}, the returned
object is a list with the predictions split by groups; else the
returned value is a vector. Defaults to \Co{FALSE}.
\item[\Co{pool:}]
an optional logical value indicating whether a pooled
estimate of the residual standard error should be used. Default is
\Co{attr(object, "pool")}.
\item[\Co{se.fit:}]
an optional logical value indicating whether pointwise
standard errors should be computed along with the
predictions. Default is \Co{FALSE}.
\end{Argument}
\Paragraph{VALUE}
a list with components given by the predictions (and, optionally, the
standard errors for the predictions) from each \Co{lm}
component of \Co{object},  a vector with the predictions from all 
\Co{lm} components of \Co{object}, or a data frame with columns
given by the predictions and their corresponding standard errors.
\Paragraph{SEE ALSO}
\Co{lmList}, \Co{predict.lm}
\need 15pt
\Paragraph{EXAMPLE}
\vspace{-16pt}
\begin{Example}
fm1 <- lmList(distance {\Twiddle} age | Subject, Orthodont)
predict(fm1, se.fit = TRUE)
\end{Example}
\end{Helpfile}
\begin{Helpfile}{predict.lme}{Predictions from an lme Object}
The predictions at level i are obtained by adding together the
population predictions (based only on the fixed effects estimates)
and the estimated contributions of the random effects to the
predictions at grouping levels less or equal to i. The resulting
values estimate the best linear unbiased predictions (BLUPs) at level
i. If group values not included in the original grouping factors
are present in \Co{newdata}, the corresponding predictions will be
set to \Co{NA} for levels greater or equal to the level at which the
unknown groups occur.
\begin{Example}
predict(object, newdata, level, asList, na.action)
\end{Example}
\begin{Argument}{ARGUMENTS}
\item[\Co{object:}]
an object inheriting from class \Co{lme}, representing
a fitted linear mixed-effects model.
\item[\Co{newdata:}]
an optional data frame to be used for obtaining the
predictions. All variables used in the fixed and random effects
models, as well as the grouping factors, must be present in the data
frame. If missing, the fitted values are returned.
\item[\Co{level:}]
an optional integer vector giving the level(s) of grouping
to be used in obtaining the predictions. Level values increase from
outermost to innermost grouping, with level zero corresponding to the
population predictions. Defaults to the highest or innermost level of
grouping.
\item[\Co{asList:}]
an optional logical value. If \Co{TRUE} and a single
value is given in \Co{level}, the returned object is a list with
the predictions split by groups; else the returned value is
either a vector or a data frame, according to the length of
\Co{level}. 
\item[\Co{na.action:}]
a function that indicates what should happen when
\Co{newdata} contains \Co{NA}s.  The default action
(\Co{na.fail}) causes the function to print an error message and
terminate if there are any incomplete observations.
\end{Argument}
\Paragraph{VALUE}
if a single level of grouping is specified in \Co{level}, the
returned value is either a list with the predictions split by groups
(\Co{asList = TRUE}) or a vector with the predictions
(\Co{asList = FALSE}); else, when multiple grouping levels are
specified in \Co{level}, the returned object is a data frame with
columns given by the predictions at different levels and the grouping
factors.
\Paragraph{SEE ALSO}
\Co{lme}, \Co{fitted.lme}
\need 15pt
\Paragraph{EXAMPLE}
\vspace{-16pt} 
\begin{Example}
fm1 <- lme(distance \Twiddle age, Orthodont, random = \Twiddle age | Subject)
newOrth <- data.frame(Sex = c("Male","Male","Female","Female","Male","Male"),
                      age = c(15, 20, 10, 12, 2, 4),
                      Subject = c("M01","M01","F30","F30","M04","M04"))
predict(fm1, newOrth, level = 0:1)
\end{Example}
\end{Helpfile}
\begin{Helpfile}{predict.nlme}{Predictions from an nlme Object}
The predictions at level i are obtained by adding together the
contributions from the estimated fixed effects and the estimated
random effects at levels less or equal to i and evaluating the
model function at the resulting estimated parameters. If group values
not included in the original grouping factors  are present in
\Co{newdata}, the corresponding predictions will be  set to
\Co{NA} for levels greater or equal to the level at which the 
unknown groups occur.
\begin{Example}
predict(object, newdata, level, asList, na.action, naPattern)
\end{Example}
\begin{Argument}{ARGUMENTS}
\item[\Co{object:}]
an object inheriting from class \Co{nlme}, representing
a fitted nonlinear mixed-effects model.
\item[\Co{newdata:}]
an optional data frame to be used for obtaining the
predictions. All variables used in the nonlinear model, the fixed and
the random effects models, as well as the grouping factors, must be
present in the data frame. If missing, the fitted values are returned.
\item[\Co{level:}]
an optional integer vector giving the level(s) of grouping
to be used in obtaining the predictions. Level values increase from
outermost to innermost grouping, with level zero corresponding to the
population predictions. Defaults to the highest or innermost level of
grouping.
\item[\Co{asList:}]
an optional logical value. If \Co{TRUE} and a single
value is given in \Co{level}, the returned object is a list with
the predictions split by groups; else the returned value is
either a vector or a data frame, according to the length of
\Co{level}. 
\item[\Co{na.action:}]
a function that indicates what should happen when
\Co{newdata} contains \Co{NA}s.  The default action
(\Co{na.fail}) causes the function to print an error message and
terminate if there are any incomplete observations.
\item[\Co{naPattern:}]
an expression or formula object, specifying which returned
values are to be regarded as missing.
\end{Argument}
\Paragraph{VALUE}
if a single level of grouping is specified in \Co{level}, the
returned value is either a list with the predictions split by groups
(\Co{asList = TRUE}) or a vector with the predictions
(\Co{asList = FALSE}); else, when multiple grouping levels are
specified in \Co{level}, the returned object is a data frame with
columns given by the predictions at different levels and the grouping
factors.
\Paragraph{SEE ALSO}
\Co{nlme}, \Co{fitted.nlme}
\need 15pt
\Paragraph{EXAMPLE}
\vspace{-16pt}
\begin{Example}
fm1 <- nlme(weight {\Twiddle} SSlogis(Time, Asym, xmid, scal), 
            data = Soybean, fixed = Asym + xmid + scal {\Twiddle} 1, 
            start = c(18, 52, 7.5))
newSoybean <- data.frame(Time = c(10,30,50,80,100),
     Plot = c("1988F1", "1988F1","1988F1", "1988F1","1988F1"))
predict(fm1, newSoybean, level = 0:1)
\end{Example}
\end{Helpfile}
\begin{Helpfile}{print.anova.lme}{Print an anova.lme Object}
When only one fitted model object is used in the call to
\Co{anova.lme}, a data frame with the estimated values, the
approximate standard errors, the z-ratios, and the  p-values for the
fixed effects is printed. Otherwise, when multiple fitted objects are
being compared, a data frame with the degrees of freedom, the
(restricted) log-likelihood, the Akaike Information Criterion (AIC),
and the Bayesian Information Criterion (BIC) of each fitted model
object is printed. If included in \Co{x}, likelihood ratio
statistics, with associated p-values, are included in the output.
\begin{Example}
print(x, verbose)
\end{Example}
\begin{Argument}{ARGUMENTS}
\item[\Co{x:}]
an object inheriting from class \Co{anova.lme},
generally obtained by applying the \Co{anova.lme} method to an
\Co{lme} object.
\item[\Co{verbose:}]
an optional logical value. If \Co{TRUE}, the calling
sequences for each fitted model object are printed with the rest of
the output, being omitted if \Co{verbose = FALSE}. Defaults to
\Co{attr(x, "verbose")}.
\end{Argument}
\Paragraph{SEE ALSO}
\Co{anova.lme}, \Co{lme}
\need 15pt
\Paragraph{EXAMPLE}
\vspace{-16pt} 
\begin{Example}
fm1 <- lme(distance ~ age * Sex, Orthodont, random = ~ age | Subject)
print(anova(fm1))           # single argument form
fm2 <- update(fm1, random = ~ 1 | Subject)
print(anova(fm1, fm2))      # multiple argument form
\end{Example}
\end{Helpfile}
\begin{Helpfile}{print.corStruct}{Print a corStruct Object}
If \Co{x} has been initialized, its coefficients are printed in
constrained form.
\begin{Example}
print(x, ...)
\end{Example}
\begin{Argument}{ARGUMENTS}
\item[\Co{x:}]
an object inheriting from class \Co{corStruct}, representing
a correlation structure.
\item[\Co{...:}]
optional arguments passed to \Co{print.default}; see
the documentation on that method function.
\end{Argument}
\Paragraph{VALUE}
the printed coefficients of \Co{x} in constrained form.
\Paragraph{SEE ALSO}
\Co{print.default}, \Co{coef.corStruct}
\need 15pt
\Paragraph{EXAMPLE}
\vspace{-16pt} 
\begin{Example}
cs1 <- corAR1(0.3)
print(cs1)
\end{Example}
\end{Helpfile}
\begin{Helpfile}{print.gls}{Print a gls Object}
Information describing the fitted linear model represented by \Co{x}
is printed. This includes the coefficients, correlation and variance
function parameters, if any are present, and the residual standard
error.
\begin{Example}
print(x, ...)
\end{Example}
\begin{Argument}{ARGUMENTS}
\item[\Co{x:}]
an object inheriting from class \Co{gls}, representing
a generalized least squares fitted linear model.
\item[\Co{...:}]
optional arguments passed to \Co{print.default}; see
the documentation on that method function.
\end{Argument}
\Paragraph{SEE ALSO}
\Co{gls}, \Co{print.summary.gls}
\need 15pt
\Paragraph{EXAMPLE}
\vspace{-16pt} 
\begin{Example}
fm1 <- gls(follicles \Twiddle sin(2*pi*Time) + cos(2*pi*Time), Ovary,
           correlation = corAR1(form = \Twiddle 1 | Mare))
print(fm1)
\end{Example}
\end{Helpfile}
\begin{Helpfile}{print.groupedData}{Print a groupedData Object}
Prints the display formula and the data frame associated with
\Co{object}.
\begin{Example}
print(x, ...)
\end{Example}
\begin{Argument}{ARGUMENTS}
\item[\Co{x:}]
an object inheriting from class \Co{groupedData}.
\item[\Co{...:}]
optional arguments passed to \Co{print.default}; see
the documentation on that method function.
\end{Argument}
\Paragraph{SEE ALSO}
\Co{groupedData}
\need 15pt
\Paragraph{EXAMPLE}
\vspace{-16pt} 
\begin{Example}
print(Orthodont)
\end{Example}
\end{Helpfile}
\begin{Helpfile}{print.intervals.gls}{Print an intervals.gls Object}
The individual components of \Co{x} are printed.
\begin{Example}
print(x, ...)
\end{Example}
\begin{Argument}{ARGUMENTS}
\item[\Co{x:}]
an object inheriting from class \Co{intervals.gls},
representing a list of data frames with confidence intervals and
estimates for the coefficients in the \Co{gls} object that produced
\Co{x}.
\item[\Co{...:}]
optional arguments passed to \Co{print.default}; see
the documentation on that method function.
\end{Argument}
\Paragraph{SEE ALSO}
\Co{intervals.gls}
\need 15pt
\Paragraph{EXAMPLE}
\vspace{-16pt} 
\begin{Example}
fm1 <- gls(follicles \Twiddle sin(2*pi*Time) + cos(2*pi*Time), Ovary,
           correlation = corAR1(form = \Twiddle 1 | Mare))
print(intervals(fm1))
\end{Example}
\end{Helpfile}
\begin{Helpfile}{print.intervals.lme}{Print an intervals.lme Object}
The individual components of \Co{x} are printed.
\begin{Example}
print(x, ...)
\end{Example}
\begin{Argument}{ARGUMENTS}
\item[\Co{x:}]
an object inheriting from class \Co{intervals.lme},
representing a list of data frames with confidence intervals and
estimates for the coefficients in the \Co{lme} object that produced
\Co{x}.
\item[\Co{...:}]
optional arguments passed to \Co{print.default}; see
the documentation on that method function.
\end{Argument}
\Paragraph{SEE ALSO}
\Co{intervals.lme}
\need 15pt
\Paragraph{EXAMPLE}
\vspace{-16pt} 
\begin{Example}
fm1 <- lme(distance \Twiddle age, Orthodont, random = \Twiddle age | Subject)
print(intervals(fm1))
\end{Example}
\end{Helpfile}
\begin{Helpfile}{print.lmList}{Print an lmList Object}
Information describing the individual \Co{lm} fits corresponding to
\Co{object} is printed. This includes the estimated coefficients and
the residual standard error.
\begin{Example}
print(x, pool, ...)
\end{Example}
\begin{Argument}{ARGUMENTS}
\item[\Co{x:}]
an object inheriting from class \Co{lmList}, representing
a list of fitted \Co{lm} objects.
\item[\Co{pool:}]
an optional logical value indicating whether a pooled
estimate of the residual standard error should be used. Default is
\Co{attr(object, "pool")}.
\item[\Co{...:}]
optional arguments passed to \Co{print.default}; see
the documentation on that method function.
\end{Argument}
\Paragraph{SEE ALSO}
\Co{lmList}
\need 15pt
\Paragraph{EXAMPLE}
\vspace{-16pt}
\begin{Example}
fm1 <- lmList(Orthodont)
print(fm1)
\end{Example}
\end{Helpfile}
\begin{Helpfile}{print.lme}{Print an lme Object}
Information describing the fitted linear mixed-effects model
represented by \Co{x} is printed. This includes the fixed
effects, the standard deviations and correlations for the
random effects, the within-group correlation and variance
function parameters, if any are present, and the within-group standard
deviation.
\begin{Example}
print(x, ...)
\end{Example}
\begin{Argument}{ARGUMENTS}
\item[\Co{x:}]
an object inheriting from class \Co{lme}, representing
a fitted linear mixed-effects model.
\item[\Co{...:}]
optional arguments passed to \Co{print.default}; see
the documentation on that method function.
\end{Argument}
\Paragraph{SEE ALSO}
\Co{lme}, \Co{print.summary.lme}
\need 15pt
\Paragraph{EXAMPLE}
\vspace{-16pt} 
\begin{Example}
fm1 <- lme(distance \Twiddle age, Orthodont, random = \Twiddle age | Subject)
print(fm1)
\end{Example}
\end{Helpfile}
\begin{Helpfile}{print.modelStruct}{Print a modelStruct Object}
This method function applies \Co{print} to each element of
\Co{object}.
\begin{Example}
print(x, ...)
\end{Example}
\begin{Argument}{ARGUMENTS}
\item[\Co{x:}]
an object inheriting from class \Co{modelStruct},
representing a list of model components, such as \Co{corStruct} and
\Co{varFunc} objects.
\item[\Co{...:}]
optional arguments passed to \Co{print.default}; see
the documentation on that method function.
\end{Argument}
\Paragraph{VALUE}
the printed elements of \Co{object}.
\Paragraph{SEE ALSO}
\Co{print}
\need 15pt
\Paragraph{EXAMPLE}
\vspace{-16pt} 
\begin{Example}
lms1 <- lmeStruct(reStruct = reStruct(pdDiag(diag(2), \Twiddle age)),
   corStruct = corAR1(0.3))
print(lms1)
\end{Example}
\end{Helpfile}
\begin{Helpfile}{print.pdMat}{Print a pdMat Object}
Print the standard deviations and correlations (if any) associated the
positive-definite matrix represented by \Co{x} (considered as a
variance-covariance matrix).
\begin{Example}
print(x, ...)
\end{Example}
\begin{Argument}{ARGUMENTS}
\item[\Co{object:}]
an object inheriting from class \Co{pdMat}, representing
a positive definite matrix.
\item[\Co{...:}]
optional arguments passed to \Co{print.default}; see
the documentation on that method function.
\end{Argument}
\Paragraph{SEE ALSO}
\Co{print.summary.pdMat}
\need 15pt
\Paragraph{EXAMPLE}
\vspace{-16pt} 
\begin{Example}
pd1 <- pdSymm(diag(1:3), nam = c("A","B","C"))
print(pd1)
\end{Example}
\end{Helpfile}
\begin{Helpfile}{print.reStruct}{Print an reStruct Object}
Each \Co{pdMat} component of \Co{object} is printed, together with
its formula and the associated grouping level.
\begin{Example}
print(x, sigma, reEstimates, verbose=F, ...)
\end{Example}
\begin{Argument}{ARGUMENTS}
\item[\Co{x:}]
an object inheriting from class \Co{reStruct},
representing a random effects structure and consisting of a list of
\Co{pdMat} objects.
\item[\Co{sigma:}]
an optional numeric value used as a multiplier for
the square-root factors of the \Co{pdMat} components (usually the
estimated within-group standard deviation from a mixed-effects
model). Defaults to 1.
\item[\Co{reEstimates:}]
an optional list with the random effects estimates
for each level of grouping. Only used when \Co{verbose = TRUE}.
\item[\Co{verbose:}]
an optional logical value determining if the random
effects estimates should be printed. Defaults to \Co{FALSE}.
\item[\Co{...:}]
optional arguments passed to \Co{print.default}; see
the documentation on that method function.
\end{Argument}
\Paragraph{SEE ALSO}
\Co{reStruct}
\need 15pt
\Paragraph{EXAMPLE}
\vspace{-16pt} 
\begin{Example}
rs1 <- reStruct(list(Dog = \Twiddle day, Side = \Twiddle 1), data = Pixel)
matrix(rs1) <- list(diag(2), 3)
print(rs1)
\end{Example}
\end{Helpfile}
\begin{Helpfile}{print.summary.corStruct}{Print summary.corStruct}
This method function prints the constrained coefficients of
an initialized \Co{corStruct} object, with a header specifying the
type of correlation structure associated with the object.
\begin{Example}
print(x, ...)
\end{Example}
\begin{Argument}{ARGUMENTS}
\item[\Co{x:}]
an object inheriting from class \Co{summary.corStruct},
generally resulting from applying \Co{summary} to an object
inheriting from class \Co{corStruct}.
\item[\Co{...:}]
optional arguments passed to \Co{print.default}; see
the documentation on that method function.
\end{Argument}
\Paragraph{VALUE}
the printed coefficients of \Co{x} in constrained form, with a
header specifying the associated correlation structure type.
\Paragraph{SEE ALSO}
\Co{summary.corStruct}
\need 15pt
\Paragraph{EXAMPLE}
\vspace{-16pt} 
\begin{Example}
cs1 <- corAR1(0.3)
print(summary(cs1))
\end{Example}
\end{Helpfile}
\begin{Helpfile}{print.summary.gls}{Print a summary.gls Object}
Information summarizing the fitted linear model represented by
\Co{x} is printed. This includes the AIC, BIC, and 
log-likelihood at convergence, the coefficient estimates and
their respective standard errors,  correlation and variance
function parameters, if any are present, and the residual standard
error.
\begin{Example}
print(x, verbose, ...)
\end{Example}
\begin{Argument}{ARGUMENTS}
\item[\Co{x:}]
an object inheriting from class \Co{summary.gls},
representing a summarized \Co{gls} object.
\item[\Co{verbose:}]
an optional logical value used to control the amount of
printed output. Defaults to \Co{FALSE}.
\item[\Co{...:}]
optional arguments passed to \Co{print.default}; see
the documentation on that method function.
\end{Argument}
\Paragraph{SEE ALSO}
\Co{summary.gls}, \Co{gls}
\need 15pt
\Paragraph{EXAMPLE}
\vspace{-16pt} 
\begin{Example}
fm1 <- gls(follicles \Twiddle sin(2*pi*Time) + cos(2*pi*Time), Ovary,
           correlation = corAR1(form = \Twiddle 1 | Mare))
print(summary(fm1))
\end{Example}
\end{Helpfile}
\begin{Helpfile}{print.summary.lmList}{Print a summary.lmList Object}
Information summarizing the individual \Co{lm} fitted objects
corresponding to \Co{x} is printed. This includes the estimated
coefficients and their respective standard errors, t-values, and
p-values.
\begin{Example}
print(x, ...)
\end{Example}
\begin{Argument}{ARGUMENTS}
\item[\Co{x:}]
an object inheriting from class \Co{summary.lmList},
representing a summarized \Co{lme} object.
\item[\Co{...:}]
optional arguments passed to \Co{print.default}; see
the documentation on that method function.
\end{Argument}
\Paragraph{SEE ALSO}
\Co{summary.lmList}, \Co{lmList}
\need 15pt
\Paragraph{EXAMPLE}
\vspace{-16pt}
\begin{Example}
fm1 <- lmList(distance {\Twiddle} age | Subject, Orthodont)
print(summary(fm1))
\end{Example}
\end{Helpfile}
\begin{Helpfile}{print.summary.lme}{Print a summary.lme Object}
Information summarizing the fitted linear mixed-effects model
represented by \Co{x} is printed. This includes the AIC, BIC, and
log-likelihood at convergence, the fixed effects estimates and
respective standard errors, the standard deviations and correlations
for the random effects, the within-group correlation and variance
function parameters, if any are present, and the within-group standard
deviation.
\begin{Example}
print(x, verbose, ...)
\end{Example}
\begin{Argument}{ARGUMENTS}
\item[\Co{x:}]
an object inheriting from class \Co{summary.lme},
representing a summarized \Co{lme} object.
\item[\Co{verbose:}]
an optional logical value used to control the amount of
printed output. Defaults to \Co{FALSE}.
\item[\Co{...:}]
optional arguments passed to \Co{print.default}; see
the documentation on that method function.
\end{Argument}
\Paragraph{SEE ALSO}
\Co{summary.lme}, \Co{lme}
\need 15pt
\Paragraph{EXAMPLE}
\vspace{-16pt} 
\begin{Example}
fm1 <- lme(distance \Twiddle age, Orthodont, random = \Twiddle age | Subject)
print(summary((fm1)))
\end{Example}
\end{Helpfile}
\begin{Helpfile}{print.summary.modelStruct}{Print summary.modelStruct}
This method function prints the constrained coefficients of
an initialized \Co{modelStruct} object, with a header specifying the
type of correlation structure associated with the object.
\begin{Example}
print(x, ...)
\end{Example}
\begin{Argument}{ARGUMENTS}
\item[\Co{x:}]
an object inheriting from class \Co{summary.modelStruct},
generally resulting from applying \Co{summary} to an object
inheriting from class \Co{modelStruct}.
\item[\Co{...:}]
optional arguments passed to \Co{print.default}; see
the documentation on that method function.
\end{Argument}
\Paragraph{VALUE}
the printed coefficients of \Co{x} in constrained form, with a
header specifying the associated correlation structure type.
\Paragraph{SEE ALSO}
\Co{summary.modelStruct}
\need 15pt
\Paragraph{EXAMPLE}
\vspace{-16pt} 
\begin{Example}
cs1 <- corAR1(0.3)
print(summary(cs1))
\end{Example}
\end{Helpfile}
\begin{Helpfile}{print.summary.pdMat}{Print summary.pdMat}
The standard deviations and correlations associated with the
positive-definite matrix represented by \Co{object} (considered as a
variance-covariance matrix) are printed, together with the formula and
the grouping level associated \Co{object}, if any are present.
\begin{Example}
print(x, sigma, rdig, Level, resid, ...)
\end{Example}
\begin{Argument}{ARGUMENTS}
\item[\Co{x:}]
an object inheriting from class \Co{summary.pdMat},
generally resulting from applying \Co{summary} to an object
inheriting from class \Co{pdMat}.
\item[\Co{sigma:}]
an optional numeric value used as a multiplier for
the square-root factor of the positive-definite matrix represented by
\Co{object} (usually the estimated within-group standard deviation
from a mixed-effects model). Defaults to 1.
\item[\Co{rdig:}]
an optional integer value with the number of significant
digits to be used in printing correlations. Defaults to 3.
\item[\Co{Level:}]
an optional character string with a description of the
grouping level associated with \Co{object} (generally corresponding
to levels of grouping in a mixed-effects model). Defaults to NULL.
\item[\Co{resid:}]
an optional logical value. If \Co{TRUE} an extra row
with the \Co{"residual"} standard deviation given in \Co{sigma}
will be included in the output. Defaults to \Co{FALSE}.
\item[\Co{...:}]
optional arguments passed to \Co{print.default}; see
the documentation on that method function.
\end{Argument}
\Paragraph{SEE ALSO}
\Co{summary.pdMat},\Co{pdMat}
\need 15pt
\Paragraph{EXAMPLE}
\vspace{-16pt} 
\begin{Example}
pd1 <- pdCompSymm(3 * diag(3) + 1, form = \Twiddle age + age^2,
         data = Orthodont)
print(summary(pd1), sigma = 1.2, resid = TRUE)
\end{Example}
\end{Helpfile}
\begin{Helpfile}{print.summary.varFunc}{Print summary.varFunc}
The variance function structure description, the formula and the
coefficients  associated with \Co{x} are printed.
\begin{Example}
print(x, header, ...)
\end{Example}
\begin{Argument}{ARGUMENTS}
\item[\Co{x:}]
an object inheriting from class \Co{varFunc}, representing a
variance function structure.
\item[\Co{header:}]
an optional logical value controlling whether a header
should be included with the rest of the output. Defaults to
\Co{TRUE}.
\item[\Co{...:}]
optional arguments passed to \Co{print.default}; see
the documentation on that method function.
\end{Argument}
\Paragraph{SEE ALSO}
\Co{summary.varFunc}
\need 15pt
\Paragraph{EXAMPLE}
\vspace{-16pt} 
\begin{Example}
vf1 <- varPower(0.3, form = \Twiddle age)
vf1 <- initialize(vf1, Orthodont)
print(summary(vf1))
\end{Example}
\end{Helpfile}
\begin{Helpfile}{print.varFunc}{Print a varFunc Object}
The class and the coefficients associated with \Co{x} are printed.
\begin{Example}
print(x, ...)
\end{Example}
\begin{Argument}{ARGUMENTS}
\item[\Co{x:}]
an object inheriting from class \Co{varFunc}, representing a
variance function structure.
\item[\Co{...:}]
optional arguments passed to \Co{print.default}; see
the documentation on that method function.
\end{Argument}
\Paragraph{SEE ALSO}
\Co{summary.varFunc},
\Co{print.summary.varFunc}
\need 15pt
\Paragraph{EXAMPLE}
\vspace{-16pt} 
\begin{Example}
vf1 <- varPower(0.3, form = \Twiddle age)
vf1 <- initialize(vf1, Orthodont)
print(vf1)
\end{Example}
\end{Helpfile}
\begin{Helpfile}{pruneLevels}{Prune Factor Levels}
The \Co{levels} attribute of \Co{object} are pruned to contain
only the levels occurring in the factor.
\begin{Example}
pruneLevels(object)
\end{Example}
\begin{Argument}{ARGUMENTS}
\item[\Co{object:}]
an object inheriting from class \Co{factor}.
\end{Argument}
\Paragraph{VALUE}
an object identical to \Co{object}, but with the \Co{levels}
attribute containing only value occurring in the factor.
\Paragraph{SEE ALSO}
\Co{factor}, \Co{ordered}
\need 15pt
\Paragraph{EXAMPLE}
\vspace{-16pt} 
\begin{Example}
f1 <- factor(c(1,1,2,3,3,4,5))
levels(f1)
f2 <- f1[4:7]
levels(f2)
levels(pruneLevels(f2))
\end{Example}
\end{Helpfile}
\begin{Helpfile}{qqnorm.gls}{Normal Plot of gls Residuals}
Diagnostic plots for assessing the normality of residuals the
generalized least squares fit are obtained. The \Co{form} argument
gives considerable flexibility in the type of plot specification. A
conditioning expression (on the right side of a \Co{|} operator)
always implies that different panels are used for each level of the
conditioning factor, according to a Trellis display.
\begin{Example}
qqnorm(object, form, abline, id, idLabels, grid, ...)
\end{Example}
\begin{Argument}{ARGUMENTS}
\item[\Co{object:}]
an object inheriting from class \Co{gls}, representing
a generalized least squares fitted model.
\item[\Co{form:}]
an optional one-sided formula specifying the desired type of
plot. Any variable present in the original data frame used to obtain
\Co{object} can be referenced. In addition, \Co{object} itself
can be referenced in the formula using the symbol
\Co{"."}. Conditional expressions on the right of a \Co{|}
operator can be used to define separate panels in a Trellis
display. The expression on the right hand side of \Co{form} and to
the left of a \Co{|} operator must evaluate to a residuals
vector. Default is \Co{{\Twiddle} resid(., type = "p")}, 
corresponding to a normal plot of the standardized residuals.
\item[\Co{abline:}]
an optional numeric value, or numeric vector of length
two. If given as a single value, a horizontal line will be added to the
plot at that coordinate; else, if given as a vector, its values are
used as the intercept and slope for a line added to the plot. If
missing, no lines are added to the plot.
\item[\Co{id:}]
an optional numeric value, or one-sided formula. If given as
a value, it is used as a significance level for a two-sided outlier
test for the standardized residuals (random effects). Observations with
absolute standardized residuals (random effects) greater than the
1 - value/2 quantile of the standard normal distribution are
identified in the plot using \Co{idLabels}. If given as a one-sided
formula, its right hand side must evaluate to a  logical, integer, or
character vector which is used to identify observations in the
plot. If missing, no observations are identified.
\item[\Co{idLabels:}]
an optional vector, or one-sided formula. If given as a
vector, it is converted to character and used to label the
observations identified according to \Co{id}. If given as a
one-sided formula, its right hand side must evaluate to a vector
which is converted to character and used to label the identified
observations. Default is the innermost grouping factor.
\item[\Co{grid:}]
an optional logical value indicating whether a grid should
be added to plot. Defaults to \Co{FALSE}.
\item[\Co{...:}]
optional arguments passed to the Trellis plot function.
\end{Argument}
\Paragraph{VALUE}
a diagnostic Trellis plot for assessing normality of residuals.
\Paragraph{NOTE} This function requires the \Co{trellis} library.
\Paragraph{SEE ALSO}
\Co{gls}, \Co{plot.gls}
\need 15pt
\Paragraph{EXAMPLE}
\vspace{-16pt}
\begin{Example}
fm1 <- gls(follicles {\Twiddle} sin(2*pi*Time) + cos(2*pi*Time), Ovary,
           correlation = corAR1(form = {\Twiddle} 1 | Mare))
qqnorm(fm1, abline = c(0,1))
\end{Example}
\end{Helpfile}
\begin{Helpfile}{qqnorm.lme}{Normal Plot of lme Residuals or Random Effects}
Diagnostic plots for assessing the normality of residuals and random
effects in the linear mixed-effects fit are obtained. The
\Co{form} argument gives considerable flexibility in the type of
plot specification. A conditioning expression (on the right side of a
\Co{|} operator) always implies that different panels are used for
each level of the conditioning factor, according to a Trellis
display.
\begin{Example}
qqnorm(object, form, abline, id, idLabels, grid, ...)
\end{Example}
\begin{Argument}{ARGUMENTS}
\item[\Co{object:}]
an object inheriting from class \Co{lme}, representing
a fitted linear mixed-effects model.
\item[\Co{form:}]
an optional one-sided formula specifying the desired type of
plot. Any variable present in the original data frame used to obtain
\Co{object} can be referenced. In addition, \Co{object} itself
can be referenced in the formula using the symbol
\Co{"."}. Conditional expressions on the right of a \Co{|}
operator can be used to define separate panels in a Trellis
display. The expression on the right hand side of \Co{form} and to
the left of a \Co{|} operator must evaluate to a residuals vector,
or a random effects matrix. Default is \Co{{\Twiddle} resid(., type = "p")},
corresponding to a normal plot of the standardized residuals
evaluated at the innermost level of nesting.
\item[\Co{abline:}]
an optional numeric value, or numeric vector of length
two. If given as a single value, a horizontal line will be added to the
plot at that coordinate; else, if given as a vector, its values are
used as the intercept and slope for a line added to the plot. If
missing, no lines are added to the plot.
\item[\Co{id:}]
an optional numeric value, or one-sided formula. If given as
a value, it is used as a significance level for a two-sided outlier
test for the standardized residuals (random effects). Observations with
absolute standardized residuals (random effects) greater than the
1 - value/2 quantile of the standard normal distribution are
identified in the plot using \Co{idLabels}. If given as a one-sided
formula, its right hand side must evaluate to a  logical, integer, or
character vector which is used to identify observations in the
plot. If missing, no observations are identified.
\item[\Co{idLabels:}]
an optional vector, or one-sided formula. If given as a
vector, it is converted to character and used to label the
observations identified according to \Co{id}. If given as a
one-sided formula, its right hand side must evaluate to a vector
which is converted to character and used to label the identified
observations. Default is the innermost grouping factor.
\item[\Co{grid:}]
an optional logical value indicating whether a grid should
be added to plot. Defaults to \Co{FALSE}.
\item[\Co{...:}]
optional arguments passed to the Trellis plot function.
\end{Argument}
\Paragraph{VALUE}
a diagnostic Trellis plot for assessing normality of residuals or
random effects.
\Paragraph{NOTE} This function requires the \Co{trellis} library.
\Paragraph{SEE ALSO}
\Co{lme}, \Co{plot.lme}
\need 15pt
\Paragraph{EXAMPLE}
\vspace{-16pt}
\begin{Example}
fm1 <- lme(distance {\Twiddle} age, Orthodont, random = {\Twiddle} age | Subject)
# normal plot of standardized residuals by gender
qqnorm(fm1, {\Twiddle} resid(., type = "p") | Sex, abline = c(0, 1))
# normal plots of random effects
qqnorm(fm1, {\Twiddle}ranef(.))
\end{Example}
\end{Helpfile}
\begin{Helpfile}{random.effects}{Extract Random Effects}
This function is generic; method functions can be written to handle
specific classes of objects. Classes which already have methods for
this function include \Co{lmList} and \Co{lme}.
\begin{Example}
random.effects(object, ...)
ranef(object, ...)
\end{Example}
\begin{Argument}{ARGUMENTS}
\item[\Co{object:}]
any fitted model object from which random effects
estimates can be extracted.
\item[\Co{...:}]
some methods for this generic function require additional
arguments.
\end{Argument}
\Paragraph{VALUE}
will depend on the method function used; see the appropriate documentation.
\Paragraph{SEE ALSO}
\Co{ranef.lmList},\Co{ranef.lme}
\need 15pt
\Paragraph{EXAMPLE}
\vspace{-16pt}
\begin{Example}
## see the method function documentation
\end{Example}
\end{Helpfile}
\begin{Helpfile}{ranef}{Extract Random Effects}
This function is generic; method functions can be written to handle
specific classes of objects. Classes which already have methods for
this function include \Co{lmList} and \Co{lme}.
\begin{Example}
ranef(object, ...)
\end{Example}
\begin{Argument}{ARGUMENTS}
\item[\Co{object:}]
any fitted model object from which random effects
estimates can be extracted.
\item[\Co{...:}]
some methods for this generic function require additional
arguments.
\end{Argument}
\Paragraph{VALUE}
will depend on the method function used; see the appropriate documentation.
\Paragraph{SEE ALSO}
\Co{ranef.lmList},
\Co{ranef.lme}
\need 15pt
\Paragraph{EXAMPLE}
\vspace{-16pt} 
\begin{Example}
## see the method function documentation
\end{Example}
\end{Helpfile}
\begin{Helpfile}{ranef.lmList}{lmList Random Effects}
The difference between the individual \Co{lm} components
coefficients and their average is calculated.
\begin{Example}
ranef(object)
\end{Example}
\begin{Argument}{ARGUMENTS}
\item[\Co{object:}]
an object inheriting from class \Co{lmList}, representing
a list of \Co{lm} objects with a common model.
\end{Argument}
\Paragraph{VALUE}
a vector with the differences between the individual \Co{lm}
coefficients in \Co{object} and their average.
\Paragraph{SEE ALSO}
\Co{lmList}, \Co{fixef.lmList}
\need 15pt
\Paragraph{EXAMPLE}
\vspace{-16pt} 
\begin{Example}
fm1 <- lmList(distance \Twiddle age, Orthodont, groups = \Twiddle Subject)
ranef(fm1)
\end{Example}
\end{Helpfile}
\begin{Helpfile}{ranef.lme}{lme Random Effects}
The estimated random effects at level i are represented as a
data frame with rows given by the different groups at that level and
columns given by the random effects. If a single level of
grouping is specified, the returned object is a data frame; else, the
returned object is a list of such data frames. Optionally, the
returned data frame(s) may be augmented with covariates summarized
over groups.
\begin{Example}
ranef(object, augFrame, level, data, which, FUN, standard, 
               omitGroupingFactor)
\end{Example}
\begin{Argument}{ARGUMENTS}
\item[\Co{object:}]
an object inheriting from class \Co{lme}, representing
a fitted linear mixed-effects model.
\item[\Co{augFrame:}]
an optional logical value. If \Co{TRUE}, the returned
data frame is augmented with variables defined in \Co{data}; else,
if \Co{FALSE}, only the coefficients are returned. Defaults to
\Co{FALSE}.
\item[\Co{level:}]
an optional vector of positive integers giving the levels
of grouping to be used in extracting the random effects from an
object with multiple nested grouping levels. Defaults to all levels
of grouping.
\item[\Co{data:}]
an optional data frame with the variables to be used for
augmenting the returned data frame when `augFrame =
     TRUE'. Defaults to the data frame used to fit \Co{object}.
\item[\Co{which:}]
an optional positive integer vector specifying which
columns of \Co{data} should be used in the augmentation of the
returned data frame. Defaults to all columns in \Co{data}.
\item[\Co{FUN:}]
an optional summary function or a list of summary functions
to be applied to group-varying variables, when collapsing \Co{data}
by groups.  Group-invariant variables are always summarized by the
unique value that they assume within that group. If \Co{FUN} is a
single function it will be applied to each non-invariant variable by
group to produce the summary for that variable.  If \Co{FUN} is a
list of functions, the names in the list should designate classes of
variables in the frame such as \Co{ordered}, \Co{factor}, or
\Co{numeric}.  The indicated function will be applied to any
group-varying variables of that class.  The default functions to be
used are \Co{mean} for numeric factors, and \Co{Mode} for both
\Co{factor} and \Co{ordered}.  The \Co{Mode} function, defined
internally in \Co{gsummary}, returns the modal or most popular
value of the variable.  It is different from the \Co{mode} function
that returns the S-language mode of the variable.
\item[\Co{standard:}]
an optional logical value indicating whether the
estimated random effects should be "standardized" (i.e.\ divided by
the corresponding estimated standard error). Defaults to
\Co{FALSE}.
\item[\Co{omitGroupingFactor:}]
an optional logical value.  When \Co{TRUE}
the grouping factor itself will be omitted from the group-wise
summary of \Co{data} but the levels of the grouping factor will
continue to be used as the row names for the returned data frame.
Defaults to \Co{FALSE}.
\end{Argument}
\Paragraph{VALUE}
a data frame, or list of data frames, with the estimated 
random effects at the grouping level(s) specified in \Co{level} and,
optionally, other covariates  summarized over groups. The returned
object inherits from classes \Co{ranef.lme} and
\Co{data.frame}.
\Paragraph{SEE ALSO}
\Co{lme}, \Co{fixef.lme},
\Co{coef.lme}, \Co{plot.ranef.lme},
\Co{gsummary}
\need 15pt
\Paragraph{EXAMPLE}
\vspace{-16pt} 
\begin{Example}
fm1 <- lme(distance \Twiddle age, Orthodont, random = \Twiddle age | Subject)
ranef(fm1)
ranef(fm1, augFrame = TRUE)
\end{Example}
\end{Helpfile}
\begin{Helpfile}{recalc}{Recalculate Condensed Linear Model Object}
This function is generic; method functions can be written to handle
specific classes of objects. Classes which already have methods for
this function include: \Co{corStruct}, \Co{modelStruct},
\Co{reStruct}, and \Co{varFunc}.
\begin{Example}
recalc(object, conLin)
\end{Example}
\begin{Argument}{ARGUMENTS}
\item[\Co{object:}]
any object which induces a recalculation of the condensed
linear model object \Co{conLin}.
\item[\Co{conLin:}]
a condensed linear model object, consisting of a list
with components \Co{"Xy"}, corresponding to a regression matrix
(\Co{X}) combined with a response vector (\Co{y}), and
\Co{"logLik"}, corresponding to the log-likelihood of the
underlying model.
\item[\Co{...:}]
some methods for this generic function may require
additional arguments.
\end{Argument}
\Paragraph{VALUE}
the recalculated condensed linear model object.
\Paragraph{NOTE} This function is only used inside model fitting functions
which require recalculation of a condensed linear model object, like
\Co{lme} and \Co{gls}.
\need 15pt
\Paragraph{EXAMPLE}
\vspace{-16pt} 
\begin{Example}
## see the method function documentation
\end{Example}
\end{Helpfile}
\begin{Helpfile}{recalc.corStruct}{Recalculate for corStruct Object}
This method function pre-multiples the \Co{"Xy"} component of
\Co{conLin} by the transpose square-root factor(s) of the
correlation matrix (matrices) associated with \Co{object} and adds
the log-likelihood contribution of \Co{object}, given by
\Co{logLik(object)}, to the \Co{"logLik"} component of
\Co{conLin}.
\begin{Example}
recalc(object, conLin)
\end{Example}
\begin{Argument}{ARGUMENTS}
\item[\Co{object:}]
an object inheriting from class \Co{corStruct},
representing a correlation structure.
\item[\Co{conLin:}]
a condensed linear model object, consisting of a list
with components \Co{"Xy"}, corresponding to a regression matrix
(\Co{X}) combined with a response vector (\Co{y}), and
\Co{"logLik"}, corresponding to the log-likelihood of the
underlying model.
\end{Argument}
\Paragraph{VALUE}
the recalculated condensed linear model object.
\Paragraph{NOTE} This method function is only used inside model fitting functions
which allow correlated error terms, like \Co{lme} and \Co{gls}.
\Paragraph{SEE ALSO}
\Co{corFactor}, \Co{logLik.corStruct}
\end{Helpfile}
\begin{Helpfile}{recalc.modelStruct}{Recalculate for modelStruct Object}
This method function recalculates the condensed linear model object
using each element of \Co{object} sequentially from last to first.
\begin{Example}
recalc(object, conLin)
\end{Example}
\begin{Argument}{ARGUMENTS}
\item[\Co{object:}]
an object inheriting from class \Co{modelStruct},
representing a list of model components, such as \Co{corStruct} and
\Co{varFunc} objects.
\item[\Co{conLin:}]
an optional  condensed linear model object, consisting of
a list with components \Co{"Xy"}, corresponding to a regression
matrix (\Co{X}) combined with a response vector (\Co{y}), and 
\Co{"logLik"}, corresponding to the log-likelihood of the
underlying model. Defaults to \Co{attr(object, "conLin")}.
\end{Argument}
\Paragraph{VALUE}
the recalculated condensed linear model object.
\Paragraph{NOTE} This method function is generally only used inside model fitting
functions like \Co{lme} and \Co{gls}, which allow model
components, such as correlated error terms and variance functions.
\Paragraph{SEE ALSO}
\Co{recalc.corStruct}, \Co{recalc.reStruct},
\Co{recalc.varFunc}
\end{Helpfile}
\begin{Helpfile}{recalc.reStruct}{Recalculate for reStruct Object}
The log-likelihood, or restricted log-likelihood, of the
Gaussian linear mixed-effects model represented by \Co{object} and
\Co{conLin} (assuming spherical within-group covariance structure),
evaluated at \Co{coef(object)} is calculated and added to the
\Co{logLik} component of \Co{conLin}. The \Co{settings}
attribute of \Co{object} determines whether the log-likelihood, or
the restricted log-likelihood, is to be calculated. The computational
methods for the (restricted) log-likelihood calculations are described
in Bates and Pinheiro (1998).
\begin{Example}
recalc(object, conLin)
\end{Example}
\begin{Argument}{ARGUMENTS}
\item[\Co{object:}]
an object inheriting from class \Co{reStruct},
representing a random effects structure and consisting of a list of
\Co{pdMat} objects.
\item[\Co{conLin:}]
a condensed linear model object, consisting of a list
with components \Co{"Xy"}, corresponding to a regression matrix
(\Co{X}) combined with a response vector (\Co{y}), and
\Co{"logLik"}, corresponding to the log-likelihood of the
underlying model.
\end{Argument}
\Paragraph{VALUE}
the condensed linear model with its \Co{logLik} component updated.
\Paragraph{REFERENCES}
Bates, D.M. and Pinheiro, J.C. (1998) "Computational methods for
multilevel models" available in PostScript or PDF formats at \\
http://nlme.stat.wisc.edu
\Paragraph{SEE ALSO}
\Co{reStruct}, \Co{logLik},
\Co{lme}
\end{Helpfile}
\begin{Helpfile}{recalc.varFunc}{Recalculate for varFunc Object}
This method function pre-multiples the \Co{"Xy"} component of
\Co{conLin} by a diagonal matrix with diagonal elements given by the
weights corresponding to the variance structure represented by
\Co{object}e and adds the log-likelihood contribution of
\Co{object}, given by \Co{logLik(object)}, to the \Co{"logLik"}
component of \Co{conLin}.
\begin{Example}
recalc(object, conLin)
\end{Example}
\begin{Argument}{ARGUMENTS}
\item[\Co{object:}]
an object inheriting from class \Co{varFunc},
representing a variance function structure.
\item[\Co{conLin:}]
a condensed linear model object, consisting of a list
with components \Co{"Xy"}, corresponding to a regression matrix
(\Co{X}) combined with a response vector (\Co{y}), and
\Co{"logLik"}, corresponding to the log-likelihood of the
underlying model.
\end{Argument}
\Paragraph{VALUE}
the recalculated condensed linear model object.
\Paragraph{NOTE} This method function is only used inside model fitting functions
which allow heteroscedastic error terms, like \Co{lme} and
\Co{gls}.
\Paragraph{SEE ALSO}
\Co{varWeights}, \Co{logLik.varFunc}
\end{Helpfile}
\begin{Helpfile}{residuals.gls}{Extract gls Residuals}
The residuals for the linear model represented by \Co{object}
are extracted.
\begin{Example}
residuals(object, type)
\end{Example}
\begin{Argument}{ARGUMENTS}
\item[\Co{object:}]
an object inheriting from class \Co{gls}, representing
a generalized least squares fitted linear model.
\item[\Co{type:}]
an optional character string specifying the type of
residuals to be used. If \Co{"response"}, the "raw" residuals
(observed - fitted) are used; else, if \Co{"pearson"}, the
standardized residuals (raw residuals divided by the corresponding
standard errors) are used; else, if \Co{"normalized"}, the
normalized residuals (standardized residuals pre-multiplied by the
inverse square-root factor of the estimated error correlation
matrix) are used. Partial matching of arguments is used, so only the
first character needs to be provided. Defaults to \Co{"pearson"}.
\end{Argument}
\Paragraph{VALUE}
a vector with the residuals for the linear model represented by
\Co{object}.
\Paragraph{SEE ALSO}
\Co{gls}, \Co{fitted.gls}
\need 15pt
\Paragraph{EXAMPLE}
\vspace{-16pt}
\begin{Example}
fm1 <- gls(follicles {\Twiddle} sin(2*pi*Time) + cos(2*pi*Time), Ovary,
           correlation = corAR1(form = {\Twiddle} 1 | Mare))
residuals(fm1)
\end{Example}
\end{Helpfile}
\begin{Helpfile}{residuals.gnlsStruct}{Calculate gnlsStruct Residuals}
The residuals for the nonlinear model represented by \Co{object}
are extracted.
\begin{Example}
fitted(object)
\end{Example}
\begin{Argument}{ARGUMENTS}
\item[\Co{object:}]
an object inheriting from class \Co{gnlsStruct},
representing a list of model components, such as
\Co{corStruct} and \Co{varFunc} objects, and attributes
specifying the underlying nonlinear model and the response variable.
\end{Argument}
\Paragraph{VALUE}
a vector with the residuals for the nonlinear model represented by
\Co{object}.
\Paragraph{NOTE} This method function is generally only used inside \Co{gnls} and 
\Co{residuals.gnls}.
\Paragraph{SEE ALSO}
\Co{gnls}, \Co{residuals.gnlsStruct},
\Co{fitted.gnlsStruct}
\end{Helpfile}
\begin{Helpfile}{residuals.lmList}{Extract lmList Residuals}
The residuals are extracted from each \Co{lm} component of
\Co{object} and arranged into a list with as many components as
\Co{object}, or combined into a single vector.
\begin{Example}
residuals(object, type, subset, asList)
\end{Example}
\begin{Argument}{ARGUMENTS}
\item[\Co{object:}]
an object inheriting from class \Co{lmList}, representing
a list of \Co{lm} objects with a common model.
\item[\Co{subset:}]
an optional character or integer vector naming the
\Co{lm} components of \Co{object} from which the residuals 
are to be extracted. Default is \Co{NULL}, in which case all
components are used.
\item[\Co{type:}]
an optional character string specifying the type of
residuals to be extracted. Options include \Co{"response"} for the
"raw" residuals (observed - fitted), \Co{"pearson"} for the
standardized residuals (raw residuals divided by the estimated
residual standard error) using different standard errors for each
\Co{lm} fit, and \Co{"pooled.pearson"} for the standardized
residuals using a pooled estimate of the residual standard
error. Partial matching of arguments is used, so only the first 
character needs to be provided. Defaults to \Co{"response"}.
\item[\Co{asList:}]
an optional logical value. If \Co{TRUE}, the returned
object is a list with the residuals split by groups; else the
returned value is a vector. Defaults to \Co{FALSE}.
\end{Argument}
\Paragraph{VALUE}
a list with components given by the residuals of each \Co{lm}
component of \Co{object}, or a vector with the residuals for all
\Co{lm} components of \Co{object}.
\Paragraph{SEE ALSO}
\Co{lmList}, \Co{fitted.lmList}
\need 15pt
\Paragraph{EXAMPLE}
\vspace{-16pt}
\begin{Example}
fm1 <- lmList(distance {\Twiddle} age | Subject, Orthodont)
residuals(fm1)
\end{Example}
\end{Helpfile}
\begin{Helpfile}{residuals.lme}{Extract lme Residuals}
The residuals at level i are obtained by subtracting the fitted
levels at that level from the response vector (and dividing
by the estimated within-group standard error, if
\Co{type="pearson"}). The fitted values at level i are obtained
by adding together the population fitted values (based only on the
fixed effects estimates) and the estimated contributions of the random
effects to the fitted values at grouping levels less or equal to
i.
\begin{Example}
residuals(object, level, type, asList)
\end{Example}
\begin{Argument}{ARGUMENTS}
\item[\Co{object:}]
an object inheriting from class \Co{lme}, representing
a fitted linear mixed-effects model.
\item[\Co{level:}]
an optional integer vector giving the level(s) of grouping
to be used in extracting the residuals from \Co{object}. Level
values increase from outermost to innermost grouping, with
level zero corresponding to the population residuals. Defaults to
the highest or innermost level of grouping.
\item[\Co{type:}]
an optional character string specifying the type of
residuals to be used. If \Co{"response"}, the "raw" residuals
(observed - fitted) are used; else, if \Co{"pearson"}, the
standardized residuals (raw residuals divided by the corresponding
standard errors) are used; else, if \Co{"normalized"}, the
normalized residuals (standardized residuals pre-multiplied by the
inverse square-root factor of the estimated error correlation
matrix) are used. Partial matching of arguments is used, so only the
first character needs to be provided. Defaults to \Co{"pearson"}.
\item[\Co{asList:}]
an optional logical value. If \Co{TRUE} and a single
value is given in \Co{level}, the returned object is a list with
the residuals split by groups; else the returned value is
either a vector or a data frame, according to the length of
\Co{level}. Defaults to \Co{FALSE}.
\end{Argument}
\Paragraph{VALUE}
if a single level of grouping is specified in \Co{level}, the
returned value is either a list with the residuals split by groups
(\Co{asList = TRUE}) or a vector with the residuals
(\Co{asList = FALSE}); else, when multiple grouping levels are
specified in \Co{level}, the returned object is a data frame with
columns given by the residuals at different levels and the grouping
factors.
\Paragraph{REFERENCES}
Bates, D.M. and Pinheiro, J.C. (1998) "Computational methods for
multilevel models" available in PostScript or PDF formats at
http://nlme.stat.wisc.edu
\Paragraph{SEE ALSO}
\Co{lme}, \Co{fitted.lme}
\need 15pt
\Paragraph{EXAMPLE}
\vspace{-16pt}
\begin{Example}
fm1 <- lme(distance {\Twiddle} age + Sex, data = Orthodont, random = {\Twiddle} 1)
residuals(fm1, level = 0:1)
\end{Example}
\end{Helpfile}
\begin{Helpfile}{residuals.lmeStruct}{Calculate lmeStruct Residuals}
The residuals at level i are obtained by subtracting the fitted
values at that level from the response vector. The fitted values at
level i are obtained by adding together the population fitted
values (based only on the fixed effects estimates) and the estimated
contributions of the random effects to the fitted values at grouping
levels less or equal to i.
\begin{Example}
residuals(object, levels, lmeFit, conLin)
\end{Example}
\begin{Argument}{ARGUMENTS}
\item[\Co{object:}]
an object inheriting from class \Co{lmeStruct},
representing a list of linear mixed-effects model components, such as
\Co{reStruct}, \Co{corStruct}, and \Co{varFunc} objects.
\item[\Co{level:}]
an optional integer vector giving the level(s) of grouping
to be used in extracting the residuals from \Co{object}. Level
values increase from outermost to innermost grouping, with
level zero corresponding to the population fitted values. Defaults to
the highest or innermost level of grouping.
\item[\Co{lmeFit:}]
an optional list with components \Co{beta} and \Co{b}
containing respectively the fixed effects estimates and the random
effects estimates to be used to calculate the residuals. Defaults
to \Co{attr(object, "lmeFit")}.
\item[\Co{conLin:}]
an optional condensed linear model object, consisting of
a list with components \Co{"Xy"}, corresponding to a regression
matrix (\Co{X}) combined with a response vector (\Co{y}), and 
\Co{"logLik"}, corresponding to the log-likelihood of the
underlying lme model. Defaults to \Co{attr(object, "conLin")}.
\end{Argument}
\Paragraph{VALUE}
if a single level of grouping is specified in \Co{level},
the returned value is a vector with the residuals at the desired
level; else, when multiple grouping levels are specified in
\Co{level}, the returned object is a matrix with 
columns given by the residuals at different levels.
\Paragraph{NOTE} This method function is generally only used inside the \Co{lme}
function.
\Paragraph{REFERENCES}
Bates, D.M. and Pinheiro, J.C. (1998) "Computational methods for
multilevel models" available in PostScript or PDF formats at
http://nlme.stat.wisc.edu
\Paragraph{SEE ALSO}
\Co{lme}, \Co{residuals.lme},
\Co{fitted.lmeStruct}
\end{Helpfile}
\begin{Helpfile}{residuals.nlmeStruct}{Calculate nlmeStruct Residuals}
The residuals at level i are obtained by subtracting the fitted
values at that level from the response vector. The fitted values at
level i are obtained by adding together the contributions from
the estimated fixed effects and the estimated random effects at levels
less or equal to i and evaluating the model function at the
resulting estimated parameters.
\begin{Example}
residuals(object, levels, conLin)
\end{Example}
\begin{Argument}{ARGUMENTS}
\item[\Co{object:}]
an object inheriting from class \Co{nlmeStruct},
representing a list of mixed-effects model components, such as
\Co{reStruct}, \Co{corStruct}, and \Co{varFunc} objects.
\item[\Co{level:}]
an optional integer vector giving the level(s) of grouping
to be used in extracting the residuals from \Co{object}. Level
values increase from outermost to innermost grouping, with
level zero corresponding to the population fitted values. Defaults to
the highest or innermost level of grouping.
\item[\Co{conLin:}]
an optional condensed linear model object, consisting of
a list with components \Co{"Xy"}, corresponding to a regression
matrix (\Co{X}) combined with a response vector (\Co{y}), and 
\Co{"logLik"}, corresponding to the log-likelihood of the
underlying nlme model. Defaults to \Co{attr(object, "conLin")}.
\end{Argument}
\Paragraph{VALUE}
if a single level of grouping is specified in \Co{level},
the returned value is a vector with the residuals at the desired
level; else, when multiple grouping levels are specified in
\Co{level}, the returned object is a matrix with 
columns given by the residuals at different levels.
\Paragraph{NOTE} This method function is generally only used inside
the \Co{nlme} function 
\Paragraph{REFERENCES}
Bates, D.M. and Pinheiro, J.C. (1998) "Computational methods for
multilevel models" available in PostScript or PDF formats at
http://nlme.stat.wisc.edu
\Paragraph{SEE ALSO}
\Co{nlme}, \Co{fitted.nlmeStruct}
\end{Helpfile}
\begin{Helpfile}{reStruct}{Random Effects Structure}
This function is a constructor for the \Co{reStruct} class,
representing a random effects structure and consisting of a list of
\Co{pdMat} objects, plus a \Co{settings} attribute containing
information for the optimization algorithm used to fit the associated
mixed-effects model.
\begin{Example}
reStruct(object, pdClass, REML, data)
\end{Example}
\begin{Argument}{ARGUMENTS}
\item[\Co{object:}]
any of the following: (i) a one-sided formula of the form
\Co{\Twiddle x1+...+xn | g1/.../gm}, with \Co{x1+...+xn} specifying the
model for the random effects and \Co{g1/.../gm} the grouping
structure (\Co{m} may be equal to 1, in which case no \Co{/} is
required). The random effects formula will be repeated for all levels
of grouping, in the case of multiple levels of grouping; (ii) a list of
one-sided formulas of the form \Co{\Twiddle x1+...+xn | g}, with possibly
different random effects models for each grouping level. The order of
nesting will be assumed the same as the order of the elements in the
list; (iii) a one-sided formula of the form \Co{\Twiddle x1+...+xn}, or a
\Co{pdMat} object with a formula (i.e.\ a non-\Co{NULL} value for
\Co{formula(object)}), or a list of such formulas or \Co{pdMat}
objects. In this case, the grouping structure formula will be derived
from the data used to to fit the mixed-effects model, which should
inherit from class \Co{groupedData}; (iv) a named list of formulas or
\Co{pdMat} objects as in (iii), with the grouping factors as
names. The order of nesting will be assumed the same as the order of
the order of the elements in the list; (v) an \Co{reStruct} object.
\item[\Co{pdClass:}]
an optional character string with the name of the
\Co{pdMat} class to be used for the formulas in
\Co{object}. Defaults to \Co{"pdSymm"} which corresponds to a
general positive-definite matrix.
\item[\Co{REML:}]
an optional logical value. If \Co{TRUE}, the associated
mixed-effects model will be fitted using restricted maximum
likelihood; else, if \Co{FALSE}, maximum likelihood will be
used. Defaults to \Co{FALSE}.
\item[\Co{data:}]
an optional data frame in which to evaluate the variables
used in the random effects formulas in \Co{object}. It is used to
obtain the levels for \Co{factors}, which affect the dimensions and
the row/column names of the underlying \Co{pdMat} objects. If
\Co{NULL}, no attempt is made to obtain information on
\Co{factors} appearing the random effects model. Defaults to parent
frame from which the function was called.
\end{Argument}
\Paragraph{VALUE}
an object inheriting from class \Co{reStruct}, representing a random
effects structure.
\Paragraph{SEE ALSO}
\Co{pdMat}, \Co{lme},
\Co{groupedData}
\need 15pt
\Paragraph{EXAMPLE}
\vspace{-16pt} 
\begin{Example}
rs1 <- reStruct(list(Dog = \Twiddle day, Side = \Twiddle 1), data = Pixel)
rs1
\end{Example}
\end{Helpfile}
\begin{Helpfile}{selfStart}{Construct Self-starting Nonlinear Models}
This function is generic; methods functions can be written to handle
specific classes of objects. Available methods include
\Co{selfStart.default} and \Co{selfStart.formula}.
See the documentation on the appropriate method function.
\begin{Example}
selfStart(model, initial, parameters, template)
\end{Example}
\Paragraph{VALUE}
a function object of the \Co{selfStart} class.
\Paragraph{SEE ALSO}
\Co{selfStart.default}, \Co{selfStart.formula}
\vspace{-16pt} 
\begin{Example}
## see documentation for the methods
\end{Example}
\end{Helpfile}
\begin{Helpfile}{selfStart.default}{Construct Self-starting Nonlinear Models}
A method for the generic function \Co{selfStart} for function objects.
\begin{Example}
selfStart(model, initial, parameters, template)
\end{Example}
\begin{Argument}{ARGUMENTS}
\item[\Co{model:}]
a function object defining a nonlinear model.
\item[\Co{initial:}]
a function object, with three arguments: \Co{mCall},
\Co{data}, and \Co{LHS}, representing, respectively, the
expression on the right hand side of \Co{model}, a data frame in
which to interpret the variables in \Co{mCall} and \Co{LHS}, and
a name, or expression, representing the variable to be used as the
"response" in the initial values calculations. It should return
initial values for the parameters on the right hand side of
\Co{model}. 
\item[\Co{parameters, template:}]
these arguments are included to keep
consistency with the call to the generic function, but are not used
in the \Co{default} method. See the documentation on
\Co{selfStart.formula}.
\end{Argument}
\Paragraph{VALUE}
a function object of class \Co{selfStart}, corresponding to a
self-starting nonlinear model function. An \Co{initial} attribute
(defined by the \Co{initial} argument) is added to the function to
calculate starting estimates for the parameters in the model
automatically.
\Paragraph{SEE ALSO}
\Co{selfStart.formula}
\need 15pt
\Paragraph{EXAMPLE}
\vspace{-16pt}
\begin{Example}
# \Co{first.order.log.model} is a function object defining a 
# first order compartment model 
# \Co{first.order.log.initial} is a function object which calculates 
# initial values for the parameters in \Co{first.order.log.model}
# self-starting first order compartment model
SSfol <- selfStart(first.order.log.model, first.order.log.initial)
\end{Example}
\end{Helpfile}
\begin{Helpfile}{selfStart.formula}{Construct Self-starting Nonlinear Models}
A method for the generic function \Co{selfStart} for formula objects.
\begin{Example}
selfStart(model, initial, parameters, template)
\end{Example}
\begin{Argument}{ARGUMENTS}
\item[\Co{model:}]
a nonlinear formula object of the form \Co{{\Twiddle}expression}.
\item[\Co{initial:}]
a function object, with three arguments: \Co{mCall},
\Co{data}, and \Co{LHS}, representing, respectively, the
expression on the right hand side of \Co{model}, a data frame in
which to interpret the variables in \Co{mCall} and \Co{LHS}, and
a name, or expression, representing the variable to be used as the
"response" in the initial values calculations. It should return
initial values for the parameters on the right hand side of
\Co{model}.
\item[\Co{parameters:}]
a character vector specifying the terms on the right
hand side of \Co{model} for which initial estimates should be
calculated. Passed as the \Co{namevec} argument to the \Co{deriv}
function.
\item[\Co{template:}]
an optional prototype for the calling sequence of the
returned object, passed as the \Co{function.arg} argument to the
\Co{deriv} function. By default, a template is generated with the
covariates in \Co{model} coming first and the parameters in
\Co{model} coming last in the calling sequence.
\end{Argument}
\Paragraph{VALUE}
a function object of class \Co{selfStart}, obtained by applying
\Co{deriv} to the right hand side of the \Co{model} formula. An
\Co{initial} attribute (defined by the \Co{initial} argument) is
added to the function to calculate starting estimates for the
parameters in the model automatically.
\Paragraph{SEE ALSO}
\Co{selfStart.default}, \Co{deriv}
\need 15pt
\Paragraph{EXAMPLE}
\vspace{-16pt}
\begin{Example}
## self-starting logistic model
SSlogis <- selfStart({\Twiddle} Asym/(1 + exp((xmid - x)/scal)),
  function(mCall, data, LHS)
  \{
    xy <- sortedXyData(mCall[["x"]], LHS, data)
    if(nrow(xy) < 4) \{
      stop("Too few distinct x values to fit a logistic")
    \}
    z <- xy[["y"]]
    if (min(z) <= 0) \{ z <- z + 0.05 * max(z) \} # avoid zeroes
    z <- z/(1.05 * max(z))        # scale to within unit height
    xy[["z"]] <- log(z/(1 - z))        # logit transformation
    aux <- coef(lm(x {\Twiddle} z, xy))
    parameters(xy) <- list(xmid = aux[1], scal = aux[2])
    pars <- as.vector(coef(nls(y {\Twiddle} 1/(1 + exp((xmid - x)/scal)), 
                   data = xy, algorithm = "plinear")))
    value <- c(pars[3], pars[1], pars[2])
    names(value) <- mCall[c("Asym", "xmid", "scal")]
    value
  \}, c("Asym", "xmid", "scal"))
\end{Example}
\end{Helpfile}
\begin{Helpfile}{simulate.lme}{simulate lme models}
The model \Co{m1} is fit to the data.  Using
the fitted values of the parameters, \Co{nsim} new data vectors from
this model are simulated.  Both \Co{m1} and \Co{m2} are fit by
maximum likelihood (ML) and/or by restricted maximum likelihood (REML)
to each of the simulated data vectors.
\begin{Example}
simulate.lme(m1, m2, Random.seed, method, nsim, niterEM, useGen)
\end{Example}
\begin{Argument}{ARGUMENTS}
\item[\Co{m1:}]
an object inheriting from class \Co{lme}, representing a fitted
linear mixed-effects model, or a list containing an lme model
specification.  If given as a list, it should contain
components \Co{fixed}, \Co{data}, and \Co{random}
with values suitable for a call to \Co{lme}. This argument
defines the null model.
\item[\Co{m2:}]
an \Co{lme} object, or a list, like \Co{m1} containing a second
lme model specification. This argument defines the alternative model.
If given as a list, only those parts of the specification that
change between model \Co{m1} and \Co{m2} need to be specified.
\item[\Co{Random.seed:}]
an optional vector to seed the random number generator so as to
reproduce a simulation.  This vector should be the same form as the
\Co{.Random.seed} object.
\item[\Co{method:}]
an optional character array.  If it includes \Co{"REML"} the models
are fit by maximizing the restricted log-likelihood. If it includes
\Co{"ML"} the log-likelihood is maximized.  Defaults to
\Co{c("REML", "ML")}, in which case both methods are used.
\item[\Co{nsim:}]
an optional positive integer specifying the number of simulations to
perform.  Defaults to 1000.
\item[\Co{niterEM:}]
an optional integer vector of length 2 giving the number of
iterations of the EM algorithm to apply when fitting the \Co{m1}
and \Co{m2} to each simulated set of data. Defaults to
\Co{c(40,200)}. 
\item[\Co{useGen:}]
an optional logical value. If \Co{TRUE}, numerical derivatives are
used to obtain the gradient and the Hessian of the log-likelihood in
the optimization algorithm in the \Co{ms} function. If
\Co{FALSE}, the default algorithm in \Co{ms} for functions that
do not incorporate gradient and Hessian attributes is used. Default
depends on the \Co{pdMat} classes used in \Co{m1} and \Co{m2}:
if both are standard classes (see \Co{pdClasses}) then
defaults to \Co{TRUE}, otherwise defaults to \Co{FALSE}.
\end{Argument}
\Paragraph{VALUE}
an object of class \Co{simulate.lme} with components \Co{null} and
\Co{alt}.  Each of these has components \Co{ML} and/or \Co{REML}
which are matrices.  An attribute called \Co{Random.seed} contains
the seed that was used for the random number generator.
\Paragraph{SEE ALSO}
\Co{lme}
\need 15pt
\Paragraph{EXAMPLE}
\vspace{-16pt} 
\begin{Example}
orthSim <-
 simulate.lme(m1 = list(fixed = distance {\Twiddle} age, data = Orthodont,
                        random = {\Twiddle} 1 | Subject),
              m2 = list(random = {\Twiddle} age | Subject))
\end{Example}
\end{Helpfile}
\begin{Helpfile}{solve.pdMat}{Calculate Inverse of a Positive-Definite Matrix}
The positive-definite matrix represented by \Co{a} is inverted and
assigned to \Co{a}.
\begin{Example}
solve(a, b, tol)
\end{Example}
\begin{Argument}{ARGUMENTS}
\item[\Co{a:}]
an object inheriting from class \Co{pdMat}, representing
a positive definite matrix.
\item[\Co{b:}]
this argument is only included for consistency with the generic
function and is not used in this method function.
\item[\Co{tol:}]
an optional numeric value for the tolerance used in the
numerical algorithm. Defaults to \Co{1e-7}.
\end{Argument}
\Paragraph{VALUE}
a \Co{pdMat} object similar to \Co{a}, but with coefficients
corresponding to the inverse of the positive-definite matrix
represented by \Co{a}.
\Paragraph{SEE ALSO}
\Co{pdMat}
\need 15pt
\Paragraph{EXAMPLE}
\vspace{-16pt} 
\begin{Example}
pd1 <- pdCompSymm(3 * diag(3) + 1)
solve(pd1)
\end{Example}
\end{Helpfile}
\begin{Helpfile}{solve.reStruct}{Apply Solve to an reStruct Object}
\Co{Solve} is applied to each \Co{pdMat} component of \Co{a},
which results in inverting the positive-definite matrices they
represent.
\begin{Example}
solve(a, b, tol)
\end{Example}
\begin{Argument}{ARGUMENTS}
\item[\Co{a:}]
an object inheriting from class \Co{reStruct},
representing a random effects structure and consisting of a list of
\Co{pdMat} objects.
\item[\Co{b:}]
this argument is only included for consistency with the
generic function and is not used in this method function.
\item[\Co{tol:}]
an optional numeric value for the tolerance used in the
numerical algorithm. Defaults to \Co{1e-7}.
\end{Argument}
\Paragraph{VALUE}
an \Co{reStruct} object similar to \Co{a}, but with the
\Co{pdMat} components representing the inverses of the
matrices represented by the components of \Co{a}.
\Paragraph{SEE ALSO}
\Co{solve.pdMat}, \Co{reStruct}
\need 15pt
\Paragraph{EXAMPLE}
\vspace{-16pt} 
\begin{Example}
rs1 <- reStruct(list(A = pdSymm(diag(1:3), form = \Twiddle Score),
  B = pdDiag(2 * diag(4), form = \Twiddle Educ)))
solve(rs1)
\end{Example}
\end{Helpfile}
\begin{Helpfile}{sortedXyData}{Create a sortedXyData object}
This is constructor function for the class of \Co{sortedXyData}
objects.  These objects are mostly used in the \Co{initial}
function for a self-starting nonlinear regression model, which will be
of the \Co{selfStart} class.
\begin{Example}
sortedXyData(x, y, data)
\end{Example}
\begin{Argument}{ARGUMENTS}
\item[\Co{x:}]
a numeric vector or an expression that will evaluate in
\Co{data} to a numeric vector 
\item[\Co{y:}]
a numeric vector or an expression that will evaluate in
\Co{data} to a numeric vector 
\item[\Co{data:}]
an optional data frame in which to evaluate expressions
for \Co{x} and \Co{y}, if they are given as expressions 
\end{Argument}
\Paragraph{VALUE}
A \Co{sortedXyData} object. This is a data frame with exactly
two numeric columns, named \Co{x} and \Co{y}.  The rows are
sorted so the \Co{x} column is in increasing order.  Duplicate
\Co{x} values are eliminated by averaging the corresponding \Co{y}
values.
\Paragraph{SEE ALSO}
\Co{selfStart}, \Co{NLSstClosestX},
\Co{NLSstLfAsymptote}, \Co{NLSstRtAsymptote}
\need 15pt
\Paragraph{EXAMPLE}
\vspace{-16pt}
\begin{Example}
DNase.2 <- DNase[ DNase\$Run == "2", ]
sortedXyData( expression(log(conc)), expression(density), DNase.2 )
\end{Example}
\end{Helpfile}
\begin{Helpfile}{splitFormula}{Split a Formula}
Splits the right hand side of \Co{form} into a list of subformulas
according to the presence of \Co{sep}. The left hand side of
\Co{form}, if present, will be ignored. The length of the returned
list will be equal to the number of occurrences of \Co{sep} in
\Co{form} plus one.
\begin{Example}
splitFormula(frm, sep)
\end{Example}
\begin{Argument}{ARGUMENTS}
\item[\Co{form:}]
a \Co{formula} object.
\item[\Co{sep:}]
an optional character string specifying the separator to be
used for splitting the formula. Defaults to \Co{"/"}. 
\end{Argument}
\Paragraph{VALUE}
a list of formulas, corresponding to the split of \Co{form}
according to \Co{sep}.
\Paragraph{SEE ALSO}
\Co{formula}
\need 15pt
\Paragraph{EXAMPLE}
\vspace{-16pt} 
\begin{Example}
splitFormula(\Twiddle g1/g2/g3)
\end{Example}
\end{Helpfile}
\begin{Helpfile}{SSasymp}{Asymptotic regression model}
This \Co{selfStart} model evaluates the asymptotic regression
function and its gradient.  It has an \Co{initial} attribute that
will evaluate initial estimates of the parameters \Co{Asym}, \Co{R0},
and \Co{lrc} for a given set of data.
\begin{Example}
SSasymp(input, Asym, R0, lrc)
\end{Example}
\begin{Argument}{ARGUMENTS}
\item[\Co{input:}]
a numeric vector of values at which to evaluate the model
\item[\Co{Asym:}]
a numeric parameter representing the horizontal asymptote on
the right side (very large values of \Co{input})
\item[\Co{R0:}]
a numeric parameter representing the response when
\Co{input} is zero.
\item[\Co{lrc:}]
a numeric parameter representing the natural logarithm of
the rate constant
\end{Argument}
\Paragraph{VALUE}
a numeric vector of the same length as \Co{input}.  It is the value of
the expression $Asym+(R0-Asym)\exp(-\exp(lrc)input)$.  If all of
the arguments \Co{Asym}, \Co{R0}, and \Co{lrc} are
names of objects, as opposed to expressions or explicit numerical
values, the gradient matrix with respect to these names is attached as
an attribute named \Co{gradient}.
\Paragraph{SEE ALSO}
\Co{nls}, \Co{selfStart}
\need 15pt
\Paragraph{EXAMPLE}
\vspace{-16pt} 
\begin{Example}
Lob.329 <- Loblolly[ Loblolly\$Seed == "329", ]
SSasymp( Lob.329\$age, 100, -8.5, -3.2 )  # response only
Asym <- 100
resp0 <- -8.5
lrc <- -3.2
SSasymp( Lob.329\$age, Asym, resp0, lrc ) # response and gradient
\end{Example}
\end{Helpfile}
\begin{Helpfile}{SSasympOff}{Asymptotic Regression Model with an Offset}
This \Co{selfStart} model evaluates the alternative asymptotic
regression function and its gradient.  It has an \Co{initial}
attribute that will evaluate initial estimates of the parameters
\Co{Asym}, \Co{lrc}, and \Co{c0}  for a given set of data.
\begin{Example}
SSasympOff(input, Asym, lrc, c0)
\end{Example}
\begin{Argument}{ARGUMENTS}
\item[\Co{input:}]
a numeric vector of values at which to evaluate the model.
\item[\Co{Asym:}]
a numeric parameter representing the horizontal asymptote on
the right side (very large values of \Co{input}).
\item[\Co{lrc:}]
a numeric parameter representing the natural logarithm of
the rate constant.
\item[\Co{c0:}]
a numeric parameter representing the \Co{input} when
response is zero.
\end{Argument}
\Paragraph{VALUE}
a numeric vector of the same length as \Co{input}.  It is the value of
the expression $Asym\{1 - \exp[-\exp(lrc)(input - c0)]\}$.  If all of
the arguments \Co{Asym}, \Co{lrc}, and \Co{c0} are
names of objects, as opposed to expressions or explicit numerical
values, the gradient matrix with respect to these names is attached as
an attribute named \Co{gradient}.
\Paragraph{SEE ALSO}
\Co{nls}, \Co{selfStart}
\need 15pt
\Paragraph{EXAMPLE}
\vspace{-16pt} 
\begin{Example}
CO2.Qn1 <- CO2[CO2\$Plant == "Qn1", ]
SSasympOff( CO2.Qn1\$conc, 32, 43, -4 )  # response only
Asym <- 32
lrc <- -4
c0 <- 43
SSasympOff( CO2.Qn1\$conc, Asym, lrc, c0 ) # response and gradient
\end{Example}
\end{Helpfile}
\begin{Helpfile}{SSasympOrig}{Asypmtotic Regression Model through the Origin}
This \Co{selfStart} model evaluates the asymptotic regression
function through the origin and its gradient.  It has an
\Co{initial} attribute that will evaluate initial estimates of the
parameters \Co{Asym} and \Co{lrc} for a given set of data.
\begin{Example}
SSasympOrig(input, Asym, lrc)
\end{Example}
\begin{Argument}{ARGUMENTS}
\item[\Co{input:}]
a numeric vector of values at which to evaluate the model.
\item[\Co{Asym:}]
a numeric parameter representing the horizontal asymptote.
\item[\Co{lrc:}]
a numeric parameter representing the natural logarithm of
the rate constant.
\end{Argument}
\Paragraph{VALUE}
a numeric vector of the same length as \Co{input}.  It is the value of
the expression $Asym\{1 - \exp[-\exp(lrc) \cdot input]\}$.  If all of
the arguments \Co{Asym} and \Co{lrc} are
names of objects, as opposed to expressions or explicit numerical
values, the gradient matrix with respect to these names is attached as
an attribute named \Co{gradient}.
\Paragraph{SEE ALSO}
\Co{nls}, \Co{selfStart}
\need 15pt
\Paragraph{EXAMPLE}
\vspace{-16pt} 
\begin{Example}
Lob.329 <- Loblolly[ Loblolly\$Seed == "329", ]
SSasympOrig( Lob.329\$age, 100, -3.2 )  # response only
Asym <- 100
lrc <- -3.2
SSasympOrig( Lob.329\$age, Asym, lrc ) # response and gradient
\end{Example}
\end{Helpfile}
\begin{Helpfile}{SSbiexp}{Biexponential model}
This \Co{selfStart} model evaluates the biexponential model function
and its gradient.  It has an \Co{initial} attribute that 
will evaluate initial estimates of the parameters \Co{A1}, \Co{lrc1},
\Co{A2}, and \Co{lrc2} for a given set of data.
\begin{Example}
SSbiexp(input, A1, lrc1, A2, lrc2)
\end{Example}
\begin{Argument}{ARGUMENTS}
\item[\Co{input:}]
a numeric vector of values at which to evaluate the model.
\item[\Co{A1:}]
a numeric parameter representing the multiplier of the first
exponential.
\item[\Co{lrc1:}]
a numeric parameter representing the natural logarithm of
the rate constant of the first exponential.
\item[\Co{A2:}]
a numeric parameter representing the multiplier of the second
exponential.
\item[\Co{lrc2:}]
a numeric parameter representing the natural logarithm of
the rate constant of the second exponential.
\end{Argument}
\Paragraph{VALUE}
a numeric vector of the same length as \Co{input}.  It is the value of
the expression
$A1\exp[-\exp(lrc1) \cdot input]+A2\exp[-\exp(lrc2) \cdot input]$.
If all of the arguments \Co{A1}, \Co{lrc1}, \Co{A2}, and
\Co{lrc2} are names of objects, as opposed to expressions or
explicit numerical values, the gradient matrix with respect to these
names is attached as an attribute named \Co{gradient}.
\Paragraph{SEE ALSO}
\Co{nls}, \Co{selfStart}
\need 15pt
\Paragraph{EXAMPLE}
\vspace{-16pt} 
\begin{Example}
Indo.1 <- Indometh[Indometh\$Subject == 1, ]
SSbiexp( Indo.1\$time, 3, 1, 0.6, -1.3 )  # response only
A1 <- 3
lrc1 <- 1
A2 <- 0.6
lrc2 <- -1.3
SSbiexp( Indo.1\$time, A1, lrc1, A2, lrc2 ) # response and gradient
\end{Example}
\end{Helpfile}
\begin{Helpfile}{SSfol}{First-order Compartment Model}
This \Co{selfStart} model evaluates the first-order compartment
function and its gradient.  It has an \Co{initial} attribute that 
will evaluate initial estimates of the parameters \Co{lKe}, \Co{lKa},
and \Co{lCl} for a given set of data.
\begin{Example}
SSfol(Dose, input, lKe, lKa, lCl)
\end{Example}
\begin{Argument}{ARGUMENTS}
\item[\Co{Dose:}]
a numeric value representing the initial dose.
\item[\Co{input:}]
a numeric vector at which to evaluate the model.
\item[\Co{lKe:}]
a numeric parameter representing the natural logarithm of
the elimination rate constant.
\item[\Co{lKe:}]
a numeric parameter representing the natural logarithm of
the absorption rate constant.
\item[\Co{lCl:}]
a numeric parameter representing the natural logarithm of
the clearance.
\end{Argument}
\Paragraph{VALUE}
a numeric vector of the same length as \Co{input}.  It is the value of
the expression $$\frac{Dose\exp(lKe+lKa-lCl)}{\exp(lKa) - \exp(lKe)}
\left\{\exp[-\exp(lKe) \cdot input] - \exp[-\exp(lKa)\cdot
input]\right\}$$.  If all of 
the arguments \Co{lKe}, \Co{lKa}, and \Co{lCl} are
names of objects, as opposed to expressions or explicit numerical
values, the gradient matrix with respect to these names is attached as
an attribute named \Co{gradient}.
\Paragraph{SEE ALSO}
\Co{nls}, \Co{selfStart}
\need 15pt
\Paragraph{EXAMPLE}
\vspace{-16pt} 
\begin{Example}
Theoph.1 <- Theoph[ Theoph\$Subject == 1, ]
# response only
SSfol( Theoph.1\$Dose, Theoph.1\$Time, -2.5, 0.5, -3 )  
lKe <- -2.5
lKa <- 0.5
lCl <- -3
# response and gradient
SSfol( Theoph.1\$Dose, Theoph.1\$Time, lKe, lKa, lCl ) 
\end{Example}
\end{Helpfile}
\begin{Helpfile}{SSfpl}{Four-parameter Logistic Model}
This \Co{selfStart} model evaluates the four-parameter logistic
function and its gradient.  It has an \Co{initial} attribute that
will evaluate initial estimates of the parameters \Co{A}, \Co{B},
\Co{xmid}, and \Co{scal} for a given set of data.
\begin{Example}
SSfpl(input, A, B, xmid, scal)
\end{Example}
\begin{Argument}{ARGUMENTS}
\item[\Co{input:}]
a numeric vector of values at which to evaluate the model.
\item[\Co{A:}]
a numeric parameter representing the horizontal asymptote on
the left side (very small values of \Co{input}).
\item[\Co{B:}]
a numeric parameter representing the horizontal asymptote on
the right side (very large values of \Co{input}).
\item[\Co{xmid:}]
a numeric parameter representing the \Co{input} value at the
inflection point of the curve.  The value of \Co{SSfpl} will be
midway between \Co{A} and \Co{B} at \Co{xmid}.
\item[\Co{scal:}]
a numeric scale parameter on the \Co{input} axis.
\end{Argument}
\Paragraph{VALUE}
a numeric vector of the same length as \Co{input}.  It is the value of
the expression $A+(B-A)/\{1+\exp[(xmid-input)/scal]\}$.  If all of
the arguments \Co{A}, \Co{B}, \Co{xmid}, and \Co{scal} are
names of objects, as opposed to expressions or explicit numerical
values, the gradient matrix with respect to these names is attached as
an attribute named \Co{gradient}.
\Paragraph{SEE ALSO}
\Co{nls}, \Co{selfStart}
\need 15pt
\Paragraph{EXAMPLE}
\vspace{-16pt} 
\begin{Example}
Chick.1 <- ChickWeight[ChickWeight\$Chick == 1, ]
SSfpl( Chick.1\$Time, 13, 368, 14, 6 )  # response only
A <- 13
B <- 368
xmid <- 14
scal <- 6
SSfpl( Chick.1\$Time, A, B, xmid, scal ) # response and gradient
\end{Example}
\end{Helpfile}
\begin{Helpfile}{SSlogis}{Logistic model}
This \Co{selfStart} model evaluates the logistic
function and its gradient.  It has an \Co{initial} attribute that
will evaluate initial estimates of the parameters \Co{Asym},
\Co{xmid}, and \Co{scal} for a given set of data.
\begin{Example}
SSfpl(input, Asym, xmid, scal)
\end{Example}
\begin{Argument}{ARGUMENTS}
\item[\Co{input:}]
a numeric vector of values at which to evaluate the model.
\item[\Co{Asym:}]
a numeric parameter representing the asymptote.
\item[\Co{xmid:}]
a numeric parameter representing the \Co{x} value at the
inflection point of the curve.  The value of \Co{SSlogis} will be
\Co{Asym/2} at \Co{xmid}.
\item[\Co{scal:}]
a numeric scale parameter on the \Co{input} axis.
\end{Argument}
\Paragraph{VALUE}
a numeric vector of the same length as \Co{input}.  It is the value of
the expression $Asym/\{1+\exp[(xmid-input)/scal]\}$.  If all of
the arguments \Co{Asym}, \Co{xmid}, and \Co{scal} are
names of objects, as opposed to expressions or explicit numerical
values, the gradient matrix with respect to these names is attached as
an attribute named \Co{gradient}.
\Paragraph{SEE ALSO}
\Co{nls}, \Co{selfStart}
\need 15pt
\Paragraph{EXAMPLE}
\vspace{-16pt} 
\begin{Example}
Chick.1 <- ChickWeight[ChickWeight\$Chick == 1, ]
SSlogis( Chick.1\$Time, 368, 14, 6 )  # response only
Asym <- 368
xmid <- 14
scal <- 6
SSlogis( Chick.1\$Time, Asym, xmid, scal ) # response and gradient
\end{Example}
\end{Helpfile}
\begin{Helpfile}{SSmicmen}{Michaelis-Menten model}
This \Co{selfStart} model evaluates the Michaelis-Menten model and
its gradient.  It has an \Co{initial} attribute that
will evaluate initial estimates of the parameters \Co{Vm} and \Co{K}
\begin{Example}
SSmicmen(input, Vm, K)
\end{Example}
\begin{Argument}{ARGUMENTS}
\item[\Co{input:}]
a numeric vector of values at which to evaluate the model.
\item[\Co{Vm:}]
a numeric parameter representing the maximum value of the response.
\item[\Co{K:}]
a numeric parameter representing the \Co{input} value at
which half the maximum response is attained.  In the field of enzyme
kinetics this is called the Michaelis parameter.
\end{Argument}
\Paragraph{VALUE}
a numeric vector of the same length as \Co{input}.  It is the value of
the expression $Vm \cdot input/(K+input)$.  If both
the arguments \Co{Vm} and \Co{K} are
names of objects, as opposed to expressions or explicit numerical
values, the gradient matrix with respect to these names is attached as
an attribute named \Co{gradient}.
\Paragraph{SEE ALSO}
\Co{nls}, \Co{selfStart}
\need 15pt
\Paragraph{EXAMPLE}
\vspace{-16pt} 
\begin{Example}
PurTrt <- Puromycin[ Puromycin\$state == "treated", ]
SSmicmen( PurTrt\$conc, 200, 0.05 )  # response only
Vm <- 200
K <- 0.05
SSmicmen( PurTrt\$conc, Vm, K ) # response and gradient
\end{Example}
\end{Helpfile}
\begin{Helpfile}{summary.corStruct}{Summarize a corStruct Object}
This method function prepares \Co{object} to be printed using the
\Co{print.summary} method, by changing its class and adding a
\Co{structName} attribute to it.
\begin{Example}
summary(object, structName)
\end{Example}
\begin{Argument}{ARGUMENTS}
\item[\Co{object:}]
an object inheriting from class \Co{corStruct},
representing a correlation structure.
\item[\Co{structName:}]
an optional character string defining the type of
correlation structure associated with \Co{object}, to be used in
the \Co{print.summary} method. Defaults to
\Co{class(object)[1]}.
\end{Argument}
\Paragraph{VALUE}
an object identical to \Co{object}, but with its class changed to
\Co{summary.corStruct} and an additional attribute
\Co{structName}. The returned value inherits from the same classes
as \Co{object}.
\Paragraph{SEE ALSO}
\Co{print.summary.corStruct}
\need 15pt
\Paragraph{EXAMPLE}
\vspace{-16pt} 
\begin{Example}
cs1 <- corAR1(0.2)
summary(cs1)
\end{Example}
\end{Helpfile}
\begin{Helpfile}{summary.gls}{Summarize a gls Object}
Additional information about the linear model fit represented
by \Co{object} is extracted and included as components of
\Co{object}. The returned object is suitable for printing with the
\Co{print.summary.gls} method.
\begin{Example}
summary(object, verbose)
\end{Example}
\begin{Argument}{ARGUMENTS}
\item[\Co{object:}]
an object inheriting from class \Co{gls}, representing
a generalized least squares fitted linear model.
\item[\Co{verbose:}]
an optional logical value used to control the amount of
output in the \Co{print.summary.gls} method. Defaults to
\Co{FALSE}.
\end{Argument}
\Paragraph{VALUE}
an object inheriting from class \Co{summary.gls} with all components
included in \Co{object} (see \Co{glsObject} for a full
description of the components) plus the following components:
\begin{Argument}{}
\vspace{-16pt} 
\item[\Co{corBeta:}]
approximate correlation matrix for the coefficients
estimates
\item[\Co{tTable:}]
a data frame with columns \Co{Value},
\Co{Std.\ Error}, \Co{t-value}, and \Co{p-value} representing
respectively the coefficients estimates, their approximate standard
errors, the ratios between the estimates and their standard errors,
and the associated p-value under a t approximation. Rows
correspond to the different coefficients.
\item[\Co{residuals:}]
if more than five observations are used in the
\Co{gls} fit, a vector with the minimum, 25
quantile, and maximum of the residuals distribution; else the
residuals.
\item[\Co{AIC:}]
the Akaike Information Criterion corresponding to
\Co{object}.
\item[\Co{BIC:}]
the Bayesian Information Criterion corresponding to
\Co{object}.
\end{Argument}
\Paragraph{SEE ALSO}
\Co{gls}, \Co{AIC}, \Co{BIC},
\Co{print.summary.gls}
\need 15pt
\Paragraph{EXAMPLE}
\vspace{-16pt} 
\begin{Example}
fm1 <- gls(follicles \Twiddle sin(2*pi*Time) + cos(2*pi*Time), Ovary,
           correlation = corAR1(form = \Twiddle 1 | Mare))
summary(fm1)
\end{Example}
\end{Helpfile}
\begin{Helpfile}{summary.lmList}{Summarize an lmList Object}
The \Co{summary.lm} method is applied to each \Co{lm} component of
\Co{object} to produce summary information on the individual fits,
which is organized into a list of summary statistics. The returned
object is suitable for printing with the \Co{print.summary.lmList}
method.
\begin{Example}
summary(object, pool)
\end{Example}
\begin{Argument}{ARGUMENTS}
\item[\Co{object:}]
an object inheriting from class \Co{lmList}, representing
a list of \Co{lm} fitted objects.
\item[\Co{pool:}]
an optional logical value indicating whether a pooled
estimate of the residual standard error should be used. Default is
\Co{attr(object, "pool")}.
\end{Argument}
\Paragraph{VALUE}
a list with summary statistics obtained by applying \Co{summary.lm}
to the elements of \Co{object}, inheriting from class
\Co{summary.lmList}. The components of \Co{value} are:
\begin{Argument}{}
\vspace{-16pt} 
\item[\Co{call:}]
a list containing an image of the \Co{lmList} call that
produced \Co{object}.
\item[\Co{coefficients:}]
a three dimensional array with summary information
on the \Co{lm} coefficients. The first dimension corresponds to
the names of the \Co{object} components, the second dimension is
given by   \Co{"Value"}, \Co{"Std. Error"}, \Co{"t value"},
and \Co{"Pr(>|t|)"}, corresponding, respectively, to the
coefficient estimates and their associated standard errors,
t-values, and p-values. The third dimension is given by the
coefficients names.
\item[\Co{correlation:}]
a three dimensional array with the 
correlations between the individual \Co{lm} coefficient
estimates. The first dimension corresponds to the names of the
\Co{object} components. The third dimension is given by the
coefficients names. For each coefficient, the rows of the associated
array give the correlations between that coefficient and the
remaining coefficients, by \Co{lm} component.
\item[\Co{cov.unscaled:}]
a three dimensional array with the unscaled
variances/covariances for the individual \Co{lm} coefficient
estimates (giving the estimated variance/covariance for the
coefficients, when multiplied by the estimated residual errors). The
first dimension corresponds to the names of the \Co{object}
components. The third dimension is given by the
coefficients names. For each coefficient, the rows of the associated
array give the unscaled covariances between that coefficient and the
remaining coefficients, by \Co{lm} component.
\item[\Co{df:}]
an array with the number of degrees of freedom for the model
and for residuals, for each \Co{lm} component.
\item[\Co{df.residual:}]
the total number of degrees of freedom for
residuals, corresponding to the sum of residuals df of all \Co{lm}
components.
\item[\Co{fstatistics:}]
an array with the F test statistics and
corresponding degrees of freedom, for each \Co{lm} component.
\item[\Co{pool:}]
the value of the \Co{pool} argument to the function.
\item[\Co{r.squared:}]
a vector with the multiple R-squared statistics for
each \Co{lm} component.
\item[\Co{residuals:}]
a list with components given by the residuals from
individual \Co{lm} fits.
\item[\Co{RSE:}]
the pooled estimate of the residual standard error.
\item[\Co{sigma:}]
a vector with the residual standard error estimates for
the individual \Co{lm} fits.
\item[\Co{terms:}]
the terms object used in fitting the individual \Co{lm}
components.
\end{Argument}
\Paragraph{SEE ALSO}
\Co{lmList}, \Co{print.summary.lmList}
\need 15pt
\Paragraph{EXAMPLE}
\vspace{-16pt}
\begin{Example}
fm1 <- lmList(distance {\Twiddle} age | Subject, Orthodont)
summary(fm1)
\end{Example}
\end{Helpfile}
\begin{Helpfile}{summary.lme}{Summarize an lme Object}
Additional information about the linear mixed-effects fit represented
by \Co{object} is extracted and included as components of
\Co{object}. The returned object is suitable for printing with the
\Co{print.summary.lme} method.
\begin{Example}
summary(object, verbose)
\end{Example}
\begin{Argument}{ARGUMENTS}
\item[\Co{object:}]
an object inheriting from class \Co{lme}, representing
a fitted linear mixed-effects model.
\item[\Co{verbose:}]
an optional logical value used to control the amount of
output in the \Co{print.summary.lme} method. Defaults to
\Co{FALSE}.
\end{Argument}
\Paragraph{VALUE}
an object inheriting from class \Co{summary.lme} with all components
included in \Co{object} (see \Co{lmeObject} for a full
description of the components) plus the following components:
\begin{Argument}{}
\vspace{-16pt} 
\item[\Co{corFixed:}]
approximate correlation matrix for the fixed effects
estimates 
\item[\Co{zTable:}]
a data frame with columns \Co{Value},
\Co{Std.\ Error}, \Co{z-value}, and \Co{p-value} representing
respectively the fixed effects estimates, their approximate standard
errors, the ratios between the estimates and their standard errors,
and the associated p-value under a normal approximation. Rows
correspond to the different fixed effects.
\item[\Co{residuals:}]
if more than five observations are used in the
\Co{lme} fit, a vector with the minimum, 25
quantile, and maximum of the innermost grouping level residuals
distribution; else the innermost grouping level residuals.
\item[\Co{AIC:}]
the Akaike Information Criterion corresponding to
\Co{object}.
\item[\Co{BIC:}]
the Bayesian Information Criterion corresponding to
\Co{object}.
\end{Argument}
\Paragraph{SEE ALSO}
\Co{lme}, \Co{AIC}, \Co{BIC},
\Co{print.summary.lme}
\need 15pt
\Paragraph{EXAMPLE}
\vspace{-16pt} 
\begin{Example}
fm1 <- lme(distance \Twiddle age, Orthodont, random = \Twiddle age | Subject)
summary(fm1)
\end{Example}
\end{Helpfile}
\begin{Helpfile}{summary.modelStruct}{Summarize modelStruct}
This method function applies \Co{summary} to each element of
\Co{object}.
\begin{Example}
summary(object)
\end{Example}
\begin{Argument}{ARGUMENTS}
\item[\Co{object:}]
an object inheriting from class \Co{modelStruct},
representing a list of model components, such as \Co{corStruct} and
\Co{varFunc} objects.
\end{Argument}
\Paragraph{VALUE}
a list with elements given by the summarized components of
\Co{object}. The returned value is of class
\Co{summary.modelStruct}, also inheriting from the same classes as
\Co{object}.
\Paragraph{SEE ALSO}
\Co{print.summary.modelStruct}
\need 15pt
\Paragraph{EXAMPLE}
\vspace{-16pt} 
\begin{Example}
lms1 <- lmeStruct(reStruct = reStruct(pdDiag(diag(2), \Twiddle age)),
   corStruct = corAR1(0.3))
summary(lms1)
\end{Example}
\end{Helpfile}
\begin{Helpfile}{summary.nlsList}{Summarize an nlsList Object}
The \Co{summary.nls} method is applied to each \Co{nls} component of
\Co{object} to produce summary information on the individual fits,
which is organized into a list of summary statistics. The returned
object is suitable for printing with the \Co{print.summary.nlsList}
method.
\begin{Example}
summary(object, pool)
\end{Example}
\begin{Argument}{ARGUMENTS}
\item[\Co{object:}]
an object inheriting from class \Co{nlsList},
representing a list of \Co{nls} fitted objects.
\item[\Co{pool:}]
an optional logical value indicating whether a pooled
estimate of the residual standard error should be used. Default is
\Co{attr(object, "pool")}.
\end{Argument}
\Paragraph{VALUE}
a list with summary statistics obtained by applying \Co{summary.nls}
to the elements of \Co{object}, inheriting from class
\Co{summary.nlsList}. The components of \Co{value} are:
\begin{Argument}{}
\vspace{-16pt} 
\item[\Co{call:}]
a list containing an image of the \Co{nlsList} call that
produced \Co{object}.
\item[\Co{parameters:}]
a three dimensional array with summary information
on the \Co{nls} coefficients. The first dimension corresponds to
the names of the \Co{object} components, the second dimension is
given by   \Co{"Value"}, \Co{"Std. Error"}, \Co{"t value"},
and \Co{"Pr(>|t|)"}, corresponding, respectively, to the
coefficient estimates and their associated standard errors,
t-values, and p-values. The third dimension is given by the
coefficients names.
\item[\Co{correlation:}]
a three dimensional array with the 
correlations between the individual \Co{nls} coefficient
estimates. The first dimension corresponds to the names of the
\Co{object} components. The third dimension is given by the
coefficients names. For each coefficient, the rows of the associated
array give the correlations between that coefficient and the
remaining coefficients, by \Co{nls} component.
\item[\Co{cov.unscaled:}]
a three dimensional array with the unscaled
variances/covariances for the individual \Co{lm} coefficient
estimates (giving the estimated variance/covariance for the
coefficients, when multiplied by the estimated residual errors). The
first dimension corresponds to the names of the \Co{object}
components. The third dimension is given by the
coefficients names. For each coefficient, the rows of the associated
array give the unscaled covariances between that coefficient and the
remaining coefficients, by \Co{nls} component.
\item[\Co{df:}]
an array with the number of degrees of freedom for the model
and for residuals, for each \Co{nls} component.
\item[\Co{df.residual:}]
the total number of degrees of freedom for
residuals, corresponding to the sum of residuals df of all \Co{nls}
components.
\item[\Co{pool:}]
the value of the \Co{pool} argument to the function.
\item[\Co{RSE:}]
the pooled estimate of the residual standard error.
\item[\Co{sigma:}]
a vector with the residual standard error estimates for
the individual \Co{lm} fits.
\end{Argument}
\Paragraph{SEE ALSO}
\Co{nlsList}, \Co{summary.nls}
\need 15pt
\Paragraph{EXAMPLE}
\vspace{-16pt}
\begin{Example}
fm1 <- nlsList(weight {\Twiddle} SSlogis(Time, Asym, xmid, scal) | Plot, 
               Soybean)
summary(fm1)
\end{Example}
\end{Helpfile}
\begin{Helpfile}{summary.pdMat}{Summarize a pdMat Object}
Attributes \Co{structName} and \Co{noCorrelation}, with the values
of the corresponding arguments to the method function, are appended to
\Co{object} and its class is changed to \Co{summary.pdMat}.
\begin{Example}
summary(object, structName, noCorrelation)
\end{Example}
\begin{Argument}{ARGUMENTS}
\item[\Co{object:}]
an object inheriting from class \Co{pdMat}, representing
a positive definite matrix.
\item[\Co{structName:}]
an optional character string with a description of
the \Co{pdMat} class. Default depends on the method function:
\Co{"Blocked"} for \Co{pdBlocked},
\Co{"Compound Symmetry"} for \Co{pdCompSymm}, \Co{"Diagonal"}
for \Co{pdDiag}, \Co{"Multiple of an Identity"} for
\Co{pdIdent}, `"General Positive-Definite, Natural
     Parametrization"' for \Co{pdNatural}, `"General
     Positive-Definite"' for \Co{pdSymm}, and
\Co{data.class(object)} for \Co{pdMat}.
\item[\Co{noCorrelation:}]
an optional logical value indicating whether
correlations are to be printed in \Co{print.summary.pdMat}. Default
depends on the method function: \Co{FALSE} for \Co{pdDiag} and
\Co{pdIdent}, and \Co{TRUE} for all other classes.
\end{Argument}
\Paragraph{VALUE}
an object similar to \Co{object}, with additional attributes
\Co{structName} and \Co{noCorrelation}, inheriting from class
\Co{summary.pdMat}.
\Paragraph{SEE ALSO}
\Co{print.summary.pdMat}, \Co{pdMat}
\need 15pt
\Paragraph{EXAMPLE}
\vspace{-16pt} 
\begin{Example}
summary(pdSymm(diag(4)))
\end{Example}
\end{Helpfile}
\begin{Helpfile}{summary.varFunc}{Summarize varFunc Object}
A \Co{structName} attribute, with the value of corresponding
argument, is appended to \Co{object} and its class is changed to
\Co{summary.varFunc}.
\begin{Example}
summary(object, structName)
\end{Example}
\begin{Argument}{ARGUMENTS}
\item[\Co{object:}]
an object inheriting from class \Co{varFunc},
representing a variance function structure.
\item[\Co{structName:}]
an optional character string with a description of
the \Co{varFunc} class. Default depends on the method function:
\Co{"Combination of variance functions"} for \Co{varComb},
\Co{"Constant plus power of covariate"} for \Co{varConstPower},
\Co{"Exponential of variance covariate"} for \Co{varExp},
\Co{"Different standard deviations per stratum"} for \Co{varIdent},
\Co{"Power of variance covariate"} for \Co{varPower}, and
\Co{data.class(object)} for \Co{varFunc}.
\end{Argument}
\Paragraph{VALUE}
an object similar to \Co{object}, with an additional attribute
\Co{structName}, inheriting from class \Co{summary.varFunc}.
\Paragraph{SEE ALSO}
\Co{print.summary.varFunc}
\need 15pt
\Paragraph{EXAMPLE}
\vspace{-16pt} 
\begin{Example}
vf1 <- varPower(0.3, form = \Twiddle age)
vf1 <- initialize(vf1, Orthodont)
summary(vf1)
\end{Example}
\end{Helpfile}
\begin{Helpfile}{update.gls}{Update a gls Object}
The non-missing arguments in the call to the \Co{update.gls} method
replace the corresponding arguments in the original call used to
produce \Co{object} and \Co{gls} is used with the modified call to
produce an updated fitted object.
\begin{Example}
update(object, model, data, correlation, weights, subset, method,
       na.action, control)
\end{Example}
\begin{Argument}{ARGUMENTS}
\item[\Co{object:}]
an object inheriting from class \Co{gls}, representing
a generalized least squares fitted linear model.
\item[\Co{other arguments:}]
defined as in \Co{gls}. See the
documentation on that function for descriptions of and default values
for these arguments.
\end{Argument}
\Paragraph{VALUE}
an updated \Co{gls} object.
\Paragraph{SEE ALSO}
\Co{gls}
\need 15pt
\Paragraph{EXAMPLE}
\vspace{-16pt} 
\begin{Example}
fm1 <- gls(follicles \Twiddle sin(2*pi*Time) + cos(2*pi*Time), Ovary,
           correlation = corAR1(form = \Twiddle 1 | Mare))
fm2 <- update(fm1, weights = varPower())
\end{Example}
\end{Helpfile}
\begin{Helpfile}{update.gnls}{Update a gnls Object}
The non-missing arguments in the call to the \Co{update.gnls} method
replace the corresponding arguments in the original call used to
produce \Co{object} and \Co{gnls} is used with the modified call to
produce an updated fitted object.
\begin{Example}
update(object, model, data, params, start, correlation, weights, 
       subset, na.action, naPattern, control, verbose) 
\end{Example}
\begin{Argument}{ARGUMENTS}
\item[\Co{object:}]
an object inheriting from class \Co{gnls}, representing
a generalized nonlinear least squares fitted model.
\item[\Co{other arguments:}]
defined as in \Co{gnls}. See the
documentation on that function for descriptions of and default values
for these arguments.
\end{Argument}
\Paragraph{VALUE}
an updated \Co{gnls} object.
\Paragraph{SEE ALSO}
\Co{gnls}
\need 15pt
\Paragraph{EXAMPLE}
\vspace{-16pt}
\begin{Example}
fm1 <- gnls(weight {\Twiddle} SSlogis(Time, Asym, xmid, scal), Soybean,
            weights = varPower())
fm2 <- update(fm1, correlation = corAR1())
\end{Example}
\end{Helpfile}
\begin{Helpfile}{update.groupedData}{Update a groupedData Object}
The non-missing arguments in the call to the \Co{update.groupedData}
method replace the corresponding arguments in the original call used to
produce \Co{object} and \Co{groupedData} is used with the modified
call to produce an updated fitted object.
\begin{Example}
update(object, formula, data, order.groups, FUN, outer, inner, 
       labels, units)
\end{Example}
\begin{Argument}{ARGUMENTS}
\item[\Co{object:}]
an object inheriting from class \Co{groupedData}.
\item[\Co{other arguments:}]
defined as in \Co{groupedData}. See the
documentation on that function for descriptions of and default values
for these arguments.
\end{Argument}
\Paragraph{VALUE}
an updated \Co{groupedData} object.
\Paragraph{SEE ALSO}
\Co{groupedData}
\need 15pt
\Paragraph{EXAMPLE}
\vspace{-16pt} 
\begin{Example}
Orthodont2 <- update(Orthodont, FUN = mean)
\end{Example}
\end{Helpfile}
\begin{Helpfile}{update.lmList}{Update an lmList Object}
The non-missing arguments in the call to the \Co{update.lmList} method
replace the corresponding arguments in the original call used to
produce \Co{object} and \Co{lmList} is used with the modified call to
produce an updated fitted object.
\begin{Example}
update(object, formula, data, level, subset, na.action, 
       control, pool)
\end{Example}
\begin{Argument}{ARGUMENTS}
\item[\Co{object:}]
an object inheriting from class \Co{lmList}, representing
a list of \Co{lm} fitted objects.
\item[\Co{formula:}]
a two-sided linear formula with the common model for the individuals
\Co{lm} fits.
\item[\Co{other arguments:}]
defined as in \Co{lmList}. See the
documentation on that function for descriptions of and default values
for these arguments.
\end{Argument}
\Paragraph{VALUE}
an updated \Co{lmList} object.
\Paragraph{SEE ALSO}
\Co{lmList}
\need 15pt
\Paragraph{EXAMPLE}
\vspace{-16pt}
\begin{Example}
fm1 <- lmList(Orthodont)
fm2 <- update(fm1, distance {\Twiddle} I(age - 11))
\end{Example}
\end{Helpfile}
\begin{Helpfile}{update.lme}{Update an lme Object}
The non-missing arguments in the call to the \Co{update.lme} method
replace the corresponding arguments in the original call used to
produce \Co{object} and \Co{lme} is used with the modified call to
produce an updated fitted object.
\begin{Example}
update(object, fixed, data, random, correlation, weights, 
       subset, method, na.action, control)
\end{Example}
\begin{Argument}{ARGUMENTS}
\item[\Co{object:}]
an object inheriting from class \Co{lme}, representing
a fitted linear mixed-effects model.
\item[\Co{other arguments:}]
defined as in \Co{lme}. See the
documentation on that function for descriptions of and default values
for these arguments.
\end{Argument}
\Paragraph{VALUE}
an updated \Co{lme} object.
\Paragraph{SEE ALSO}
\Co{lme}
\need 15pt
\Paragraph{EXAMPLE}
\vspace{-16pt} 
\begin{Example}
fm1 <- lme(distance \Twiddle age, Orthodont, random = \Twiddle age | Subject)
fm2 <- update(fm1, distance \Twiddle age * Sex)
\end{Example}
\end{Helpfile}
\begin{Helpfile}{update.modelStruct}{Update a modelStruct Object}
This method function updates each element of \Co{object}, allowing
the access to \Co{data}.
\begin{Example}
update(object, data)
\end{Example}
\begin{Argument}{ARGUMENTS}
\item[\Co{object:}]
an object inheriting from class \Co{modelStruct},
representing a list of model components, such as \Co{corStruct} and
\Co{varFunc} objects.
\item[\Co{data:}]
a data frame in which to evaluate the variables needed for
updating the elements of \Co{object}. 
\end{Argument}
\Paragraph{VALUE}
an object similar to \Co{object} (same class, length, and names),
but with updated elements.
\Paragraph{NOTE} This method function is generally only used inside model fitting
functions like \Co{lme} and \Co{gls}, which allow model
components, such as variance functions.
\end{Helpfile}
\begin{Helpfile}{update.nlme}{Update an nlme Object}
The non-missing arguments in the call to the \Co{update.nlme} method
replace the corresponding arguments in the original call used to
produce \Co{object} and \Co{nlme} is used with the modified call to
produce an updated fitted object.
\begin{Example}
update(object, model, data, fixed, random, groups, start,
       correlation, weights, subset, method, na.action, 
       naPattern, control, verbose)
\end{Example}
\begin{Argument}{ARGUMENTS}
\item[\Co{object:}]
an object inheriting from class \Co{nlme}, representing
a fitted nonlinear mixed-effects model.
\item[\Co{other arguments:}]
defined as in \Co{nlme}. See the
documentation on that function for descriptions of and default values
for these arguments.
\end{Argument}
\Paragraph{VALUE}
an updated \Co{nlme} object.
\Paragraph{SEE ALSO}
\Co{nlme}
\need 15pt
\Paragraph{EXAMPLE}
\vspace{-16pt}
\begin{Example}
fm1 <- nlme(weight {\Twiddle} SSlogis(Time, Asym, xmid, scal), 
            data = Soybean, fixed = Asym + xmid + scal {\Twiddle} 1, 
            start = c(18, 52, 7.5))
fm2 <- update(fm1, weights = varPower())
\end{Example}
\end{Helpfile}
\begin{Helpfile}{update.nlsList}{Update an nlsList Object}
The non-missing arguments in the call to the \Co{update.nlsList} method
replace the corresponding arguments in the original call used to
produce \Co{object} and \Co{nlsList} is used with the modified call to
produce an updated fitted object.
\begin{Example}
update(object, model, data, start, control, level, subset, 
       na.action, control, pool)
\end{Example}
\begin{Argument}{ARGUMENTS}
\item[\Co{object:}]
an object inheriting from class \Co{nlsList}, representing
a list of fitted \Co{nls} objects.
\item[\Co{other arguments:}]
defined as in \Co{nlsList}. See the
documentation on that function for descriptions of and default values
for these arguments.
\end{Argument}
\Paragraph{VALUE}
an updated \Co{nlsList} object.
\Paragraph{SEE ALSO}
\Co{nlsList}
\need 15pt
\Paragraph{EXAMPLE}
\vspace{-16pt}
\begin{Example}
fm1 <- nlsList(weight {\Twiddle} SSlogis(Time, Asym, xmid, scal) | Plot, 
               Soybean)
fm2 <- update(fm1, start = list(Asym = 23, xmid = 57, scal = 9))
\end{Example}
\end{Helpfile}
\begin{Helpfile}{update.varFunc}{Update varFunc Object}
If the \Co{formula(object)} includes a \Co{"."} term, representing
a fitted object, the variance covariate needs to be updated upon
completion of an optimization cycle (in which the variance function
weights are kept fixed). This method function allows a reevaluation of
the variance covariate using the current fitted object and,
optionally, other variables in the original data.
\begin{Example}
update(object, data)
\end{Example}
\begin{Argument}{ARGUMENTS}
\item[\Co{object:}]
an object inheriting from class \Co{varFunc},
representing a variance function structure.
\item[\Co{data:}]
a list with a component named \Co{"."} with the current
version of the fitted object (from which fitted values, coefficients,
and residuals can be extracted) and, if necessary, other variables
used to evaluate the variance covariate(s).
\end{Argument}
\Paragraph{VALUE}
if \Co{formula(object)} includes a \Co{"."} term, an
\Co{varFunc} object similar to \Co{object}, but with the 
variance covariate reevaluated at the current fitted object value;
else \Co{object} is returned unchanged.
\Paragraph{SEE ALSO}
\Co{needUpdate}, \Co{covariate<-.varFunc}
\end{Helpfile}
\begin{Helpfile}{varClasses}{Variance Function Classes}
Standard classes of variance function structures (\Co{varFunc}) 
available in the \Co{nlme} library. Covariates included in the
variance function, denoted by variance covariates, may involve
functions of the fitted model object, such as the fitted values and
the residuals. Different coefficients may be assigned to the levels of
a classification factor.
\begin{Argument}{STANDARD CLASSES}
\item[\Co{varExp:}]
exponential of a variance covariate.
\item[\Co{varPower:}]
power of a variance covariate.
\item[\Co{varConstPower:}]
constant plus power of a variance covariate.
\item[\Co{varIdent:}]
constant variance(s), generally used to allow
different variances according to the levels of a classification
factor.
\item[\Co{varFixed:}]
fixed weights, determined by a variance covariate.
\item[\Co{varComb:}]
combination of variance functions.
\Paragraph{NOTE} Users may define their own \Co{varFunc} classes by specifying a
\Co{constructor} function and, at a minimum, methods for the
functions \Co{coef}, \Co{coef<-}, and \Co{initialize}. For
examples of these functions, see the methods for class
\Co{varPower}.
\end{Argument}
\Paragraph{SEE ALSO}
\Co{varExp}, \Co{varPower},
\Co{varConstPower}, \Co{varIdent},
\Co{varFixed}, \Co{varComb}
\end{Helpfile}
\begin{Helpfile}{varComb}{Combination of Variance Functions}
This function is a constructor for the \Co{varComb} class,
representing a combination of variance functions. The corresponding
variance function is equal to the product of the variance functions of
the \Co{varFunc} objects listed in \Co{...}.
\begin{Example}
varComb(...)
\end{Example}
\begin{Argument}{ARGUMENTS}
\item[\Co{...:}]
objects inheriting from class \Co{varFunc} representing
variance function structures.
\end{Argument}
\Paragraph{VALUE}
a \Co{varComb} object representing a combination of variance
functions, also inheriting from class \Co{varFunc}.
\Paragraph{SEE ALSO}
\Co{varWeights.varComb}, \Co{coef.varComb}
\need 15pt
\Paragraph{EXAMPLE}
\vspace{-16pt} 
\begin{Example}
vf1 <- varComb(varIdent(form = \Twiddle 1|Sex), varPower())
\end{Example}
\end{Helpfile}
\begin{Helpfile}{varConstPower}{Constant Plus Power Variance Function}
This function is a constructor for the \Co{varConstPower} class,
representing a constant plus power variance function
structure. Letting $v$ denote the variance covariate and
$\sigma^2(v)$ denote the variance function evaluated at
$v$, the constant plus power variance function is defined as
$\sigma^2(v) = \left(\theta_1 + |v|^{\theta_2}\right)^2$,
where $\theta_1$, $\theta_2$ are the variance  
function coefficients. When a grouping factor is present, different
$\theta_1$, $\theta_2$ are used for each factor level.
\begin{Example}
varConstPower(const, power, form, fixed)
\end{Example}
\begin{Argument}{ARGUMENTS}
\item[\Co{const, power:}]
optional numeric vectors, or lists of numeric
values, with, respectively, the coefficients for the constant 
and the power terms. Both arguments must have length one, unless a
grouping factor is specified in \Co{form}. If either argument has
length greater than one, it must have names which identify its
elements to the levels of the grouping factor defined in
\Co{form}. If a grouping factor is present in 
\Co{form} and the argument has length one, its value will be
assigned to all grouping levels. Only positive values are allowed
for \Co{const}. Default is \Co{numeric(0)}, which
results in a vector of zeros of appropriate length being assigned to
the coefficients when \Co{object} is initialized (corresponding
to constant variance equal to one).
\item[\Co{form:}] an optional one-sided formula of the form
  \Co{\Twiddle v}, or \Co{\Twiddle v | g}, specifying a variance
  covariate \Co{v} and, optionally, a grouping factor \Co{g} for the
  coefficients. The variance covariate must evaluate to a numeric
  vector and may involve expressions using \Co{"."}, representing a
  fitted model object from which fitted values (\Co{fitted(.)}) and
  residuals (\Co{resid(.)}) can be extracted (this allows the variance
  covariate to be updated during the optimization of an objective
  function). When a grouping factor is present in \Co{form}, a
  different coefficient value is used for each of its levels. Several
  grouping variables may be simultaneously specified, separated by the
  \Co{*} operator, like in \Co{{\Twiddle} v | g1 * g2 * g3}. In this
  case, the levels of each grouping variable are pasted together and
  the resulting factor is used to group the observations. Defaults to
  \Co{\Twiddle fitted(.)} representing a variance covariate given by
  the fitted values of a fitted model object and no grouping factor.
\item[\Co{fixed:}]
an optional list with components \Co{const} and/or
\Co{power}, consisting of numeric vectors, or lists of numeric
values, specifying the values at which some or all of the
coefficients in the variance function should be fixed. If a grouping
factor is specified in \Co{form}, the components of \Co{fixed}
must have names identifying which coefficients are to be
fixed. Coefficients included in \Co{fixed} are not allowed to vary
during the optimization of an objective function. Defaults to
\Co{NULL}, corresponding to no fixed coefficients.
\end{Argument}
\Paragraph{VALUE}
a \Co{varConstPower} object representing a constant plus power
variance function structure, also inheriting from class
\Co{varFunc}.
\Paragraph{SEE ALSO}
\Co{varWeights.varFunc}, \Co{coef.varConstPower}
\need 15pt
\Paragraph{EXAMPLE}
\vspace{-16pt} 
\begin{Example}
vf1 <- varConstPower(1.2, 0.2, form = \Twiddle age|Sex)
\end{Example}
\end{Helpfile}
\begin{Helpfile}{VarCorr}{Extract variance and correlation components}
This function calculates the estimated variances, standard
deviations, and correlations between the random-effects terms in a
linear mixed-effects model, of class \Co{lme}, or a nonlinear
mixed-effects model, of class \Co{nlme}. The within-group error
variance and standard deviation are also calculated.
\begin{Example}
VarCorr(object, sigma, rdig)
\end{Example}
\begin{Argument}{ARGUMENTS}
\item[\Co{object:}]
a fitted model object, usually an object inheriting from
class \Co{lme}. 
\item[\Co{sigma:}]
an optional numeric value used as a multiplier for the
standard deviations. Default is \Co{1}.
\item[\Co{rdig:}]
an optional integer value specifying the number of digits
used to represent correlation estimates. Default is \Co{3}.
\end{Argument}
\Paragraph{VALUE}
a matrix with the estimated variances, standard deviations, and
correlations for the random effects. The first two columns, named
\Co{Variance} and \Co{StdDev}, give, respectively, the variance
and the standard deviations. If there are correlation components in
the random effects model, the third column, named \Co{Corr},
and the remaining unnamed columns give the estimated correlations
among random effects within the same level of grouping. The
within-group error variance and standard deviation are included as
the last row in the matrix.
\need 15pt
\Paragraph{EXAMPLE}
\vspace{-16pt} 
\begin{Example}
fm1 <- lme(distance {\Twiddle} age, data = Orthodont, random = {\Twiddle}age)
VarCorr(fm1)
\end{Example}
\end{Helpfile}
\begin{Helpfile}{varExp}{Exponential Variance Function}
This function is a constructor for the \Co{varExp} class,
representing an exponential variance function structure. Letting
$v$ denote the variance covariate and $\sigma^2(v)$
denote the variance function evaluated at $v$, the exponential
variance function is defined as $\sigma^2(v) = \exp(2\theta v)$, where
$\theta$ is the variance function coefficient. When a grouping factor
is present, a different $\theta$ is used for each factor level.
\begin{Example}
varExp(value, form, fixed)
\end{Example}
\begin{Argument}{ARGUMENTS}
\item[\Co{value:}]
an optional numeric vector, or list of numeric values,
with the variance function coefficients. \Co{Value} must have
length one, unless a grouping factor is specified in \Co{form}.
If \Co{value} has length greater than one, it must have names
which identify its elements to the levels of the grouping factor
defined in \Co{form}. If a grouping factor is present in
\Co{form} and \Co{value} has length one, its value will be
assigned to all grouping levels. Default is \Co{numeric(0)}, which
results in a vector of zeros of appropriate length being assigned to
the coefficients when \Co{object} is initialized (corresponding
to constant variance equal to one).
\item[\Co{form:}] an optional one-sided formula of the form
  \Co{\Twiddle v}, or \Co{\Twiddle v | g}, specifying a variance
  covariate \Co{v} and, optionally, a grouping factor \Co{g} for the
  coefficients. The variance covariate must evaluate to a numeric
  vector and may involve expressions using \Co{"."}, representing a
  fitted model object from which fitted values (\Co{fitted(.)}) and
  residuals (\Co{resid(.)}) can be extracted (this allows the variance
  covariate to be updated during the optimization of an objective
  function). When a grouping factor is present in \Co{form}, a
  different coefficient value is used for each of its levels. Several
  grouping variables may be simultaneously specified, separated by the
  \Co{*} operator, like in \Co{{\Twiddle} v | g1 * g2 * g3}. In this
  case, the levels of each grouping variable are pasted together and
  the resulting factor is used to group the observations. Defaults to
  \Co{\Twiddle fitted(.)} representing a variance covariate given by
  the fitted values of a fitted model object and no grouping factor.
\item[\Co{fixed:}]
an optional numeric vector, or list of numeric values,
specifying the values at which some or all of the  coefficients in
the variance function should be fixed. If a grouping factor is
specified in \Co{form}, \Co{fixed} must have names identifying
which coefficients are to be fixed. Coefficients included in
\Co{fixed} are not allowed to vary during the optimization of an
objective function. Defaults to \Co{NULL}, corresponding to no
fixed coefficients.
\end{Argument}
\Paragraph{VALUE}
a \Co{varExp} object representing an exponential variance function
structure, also inheriting from class \Co{varFunc}.
\Paragraph{SEE ALSO}
\Co{varWeights.varFunc}, \Co{coef.varExp}
\need 15pt
\Paragraph{EXAMPLE}
\vspace{-16pt} 
\begin{Example}
vf1 <- varExp(0.2, form = \Twiddle age|Sex)
\end{Example}
\end{Helpfile}
\begin{Helpfile}{varFixed}{Fixed Variance Function}
This function is a constructor for the \Co{varFixed} class,
representing a variance function with fixed variances. Letting $v$
denote the variance covariate defined in \Co{value}, the variance
function $\sigma^2(v)$ for this class is
$\sigma^2(v)=|v|$. The variance covariate $v$ is
evaluated once at initialization and remains fixed thereafter. No
coefficients are required to represent this variance function.
\begin{Example}
varFixed(value)
\end{Example}
\begin{Argument}{ARGUMENTS}
\item[\Co{value:}]
a one-sided formula of the form \Co{\Twiddle v} specifying a
variance covariate \Co{v}. Grouping factors are ignored.
\end{Argument}
\Paragraph{VALUE}
a \Co{varFixed} object representing a fixed variance function
structure, also inheriting from class \Co{varFunc}.
\Paragraph{SEE ALSO}
\Co{varWeights.varFunc}, \Co{varFunc}
\need 15pt
\Paragraph{EXAMPLE}
\vspace{-16pt} 
\begin{Example}
vf1 <- varFixed(\Twiddle age)
\end{Example}
\end{Helpfile}
\begin{Helpfile}{varFunc}{Variance Function Structure}
If \Co{object} is a one-sided formula, it is used as the argument to
\Co{varFixed} and the resulting object is returned. Else, if
\Co{object} inherits from class \Co{varFunc}, it is returned
unchanged.
\begin{Example}
varFunc(object)
\end{Example}
\begin{Argument}{ARGUMENTS}
\item[\Co{object:}]
either an one-sided formula specifying a variance
covariate, or an object inheriting from class \Co{varFunc},
representing a variance function structure.
\end{Argument}
\Paragraph{VALUE}
an object from class \Co{varFunc}, representing a variance function
structure.
\Paragraph{SEE ALSO}
\Co{varFixed}, \Co{varWeights.varFunc},
\Co{coef.varFunc}
\need 15pt
\Paragraph{EXAMPLE}
\vspace{-16pt} 
\begin{Example}
vf1 <- varFunc(\Twiddle age)
\end{Example}
\end{Helpfile}
\begin{Helpfile}{varIdent}{Different Variances per Group}
This function is a constructor for the \Co{varIdent} class,
representing a constant variance function structure. If no grouping
factor is present in \Co{form}, the variance function is constant
and equal to one, and no coefficients required to represent it. When
\Co{form} includes a grouping factor with $M > 1$ levels, the
variance function allows $M$ different variances, one  for each level of
the factor. For identifiability reasons, the coefficients of the
variance function represent the ratios between the variances and a
reference variance (corresponding to a reference group
level). Therefore, only $M-1$ coefficients are needed to represent
the variance function. By default, if the elements in \Co{value} are
unnamed, the first group level is taken as the reference level.
\begin{Example}
varIdent(value, form, fixed)
\end{Example}
\begin{Argument}{ARGUMENTS}
\item[\Co{value:}]
an optional numeric vector, or list of numeric values,
with the variance function coefficients. If no grouping factor is
present in \Co{form}, this argument is ignored, as the resulting
variance function contains no coefficients. If \Co{value} has
length one, its value is repeated for all coefficients in the
variance function. If \Co{value} has length greater than one, it
must have length equal to the number of grouping levels minus one
and names which identify its elements to the levels of the grouping
factor. Only positive values are allowed for this argument. Default
is \Co{numeric(0)}, which results in a vector of zeros of
appropriate length being assigned to the coefficients when
\Co{object} is initialized (corresponding to constant variance 
equal to one).
\item[\Co{form:}] an optional one-sided formula of the form
  \Co{\Twiddle v}, or \Co{\Twiddle v | g}, specifying a variance
  covariate \Co{v} and, optionally, a grouping factor \Co{g} for the
  coefficients. The variance covariate is ignored in this variance
  function. When a grouping factor is present in \Co{form}, a
  different coefficient value is used for each of its levels less one
  reference level (see description section below). Several grouping
  variables may be simultaneously specified, separated by the \Co{*}
  operator, like in \Co{{\Twiddle} v | g1 * g2 * g3}. In this case,
  the levels of each grouping variable are pasted together and the
  resulting factor is used to group the observations. Defaults to
  \Co{\Twiddle 1}.
\item[\Co{fixed:}]
an optional numeric vector, or list of numeric values,
specifying the values at which some or all of the  coefficients in
the variance function should be fixed. It must have names
identifying which coefficients are to be fixed. Coefficients
included in \Co{fixed} are not allowed to vary during the
optimization of an objective function. Defaults to \Co{NULL},
corresponding to no fixed coefficients.
\end{Argument}
\Paragraph{VALUE}
a \Co{varIdent} object representing a constant variance function
structure, also inheriting from class \Co{varFunc}.
\Paragraph{SEE ALSO}
\Co{varWeights.varFunc}, \Co{coef.varIdent}
\need 15pt
\Paragraph{EXAMPLE}
\vspace{-16pt} 
\begin{Example}
vf1 <- varIdent(c(F = 0.5), form = \Twiddle 1 | Sex)
\end{Example}
\end{Helpfile}
\begin{Helpfile}{Variogram}{Calculate Semi-Variogram}
This function is generic; method functions can be written to handle
specific classes of objects. Classes which already have methods for
this function include \Co{default}, \Co{gls} and \Co{lme}. See
the appropriate method documentation for a description of the
arguments.
\begin{Example}
Variogram(object, distance, ...)
\end{Example}
\Paragraph{VALUE}
will depend on the method function used; see the appropriate documentation.
\Paragraph{REFERENCES}
Cressie, N.A.C. (1993), "Statistics for Spatial Data", J. Wiley \& Sons.\\
Diggle, P.J., Liang, K.Y. and Zeger, S. L. (1994) "Analysis of
Longitudinal Data", Oxford University Press Inc. 
\Paragraph{SEE ALSO}
\Co{Variogram.default},\Co{Variogram.gls},
\Co{Variogram.lme}, \Co{plot.Variogram}
\need 15pt
\Paragraph{EXAMPLE}
\vspace{-16pt}
\begin{Example}
## see the method function documentation
\end{Example}
\end{Helpfile}
\begin{Helpfile}{Variogram.corExp}{Calculate Semi-Variogram for a corExp Object}
  This method function calculates the semi-variogram values
  corresponding to the Exponential correlation model, using the
  estimated coefficients corresponding to \Co{object}, at the
  distances defined by \Co{distance}.
\begin{Example}
Variogram(object, distance, sig2, length.out)
\end{Example}
\begin{Argument}{ARGUMENTS}
\item[\Co{object:}]
an object inheriting from class \Co{corExp},
representing an exponential spatial correlation structure.
\item[\Co{distance:}]
an optional numeric vector with the distances at
which the semi-variogram is to be calculated. Defaults to
\Co{NULL}, in which case a sequence of length \Co{length.out}
between the minimum and maximum values of
\Co{getCovariate(object)} is used.
\item[\Co{sig2:}]
an optional numeric value representing the process
variance. Defaults to \Co{1}.
\item[\Co{length.out:}]
an optional integer specifying the length of the
sequence of distances to be used for calculating the semi-variogram,
when \Co{distance = NULL}. Defaults to \Co{50}.
\end{Argument}
\Paragraph{VALUE}
a data frame with columns \Co{variog} and \Co{dist} representing,
respectively, the semi-variogram values and the corresponding
distances. The returned value inherits from class \Co{Variogram}.
\Paragraph{REFERENCES}
Cressie, N.A.C. (1993), "Statistics for Spatial Data", J. Wiley \& Sons.
\Paragraph{SEE ALSO}
\Co{corExp}, \Co{plot.Variogram}
\need 15pt
\Paragraph{EXAMPLE}
\vspace{-16pt}
\begin{Example}
cs1 <- corExp(3, form = {\Twiddle} Time | Rat)
cs1 <- initialize(cs1, BodyWeight)
Variogram(cs1)[1:10,]
\end{Example}
\end{Helpfile}
\begin{Helpfile}{Variogram.corGaus}{Calculate Semi-Variogram for a corGaus Object}
This method function calculates the semi-variogram values
corresponding to the Gaussian correlation model, using the estimated
coefficients corresponding to \Co{object}, at the distances defined
by \Co{distance}.
\begin{Example}
Variogram(object, distance, sig2, length.out)
\end{Example}
\begin{Argument}{ARGUMENTS}
\item[\Co{object:}]
an object inheriting from class \Co{corGaus},
representing an Gaussian spatial correlation structure.
\item[\Co{distance:}]
an optional numeric vector with the distances at
which the semi-variogram is to be calculated. Defaults to
\Co{NULL}, in which case a sequence of length \Co{length.out}
between the minimum and maximum values of
\Co{getCovariate(object)} is used.
\item[\Co{sig2:}]
an optional numeric value representing the process
variance. Defaults to \Co{1}.
\item[\Co{length.out:}]
an optional integer specifying the length of the
sequence of distances to be used for calculating the semi-variogram,
when \Co{distance = NULL}. Defaults to \Co{50}.
\end{Argument}
\Paragraph{VALUE}
a data frame with columns \Co{variog} and \Co{dist} representing,
respectively, the semi-variogram values and the corresponding
distances. The returned value inherits from class \Co{Variogram}.
\Paragraph{REFERENCES}
Cressie, N.A.C. (1993), "Statistics for Spatial Data", J. Wiley \& Sons.
\Paragraph{SEE ALSO}
\Co{corGaus}, \Co{plot.Variogram}
\need 15pt
\Paragraph{EXAMPLE}
\vspace{-16pt}
\begin{Example}
cs1 <- corGaus(3, form = {\Twiddle} Time | Rat)
cs1 <- initialize(cs1, BodyWeight)
Variogram(cs1)[1:10,]
\end{Example}
\end{Helpfile}
\begin{Helpfile}{Variogram.corLin}{Calculate Semi-Variogram for a corLin Object}
This method function calculates the semi-variogram values
corresponding to the Linear correlation model, using the estimated
coefficients corresponding to \Co{object}, at the distances defined
by \Co{distance}.
\begin{Example}
Variogram(object, distance, sig2, length.out)
\end{Example}
\begin{Argument}{ARGUMENTS}
\item[\Co{object:}]
an object inheriting from class \Co{corLin},
representing an Linear spatial correlation structure.
\item[\Co{distance:}]
an optional numeric vector with the distances at
which the semi-variogram is to be calculated. Defaults to
\Co{NULL}, in which case a sequence of length \Co{length.out}
between the minimum and maximum values of
\Co{getCovariate(object)} is used.
\item[\Co{sig2:}]
an optional numeric value representing the process
variance. Defaults to \Co{1}.
\item[\Co{length.out:}]
an optional integer specifying the length of the
sequence of distances to be used for calculating the semi-variogram,
when \Co{distance = NULL}. Defaults to \Co{50}.
\end{Argument}
\Paragraph{VALUE}
a data frame with columns \Co{variog} and \Co{dist} representing,
respectively, the semi-variogram values and the corresponding
distances. The returned value inherits from class \Co{Variogram}.
\Paragraph{REFERENCES}
Cressie, N.A.C. (1993), "Statistics for Spatial Data", J. Wiley \& Sons.
\Paragraph{SEE ALSO}
\Co{corLin}, \Co{plot.Variogram}
\need 15pt
\Paragraph{EXAMPLE}
\vspace{-16pt}
\begin{Example}
cs1 <- corLin(15, form = {\Twiddle} Time | Rat)
cs1 <- initialize(cs1, BodyWeight)
Variogram(cs1)[1:10,]
\end{Example}
\end{Helpfile}
\begin{Helpfile}{Variogram.corRatio}{Calculate Semi-Variogram for a
    corRatio Object} 
  This method function calculates the semi-variogram values
  corresponding to the Rational Quadratic correlation model, using the
  estimated coefficients corresponding to \Co{object}, at the
  distances defined by \Co{distance}.
\begin{Example}
Variogram(object, distance, sig2, length.out)
\end{Example}
\begin{Argument}{ARGUMENTS}
\item[\Co{object:}]
an object inheriting from class \Co{corRatio},
representing an Rational Quadratic spatial correlation structure.
\item[\Co{distance:}]
an optional numeric vector with the distances at
which the semi-variogram is to be calculated. Defaults to
\Co{NULL}, in which case a sequence of length \Co{length.out}
between the minimum and maximum values of
\Co{getCovariate(object)} is used.
\item[\Co{sig2:}]
an optional numeric value representing the process
variance. Defaults to \Co{1}.
\item[\Co{length.out:}]
an optional integer specifying the length of the
sequence of distances to be used for calculating the semi-variogram,
when \Co{distance = NULL}. Defaults to \Co{50}.
\end{Argument}
\Paragraph{VALUE}
a data frame with columns \Co{variog} and \Co{dist} representing,
respectively, the semi-variogram values and the corresponding
distances. The returned value inherits from class \Co{Variogram}.
\Paragraph{REFERENCES}
Cressie, N.A.C. (1993), "Statistics for Spatial Data", J. Wiley \& Sons.
\Paragraph{SEE ALSO}
\Co{corRatio}, \Co{plot.Variogram}
\need 15pt
\Paragraph{EXAMPLE}
\vspace{-16pt}
\begin{Example}
cs1 <- corRatio(7, form = {\Twiddle} Time | Rat)
cs1 <- initialize(cs1, BodyWeight)
Variogram(cs1)[1:10,]
\end{Example}
\end{Helpfile}
\begin{Helpfile}{Variogram.corSpher}{Calculate Semi-Variogram for a
    corSpher Object} This method function calculates the
  semi-variogram values corresponding to the Spherical correlation
  model, using the estimated coefficients corresponding to
  \Co{object}, at the distances defined by \Co{distance}.
\begin{Example}
Variogram(object, distance, sig2, length.out)
\end{Example}
\begin{Argument}{ARGUMENTS}
\item[\Co{object:}]
an object inheriting from class \Co{corSpher},
representing an Spherical spatial correlation structure.
\item[\Co{distance:}]
an optional numeric vector with the distances at
which the semi-variogram is to be calculated. Defaults to
\Co{NULL}, in which case a sequence of length \Co{length.out}
between the minimum and maximum values of
\Co{getCovariate(object)} is used.
\item[\Co{sig2:}]
an optional numeric value representing the process
variance. Defaults to \Co{1}.
\item[\Co{length.out:}]
an optional integer specifying the length of the
sequence of distances to be used for calculating the semi-variogram,
when \Co{distance = NULL}. Defaults to \Co{50}.
\end{Argument}
\Paragraph{VALUE}
a data frame with columns \Co{variog} and \Co{dist} representing,
respectively, the semi-variogram values and the corresponding
distances. The returned value inherits from class \Co{Variogram}.
\Paragraph{REFERENCES}
Cressie, N.A.C. (1993), "Statistics for Spatial Data", J. Wiley \& Sons.
\Paragraph{SEE ALSO}
\Co{corSpher}, \Co{plot.Variogram}
\need 15pt
\Paragraph{EXAMPLE}
\vspace{-16pt}
\begin{Example}
cs1 <- corSpher(15, form = {\Twiddle} Time | Rat)
cs1 <- initialize(cs1, BodyWeight)
Variogram(cs1)[1:10,]
\end{Example}
\end{Helpfile}
\begin{Helpfile}{Variogram.default}{Calculate Semi-Variogram}
  This method function calculates the semi-variogram for an arbitrary
  vector \Co{object}, according to the distances in \Co{distance}. For
  each pair of elements $x, y$ in \Co{object}, the corresponding
  semi-variogram is $(x-y)^2/2$.  The semi-variogram is useful for
  identifying and modeling spatial correlation structures in
  observations with constant expectation and constant variance.
\begin{Example}
Variogram(object, distance)
\end{Example}
\begin{Argument}{ARGUMENTS}
\item[\Co{object:}]
a numeric vector with the values to be used in
calculating the semi-variogram, usually a residual vector from a
fitted model.
\item[\Co{distance:}]
a numeric vector with the pairwise distances
corresponding to the elements of \Co{object}. The order of the
elements in \Co{distance} must correspond to the pairs
\Co{(1,2), (1,3), ..., (n-1,n)}, with \Co{n} representing the
length of \Co{object}, and must have length \Co{n(n-1)/2}.
\end{Argument}
\Paragraph{VALUE}
a data frame with columns \Co{variog} and \Co{dist} representing,
respectively, the semi-variogram values and the corresponding
distances. The returned value inherits from class \Co{Variogram}.
\Paragraph{REFERENCES}
Cressie, N.A.C. (1993), "Statistics for Spatial Data", J. Wiley \& Sons.\\
Diggle, P.J., Liang, K.Y. and Zeger, S. L. (1994) "Analysis of
Longitudinal Data", Oxford University Press Inc. 
\Paragraph{SEE ALSO}
\Co{Variogram.gls}, \Co{Variogram.lme},
\Co{plot.Variogram}
\need 15pt
\Paragraph{EXAMPLE}
\vspace{-16pt}
\begin{Example}
fm1 <- lm(follicles {\Twiddle} sin(2 * pi * Time) + cos(2 * pi * Time), 
         Ovary, subset = Mare == 1)
Variogram(resid(fm1), dist(1:29))[1:10,]
\end{Example}
\end{Helpfile}
\begin{Helpfile}{Variogram.gls}{Calculate Semi-Variogram of gls Residuals}
  This method function calculates the semi-variogram for the residuals
  from an \Co{gls} fit. The semi-variogram values are calculated for
  pairs of residuals within the same group level, if a grouping factor
  is present. If \Co{collapse} is different from \Co{"none"}, the
  individual semi-variogram values are collapsed using either a robust
  estimator (\Co{robust = TRUE}) defined in Cressie (1993), or the
  average of the values within the same distance interval. The
  semi-variogram is useful for modelling the error term correlation
  structure.
\begin{Example}
Variogram(object, distance, form, resType, data, na.action, maxDist,
          length.out, collapse, nint, breaks, robust, metric)
\end{Example}
\begin{Argument}{ARGUMENTS}
\item[\Co{object:}]
an object inheriting from class \Co{gls}, representing
a generalized least squares fitted model.
\item[\Co{distance:}]
an optional numeric vector with the distances between
residual pairs. If a grouping variable is present, only the
distances between residual pairs within the same group should be
given. If missing, the distances are calculated based on the
values of the arguments \Co{form}, \Co{data}, and
\Co{metric}, unless \Co{object} includes a \Co{corSpatial}
element, in which case the associated covariate (obtained with the
\Co{getCovariate} method) is used.
\item[\Co{form:}]
an optional one-sided formula specifying the covariate(s)
to be used for calculating the distances between residual pairs and,
optionally, a grouping factor for partitioning the residuals (which
must appear to the right of a \Co{|} operator in
\Co{form}). Default is \Co{{\Twiddle}1}, implying that the observation 
order within the groups is used to obtain the distances.
\item[\Co{resType:}]
an optional character string specifying the type of
residuals to be used. If \Co{"response"}, the "raw" residuals
(observed - fitted) are used; else, if \Co{"pearson"}, the
standardized residuals (raw residuals divided by the corresponding
standard errors) are used; else, if \Co{"normalized"}, the
normalized residuals (standardized residuals pre-multiplied by the
inverse square-root factor of the estimated error correlation
matrix) are used. Partial matching of arguments is used, so only the
first character needs to be provided. Defaults to \Co{"pearson"}.
\item[\Co{data:}]
an optional data frame in which to interpret the variables
in \Co{form}. By default, the same data used to fit \Co{object}
is used.
\item[\Co{na.action:}]
a function that indicates what should happen when the
data contain \Co{NA}s. The default action (\Co{na.fail}) causes
an error message to be printed and the function to terminate, if there
are any incomplete observations.
\item[\Co{maxDist:}]
an optional numeric value for the maximum distance used
for calculating the semi-variogram between two residuals. By default
all residual pairs are included.
\item[\Co{length.out:}]
an optional integer value. When \Co{object}
includes a \Co{corSpatial} element, its semi-variogram values are
calculated and this argument is used as the \Co{length.out}
argument to the corresponding \Co{Variogram} method. Defaults to
\Co{50}.
\item[\Co{collapse:}]
an optional character string specifying the type of
collapsing to be applied to the individual semi-variogram values. If
equal to \Co{"quantiles"}, the semi-variogram values are split
according to quantiles of the distance distribution, with equal
number of observations per group, with possibly varying distance
interval lengths. Else, if \Co{"fixed"}, the semi-variogram values
are divided according to distance intervals of equal lengths, with
possibly different number of observations per interval. Else, if
\Co{"none"}, no collapsing is used and the individual
semi-variogram values are returned. Defaults to \Co{"quantiles"}.
\item[\Co{nint:}]
an optional integer with the number of intervals to be
used when collapsing the semi-variogram values. Defaults to \Co{20}.
\item[\Co{robust:}]
an optional logical value specifying if a robust
semi-variogram estimator should be used when collapsing the
individual values. If \Co{TRUE} the robust estimator is
used. Defaults to \Co{FALSE}.
\item[\Co{breaks:}]
an optional numeric vector with the breakpoints for the
distance intervals to be used in collapsing the semi-variogram
values. If not missing, the option specified in \Co{collapse} is
ignored.
\item[\Co{metric:}]
an optional character string specifying the distance
metric to be used. The currently available options are
\Co{"euclidian"} for the root sum-of-squares of distances;
\Co{"maximum"} for the maximum difference; and \Co{"manhattan"}
for the sum of the absolute differences. Partial matching of
arguments is used, so only the first three characters need to be
provided. Defaults to \Co{"euclidian"}.
\end{Argument}
\Paragraph{VALUE}
a data frame with columns \Co{variog} and \Co{dist} representing,
respectively, the semi-variogram values and the corresponding
distances. If the semi-variogram values are collapsed, an extra
column, \Co{n.pairs}, with the number of residual pairs used in each
semi-variogram calculation, is included in the returned data frame. If
\Co{object} includes a \Co{corSpatial} element, a data frame with its
corresponding semi-variogram is included in the returned value, as an
attribute \Co{"modelVariog"}. The returned value inherits from class
\Co{Variogram}.
\Paragraph{REFERENCES}
Cressie, N.A.C. (1993), "Statistics for Spatial Data", J. Wiley \& Sons.\\
Diggle, P.J., Liang, K.Y. and Zeger, S. L. (1994) "Analysis of
Longitudinal Data", Oxford University Press Inc. 
\Paragraph{SEE ALSO}
\Co{gls}, \Co{Variogram.default},
\Co{Variogram.gls}, \Co{plot.Variogram}
\need 15pt
\Paragraph{EXAMPLE}
\vspace{-16pt}
\begin{Example}
fm1 <- gls(weight {\Twiddle} Time * Diet, BodyWeight)
Variogram(fm1, form = {\Twiddle} Time | Rat, nint = 10, robust = TRUE)
\end{Example}
\end{Helpfile}
\begin{Helpfile}{Variogram.lme}{Calculate Semi-Variogram for Residuals
    from an lme Object} 
This method function calculates the semi-variogram for the
within-group residuals from an \Co{lme} fit. The semi-variogram
values are calculated for pairs of residuals within the same group. If
\Co{collapse} is different from \Co{"none"}, the individual
semi-variogram values are collapsed using either a robust estimator
(\Co{robust = TRUE}) defined in Cressie (1993), or the average of
the values within the same distance interval. The semi-variogram is
useful for modeling the error term correlation structure.
\begin{Example}
Variogram(object, distance, form, resType, data, na.action, maxDist,
          length.out, collapse, nint, breaks, robust, metric)
\end{Example}
\begin{Argument}{ARGUMENTS}
\item[\Co{object:}]
an object inheriting from class \Co{lme}, representing
a fitted linear mixed-effects model.
\item[\Co{distance:}]
an optional numeric vector with the distances between
residual pairs. If a grouping variable is present, only the
distances between residual pairs within the same group should be
given. If missing, the distances are calculated based on the
values of the arguments \Co{form}, \Co{data}, and
\Co{metric}, unless \Co{object} includes a \Co{corSpatial}
element, in which case the associated covariate (obtained with the
\Co{getCovariate} method) is used.
\item[\Co{form:}]
an optional one-sided formula specifying the covariate(s)
to be used for calculating the distances between residual pairs and,
optionally, a grouping factor for partitioning the residuals (which
must appear to the right of a \Co{|} operator in
\Co{form}). Default is \Co{{\Twiddle}1}, implying that the observation 
order within the groups is used to obtain the distances.
\item[\Co{resType:}]
an optional character string specifying the type of
residuals to be used. If \Co{"response"}, the "raw" residuals
(observed - fitted) are used; else, if \Co{"pearson"}, the
standardized residuals (raw residuals divided by the corresponding
standard errors) are used; else, if \Co{"normalized"}, the
normalized residuals (standardized residuals pre-multiplied by the
inverse square-root factor of the estimated error correlation
matrix) are used. Partial matching of arguments is used, so only the
first character needs to be provided. Defaults to \Co{"pearson"}.
\item[\Co{data:}]
an optional data frame in which to interpret the variables
in \Co{form}. By default, the same data used to fit \Co{object}
is used.
\item[\Co{na.action:}]
a function that indicates what should happen when the
data contain \Co{NA}s. The default action (\Co{na.fail}) causes
an error message to be printed and the function to terminate, if there
are any incomplete observations.
\item[\Co{maxDist:}]
an optional numeric value for the maximum distance used
for calculating the semi-variogram between two residuals. By default
all residual pairs are included.
\item[\Co{length.out:}]
an optional integer value. When \Co{object}
includes a \Co{corSpatial} element, its semi-variogram values are
calculated and this argument is used as the \Co{length.out}
argument to the corresponding \Co{Variogram} method. Defaults to
\Co{50}.
\item[\Co{collapse:}]
an optional character string specifying the type of
collapsing to be applied to the individual semi-variogram values. If
equal to \Co{"quantiles"}, the semi-variogram values are split
according to quantiles of the distance distribution, with equal
number of observations per group, with possibly varying distance
interval lengths. Else, if \Co{"fixed"}, the semi-variogram values
are divided according to distance intervals of equal lengths, with
possibly different number of observations per interval. Else, if
\Co{"none"}, no collapsing is used and the individual
semi-variogram values are returned. Defaults to \Co{"quantiles"}.
\item[\Co{nint:}]
an optional integer with the number of intervals to be
used when collapsing the semi-variogram values. Defaults to \Co{20}.
\item[\Co{robust:}]
an optional logical value specifying if a robust
semi-variogram estimator should be used when collapsing the
individual values. If \Co{TRUE} the robust estimator is
used. Defaults to \Co{FALSE}.
\item[\Co{breaks:}]
an optional numeric vector with the breakpoints for the
distance intervals to be used in collapsing the semi-variogram
values. If not missing, the option specified in \Co{collapse} is
ignored.
\item[\Co{metric:}]
an optional character string specifying the distance
metric to be used. The currently available options are
\Co{"euclidian"} for the root sum-of-squares of distances;
\Co{"maximum"} for the maximum difference; and \Co{"manhattan"}
for the sum of the absolute differences. Partial matching of
arguments is used, so only the first three characters need to be
provided. Defaults to \Co{"euclidian"}.
\end{Argument}
\Paragraph{VALUE}
a data frame with columns \Co{variog} and \Co{dist} representing,
respectively, the semi-variogram values and the corresponding
distances. If the semi-variogram values are collapsed, an extra
column, \Co{n.pairs}, with the number of residual pairs used in each
semi-variogram calculation, is included in the returned data frame. If
\Co{object} includes a \Co{corSpatial} element, a data frame with
its corresponding semi-variogram is included in the returned value, as
an attribute \Co{"modelVariog"}. The returned value inherits from
class \Co{Variogram}.
\Paragraph{REFERENCES}
Cressie, N.A.C. (1993), "Statistics for Spatial Data", J. Wiley \& Sons.
Diggle, P.J., Liang, K.Y. and Zeger, S. L. (1994) "Analysis of
Longitudinal Data", Oxford University Press Inc. 
\Paragraph{SEE ALSO}
\Co{lme}, \Co{Variogram.default},
\Co{Variogram.gls}, \Co{plot.Variogram}
\need 15pt
\Paragraph{EXAMPLE}
\vspace{-16pt}
\begin{Example}
fm1 <- lme(weight {\Twiddle} Time * Diet, BodyWeight, {\Twiddle} Time | Rat)
Variogram(fm1, form = {\Twiddle} Time | Rat, nint = 10, robust = TRUE)
\end{Example}
\end{Helpfile}
\begin{Helpfile}{varPower}{Power Variance Function}
This function is a constructor for the \Co{varPower} class,
representing a power variance function structure. Letting
$v$ denote the variance covariate and $\sigma^2(v)$
denote the variance function evaluated at $v$, the power
variance function is defined as $\sigma^2(v) = |v|^{2\theta}$, where
$\theta$ is the variance function coefficient. When a grouping factor
is present, a different $\theta$ is used for each factor level.
\begin{Example}
varPower(value, form, fixed)
\end{Example}
\begin{Argument}{ARGUMENTS}
\item[\Co{value:}]
an optional numeric vector, or list of numeric values,
with the variance function coefficients. \Co{Value} must have
length one, unless a grouping factor is specified in \Co{form}.
If \Co{value} has length greater than one, it must have names
which identify its elements to the levels of the grouping factor
defined in \Co{form}. If a grouping factor is present in
\Co{form} and \Co{value} has length one, its value will be
assigned to all grouping levels. Default is \Co{numeric(0)}, which
results in a vector of zeros of appropriate length being assigned to
the coefficients when \Co{object} is initialized (corresponding
to constant variance equal to one).
\item[\Co{form:}] an optional one-sided formula of the form
  \Co{\Twiddle v}, or \Co{\Twiddle v | g}, specifying a variance
  covariate \Co{v} and, optionally, a grouping factor \Co{g} for the
  coefficients. The variance covariate must evaluate to a numeric
  vector and may involve expressions using \Co{"."}, representing a
  fitted model object from which fitted values (\Co{fitted(.)}) and
  residuals (\Co{resid(.)}) can be extracted (this allows the variance
  covariate to be updated during the optimization of an objective
  function). When a grouping factor is present in \Co{form}, a
  different coefficient value is used for each of its levels. Several
  grouping variables may be simultaneously specified, separated by the
  \Co{*} operator, like in \Co{{\Twiddle} v | g1 * g2 * g3}. In this
  case, the levels of each grouping variable are pasted together and
  the resulting factor is used to group the observations. Defaults to
  \Co{\Twiddle fitted(.)} representing a variance covariate given by
  the fitted values of a fitted model object and no grouping factor.
\item[\Co{fixed:}]
an optional numeric vector, or list of numeric values,
specifying the values at which some or all of the  coefficients in
the variance function should be fixed. If a grouping factor is
specified in \Co{form}, \Co{fixed} must have names identifying
which coefficients are to be fixed. Coefficients included in
\Co{fixed} are not allowed to vary during the optimization of an
objective function. Defaults to \Co{NULL}, corresponding to no
fixed coefficients.
\end{Argument}
\Paragraph{VALUE}
a \Co{varPower} object representing a power variance function
structure, also inheriting from class \Co{varFunc}.
\Paragraph{SEE ALSO}
\Co{varWeights.varFunc}, \Co{coef.varPower}
\need 15pt
\Paragraph{EXAMPLE}
\vspace{-16pt} 
\begin{Example}
vf1 <- varPower(0.2, form = \Twiddle age|Sex)
\end{Example}
\end{Helpfile}
\begin{Helpfile}{varWeights}{Extract Variance Function Weights}
The inverse of the standard deviations corresponding to the variance
function structure represented by \Co{object} are returned.
\begin{Example}
varWeights(object)
\end{Example}
\begin{Argument}{ARGUMENTS}
\item[\Co{object:}]
an object inheriting from class \Co{varFunc},
representing a variance function structure.
\end{Argument}
\Paragraph{VALUE}
if \Co{object} has a \Co{weights} attribute, its value is
returned; else \Co{NULL} is returned.
\Paragraph{SEE ALSO}
\Co{logLik.varFunc}
\need 15pt
\Paragraph{EXAMPLE}
\vspace{-16pt} 
\begin{Example}
vf1 <- varPower(form=\Twiddle age)
vf1 <- initialize(vf1, Orthodont)
coef(vf1) <- 0.3
varWeights(vf1)[1:10]
\end{Example}
\end{Helpfile}
\begin{Helpfile}{varWeights.glsStruct}{glsStruct Variance Weights}
If \Co{object} includes a \Co{varStruct} component, the inverse of
the standard deviations of the variance function structure represented
by the corresponding \Co{varFunc} object are returned; else, a
vector of ones of length equal to the number of observations in the
data frame used to fit the associated linear model is returned.
\begin{Example}
varWeights(object)
\end{Example}
\begin{Argument}{ARGUMENTS}
\item[\Co{object:}]
an object inheriting from class \Co{glsStruct},
representing a list of linear model components, such as
\Co{corStruct} and \Co{varFunc} objects.
\end{Argument}
\Paragraph{VALUE}
if \Co{object} includes a \Co{varStruct} component, a vector with
the corresponding variance weights; else, or a vector of ones.
\Paragraph{SEE ALSO}
\Co{varWeights}
\end{Helpfile}
\begin{Helpfile}{varWeights.lmeStruct}{lmeStructVariance Weights}
If \Co{object} includes a \Co{varStruct} component, the inverse of
the standard deviations of the variance function structure represented
by the corresponding \Co{varFunc} object are returned; else, a
vector of ones of length equal to the number of observations in the
data frame used to fit the associated linear mixed-effects model is
returned.
\begin{Example}
varWeights(object)
\end{Example}
\begin{Argument}{ARGUMENTS}
\item[\Co{object:}]
an object inheriting from class \Co{lmeStruct},
representing a list of linear mixed-effects model components, such as
\Co{reStruct}, \Co{corStruct}, and \Co{varFunc} objects.
\end{Argument}
\Paragraph{VALUE}
if \Co{object} includes a \Co{varStruct} component, a vector with
the corresponding variance weights; else, or a vector of ones.
\Paragraph{SEE ALSO}
\Co{varWeights}
\end{Helpfile}

\end{document}


% Local Variables: 
% mode: latex
% TeX-master: t
% End: 
